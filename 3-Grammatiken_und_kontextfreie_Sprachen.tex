\section{Grammatiken und kontextfreie Sprachen}
\datenote{18.11.16}
Im folgenden Kapitel wechseln wir den Standpunkt von Spracherkennung auf \emph{Spracherzeugung}.
Das Werkzeug sind hierbei sogenannte \emph{Phasenstrukturgrammatiken}, oder kurz, \emph{Grammatiken}.
Bei Grammatiken gibt es neben dem Alphabet weitere Symbole, sogenannte \emph{Nichtterminale} oder \emph{Variablen}, und ein Regelsystem mit dem Worte, die Nichtterminale enthalten, geändert werden können.
\begin{Def}[Chomsky]
	Eine \emph{Grammatik} ist ein 4-Tupel $(N,\Sigma,P,S)$
	\begin{itemize}
	\item $N$ ist eine endliche Menge von \emph{Nichtterminalsymbolen}.
	\item $\Sigma$ ist ein Alphabet von \emph{Terminalsymbolen} (Variablen).
	\item $P\subset (N\cup\Sigma)^*N(N\cup\Sigma)^* \x (N\cup\Sigma)^*$ ist endliche Menge von Regeln, sogenannte \emph{Produktionen}.
	\item $S\in N$ ist das \emph{Startsymbol}.
	\end{itemize}
\end{Def}
\begin{Bsp}
  $\mathcal{G} = (N, \Sigma, P, E)$ mit
	\begin{align*}
		\Sigma &= \{ \mathtt{a},\mathtt{(},\mathtt{)},\mathtt{*},\mathtt{+} \}\\
		N &= \{E\}\\
		P &= \begin{aligned}[t]
      \{ E &\-> a,\\
      E &\-> (E*E),\\
      E &\-> (E+E) \}
        \end{aligned}
	\end{align*}
\end{Bsp}
  $G$ erzeugt geklammerte arithmetische Ausdrücke über der Konstante $\mathtt{a}$, wie zum Beispiel das Wort $\mathtt{(a*(a+a))}$.
  Eine Grammatik erzeugt Worte durch \emph{Ableitungen}, die wir im Folgenden definieren.
\begin{Def}[Ableitungsrelation, Ableitung, Sprache einer Grammatik][name={[Ableitungsrelation]}]
  Sei $\mathcal{G} =(N,\Sigma,P,S)$ eine Grammatik.

	Die \emph{Ableitungsrelation} zu $\mathcal{G}$ ist 
  \begin{displaymath}
    \cdot \==>_\mathcal{G} \cdot \subseteq (N\cup\Sigma)^*\x(N\cup\Sigma)^*
  \end{displaymath}
  mit $\alpha \==>_\mathcal{G} \beta$ gdw 
  \begin{itemize}
  \item $\alpha = \gamma_1\alpha'\gamma_2$,
  \item $\beta = \gamma_1\beta'\gamma_2$ und
  \item $\alpha' \to \beta' \in P$
  \end{itemize}

  Eine Folge $\alpha = \alpha_0,\ldots,\alpha_n = \beta$ heißt \emph{Ableitung von $\beta$ aus $\alpha$ in $n$ Schritten}, geschrieben $\alpha \stackrel{n}{\==>}_\mathcal{G} \beta$, gdw $\alpha_i \==>_\mathcal{G} \alpha_{i+1}$ für $0 \le i < n$.
  Jedes solche $\alpha_i$ heißt \emph{Satzform von $\mathcal{G}$}.

  Die \emph{Ableitung von $\beta$ aus $\alpha$}, geschrieben $\alpha \stackrel{*}{\==>}_\mathcal{G} \beta$, existiert gdw ein $n \in \N$ existiert, so dass $\alpha \stackrel{n}{\==>}_\mathcal{G} \beta$.
  Damit ist ,,$\stackrel{*}{\==>}_\mathcal{G}$'' die reflexive, transitive Hülle von ,,$\==>_\mathcal{G}$''.

	Die \emph{Sprache, die von $\mathcal{G}$ erzeugt wird} ist definiert als:
	\[ L(\mathcal{G}) = \{w\in\Sigma^* \mid S \stackrel{*}{\==>}_{\mathcal{G}} w \}\]
	
\end{Def}

\draftnote{18.11.16}

\begin{Bsp}
  \begin{align*}
		E &\=> a\\
		E &\=> (E*E) \=> (a*E)\\
		&\=> (a*(E+E)) \=> \dots \=> (a*(a+a))
  \end{align*}
\end{Bsp}

\begin{Bsp} $\mathcal{G}=(N,\Sigma,P,S)$ mit
	\begin{align*}
		N &= \{S,B,C\}\\
		\Sigma &= \{a,b,c\}\\8
		P &= 
		\begin{aligned}[t]
			 \{ &S \-> aSBC,\ S\->aBC,\ CB\-> BC,\ aB\-> ab,\\
			 &bB\-> bb, bC\-> bc, cC\-> cc \} & \quad\text{Startsymbol }S
		\end{aligned}\\
		L(\mathcal{G}) &=  \{ a^nb^nc^n \mid n\geq 1 \}\\
		S &\=> aBC \=> abC \=> abc \qquad S\=>a\emph{S}BC \=> aa\emph{S}BCBC\\
		S &\=> aSBC \xRightarrow{n-2} n \=> a^{n-1}S(BC)^{n-1} \=> a^n(BC)^n = a^nBCBC \dots \=> a^nBBCCBC\\
		&\=> a^nB^nC^n \xRightarrow* a^nb^nc^n
	\end{align*}
\end{Bsp}
Die Chomsky-Hierarchie teilt die Grammatiken in vier Typen unterschiedlicher Mächtigkeit ein.
\begin{Def}[Chomsky Hierarchie]\
	\begin{itemize}
	\item Jede Grammatik ist eine \emph{Typ-0 Grammatik}.
	\item Eine Grammatik ist \emph{Typ-1} oder \emph{kontextsensitiv}, falls alle Regeln expansiv sind, d.h. $\alpha\->\beta\in P$, dann ist $|\alpha|\leq |\beta|$ mit einer Ausnahme: falls $S$ nicht in einer rechten Regelseite auftritt, dann ist $S\->\Eps$ erlaubt.
	\item Eine Grammatik heißt \emph{Typ-2} oder \emph{kontextfrei}, falls alle Regeln die Form $A\->\alpha$ mit $A\in N$ und $\alpha\in(N\cup\Sigma)^*$ haben.
	\item Eine Grammatik heißt \emph{Typ-3} oder \emph{regulär}, falls alle Regeln die Form
	\begin{alignat*}{2}
		\text{Form}\quad &A\->w &\quad w&\in\Sigma^*\\
		\text{oder}\quad &A\->aB& a&\in\Sigma,\ B\in N
	\end{alignat*}
	\end{itemize}
	Eine Sprache heißt Typ-$i$ Sprache, falls $\exists$ Typ-$i$ Grammatik für sie.
\end{Def}

\begin{Beobachtung}
	Jede Typ $i+1$ Sprache ist nicht Typ-$i$ Sprache.
\end{Beobachtung}
Jede Typ-3 Grammatik ist Typ-2 Grammatik.\\
Jede Typ-2 Grammatik kann in äquivalente $\Eps$-freie Typ-2 Grammatik transformiert werden. \-> Typ-1 Grammatik. (Vgl.\ Elmination von $\Eps$-Produktionen)

Ziel: Hierarchie-Satz (Chomsky)\\
Sei $\mathcal{L}_i$ die Menge der Typ-$i$ Sprachen.\\
Es gilt $\mathcal{L}_3 \subsetneq \mathcal{L}_2 \subsetneq\mathcal{L}_1 \subsetneq\mathcal{L}_0$

\draftnote{25.11.16}
\begin{Satz}[name={[Typ-3 Sprache ist regulär]}]
	$L$ ist regulär \<=> $L$ ist Type-3 Sprache
\end{Satz}
\begin{proof}
	\begin{description}[font=\normalfont,labelwidth=\widthof{"'\=>"':},leftmargin=!]
	\item["`\=>"'] Sei $\A =(Q,\Sigma,\delta,q_0,F)$ \ac{DFA} für $L$.\\
		Konstruiere Grammatik $\mathcal{G}=(N,\Sigma,P,S)$
		\begin{alignat*}{2}
			N &=Q &\qquad S&=q_0\\
			q&\in Q\ \mathrlap{\forall a: \delta(q,a)=q'\ \curvearrowright\ q\->aq'\in P}\\
			q&\in F & q&\->\Eps\in P
		\end{alignat*}
		Zeige noch $L(\mathcal{G})=L(\A)$
	\item["`\<="'] Sei $\mathcal{G}=(N,\Sigma,P,S)$ Typ-3 Grammatik
	Def. $\A=(Q,\Sigma,\delta,q_0,F)$ \ac{NFA}
	\begin{alignat*}{2}
		Q &=N\cup\dots\\
		q_0 &=S\\
		A &\->aB\in P & \delta(A,a)&\ni B\\
		A &\->a,\dots a_n &\qquad n=0\=>A &\in F\\
		&\mathclap{\tikz[shorten >=1pt,on grid,node distance=1.5cm, every state/.style={minimum size=0pt,inner sep=0pt}]{
			\node[state,inner sep=3pt] (q0) {$A$};
			\node [above=0pt of q0.north] {$n>0$: $n$ weitere Zust"ande erforderlich, letzter Endzustand};
			\node[state] (q1) [right=of q0] {};
			\node (dots) [right=of q1] {\dots};
			\node[state,accepting] (qa) [right=of dots] {};
			\path[-stealth] (q0) edge node[auto] {$a_1$} (q1)
				(q1)   edge node[auto] {$a_2$} (dots)
				(dots) edge node[auto] {$a_n$} (qa);
		}}\\
		\text{Zeige noch }\quad
		L(\mathcal{G})&=L(\A) \tag*{\qedhere}
	\end{alignat*}
	\end{description}
\end{proof}


\begin{lemma}
	$\mathcal{L}_3 \subsetneq \mathcal{L}_2$
\end{lemma}
\begin{proof}
	Betrachte $L=\{a^nb^n \mid n\in\N \}$\\
	bekannt: $L$ nicht regulär.\\
	Aber $\exists$ Typ-2 Grammatik für $L$:
	\[ \mathcal{G} = (\{S\}, \{a,b\}, \{S\->\Eps, S\->aSb \}, S) \]
	Zeige (selbst) $L=L(\mathcal{G})$
\end{proof}

\datenote{30.11.16}
\subsection{Kontextfreie Sprachen}
Hier sind einige Beispiele kontextfreier Sprachen und Grammatiken:
\begin{itemize}
\item Arithmetische Ausdrücke: $(\{E\}, \{a, \mathnormal{+}, \mathnormal{*}, \mathnormal{(}, \mathnormal{)}\}, P, E)$ mit
  \begin{displaymath}
    P =
    \begin{aligned}[t]
      \{ E &\->a, \\
        E &\->(E+E), \\
        E &\->(E*E) \}
    \end{aligned}
  \end{displaymath}
\item Syntax von Programmiersprachen%\\[\abovedisplayskip]
	\begin{center}
		\begin{tabular}[t]{r@{ }l}
			<Stmt> &\-> <Var> = <Exp>\\
			&| <Stmt>\,;\,<Stmt>\\
			&| if\,(<Exp>) <Stmt>\\
			&\phantom{|} else <Stmt>\\
			&| while\,(<Exp>) <Stmt>
		\end{tabular}
	\end{center}%\\[\belowdisplayskip]
	Hier: Nichtterminal$\hat=$ Wort in spitzen Klammern <Stmt>;
	Terminalsymbol --- alles andere.
\item Palindrome über $\{a,b\}$
	\[ S\-> aSa \mid bSb \mid a \mid b \mid \Eps \]
\item Sprache der Worte über $\{0,1\}^*$, die gleich viele Nullen wie Einsen haben:
  \begin{displaymath}
    L = \{ w \in \Sigma^* \mid \#_0(w) = \#_1(w)\}
  \end{displaymath}
  Die Funktion $\#_a(w)$ berechnet hierbei die Anzahl der Vorkommen von $a \in \{0, 1\}$ in $w$.
  Eine Grammatik für $L$ ist $\mathcal{G} = (\{S\}, \{0,1\}, P, S)$ mit
  \begin{displaymath}
    P =
    \begin{aligned}[t]
      \{ S & \to 1S0S \\
        S & \to 0S1S \\
        S & \to \Eps
      \}
    \end{aligned}
  \end{displaymath}

  Das $L(\mathcal{G}) \subseteq L$ lässt sich per Induktion über die Länge der Ableitung von $S \stackrel{*}{\==>}_\mathcal{G} w$ zeigen.
  Der Beweis wird als Übung dem Leser überlassen.

  Wir zeigen das $L \subseteq L(\mathcal{G})$, also dass wenn $w \in L$, dann $w \in L(\mathcal{G})$ per Induktion über die Länge von $w$.
  Hierzu definieren wir die noch die Hilfsfunktion $d : \Sigma^* \to \N$ als
  \begin{align*}
    d(\Eps) &= 0 \\
    d(1w) &= d(w) + 1 \\
    d(0w) &= d(w) - 1
  \end{align*}
  Per Induktion über $w \in \Sigma^*$ lässt sich zeigen, dass $L = \{w \in \Sigma^* \mid d(w) = 0\}$ und $d(v \cdot w) = d(v) + d(w)$.
\begin{description}
\item[IA] $|w| = n = 0$, $w = \Eps$. Es gilt $\Eps \in L$, da $\#_0(\Eps) = \#_1(\Eps) = 0$.
\item[IV] $\forall n' < n: \forall w' \in \Sigma^*: falls |w| = n' \text{ und } w \in L \text{ dann } w \in L(\mathcal{G})$
\item[IS] $|w| = n > 1$, $w = aw'$, $a \in \{0,1\}$.

  Betrachte $a = 0$ (der Fall für $a = 1$ funktioniert analog).

  Da $d(w) = d(0w') = d(w') - 1 = 0$ ist $d(w') = 1$.

  Es stellt sich heraus, dass $w' = w_11w_2$, $d(w_1) = 0$ und $d(w_2) = 0$.
  \begin{itemize}
  \item[] 
    Sei $w' = a_1 \ldots a_n$.
    Betrachte die Folge $d_0,\ldots,d_n$ mit $d_0 = 0$ und $d_i = d(a_1\ldots a_i)$ für $1 \le i < n$.
    Wähle $0 \le i < n$ maximal, so dass für alle $0 \le j \le i$: $d_j < 1$.

    Da $i$ maximal ist folgt $d_{i+1} \ge 1$ und da $d_{i+1} - d_i \le 1$  folgt $d_i = 0$, $d_{i+1} = 1$ und $a_i = 1$.
    Setze also $w_1 = a_1\ldots a_{i}$.

    Es gilt also $w' = a_1\ldots a_{i}a_{i+1}w_2 = w_1w_2$ mit $d(w_1) = 0$.

    Da $d(v \cdot w) = d(v) + d(w)$ und $d(w') = d(w_11w_2) = 1$ folgt $d(w_2) = 0$.
  \end{itemize}

  Da $|w_1| < n$ und $|w_2| < n$, folgt per IV, dass $S \stackrel{*}{ \==>_{\mathcal{G}} } w_1$ und $S \stackrel{*}{\==>_{\mathcal{G}}} w_2$.

  Es folgt mit den Produktionsregeln $S \==>_{\mathcal{G}} 0S1S \stackrel{*}{\==>}_{\mathcal{G}} 0w_11S \stackrel{*}{\==>}_{\mathcal{G}} 0w_11w_2$.

\end{description}
\end{itemize}

%Abk.: \aclu{CFL}: \acs{CFL}\\
%\phantom{Abk.:} \aclu{CFG}: \acs{CFG}

\begin{Def}[name={[Ableitungsbaum]}] Sei $\mathcal{G} = (N, \Sigma, P, S)$ eine kontextfreie Grammatik.
  Definiere die Menge der \emph{Ableitungsbäume von $\mathcal{G}$ beginnend mit $A \in N$}, $\operatorname{Abl(G,A)}$, als Menge von markierten, geordneten Bäumen induktiv durch:
  \begin{itemize}
  \item[] Falls $\pi = A \to w_0A_1w_1\ldots A_nw_n \in P$ mit $A_i \in N$, $w_i \in \Sigma^*$, $0 \le i \le n$ und $\mathcal{A}_i \in \operatorname{Abl}(G, A_i)$ dann ist
    \begin{center}
			\tikz[baseline=0cm]{\Tree [.$\pi$ $\mathcal{A}_1$ \edge[draw=none]; {$\dots$} $\mathcal{A}_n$ ]} $\in \operatorname{Abl}(G, A)$
    \end{center}
    Abgekürzt: $\pi(\mathcal{A}_1, \ldots, \mathcal{A}_n)$

  \end{itemize}
    Das \emph{abgeleitete Wort zu einem $\mathcal{A} \in \operatorname{Abl}(G, A)$}, $Y(\mathcal{A})$ (,,yield'' von $\mathcal{A}$), ist definiert durch
    \begin{displaymath}
      Y\left(\tikz[baseline=-0.4cm]{\Tree [.$\pi$ $\mathcal{A}_1$ \edge[draw=none]; {$\dots$} $\mathcal{A}_n$ ]} \right) = w_oY(\mathcal{A}_1)w_1\ldots Y(\mathcal{A}_n)w_n
    \end{displaymath}
    wobei $\pi = A \to w_0A_1w_1\ldots A_nw_n \in P$
\end{Def}

\begin{Bsp}\label{bsp:Ableitungsbaum} 
  Einige Ableitungsbäume für die Grammatik $\mathcal{G} = (\{S\}, \{0,1\}, P, S)$ mit
  \begin{displaymath}
    P =
    \begin{aligned}[t]
      \{ S & \to 1S0S \\
        S & \to 0S1S \\
        S & \to \Eps
      \}
    \end{aligned}
  \end{displaymath}
  \begin{description}
  \item[$\operatorname{Abl}(\mathcal{G}, S) = $]\hfill\\
    \tikz[baseline=0cm]{\Tree [ .$S\to\Eps$ ]}
    \quad
    \tikz[baseline=0cm]{\Tree [ .$S\to0S1S$ $S\to\Eps$ $S\to\Eps$ ]}
    \quad
    \tikz[baseline=0cm]{\Tree [ .$S\to1S0S$ $S\to\Eps$ $S\to\Eps$ ]}
    \quad
    \tikz[baseline=0cm]{\Tree [ .$S\to0S1S$ $S\to\Eps$ [ .$S\to\Eps$ $S\to\Eps$ $S\to\Eps$ ] ]}
    \quad
    \dots
  \end{description}
  
	% \begin{figure}[H]\centering
	% 	\begin{subfigure}[t]{.25\linewidth}\centering
	% 		\tikz{\Tree [.$E$ $a$ ]}
	% 		\caption{$E \Rightarrow a$}
	% 	\end{subfigure}
	% 	\begin{subfigure}[t]{.4\linewidth}\centering
	% 		\begin{tikzpicture}[every tree node/.style={execute at begin node=$, execute at end node=$}]
	% 			\Tree [.E (
	% 		          [.E a ]
	% 		          +
	% 		          [.E ( [.E a ] * [.E a ] ) ]
	% 		          )
	% 		        ]
	% 		\end{tikzpicture}
	% 		\caption[$E \xRightarrow{*} (a+(a*a))$]{$E \xRightarrow{*} (a+(a*a)) \newline \longrightarrow E \Rightarrow (E+E)\Rightarrow(E+(E*E))\xRightarrow{*}$}
	% 	\end{subfigure}
	% 	\begin{subfigure}[t]{.3\linewidth}\centering
	% 		\begin{tikzpicture}[every tree node/.style={execute at begin node=$, execute at end node=$}]
	% 			\Tree [.A X_1 X_2 \edge[draw=none]; {\dots} X_n ]
	% 		\end{tikzpicture}
	% 		\caption{$A\rightarrow X_1\dots X_n$}
	% 	\end{subfigure}
	% 	\caption{Ableitungsbäume zu \autoref{bsp:Ableitungsbaum}}\label{fig:Ableitungsbäume}
	% \end{figure}
\end{Bsp}
\begin{lemma}
  Sei $\mathcal{G} = (N, \Sigma, P, S)$ eine kontextfreie Grammatik.

  \begin{displaymath}
    w \in L(\mathcal{G})  \text{ gdw } \exists \mathcal{A} \in \operatorname{Abl}(G, S) \text{ mit } Y(\mathcal{A}) = w
  \end{displaymath}
\end{lemma}

\begin{proof}[,,links nach rechts'']
  Zu zeigen:
  \begin{displaymath}
   \forall n \in \N: \forall w \in \Sigma^*: \forall A \in N: A \stackrel{n}{\==>} w \text{ dann } \exists \mathcal{A} \in \operatorname{Abl}(\mathcal{G}, A) \text{ mit } Y(\mathcal{A}) = w
\end{displaymath}
     Per Induktion über die Länge der Ableitung von $A \stackrel{n}{\==>} w \in \Sigma^*$:
   \begin{description}
   \item[IA] $n=0$ (nichts zu tun, da $A \stackrel{n}{\==>} \Eps$ unmöglich)
   \item[IS] Sei $n > 0$: $A \stackrel{n}{\==>}$ hat die Form
     \begin{displaymath}
       A {\==>} w_0A_1w_1\ldots A_nw_n \stackrel{n-1}{\==>} w
     \end{displaymath}
     Also ist $w = w_0v_1w_1\ldots v_nw_n$ mit $A_i \stackrel{n_i}{\==>} v_i$ für $1 \le i \le n$ und $n_i < n$.\footnote{Dieser Beweisschritt ist nicht ganz präzise\ldots die Ableitungen der $v_i$ sind nicht unbedingt Teil der gesamten Ableitung $A \stackrel{n}{\Longrightarrow} w$.
     Man kann aber beweisen dass eine alternative Ableitung existieren muss, bei der dies der Fall ist.}
   Nach IV ergibt sich
   \begin{displaymath}
     \exists \mathcal{A}_i \in \operatorname{Abl}(G, A_i) \text{ mit } Y(\mathcal{A}_i) = v_i
   \end{displaymath}
   Dann ist $\mathcal{A} = \pi(\mathcal{A}_1,\ldots, \mathcal{A}_n)$ mit $\pi = A \to w_0A_1w_1\ldots A_nw_n \in \operatorname{Abl}(\mathcal{G}, A)$ und $Y(\mathcal{A}) = w_0Y(\mathcal{A}_1)w_1\ldots Y(\mathcal{A}_n)w_n= w_0v_1w_1\ldots v_nw_n$.
   \end{description}
\end{proof}

\begin{proof}[,,rechts nach links'']
  Zu zeigen:
  \begin{displaymath}
    \forall n \in \N: \forall \mathcal{A} \in \operatorname{Abl}(\mathcal{G}, A): \text{falls } Y(\mathcal{A}) = w \text{ dann } A \stackrel{*}{\==>} w
  \end{displaymath}
  \begin{description}
  \item[IS]
\footnote{ Der Induktionsanfang ist bei Ableitungsbäumen ein Spezialfall des Induktionsschritts (Bäume ohne Kinder bzw.\ Regeln, die nur Terminale ableiten) und wird daher nicht gesondert aufgeführt }
    Falls für $n \ge 0$
    \begin{itemize}
    \item  $\mathcal{A} = \pi(\mathcal{A}_1, \ldots, \mathcal{A}_n)$,
    \item $\pi = w_0A_1w_1\ldots A_nw_n$,
    \item $\mathcal{A}_i \in \operatorname{Abl}(\mathcal{G}, A_i)$,
    \item $Y(\mathcal{A}) = w_0Y(\mathcal{A}_1)w_1\ldots Y(\mathcal{A}_n)w_n$
    \end{itemize}
    dann gilt per IV:\footnote{Strengenommen müsste die Folgerung, die hier mit ,,\ldots'' abgekürzt ist noch per Induktion über Ableitungen gezeigt werden} 
    \begin{align*}
      A &\==> w_0A_1\ldots A_nw_n \\
        &\stackrel{*}{\==>} w_0Y(\mathcal{A}_1)\ldots A_nw_n \\
        & \dots \\
        & \stackrel{*}{\==>} w_0Y(\mathcal{A}_1)\ldots Y(\mathcal{A}_n)w_n \\
        &= w
    \end{align*}
  \end{description}
  
\end{proof}

\begin{Def}[name={[Eindeutigkeit von \acs*{CFG} und \acs*{CFL}]}]\
	\begin{itemize}
	\item $\mathcal{G}\in$ \ac{CFG} heißt \emph{eindeutig}, falls es für jedes Wort $\in L(\mathcal{G})$ genau einen Ableitungsbaum gibt.
	\item Eine kontextfreie Sprache heißt \emph{eindeutig}, falls $\exists$ eindeutige \ac{CFG} für sie.
	\end{itemize}
\end{Def}
\begin{Bsp}
	\begin{align*}
		L &= \{ a^ib^jc^k \mid i=j\text{ oder }j=k \}\\
		S &\-> AC \mid DB\\
		A &\-> aAb \mid \Eps & D &\->aD \mid \Eps\\
		C &\->cC \mid \Eps & B &\-> bBc \mid \Eps \quad[\text{für $L$ gibt es keine eindeutige \ac{CFG}}]
	\end{align*}
	Werte der Form $a^nb^nc^n$ haben zwei Ableitungen $\curvearrowright$ Grammatik ist nicht eindeutig.
\end{Bsp}

\datenote{2.12.16}
\subsection{Die Chomsky Normalform für kontextfreie Sprachen}
Notwendig für das Pumping Lemma kontextfreier Sprachen und für einen effizienten Algorithmus für das \nameref{satz:wortproblem}.


\begin{Def}
  Eine CFG heißt \emph{separiert} wenn jede Produktion entweder die Form
  \begin{itemize}
  \item $A\-> A_1\dots A_n$ \qquad $A_i\in N,n\geq 0$, oder
  \item $A \-> a$ \quad $a\in\Sigma$
  \end{itemize}
  hat.
\end{Def}

\begin{lemma}[\textsc{Sep}]
  Zu jeder CFG gibt es eine äquivalente separierte CFG.
\end{lemma}
\begin{proof}
  Sei $\mathcal{G} = (N, \Sigma, P, S)$ eine CFG.
  Konstruiere $\mathcal{G}' = (N', \Sigma, P', S)$ mit
  \begin{itemize}
  \item $N' = N \uplus \{Y_a \mid a \in \Sigma \}$ (,,$\uplus$'' steht für ,,disjunkte Vereinigung'')
  \item $P' = \{Y_a \to a \mid a \in \Sigma \} \cup \{A \to \beta[a \to Y_a] \mid A \to \beta \in P \}$.
    Die Schreibweise $\beta[a\to Y_a]$ bedeutet hier dass alle $a \in \Sigma$, die in $\beta$ vorkommen durch $Y_a$ ersetzt werden.
  \end{itemize}
Offenbar gilt $L(\mathcal{G}) = L(\mathcal{G'})$ und $\mathcal{G}'$ ist separiert (ohne Beweis).
Der Aufwand für \textsc{Sep} beträgt $O(|\mathcal{G}|)$, wobei 
\begin{displaymath}
  |G| = \sum_{A \in N}\sum_{A \to \alpha \in P} |A\alpha|
\end{displaymath}
\end{proof}

\begin{lemma}[\textsc{Bin}]
  Zu jeder CFG gibt es eine äquivalente CFG, bei der für alle Produktionen $A \to \alpha$ gilt, dass $|\alpha| \le 2$.
\end{lemma}
\begin{proof}
  Ersetze jede Produktion der Form 
  \begin{displaymath}
    A \to X_1X_2\ldots X_n
  \end{displaymath}
  mit $X_i \in N \cup \Sigma$, $1 \le i \le n$,
  durch die Regeln 
  \begin{align*}
    A &\to X_1\langle  X_2\ldots X_n \rangle \\
    \langle  X_2\ldots X_n \rangle & \to X_2\langle  X_3\ldots X_n \rangle \\
    \vdots \\
    \langle  X_{n-1}X_n \rangle & \to X_{n-1}X_n
  \end{align*}
  Dabei sind  $\langle  X_2\ldots X_n \rangle$, \ldots, $\langle  X_{n-1}\ldots X_n \rangle$ neue Nichterminalsymbole.

  Der Aufwand von \textsc{Bin} liegt in $O(|G|)$.
\end{proof}


\begin{Def}
  Sie $\mathcal{G} = (N, \Sigma, P, S)$ eine CFG.
  Definiere die Menge $\operatorname{Nullable}(\mathcal{G}) \subseteq N$ als
  \begin{displaymath}
    \operatorname{Nullable}(\mathcal{G}) = \{ A \in N \mid A \stackrel{*}{\==>} \Eps \}
  \end{displaymath}
  Es handelt sich um die Menge an Nichtterminalen, aus denen das leere Wort abgeleitet werden kann.
\end{Def}

\begin{Satz}
  Es gibt einen Algorithmus, der $\operatorname{Nullable}(\mathcal{G})$ in $O(|\mathcal{G}|^2)$ ausrechnet.
\end{Satz}
\begin{proof}
  Definiere $M_i$ als die Menge der Nichtterminale, aus denen sich
  $\Eps$ mit einem Ableitungsbaum der Höhe $< i$ ableiten lässt:
  \begin{align*}
    M_0 &= \emptyset \\
    M_{i+1} &= M_i \cup \{ A \mid A \to \alpha \in P \text{ und }
              \alpha \in M_i^* \}
  \end{align*}
  Es gilt für alle $i \in \N$, dass $M_i \subseteq N$.
  Da $|N|$ endlich ist, existiert ein $m \in \N$, so dass
  \begin{displaymath}
    M_m = M_{m+1} = \bigcup_{i \in \N} M_i
  \end{displaymath}
  Das bedeutet, $\bigcup_{i \in \N} M_i$ lässt sich iterativ berechnen (s.u.).

  Wir zeigen nun, dass $\operatorname{Nullable}(\mathcal{G}) = \bigcup_{i \in \N} M_i$.

  \paragraph{,,$\supseteq$''}
  Es ist zu zeigen, dass $\forall i \in \N: M_i \subseteq \operatorname{Nullable}(\mathcal{G})$

  Per Induktion über $i$:
  \begin{description}
  \item[IA] $i = 0$, $M_0 = \emptyset \subseteq \operatorname{Nullable}(\mathcal{G})$
  \item[IS] $i \Rightarrow i+1$

    Wenn $A \in M_{i+1}$, dann ist
    \begin{itemize}
    \item entweder
      $A \in M_i \subseteq \operatorname{Nullable}(\mathcal{G})$ nach
      IV oder
    \item 
      es existiert $A \to A_1\ldots A_n \in P$ mit $A_j \in M_i$ für
      $1 \le j \le n$.
      Per IV existieren Ableitungen
      \begin{displaymath}
        A_j \stackrel{*}{\==>} \Eps
      \end{displaymath}
      denen kann die obige Produktion vorangestellt werden, sodass auch
      eine Ableitung von $A$ nach $\Eps$ existert.
      \begin{displaymath}
        A  \==> A_1 \ldots A_n \stackrel{*}{\==>} \underbrace{\Eps \ldots \Eps}_{n \text{ mal}} = \Eps
      \end{displaymath}
      Also gilt $A \in \operatorname{Nullable}(\mathcal{G})$
    \end{itemize}
  \end{description}

  
  \paragraph{,,$\subseteq$''}
  Wenn $A \in \operatorname{Nullable}(\mathcal{G})$, dann existiert $m
  \in \N$, so dass $A \stackrel{*}{\==>} \Eps$ mit einem
  Ableitungsbaum der Höhe $m$.
  Wir zeigen per Induktion über $m$, dass $A \in M_{m+1} \subseteq \bigcup_{i \in \N} M_i$.
  \begin{description}
  \item[IA] $m = 0$. Die Ableitung ist $A \==> \Eps$, sodass $A\in M_1$.
  \item[IS] $m > 0$.
    Die Wurzel des Ableitungsbaum muss mit $A \to A_1\ldots A_n$ ($n>0$)
    markiert sein und die Kinder sind jeweils Ableitungsbäume für
    $\Eps$ in  $Abl(\mathcal{G}, A_i)$ der Höhe $m_i \le m-1$, wobei $1 \le i \le n$.
    
    Es gilt nun per IV, dass $A_i \in M_{m_i} \subseteq M_{m-1}$.
    Somit ist $A_i \in M_{m-1}$ und damit, per Definition, $A \in M_m$.
  \end{description}
\end{proof}
Aus dem Beweis ergibt sich der folgende Algorithmus zur Berechnung von $\bigcup_{i \in \N} M_i$.
\begin{itemize}
\item Eingabe: CFG $\mathcal{G} = (N, \Sigma, P, S)$
\item Ausgabe: $\operatorname{Nullable}(\mathcal{G})$
\item[] $M = \emptyset$
\item[] $\mathit{done} = \mathtt{false}$
\item[] $\mathbf{while} (\mathit{not}~\mathit{done})$
  \begin{itemize}
  \item[] $\mathit{done} = \mathtt{true}$
  \item[] $\mathbf{for}~(A \to \alpha \in P)$
    \begin{itemize}
    \item[] $\mathbf{if}~(A \not \in M \land \alpha \in M^*)$
      \begin{itemize}
      \item[] $M = M \cup \{A \}$
      \item[] $\mathit{done} = \mathtt{false}$
      \end{itemize}
    \end{itemize}
  \end{itemize}
\item[] $\mathbf{return}~M$
\end{itemize}
\begin{Korollar}
  Es gibt einen Algorithmus, der zu einer CFG $\mathcal{G}$ berechnet ob $\Eps \in L(\mathcal{G})$.
\end{Korollar}
\begin{proof}
  Prüfe ob $S \in \operatorname{Nullable}(\mathcal{G})$
\end{proof}
\begin{lemma}[\textsc{Del}]
  Zu jeder CFG $\mathcal{G}=(N, \Sigma, P, S)$ gibt es eine äquivalente CFG $\mathcal{G'} =(N', \Sigma, P', S')$, bei der die einzige $\Eps$-Regel $S' \to \Eps$ ist (falls $\Eps \in L(\mathcal{G})$) und bei der $S'$ auf keiner rechten Seite einer Produktion vorkommt.
\end{lemma}
\begin{proof}
  \hfill
  \begin{enumerate}
  \item Erweitere $\mathcal{G}$ um ein neues Startsymbol $S'$ mit $S' \to S$ als neue Produktion.
    Dieser Schritt stellt sicher, dass $S'$ auf keiner rechten Regelseite vorkommt.
  \item Wende erst \textsc{Sep}, dann \textsc{Bin} an.
    Nun hat jede rechte Regelseite die Form $a \in \Sigma$ oder $\Eps$ oder $A$ oder $BC$.
  \item Sei $M = \operatorname{Nullable}(\mathcal{G}')$.
  \item Für alle Regeln $A \to BC \in P$:
    \begin{itemize}
    \item Falls $B \in M$ füge $A \to C$ hinzu.
    \item Falls $C \in M$ füge $A \to B$ hinzu.
    \end{itemize}
  \item Falls $S \in \operatorname{Nullable}(\mathcal{G})$ füge $S' \to \Eps$ hinzu
  \item Entferne alle Produktionen $A \to \Eps$ für $A \neq S'$.
  \end{enumerate}
\end{proof}

\begin{Def}
  Sei $\mathcal{G} = (N, \Sigma, P, S)$ eine Grammatik.
  Eine \emph{Kettenregel} von $\mathcal{G}$ ist eine Produktion in $P$ der Form $A \to B$, wobei $A,B \in N$.
\end{Def}

\datenote{7.12.16}
\begin{lemma}[\textsc{Unit}]
  Zu jeder CFG $\mathcal{G} = (N, \Sigma, P, S)$ gibt es eine äquivalente CFG $\mathcal{G}' = (N, \Sigma, P', S)$ ohne Kettenregeln.
\end{lemma}
\begin{proof}
  Setze zu Anfang $P' = P$ und eliminiere alle Kettenregeln mit folgendem Algorithmus:
  \begin{enumerate}
  \item Betrachte den gerichteten Graphen $G$ mit Knoten $N$ und Kanten $\{(A, B) \mid A,B \in N \text{ und } A \to B \in P' \}$ \label{alg:unit-step-dg}.
  \item Suche, z.B. mittels Tiefensuche, einen Zyklus in $G$ .
    Wenn kein Zyklus gefunden wurde, weiter mit Schritt \ref{alg:unit-step-dag}.
  \item Wurde der Zyklus $A_1 \to A_1 \to \ldots \to A_k \to A_1$ gefunden, dann ersetze in $P'$ alle Vorkommen von $A_j$ mit $j > 1$ durch $A_1$ (auf linker \emph{und} rechter Regelseite).
    Fahre fort mit Schritt \ref{alg:unit-step-dg}.
  \item \label{alg:unit-step-dag}
    Der Graph $G$ ist ein gerichteter, azyklischer Graph (DAG, \emph{Directed Acyclic Graph}).
    Sortiere die Knoten von $G$ topologisch als $A_1, \ldots, A_n$, so dass $A_n$ auf keiner linken Seite einer Kettenregel vorkommt.
  \item $\mathbf{for}~j = m \ldots 1$
    \begin{itemize}
    \item[] $\mathbf{for}~\mathbf{each}~A_j \to A_k \in P'$
      \begin{itemize}
      \item[] (wegen topologischer Sortierung gilt $k > j$)
      \item[] entferne $A_j \to A_k$ aus $P'$
      \item[] füge $A_j \to BC$ zu $P'$ hinzu, falls $A_k \to BC \in P'$,\quad $B,C \in N$.
      \item[] füge $A_j \to \mathtt{a}$ zu $P'$ hinzu, falls $A_k \to \mathtt{a} \in P'$,\quad $\mathtt{a} \in \Sigma$.
      \end{itemize}
    \end{itemize}
    Die äußere Schleife hat die Invariante, dass am Ende jedes Durchlaufs für $m \ge i \ge j$ keine Kettenregel $A_i \to \ldots$ in $P'$ existiert.
    Beim Verlassen der Schleife ist $j = 1$ und es existieren überhaupt keine Kettenregeln mehr.
  \end{enumerate}
\end{proof}


\begin{Def}[name={[\acs*{CFG} in \acs*{CNF}]}]
	Eine \ac{CFG} $\mathcal{G} = (N, \Sigma, P, S)$ ist in \ac{CNF}, falls jede
  Produktion die Form $A\->a$, $A\->BC$, oder $S \-> \Eps$ hat, wobei $A,B,C\in N$, $a\in\Sigma$.
  Falls $S \to \Eps \in P$, dann darf $S$ auf keiner rechten Seite einer Produktion vorkommen.
\end{Def}
\begin{proof}
    Wende der Reihe nach \textsc{Sep}, \textsc{Bin}, \textsc{Del} und \textsc{Unit} an.\footnote{Es genügt sogar nur \textsc{Del} und dann \textsc{Unit} anzuwenden, da \textsc{Del} schon \textsc{Sep} und \textsc{Bin} ausführt.}
\end{proof}

\begin{Beobachtung}
  Ist $\mathcal{G} = (N, \Sigma, P, S)$ in CNF, dann lassen sich für die Ableitungsbäume $\mathcal{A} \in \operatorname{Abl}(\mathcal{G}, S)$ folgende Eigenschaften feststellen:
  \begin{enumerate}
  \item $\mathcal{A}$ ist ein Binärbaum.
  \item Falls $Y(\mathcal{A}) = w$ dann ist die Anzahl der Blätter des Baumes $|w|$.
  \end{enumerate}
\end{Beobachtung}
	
\begin{Bemerkung} Sei $|\mathcal{G}|=\sum\limits_{A\->\alpha\in P}(|\alpha|+1)$ die Gr"o"se einer CFG.\\
Die Transformation nach \ac{CNF} benötigt Zeit $O(|\mathcal{G}|^2)$. Die Größe der \ac{CNF}-Grammatik ist $O(|\mathcal{G}|^2)$.
\end{Bemerkung}

\subsection{Pumping Lemma für kontextfreie Sprachen}
\begin{Satz}[Pumping lemma für \acs*{CFL}, uvwxy Lemma]
\label{satz:PL für CFL}
%\rlnote{Satz \# 4.2?}
	Sei $L\in {CFL}$. Dann $\exists n>0$, so dass $\forall z\in L$ mit $|z|\geq n\ \exists u,v,w,x,y:$ mit
	\begin{itemize}
	\item $z=uvwxy$
	\item $|vwx|\leq n$
	\item $|vx|\geq 1$
	\item $\forall i \in \N : uv^iwx^iy\in L$
	\end{itemize}
\end{Satz}
\begin{proof}
	Sei $\mathcal{G}= (N,\Sigma,P,S)$ in \ac{CNF} mit
    $L(\mathcal{G}) = L$.
    
	Sei $k = |N|$ und wähle $n=2^k$
	
	Betrachte den Ableitungsbaum $\mathcal{A} \in \operatorname{Abl}(\mathcal{G}, S)$ von $z$ mit $|z|\geq n$.

  $\mathcal{A}$ ist ein Binärbaum mit $|z|\geq n = 2^k$ Blättern.
  In einem solchen Bin"arbaum existiert ein Pfad $\zeta$ der Länge $\geq k$

	Auf $\zeta$ liegen $\geq k+1$ \acfp{NT}, es muss also mindestens ein Nichtterminal $A$ mehrmals vorkommen.
  Folge $\zeta$ vom Blatt Richtung Wurzel und bis sich $A$ das erste mal Wiederholt.
  Das geschieht nach $\le k$ Schritten.
  Nun teile $z = uvwxy$ wie hier skizziert:
  
			\begin{tikzpicture}[>=stealth,
				widebox/.style={draw, minimum width=.75cm, minimum height=.35cm, inner sep=0pt}
				,dot/.style={inner sep=0pt,outer sep=0pt,label={center:\scalebox{.75}{\textbullet}}}
				]
				
				\node[dot,label={above right:$S$}] (v1) at (0,-1) {};
				\node[dot] (v2) at (-0.25,-1.6) {};
				\node[dot,] (v3) at (0,-2) {};
				\node[dot] (v4) at (-0.5,-2.5) {};
				\node[dot,label={above right:$A$}] (v5) at (0,-3) {};
				\node[widebox] (v6) at (-1.5,-3.5) {$u$};
				\node[widebox] (v7) at (-.75,-3.5) {$v$};
				\node[widebox] (v8) at (0,-3.5) {$w$};
				\node[widebox] (v9) at (.75,-3.5) {$x$};
				\node[widebox] (v10) at (1.5,-3.5) {$y$};
				
				\draw  (v1) edge (v2);
				\draw  (v2) edge (v3);
				\draw  (v3) edge (v4);
				\draw  (v4) edge (v7.north west);
				\draw  (v4) edge (v5);
				\draw  (v3) edge (v10.north west);
				\draw  (v5) edge (v8.north west);
				\draw  (v5) edge (v8.north east);
				\draw  (v1) edge (v6.north west);
				\draw  (v1) edge (v10.north east);
				\node[align=left, text height=3em,inner sep=2pt] at (2.1,-1.5) {$A\rightarrow BC$~\small{ist Binärbaum}} edge[->,shorten >=.1cm] (v3);
				\node[inner sep=0pt] at (2.75,-3.5) {$A\rightarrow a$} edge[->] (v10);
				\node (v11) at (0,-4) {$z$};
				\draw[semithick]
				let \p1 = (v6.west), \p2 = (v11), \p3 = (v10.east) in 
				 (\x1,\y2) edge[{Bar[]<}-] (v11) (v11) edge[-{>Bar[]}] (\x3,\y2);
			\end{tikzpicture}
	
      Nun gilt
      \begin{itemize}
      \item $|vx|\geq 1$, da $\zeta$ entweder durch $B$ oder durch $C$ läuft. Nehme also an, $\zeta$ verläuft durch $B$.
        Somit muss $C \stackrel{*}{\Longrightarrow} x$. In CNF ist $C \stackrel{*}{\Longrightarrow} \Eps$ nicht möglich und somit ist $|x| \ge 1$.
        Der Fall, das $\zeta$ durch $C$ verläuft, ist analog.
      \item $|vwx| \le 2^k = n$ TODO
      \end{itemize}
		\begin{figure}[H]\centering
		\begin{subfigure}[b]{.29\linewidth}\centering
			\begin{tikzpicture}[node distance=.75cm, on grid,
					every node/.style={
						execute at begin node=$,
						execute at end node=$
					}
				]
				\node (0) {};
				\node (A1) [below=of 0] {A};
				\node (A2) [below=of A1] {A};
				\node (A3) [below=of A2] {A};
				
				\newlength\Offset \setlength\Offset{.75cm}
				\coordinate(p1) at ($(A2.south) - (0,\Offset)$);
				%
				\coordinate (A2l) at ($(p1) - (\Offset,0)$);
				\coordinate (A1l) at ($(A2l) - (\Offset,0)$);
				\coordinate (0l) at ($(A1l) - (\Offset,0)$);
				%
				\coordinate (A2r) at ($(p1) + (\Offset,0)$);
				\coordinate (A1r) at ($(A2r) + (\Offset,0)$);
				\coordinate (0r) at ($(A1r) + (\Offset,0)$) ;
				
				\coordinate (A3l) at ($(A3.south) - (\Offset,\Offset)$);
				\coordinate (A2l2) at ($(A3l) - (\Offset,0)$);
				%
				\coordinate (A3r) at ($(A3.south) + (\Offset,-\Offset)$);
				\coordinate (A2r2) at ($(A3r) + (\Offset,0)$);
				
				\node (dots) [below=\Offset of A3] {\vdots};
				
				\node (A4) [below=\Offset of dots] {A};
				\coordinate (A4l) at ($(A3l) - (0,2*\Offset)$);
				\coordinate (A4r) at ($(A3r) - (0,2*\Offset)$);
				
				\draw (0.south) edge (0l.center) edge (0r.center)
					(A1.south) edge (A1l.center) edge (A1r.center)
					(A2.south) edge (A2l2.center) edge (A2r2.center)
					(A3.south) edge (A3l.center) edge (A3r.center)
					(A4.south) edge (A4l.center) edge (A4r.center)
				;
				\draw (0l.center) edge node[below] {u} (A1l.center)
					(A1l.center)  edge node[below] {v} (A2l.center)
					(A2r.center)  edge node[below] {x} (A1r.center)
					(A1r.center)  edge node[below] {y} (0r.center)
					(A2l2.center) edge node[below] {v} (A3l.center)
					(A3r.center)  edge node[below] {x} (A2r2.center)
					(A4l.center)  edge node[below] {w} (A4r.center)
				;
			\end{tikzpicture}
			\caption{$uv^iwx^iy\in L$}
		\end{subfigure}
		\begin{subfigure}[b]{.28\linewidth}\centering
			\begin{tikzpicture}[node distance=.5cm, on grid,
					every node/.style={
						execute at begin node=$,
						execute at end node=$
					},
					widebox/.style={draw, minimum width=.75cm, minimum height=.35cm, inner sep=0pt}
				]
				\node (S) {S};
				\node[inner sep=2pt] (A) [below=of S.south, xshift=.2cm] {A};
				\coordinate (Ap) at (A.south west);
				\node[widebox] (w) [below=of Ap] {w};
				\node[widebox] (u) at ($(Ap)!2!(w.north west)+.5*(-.75,-.3)$) {u};
				\node[widebox] (y) at ($(Ap)!2!(w.north east)+.5*(.75,-.3)$) {y};
				
				\draw (S.south) edge (u.north west) edge (y.north east)
					(Ap) edge (u.north east) edge (y.north west) edge (w.north)
					($(S.south) - (.25,.6)$) edge (S.south) edge (Ap)
				;
			\end{tikzpicture}
			\caption{$\curvearrowright uwy\in L$}
		\end{subfigure}
		\begin{subfigure}[b]{.33\linewidth}\centering
			\begin{tikzpicture}[>=stealth,
				widebox/.style={draw, minimum width=.75cm, minimum height=.35cm, inner sep=0pt}
				,dot/.style={inner sep=0pt,outer sep=0pt,label={center:\scalebox{.75}{\textbullet}}}
				]
				
				\node[dot,label={above right:$S$}] (v1) at (0,-1) {};
				\node[dot] (v2) at (-0.25,-1.6) {};
				\node[dot,label={above right:$A$}] (v3) at (0,-2) {};
				\node[dot] (v4) at (-0.5,-2.5) {};
				\node[dot,label={above right:$A$}] (v5) at (0,-3) {};
				\node[widebox] (v6) at (-1.5,-3.5) {$u$};
				\node[widebox] (v7) at (-.75,-3.5) {$v$};
				\node[widebox] (v8) at (0,-3.5) {$w$};
				\node[widebox] (v9) at (.75,-3.5) {$x$};
				\node[widebox] (v10) at (1.5,-3.5) {$y$};
				
				\draw  (v1) edge (v2);
				\draw  (v2) edge (v3);
				\draw  (v3) edge (v4);
				\draw  (v4) edge (v7.north west);
				\draw  (v4) edge (v5);
				\draw  (v3) edge (v10.north west);
				\draw  (v5) edge (v8.north west);
				\draw  (v5) edge (v8.north east);
				\draw  (v1) edge (v6.north west);
				\draw  (v1) edge (v10.north east);
				\node[align=left, text height=3em,inner sep=2pt] at (2.1,-1.5) {$A\rightarrow BC$\small{ist Binärbaum}} edge[->,shorten >=.1cm] (v3);
				\node[inner sep=0pt] at (2.75,-3.5) {$A\rightarrow a$} edge[->] (v10);
				\node (v11) at (0,-4) {$z$};
				\draw[semithick]
				let \p1 = (v6.west), \p2 = (v11), \p3 = (v10.east) in 
				 (\x1,\y2) edge[{Bar[]<}-] (v11) (v11) edge[-{>Bar[]}] (\x3,\y2);
			\end{tikzpicture}
			\caption{$\exists$ Pfad mit Länge $\geq k$}
		\end{subfigure}
		\caption{Schema zu \autoref{satz:PL für CFL}}
	\end{figure}\vspace{-2em}\qedhere
\end{proof}

\begin{lemma} %\rlnote{Lemma \# 4.3?}
	$\mathcal{L}_2 \subsetneq \mathcal{L}_1$
\end{lemma}
\begin{proof}
	Sei $L=\{L=\{a^nb^nc^n \mid n\geq 1\}$.\\
	$L$ ist nicht kontextfrei. Verwende \ac{PL}. Angenommen $L$ sei kontextfrei.
	Sei dann $n$ die Konstante aus dem \ac{PL}.
	
	Wähle $z=a^nb^nc^n$ und somit $|z| = 3n \ge n$.
  Nach PL ist $z = uvwxy$ mit $|vx| \ge 1$ und $|vwx| \le n$.

  Durch $|vwx| \le n$ ergeben sich folgende Möglichkeiten:
  \begin{itemize}
  \item $vwx = a^j$: 
    Für $i=0$ ist $vwx=w$ mit $|w| < j$.
    Anzahl der $a$ stimmt nicht mehr mit $n$ überein.
  \item $vwx = a^kb^j$:
    \begin{itemize}
    \item Falls $v = a^{k'}$, $x = b^{j'}$: Für $i=0$ stimmt die Anzahl der $a$ nicht mehr mit $n$ überein.
    \item Falls $v$ ein Gemisch aus $a$ und $b$ enthält.
      Für $i>2$ würde eine Folge $a^n$ durch eine $b$-Folge unterbrochen.
    \item Falls $x$ ein Gemisch aus $a$ und $b$ enthält.
      Für $i>2$ würde eine Folge $a^n$ durch eine $b$-Folge unterbrochen.
    \end{itemize}
  \item $vwx = b^j$: analog Fall $a^j$
  \item $vwx = b^kc^j$: analog Fall $a^kb^j$
  \item $vwx = c^j$: analog Fall $a^j$
  \end{itemize}
  In jeder der Möglichkeiten lässt sich durch Pumpen ein Wort $w \not \in L$ finden.
  Daher kann $L$ nicht kontextfrei sein.
\end{proof}

\begin{Bsp}
  \begin{displaymath}
    L = \{ww \mid w \in \{a,b\}^*\}
  \end{displaymath}
  ist kontextsensitive aber nicht kontextfrei.

  Sei $n$ die Konstante aus dem PL

  Betrachte $z = a^nb^na^nb^n \in L$ mit $|z| = 4n \ge n$.
  Nach PL ist $z = uvwxy$ mit $|vx| \ge 1$ und $|vwx| \le n$.

  Es ergeben sich folgende Möglichkeiten:
  \begin{itemize}
  \item $vwx = a^j$, $j \le n$.
    Für $i=0$ ist $vwx=w$ mit $|w| < j$.
    Anzahl der $a$ stimmt nicht mehr mit $n$ überein.
  \item $vwx = a^kb^j$, $k+j \le n$.

    \begin{itemize}
    \item Falls $v = a^{k'}$, $x = b^{j'}$: Für $i=0$ stimmt die Anzahl der $a$ nicht mehr mit $n$ überein.
    \item Falls $v$ ein Gemisch aus $a$ und b enthält.
      Für $i>2$ würde eine Folge $a^n$ durch eine $b$-Folge unterbrochen.
    \item Falls $x$ ein Gemisch aus $a$ und b enthält.
      Für $i>2$ würde eine Folge $a^n$ durch eine $b$-Folge unterbrochen.
    \end{itemize}
  \item $vwx = b^j$ analog Fall $a^j$
  \item $vwx = b^ka^j$ analog Fall $a^kb^j$
  \end{itemize}
\end{Bsp}

\draftnote{9.12.16}
\subsection{Entscheidungsprobleme für kontextfreie Sprachen}
\begin{Satz}[name={[Wortproblem für \acs*{CFL} entscheidbar]}]
	Das Wortproblem "`$w\in L?$"' ist für \ac{CFL} entscheidbar. Falls $|w|=n$, benötigt der Algorithmus $O(n^3)$ Schritte und $O(n^2)$ Platz.
\end{Satz}
\begin{proof}
	Algorithmus \ac{CYK}.
\end{proof}
\begin{Bsp}
	$L=\{a^nb^nc^m \mid n,m\geq 1\}$ mit Grammatik (\acs{CFG})
	\begin{align*}
		S &\-> XY\\
		X &\-> ab \mid aXb\\
		Y &\-> c \mid cY
	\end{align*}
\end{Bsp}

	$\begin{aligned}[t]
		S &\-> XY\\
		X &\-> AB\\
		X &\-> AZ\\
		Z &\-> XB\\
		Y &\-> c\\
		Y &\-> CY\\
		A &\-> a\\
		B &\-> b\\
		C &\-> c
	\end{aligned}$\quad
\begin{proof}
		\ac{CYK}-Algorithmus
		\begin{description}
	    \item[Eingabe:]
		\begin{itemize}
			\item \ac{CFG} $\mathcal{G}$ in \ac{CNF}
			\item $w = a_1\dots a_n\in\Sigma^*, |w|=n$
		\end{itemize}
		\item[Ausgabe:] "`$w\in L(\mathcal{G})$"'
		\item[Idee:] Berechne eine $(n\x n)$ Matrix $M$ mit Einträgen in $\mathcal{P}(N)$ mit folgender Spezifikation:
		\end{description}
	\begin{align*}
		M_{ij} &= \{ A \mid A\xRightarrow{*} a_i\dots a_j \} && 1\leq i\leq j\leq n \quad\text{nur diese }M_{ij}\neq \varnothing\\
		& \text{falls $i=j$:}\\
		M_{ii} &= \{ A \mid A\xRightarrow{*} a_i\}\\
		&= \{A \mid A\-> a_i \} \\
		&\text{falls }1\leq i<j\leq n:\\
		M_{ij} &= \{ A \mid A\xRightarrow{*} a_i\dots a_j \}\\
		&= \{ A \mid A\=> BC \xRightarrow{*} a_i\dots a_j \}\\
		&= \{ A \mid A\-> BC \land BC \xRightarrow{*} a_i\dots a_j \}\\
		&= \{ A \mid A\-> BC \land \exists k: i\leq k<j\quad 
			\begin{aligned}[t]
				B &\xRightarrow{*} a_i\dots a_k \land{}\\
				C &\xRightarrow{*} a_{k+1}\dots a_j \}
			\end{aligned}\\
		% &= \bigcup_{k=i}^{j-1} \{ A \mid A \-> BC \land 
		% 	\begin{aligned}[t]
		% 		B &\xRightarrow{*} a_i\dots a_k \land{}\\
		% 		C &\xRightarrow{*} a_{k+1}\dots a_j \}
		% 	\end{aligned}\\
		% &= \bigcup_{k=i}^{j-1} \{ A \mid A \-> BC \land B \in M_{i,k} \land C \in M_{k+1,j} \}
	\end{align*}
	Damit: $w\in L$ genau dann, wenn $S \=>^* w$ genau dann, wenn $S \in M_{1,n}$.
	
  Beispiel der Matrix für $w = aaabbbcc$.
  Zellen, die mit ,,${\bullet}$'' markiert sind, sind leer.
  Die Reihenfolge, in der die nicht-diagonalen, nicht leeren Zellen, eingetragen wurden ist mit kleinen Zahlen.
  Beispielsweise wurde $M_{35} = \{Z\}$ mit Zahl ,,{\scriptsize 3}'' \emph{nach} $M_{78} = \{Y\}$ mit Zahl ,,{\scriptsize 1}'' eingetragen.
  Da $M_{1n} = \{S\}$ gilt $w \in L$.
  \begin{center}
		\begin{tabular}[t]{M{c}|*8{M{c}|}}
      w = & a & a & a & b & b & b & c & c \\
		\cline{2-9}
			M = & A               & \bullet & \bullet & \bullet & \bullet & X ~\text{\tiny 6}& \emph{S} ~\text{\tiny 7}& S~\text{\tiny 8}       \\
		\cline{2-9}
			\multicolumn{2}{c|}{} & A       & \bullet & \bullet & {X} ~\text{\tiny 4}& {Z} ~\text{\tiny 5}& \bullet & \bullet \\
		\cline{3-9}
			     \multicolumn{3}{c|}{}      & A       & X ~\text{\tiny 2}      & Z     ~\text{\tiny 3}  & \bullet & \bullet & \bullet \\
		\cline{4-9}
			          \multicolumn{4}{c|}{}           & B       & \bullet & \bullet & \bullet & \bullet \\
		\cline{5-9}
			               \multicolumn{5}{c|}{}                & B       & \bullet & \bullet & \bullet \\
		\cline{6-9}
			                    \multicolumn{6}{c|}{}                     & B & \bullet & \bullet \\
		\cline{7-9}
			                         \multicolumn{7}{c|}{}                          & C,Y     & Y ~\text{\tiny 1}      \\
		\cline{8-9}
			                              \multicolumn{8}{c|}{}                               & C,Y     \\
		\cline{9-9}
		\end{tabular} 
  \end{center}
\end{proof}

$\acs{CYK}(\mathcal{G},a_1,\dots,a_n)$\\
$M\ n\x n$ Matrix mit $M_{ij}=\varnothing$
\begin{lstlisting}[mathescape,morekeywords={for,do,return},morecomment={[l]{//}}]
  for i=1 .. n do  // $O(|\mathcal{G}|)\cdot O(n)$
    $M_{ii}=\{A \mid A \-> a_i \}$
  for i=n-1 ..  1 do
    for j=i+1 .. n do
      for k=i .. j-1 do  // $O(|\mathcal{G}|)\cdot O(n^3)$
        $M_{ij}=M_{ij}\cup \{ A \mid A \-> BC, B\in M_{ik}, C\in M_{k+1,j} \}$
  return $S\in M_{1n}$
\end{lstlisting}

\begin{Satz}[name={[Entscheidbarkeit des Leerheitsproblems für kontextfreie Sprachen]}] % 4.7
  \label{thm:cfl-decidable-emptyness}
    Das Leerheitsproblem ist für \ac{CFL} entscheidbar.
\end{Satz}

\begin{proof}
    Sei $\mathcal{G} = (N, \Sigma, P, S)$ eine CFG für $L$.
    Gefragt ist, ob $L(\mathcal{G}) = \{ w\in\Sigma^* \mid S\overset{*}{\=>} w \} \overset?= \varnothing
$. 
    Betrachte dazu die Menge $M$ der nützlichen Nichtterminalsymbole, aus denen ein Terminalwort herleitbar ist.
	\begin{align*}
		M &= \{ A \mid A \overset{*}{\=>} w, w\in \Sigma^* \}\\
		M_0 &= \{ A \mid A \-> w\in P, w\in \Sigma^* \}\\
		M_{i+1} &= M_i\cup \{ A \mid A \-> \alpha\in P, \alpha\in(\Sigma\cup M_i)^* \}\\
		\exists n : M_n &= M_{n+1} \overset!= M\\
		L =\varnothing &\<=> S\notin M \qedhere
	\end{align*}
	Offenbar ist $M_0 \subseteq M$ eine Approximation von $M$, $M \supseteq M_{i+1} \supseteq M_i$ verbessert $M_i$ und das jeweils nächste $M_i$ ist in Linearzeit berechenbar. Da $M \subseteq N$ endlich ist, muss es ein $n$ geben, sodass $M_n = M_{n+1}$. Für dieses $n$ kann man zeigen, dass $M = M_n$. 
\end{proof}
\begin{Bem}
	$M$ ist die Menge der nützlichen Nichterminale. Nicht nützliche Nichtterminale können entfernt werden, ohne dass $L(\mathcal{G})$ sich ändert.
\end{Bem}
\begin{Satz}[name={[Entscheidbarkeit des Endlichkeitsproblem für \acs*{CFL}]}]
	Das Endlichkeitsproblem für \ac{CFL} ist entscheidbar.
\end{Satz}
\begin{proof}
	Mit \ac{PL} analog zum Endlichkeitsproblem für reguläre Sprachen.
\end{proof}

\subsection{Abschlusseingenschaften für kontextfreie Sprachen}


\begin{Satz} % 4.7
  \label{thm:cfl-closed-reg-intersect}
	Die Menge \acs{CFL} ist abgeschlossen unter $\cup, \cdot, ^*$, jedoch \emph{nicht} unter $\cap, \overline{\phantom{A}}$.
\end{Satz}
\begin{proof}
	\begin{align*}
		\mathcal{G}_i :&= (N_i, \Sigma,P_i,S_i) \quad i=1,2\\
		\text{"`}\cup\text{"'}: N &= N_1\dotcup N_2\dotcup\{S\}\\
		P &= \{S\->S_1, S\->S_2 \} \cup P_1\cup P_2\\
		\text{"`}\cdot\text{"'}:: N &= N_1\dotcup N_2\dotcup\{S\}\\
		P &= \{ S\->S_1S_2 \} \cup P_1\cup P_2\\
		\text{"`}^*\text{"'}: N &= N_1 \dotcup \{S\}\\
		P &= \{ S\->\Eps, S\-> S_1S \} \dotcup P_1
	\end{align*}
	\begin{itemize}
	\item für $n\geq 1$
		\[ \underbrace{\{a^nb^nc^n\}}_{\notin \acs{CFL}} = \underbrace{\{a^nb^nc^m \mid n,m\geq 1 \}}_{\in \acs{CFL}} \cap \underbrace{\{a^mb^nc^n \mid m,n\geq 1 \}}_{\in \acs{CFL}} \]
		Also \ac{CFL} nicht abgeschlossen unter $\cap$.
	\item Angenommen \ac{CFG} abgeschlossen unter $\overline{\phantom{X}}$\\
		Falls $L_1,L_2\in \acs*{CFL}$, dann ist $\overline{L}_1,\overline{L}_2\in\acs*{CFL}$ nach Annahme.\\
		$\curvearrowright \overline{L}_1\cup \overline{L}_2\in\acs*{CFL}$ wegen Teil "'$\cup$"'.\\
		$\curvearrowright \overline{\overline{L}_1\cup \overline{L}_2} = L_1\cap L_2\in\acs*{CFL}\ \lightning$ zu Teil "'$\cap$"'. \qedhere
	\end{itemize}
\end{proof}

\datenote{14.12.16}
\begin{Satz} % 4.8
	Die Menge \ac{CFL} ist abgeschlossen unter ,,$\cap$ mit regulären Sprachen REG''.
  Das heißt, für alle $L \in \mathrm{CFL}$ und $R \in \mathrm{REG}$ gilt $L \cap R \in \mathrm{CFL}$.
\end{Satz}
\begin{proof}
  Sei $\mathcal{G} = (N, \Sigma, P, S)$ kontextfreie Grammatik in CNF mit $L(\mathcal{G}) = L$.

  Sei $M = (Q, \Sigma, q, \delta, q_0, F)$ NEA mit $L(M) = R$.

  Definiere $\mathcal{G}' = (N', \Sigma, P', S)$ durch
  \begin{itemize}
  \item[] $N' = Q \times N \times Q \cup \{S\}$
  \item[] $P' =
    \begin{aligned}[t]
      &\{ (p, A, q) \to a \mid A \to a \in P \text{ und } \delta(p,a) \ni q \} \\
      &\cup \{ (p, A, q) \to (p,B,q')(q',C,q) \mid A \to BC \in P \text{ und } p,q,q' \in Q\} \\
      &\cup \{ S \to (q_0, S, q) \mid q \in F \}
    \end{aligned}
    $
  \end{itemize}
  Durch die $S \to \ldots$ Regeln erzeugt $\mathcal{G}'$ offensichtlich genau die Worte, die durch die Nichtterminale $(q_0, S, q)$ für $q \in F$ erzeugt werden.
  Es bleibt zu zeigen, dass ein Nichtterminal genau die Worte aus $L$ erzeugt die gleichzeitig einen (akzeptierenden) Lauf $q_0\ldots q$ von $M$ erlauben.
  Wir zeigen eine Verallgemeinerung: für alle $p,q \in Q$ und $A \in N$ gilt
  \begin{align*}
    &(p,A,q) \stackrel{*}{\Longrightarrow}_{\mathcal{G}'} w \\
    \text{ gdw } & A \stackrel{*}{\Longrightarrow}_{\mathcal{G}} w \text{ und es existiert ein Lauf } p\ldots q \text{ von $M$ auf $w$}
  \end{align*}
  \begin{itemize}
  \item Richtung ,,links nach rechts'': (siehe Übung)
  \item Richtung ,,rechts nach links'':

    Per Induktion über den Ableitungsbaum $\mathcal{A} = \pi(\mathcal{A}_1,\ldots,\mathcal{A}_n) \in \operatorname{Abl}(\mathcal{G}, A)$ mit $Y(\mathcal{A}) = w$.
    \begin{description}
      \item[IV] $\forall 1 \le i \le n:$ wenn $w_i = Y(\mathcal{A}_i)$ mit $\mathcal{A}_i = A_i \to \ldots(\ldots)$ und es existiert ein Lauf  $p_i\ldots q_i$  von M auf $w_i$, dann $(p_i, A_i, q_i) \stackrel{*}{\Longrightarrow}_{\mathcal{G}'} w_i$.
      \item[IS] \hfill
        \begin{itemize}
        \item $\pi = A \to a$, $a \in \Sigma$.
          Es gilt $w = a$ und $pq$ ist Lauf auf $a$.
          Folglich ist $\delta(p, a) \ni q$. Damit ist $(p,A,q) \to a \in P'$, woraus direkt folgt, dass $(p,A,q) \Longrightarrow_{\mathcal{G}'} a$.
        \item $\pi = A \to BC$, $\mathcal{A} = \pi(\mathcal{A}_1, \mathcal{A}_2)$, $Y(\mathcal{A}_1) = w_1$, $Y(\mathcal{A}_2) = w_2$, \mbox{$\mathcal{A}_1 = B \to \ldots(\ldots)$} und $\mathcal{A}_2 = C \to \ldots(\ldots)$.

          Es gilt $w = w_1w_2$ und $\zeta = p\ldots q$ ist Lauf auf $w_1w_2$.
          Es existiert also $q'$ so dass $\zeta = p\ldots q' \ldots q$ und $p\ldots q'$ ist Lauf auf $w_1$ und $q' \ldots q$ ist Lauf auf $w_2$.

          Es folgt per IV, dass
          \begin{displaymath}
            (p, B, q') \stackrel{*}{\Longrightarrow}_{\mathcal{G}'} Y(\mathcal{A}_1) = w_1 \text{ und } (q', C, q) \stackrel{*}{\Longrightarrow}_{\mathcal{G}'} Y(\mathcal{A}_2) = w_2.
          \end{displaymath}
          Nun ist $(p,A,q) \to (p,B,q')(q',C,q) \in P'$ per Konstruktion, also ist folgende Ableitung möglich:
          \begin{displaymath}
            (p,A,q) \Longrightarrow_{\mathcal{G}'} (p,B,q')(q',C,q) \stackrel{*}{\Longrightarrow}_{\mathcal{G}'} w_1(q',C,q) \stackrel{*}{\Longrightarrow}_{\mathcal{G}'} w_1w_2 = w
          \end{displaymath}
        \end{itemize}
    \end{description}
  \end{itemize}
  \end{proof}
%
%Zuletzt: $\acsu{CFL}\land\acsu{REG}=\ac{REG}$
\begin{Satz}[name={[$L\subseteq R$ entscheidbar]}] %\rlwarning{Satz 4.9}
	Sei $L\in\ac{CFL}$ und $R\in\textrm{REG}$. Es ist entscheidbar, ob $L\subseteq R$.
\end{Satz}
\begin{proof}
  Es gilt $L\subseteq R \text{ gdw } L \cap\overline{R} = \emptyset$.

  Da $R \in \mathrm{REG}$ ist durch die Abschlusseigenschaften regulärer Sprachen auch $\overline{R} \in \mathrm{REG}$.
  Nach Satz \ref{thm:cfl-closed-reg-intersect} ist $L \cap\overline{R} \in \mathrm{CFL}$.
  Nach Satz \ref{thm:cfl-decidable-emptyness} ist $L \cap\overline{R} = \emptyset$ daher entscheidbar
\end{proof}