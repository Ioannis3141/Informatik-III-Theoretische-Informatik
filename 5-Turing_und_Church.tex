\datenote{20.12.17}
\section{Turingmaschinen}

\hide{
%\section{\acf{TM}}
1930er Jahre\\
Suche nach formalem Modell für maschinelle Berechenbarkeit
\begin{description}
	\item[Alan Turing:] (1912-1954) Turingmaschine 1936
	\item[Alonzo Church:] Lambdakalkül 1936
        \item[Emil Post:]  Postband 1936
	\item[Kleene, Sturgis:] partiell rekursive Funktionen
	\item[Chomsky:] Typ-0-Grammatiken 1956
\end{description}
\emph{Alan Turing:}\begin{minipage}[t]{.8\textwidth}
\begin{itemize}[parsep=0pt]
	\item Informatik, Logik
	\item Kryptographie (Enigma Entschlüsselung, Sprachverschlüsselung)
	\item KI (Turing-Test)
\end{itemize}\end{minipage}

außerdem: Turing-Award
}

\subsection{Turingmaschine \normalfont(informell)}

TODO (teilweise siehe Folien)
\hide{
Ein primitives Rechenmodell:
\begin{figure}[H]\centering
	\begin{tikzpicture}[every node/.style={block}]
		\node (1) {\blank};
		\node (2) [right=of 1] {b};
		\node (3) [right=of 2] {a};
		\node (4) [right=of 3] {n};
		\node (5) [right=of 4] {a};
		\node (6) [right=of 5] {n};
		\node (7) [right=of 6] {e};
		\node (8) [right=of 7] {\blank};
		\node (9) [right=of 8] {\blank};
		\node (10) [right=of 9] {\blank};
		\node (11) [right=of 10] {\blank};
		\node (last) [right=of 11] {\blank};
		
		\node (Kopf) [below=1.5em of 2, draw=none, text height=.5em] {Kopf}
		edge [->, shorten >=.5ex, semithick] (2);
		
		\node (TB) [below=.5em of 9, draw=none, text height=.5em,  anchor=north west] {Turingband};
		
		\draw (8.south) -- ($(8.south)-(0,1em)$) -- ($(TB.north west)-(0,.5em)$);
		
		% Open begin and end.
		\draw (1.north west) -- ++(-1cm,0) (1.south west)
		-- ++ (-1cm,0) (last.north east)
		-- ++ ( 1cm,0) (last.south east)
		-- ++ ( 1cm,0);
	\end{tikzpicture}
	\vspace{-1em}
	\caption{Turingband}
	\framebox{q} = Zustand
\end{figure}
\vspace{-.5em}
\begin{tabu}{>{\bfseries}X[.22]X[.72]}
	Turingband & \vspace{-1em}\begin{itemize}[leftmargin=1em,parsep=0pt,topsep=0pt]
	\item unendliches Band
	\item Jedes Feld enthält ein Symbol aus einem Bandalphabet $\Gamma$.
	\item uninitialisiert: $\blank\in\Gamma$ (Blank) ist ein spezielles Symbol welches ein Feld als ,,leer'' markiert
	\end{itemize}
	\\
	Kopf & \vspace{-1em}\begin{itemize}[leftmargin=1em,parsep=0pt,topsep=0pt]
	\item zeigt immer auf ein Feld
	\item nur am Kopf kann die \ac{TM} ein Zeichen lesen und schreiben
	\item kann nach rechts /links bewegt werden
	\end{itemize}\\
	Zustand & \vspace{-1em}\begin{itemize}[leftmargin=1em,parsep=0pt,topsep=0pt]
	\item kann verändert werden
	\item kann gelesen werden
	\item es gibt nur endlich viele Zustände
	\end{itemize}\\
	Turingtabelle\newline\normalfont
	\begin{tabular}{|*5{c|}}
		q & a & q' & a' & d \\\hline
		&&&&\\
		&&&&
	\end{tabular}
	& $\sim$ Programm $\sim$ Transitionsfunktion \newline
	$\rightarrow$ Wenn \ac{TM} in Zustand $q$ und Kopf liest
        gerade Symbol $a\in\Gamma$ dann wechsle in Zustand $q'$,
        schreibe $a'$ (über altes $a$) und bewege den Kopf gemäß
        $d\in\{L,R,N\}$ 
\end{tabu}\

\begin{samepage}
	\begin{Bsp*}\ 
		\vspace{-2em}
		\begin{figure}[H]
                   \begin{center}
			\begin{tikzpicture}[every node/.style={block}]
				\node (1) {\blank};
				\node (2) [right=of 1] {\blank};
				\node (3) [right=of 2] {\cancel{b}};
				\node (4) [right=of 3] {a};
				\node (5) [right=of 4] {n};
				\node (6) [right=of 5] {a};
				\node (7) [right=of 6] {n};
				\node (8) [right=of 7] {e};
				\node (9) [right=of 8] {\cancel{\blank}};
				\node (10) [right=of 9] {\blank};
				\node (last) [right=of 10] {\blank};
				
				\node (q1) [draw=none, above=-2pt of 3] {\blank};
				\node (q3) [draw=none, above=-2pt of 9] {b};
				
				% Open begin and end.
				\draw (1.north west) -- ++(-1cm,0) (1.south west)
				-- ++ (-1cm,0) (last.north east)
				-- ++ ( 1cm,0) (last.south east)
				-- ++ ( 1cm,0);
			\end{tikzpicture}
			\\[2ex]
                        \begin{displaymath}
                          \begin{array}{|c|c|c|c|c|l}
                            \hline
			  	q_0 &  x & q_x & \blank & R  & x\ne\blank\\\hline
                                q_x & \blank & q_3 & x & L\\\hline
                                q_x & y & q_x & y & R & y \ne \blank \\\hline
                                q_3 & y & q_3 & y & L & y \ne \blank\\\hline
                                q_3 & \blank & q_4 & \blank & R\\\hline
                          \end{array}
                        \end{displaymath}
                        \end{center}
			\caption{Bsp.: Turingmaschine---Füge das erste
                        Zeichen am Ende der Eingabe an}
		\end{figure}
	\end{Bsp*}
\end{samepage}
%
Was kann die \ac{TM} ausrechnen?
\begin{enumerate}
	\item Die \ac{TM} kann eine Sprache $L\subseteq\Sigma^*$ erkennen.
	\begin{itemize}
		\item Wörter müssen auf Band repräsentierbar sein $\Sigma\subseteq\Gamma\setminus\{\blank\}$
	\end{itemize}
	Ein Wort $w$ wird von einer \ac{TM} erkannt, wenn
	\begin{itemize}
		\item zu Beginn steht nur $w$ auf dem Band, alle anderen Zellen $=\blank$
		\item Kopf auf erstem Zeichen von $w$
		\item Zustand ist Startzustand $q_0$
		\item Abarbeitung der \ac{TT}
		\item Falls \ac{TM} nicht terminiert: $w\notin L$
		\item Falls \ac{TM} terminiert betrachte den errechneten Zustand $q$.\\
		Falls $q\in F$ (akzeptierender Zustand), dann $w\in L$, anderenfalls $w\notin L$
	\end{itemize}
	
	\begin{Bsp*}
		\begin{flalign*}
			\Sigma &=\{0,1\} &\\
			L &=\left\{w\in\Sigma^* \mid \,w\text{ ist Palindrom}\right\}\\
			Q &= \{q_0,q_1,q_r^0, q_r^1, {q_r^0}', {q_r^1}', q_l^0, q_l^1 \} \quad F=\{q_1\}
		\end{flalign*}
		\begin{tabular}{@{}*6{M{l}}}
			q_0      & \blank & q_1      & \blank & N & q_1 \x q_1\x N\\
			q_0      & 0      & q_r^0    & \blank & R\\
			q_0      & 1      & q_r^1    & \blank & R
			\\ \cmidrule{1-5}
			q_r^0    & \blank & q_1      & \blank & N\\
			q_r^0    & 0      & {q_r^0}' & 0      & R\\
			q_r^0    & 1      & {q_r^0}' & 1      & R\\
			{q_r^0}' & \blank & q_l^0    & \blank & L & q_l\->\text{prüfe $0$, fahre zum linken Rand und weiter mit }q_0\\
			{q_r^0}' & 0      & {q_r^0}' & 0      & R & \multirow{2}{*}{\hspace{-1em}$\begin{rcases}\\[1em]\end{rcases}$ Rechtslauf} \\
			{q_r^0}' & 1      & {q_r^0}' & 1      & R
		\end{tabular}\\[.5em]
		\begin{tabular}{@{}*6{M{l}}|}
			\multicolumn{6}{@{}l|}{Alternative 1:}\\
			\multicolumn{6}{@{}l|}{\ac{TM} hält bei jeder Eingabe an.}\\[.5em]
			q_l^0 & \blank & ---  & --- & --- & \<-\text{Halt}\\
                        q_l^0 & 0   & q_l    & \blank & L &\\
			q_l^0 & 1   & ---  & ---      & --- & \<-\text{Halt}
		\end{tabular}\quad\begin{tabular}{@{}*5{M{l}}@{ }l}
		\multicolumn{6}{@{}l}{Alternative 2:}\\
		\multicolumn{6}{@{}l}{\ac{TM} hält nur bei Palindrom an.}\\[.5em]
		q_l^0 & \blank & q_l^0 & 1      & N & \multirow{3}{*}{\scalebox{2.9}{\rotatebox[origin=rb]{-90}{$\curvearrowleftright$}}}\\
		q_l^0 & 0      & q_l   & \blank & L\\
		q_l^0 & 1      & q_l^0 & \blank & N
		\end{tabular}
	\end{Bsp*}
	
	\item Eine \ac{TM} berechnet eine \emph{partielle} Funktion $f: \Sigma^*\dashrightarrow\Sigma^*$\\
	Die Berechnung von $f(w),\ w\in\Sigma^*$
	\begin{itemize}
		\item $w$ auf leeres Band
		\item Kopf auf erstes Zeichen, Standardzustand $q_0$
		\item Abarbeitung der \ac{TT}
		\item Falls terminiert und Kopf steht auf dem ersten Symbol von $v\in\Sigma^*$\\
		Dann $f(w)=v$
	\end{itemize}
\end{enumerate}
\begin{alignat*}{3}
	\text{Schreibe}&\quad& A &\-->B &\quad& \text{totale Funktion von $A$ nach $B$}\\
	&& A&\dashrightarrow B && \text{partielle Funktion von $A$ nach $B$}
\end{alignat*}
\begin{Bsp} % 2.1
	$\Sigma=\{0,1\}$\\
	Gesucht eine \ac{TM}, die die Nachfolgerfunktion auf natürliche Zahlen in Binärdarstellung berechnet.\\
	Annahme: niederwertigste Stelle der Zahl am Anfang der Eingabe.\medskip\\
	\begin{tabular}{@{}M{l}@{ } *5{M{l}} @{ }l}
		\xrightarrow{\text{Start}} & q_0 & \blank & q_2 & 1 & L \\
		& q_0 & 0      & q_1 & 1 & L \\
		& q_1 & 1      & q_0 & 0 & R \\[.5em]
		& q_1 & \blank & q_1 & \blank & N & \<-Halt\\
		& q_1 & 0      & q_2 & 0 & L & \multirow{2}{*}{$\begin{rcases}\\[1em]\end{rcases}$ Linksmaschine}\\
		& q_1 & 1      & q_2 & 1 & L
	\end{tabular}
\end{Bsp}
}

\subsection{Formalisierung der \ac{TM}} % 2.2
\begin{Def}[name={[\acs*{TM}]}] % 2.1
	Eine \ac{TM} ist ein 7-Tupel
	\begin{equation*}
		\mathcal{M}=\left(\Sigma,Q,\Gamma,\delta,\qinit,\blank,F\right)\\
	\end{equation*}
Dabei ist
	\begin{itemize}
		\item $\Sigma$ ist Alphabet,
		\item $Q$ ist endliche Menge deren Elemente wir Zustände nennen,
		\item $\Gamma\supsetneq\Sigma$ ist Menge die wir \emph{Bandalphabet} nennen,
		\item $\delta: Q\times\Gamma\--> \Powerset(Q\times\Gamma\times\{R,L,N\})$ eine Funktion, die wir \emph{Transitionsfunktion} nennen,
		\item $\qinit\in Q$ ein Zustand, den wir \emph{Startzustand} nennen,
		\item $\blank\in\Gamma\setminus\Sigma$ ein Zeichen dass wir \emph{Blank} nennen und
		\item $F\subseteq Q$ eine Teilmenge der Zustände, deren Elemente wir \emph{akzeptierende} Zustände nennen.
	\end{itemize}
        Die TM $\mathcal{M}$ heisst \emph{deterministisch}
        (DTM), falls $\forall q\in Q, \forall a\in\Gamma, |\delta
        (q,a)| \le 1$.\\
        Ansonsten ist $\mathcal{M}$ \emph{nichtdeterministisch} (NTM).
\end{Def}
Im Folgenden sei $\mathcal{M}=(\Sigma, Q, \Gamma, \delta, \qinit, \blank, F)$ eine \ac{TM}.


Die Konfiguration einer Turingmaschine wird durch drei Dinge beschrieben:
\begin{enumerate}
 \item Der aktuelle Zustand
 \item Der Bandinhalt
 \item Die Position des Kopfes
\end{enumerate}
Die scheinbar ``natürlichste'' Beschreibung der Konfiguration wäre also ein Tripel aus
$Q\times(\mathbb{Z}\rightarrow\Gamma)\times\mathbb{Z}$.
Dies würde aber erfordern dass wir jeder Zelle des Bandes mit einer ganzen Zahl identifizieren
und wir müssten uns eine endliche Repräsentation für die Abbildung $\mathbb{Z}\rightarrow\Gamma$ überlegen.
Wir verwenden statt dessen den folgenden ``Trick'' und beschreiben eine Konfiguration durch ein Tripel $(v,q,w)$.
Dabei beschreibt
\begin{itemize}
	\item $v$ einen endlichen Suffix der Bandhälfte links des Kopfes der alle nicht-Blank Zeichen enthält,
	\item $q$ den aktuellen Zustand und
	\item $w$ einen endlichen Präfix der verbleibenden Bandhälfte (also unter Kopf und rechts davon) der alle nicht-Blank Zeichen enthält.
\end{itemize}
\begin{figure}[H]\centering
	\begin{tikzpicture}[every node/.style={block}, decoration={brace, amplitude=5pt}]
		\node (A) {$v$};
		\node (B) [right=of A] {$a$};
		\node (C) [right=of B] {$w'$};
		\node (D) [below=1em of B, draw=none] {Kopf; Zustand $q$}
		edge [->, shorten >=.5ex, semithick] (B);
		
		\draw [decorate, semithick] (B.north west) -- (C.north east)
		node [draw=none,above,midway] {$w$};
		\draw (A.north west) -- ++(-1cm,0) (A.south west)
		-- ++ (-1cm,0) (C.north east)
		-- ++ ( 1cm,0) (C.south east)
		-- ++ ( 1cm,0);
	\end{tikzpicture}
	\caption{Informelle graphische Darstellung einer Turingmaschinenkonfiguration}
\end{figure}

\begin{Def}[name={[Konfiguration einer \acs*{TM}]}] % 2.2
	Die Menge der Konfiguration einer \ac{TM} ist $\Konf(\mathcal{M})=\Gamma^*\times Q\times\Gamma^+$
	Die \emph{Startkonfiguration} bei Eingabe $w$ ist: $(\Eps,\qinit,w)$.
	Eine \emph{Haltekonfiguration} ist eine Konfiguration $(v,q,aw)$ bei der $\delta(q,a)=\emptyset$ gilt.
\end{Def}
Notation: Wenn aus dem Kontext klar wird welcher Buchstabe zu welche Komponente des Tripels gehört, dürfen wir die Klammern und Kommata auch weglassen. Wir schreiben z.B. die Startkonfiguration als \ $\qinit w$\ .
%



\begin{Def}[name={[Rechenschrittrelation]}] % 2.3
	Die \emph{Rechenschrittrelation}
	\[ {\vdash} \subseteq \Konf(\mathcal{M})\x\Konf(\mathcal{M}) \]
	ist definiert durch
	\footnote{Die beiden Spezialfälle dass $q$ ``ganz rechts'' oder ``ganz links'' steht wurde in der Vorlesung am 20.12.2017 vergessen. Die Definition wird am 10.1.2018 nochmal kurz besprochen werden.}
	\begin{alignat*}{3}
		1.&\ & v qaw &\vdash v q'a'w &\quad& \text{falls }\delta(q,a)=(q',a',N)\\
		2.&& v qaw &\vdash v a'q'w && \text{falls }\delta(q,a)=(q',a',R), w\neq \Eps\\
		&&&\phantom{{}\vdash{}} v a'q'\blank && \ruleplaceholder{\widthof{falls $\delta(q,a)=(q',a',R)$}}, w=\Eps\\
		3.&& qaw &\vdash q'\blank a'w && \delta(q,a)=(q',a',L)\\
		&& vbqaw &\vdash vq'ba'w && \ruleplaceholder{\widthof{$\delta(q,a)=(q',a',L)$}}\  b\in\Gamma
	\end{alignat*}
\end{Def}
Analog zu den vorigen Kapiteln schreiben wir
% $\vdash$ Einzelschritt, gesuchte Relation für endlich viele Schritte \smallskip\\
	${\vdash^*} \subseteq \Konf(\mathcal{M}) \x \Konf(\mathcal{M})$ 
	für die reflexive transitive Hülle von $\vdash$.
\begin{itemize}
 \item Das Symbol $\vdash$ steht also für eine Relation auf $\Konf(\mathcal{M})$ die einzelne Berechnungsschritte beschreibt und
 \item der Ausdruck $\vdash^*$ steht für eine Relation auf $\Konf(\mathcal{M}$ die eine endliche Sequenz von  Berechnungsschritten beschreibt.
\end{itemize}

 
%
%
\begin{Def}[Die von \acs*{TM} $\mathcal{M}$ akzeptierte Sprache] % 2.5
	\begin{align*}
		L(\mathcal{M})=\{ w\in\Sigma^* \mid{}
		& \qinit w \vdash^*uqv\\
		&uqv \text{ Haltekonfiguration}\\
		&q\in F\}
	\end{align*}
\end{Def}


\begin{Def}[name={[Terminierung]}]
 Eine \ac{DTM} $\mathcal{M}$ \emph{terminiert} auf Eingabe $w$ 
 falls es eine Haltekonfiguration $uq'v$ gibt sodass $\qinit w\vdash^*uq'v$ gilt.
\end{Def}

Falls eine \ac{DTM} $\mathcal{M}$ ein Wort $w$ nicht akzeptiert kann dies zwei Ursachen haben:
\begin{enumerate}
 \item $\mathcal{M}$ terminiert nicht auf $w$.
 \item Der Zustand der Haltekonfiguration von $\mathcal{M}$ auf $w$ ist nicht akzeptierend.
\end{enumerate}

% \emph{Beachte:}
% $w\notin L(\mathcal{M})\begin{casesarrows}
% \mathcal{M}\text{ kann anhalten}          \\
% \mathcal{M}\text{ kann nicht terminieren}
% \end{casesarrows}$

\begin{Def}[Die von einer \ac{DTM} $\mathcal{M}$ berechnete Funktion] % 2.6
Sei $\mathcal{M}$ eine \ac{DTM} dann ist die von $\mathcal{M}$ \emph{berechnete partielle
\footnote{Im Gegensatz zu einer Funktion muss eine partielle Funktion nicht für alle Elemente des Urbildes definiert sein.
Ist $f$ nur eine partielle Funktion von $X$ nach $Y$ so schreiben wir $f: X\nrightarrow Y$ statt $f: X\rightarrow Y$. }
Funktion} $f_\mathcal{M}:\Sigma^*\nrightarrow\Sigma^*$ wie folgt definiert.
$$
f_\mathcal{M}(w)= 
\begin{cases}
 v &\text{falls } \exists (u,q,v')\in \Konf(\mathcal{M}) \text{ sodass }
 \begin{array}{ll}
  & \qinit w \vdash^*uqv'\\
  \text{und} & uqv'\text{ ist Haltekonfiguration}\\
  \text{und} & v=\out( v')
 \end{array}
\\
 \mathit{undef.} & \text{sonst}
\end{cases}
$$
Dabei ist $\out$ die Funktion, die ein Wort auf das Eingabealphabet $\Sigma$ projeziert. 
Wie definieren $\out:\Gamma^* \-> \Sigma^*$ formal wie folgt.
	\begin{alignat*}{3}
		&\out(\Eps) &&= \Eps\\
		&\out(au) &&= a\cdot\out(u) &\quad& a\in\Sigma\\
		&\out(bu) &&= \out(u) && b\in\Gamma\setminus\Sigma
	\end{alignat*}
\emph{Beobachtung:} Die Funktion $f_\mathcal{M}(w)$ ist nicht definiert
genau dann wenn $\mathcal{M}$ angesetzt auf $w$ nicht terminiert.
\end{Def}

% 
% 	\begin{alignat*}{3}
% 		&&&f_\mathcal{M}:\Sigma^*\nrightarrow\Sigma^*\\
% 		&&&f_\mathcal{M}(w)=v\\
% 		&\text{ falls } &&\qinit w \vdash^*uqv'&\quad&\text{Haltekonf.}\\
% 		&\text{ und } &&v=\out( v')\\[.5em]
% 		&\out:\Gamma^* &&\-> \Sigma^*\\
% 		&\out(\Eps) &&= \Eps\\
% 		&\out(au) &&= a\cdot\out(u) &\quad& a\in\Sigma\\
% 		&\out(bu) &&= \out(u) && b\in\Gamma\setminus\Sigma
% 	\end{alignat*}
% \end{Def}
% \emph{Beachte:} Falls $\qinit w$ nicht terminiert, dann ist $f_\mathcal{M}(w)$ nicht definiert.
% 
% Eine \ac{TM} $\mathcal{M}$ terminiert nicht bei Eingabe $w$, falls für alle $uq'v$, so dass $\qinit w\vdash^*uq'v$\\
% $uq'v$ ist keine Haltekonfiguration.

\subsection{Techniken zur \ac{TM} Programmierung}
\datenote{10.01.2018}
\begin{itemize}
	\item Endlicher Speicher\\
	Zum Abspeichern eines Elements aus endl. Menge $A$ verwende
	\[ Q'=Q\x A \]
	\item Mehrspurmachinen
	\begin{figure}[H]\centering
		{\renewcommand{\arraystretch}{0.8}
		\begin{tabu} to .5\textwidth {X[.35]|X[.65]}
			&\\\hline
			&\\\hline
			&\\\hline
			&\\\hline
			&
		\end{tabu}}
		\caption{Mehrspurmachine}
	\end{figure}
	
	Eine $k$-Spur \ac{TM} kann gleichzeitig $k\geq 1$ Symbole $\<- \Gamma$ unter dem Kopf lesen.\\
	Kann durch Standard \ac{TM} simuliert werden:
	\[ \Gamma' = \Sigma \overset{.}{\cup} \Gamma^k\text{ mit } \blank'=\blank^k \]
	\dots vereinfacht die Programmierung\\
	\begin{Bsp*}
		Schulalg. für binäre Addition, Multiplikation
	\end{Bsp*}
	\item \emph{Mehrbandmachinenen}\\
	Eine $k$-Band \ac{TM} besitzt $k\geq1$ Bänder und $k$ Köpfe, die bei jedem Schritt lesen, schreiben und sich unabhängig voneinander bewegen.
	\[ \delta_K:Q\x\Gamma^k \-> Q\x\Gamma^k\x \{R,L,N\}^k \]
	\item keine herkömmliche \ac{TM} (für $k>1$)
	\item kann durch $2k+1$ Spur \ac{TM} simuliert werden:\\
	\begin{tabular}{lll}
		Spur\\
		1 & \ruleplaceholder[ Band 1 ]{.5\linewidth}\\
		2 & \hspace{.23\linewidth}\# Kopf für Band 1\\
		3 & \ruleplaceholder[ Band 2 ]{.5\linewidth}\\
		4 & \multicolumn1r{\# Kopf\qquad\ }\\
		& \vdots\\
		$2k$ & \hspace{.09\linewidth}\dots\hspace{.09\linewidth} \# Kopf $k$\\
		$2k+1$ & \# &\#\#\\
		& linker Rand & rechter Rand
	\end{tabular}
\end{itemize}
\vspace{1em}

\begin{Satz}[name={[Simulation von $k$-Band \acs*{TM} durch 1-Band \acs*{TM}]}]
	Eine $k$-Band \ac{TM} kann durch eine 1-Band \ac{TM} simuliert werden.\quad $M=(Q\dots)$
\end{Satz}
\begin{proof}
	Zeige: ein Schritt der $k$-Band \ac{TM} wird durch endlich viele Schritte auf einer 1-Band \ac{TM} simuliert.
	\begin{enumerate}
		\item Schritt: Kodierung der Konfiguration der $k$-Band \ac{TM}\\
		Definiere $M'$ als \ac{TM} mit $2k+1$ Spuren und $\Gamma'=\Gamma\cup\{\#\}$
		\begin{itemize}
			\item Die Spuren $1,3,\dots,2k-1$ enthalten das entspr. Band von $M$: Band $i\<->$ Spur $2i-1$
			\item Die Spuren $2,4,\dots,2k$ sind leer bis auf eine Marke \#, die auf Spur $2i$ die Position des Kopfes auf Band $i$ markiert
			\item Spur $2k+1$ enthält\\
			\#\phantom{\#} Marke für linken Rand\\
			\#\# Marke für rechten Rand\\
			Zwischen den beiden Marken befindet sich der bearbeitete Bereich des Bands. D.h. die \ac{TM} arbeitet zwischen der linken und rechten Marke und schiebt die Marken bei Bedarf weiter.
		\end{itemize}
		
		\item Schritt: Herstellen der Start-Konfiguration.\\
		Annahme: Eingabe für $M$ auf Band 1\\
		Jetzt Eingabe (für $M'$) $w = a_1\dots a_n$
		\begin{enumerate}
			\item Kopiere $w$ auf Spur 1
			\item Kopf setzen auf Spur $2,\dots,2k$ an die Position des ersten Symbols von $w$
			\item auf Spur $2k+1$: \verb*!# ##!
	\end{enumerate}
	\begin{tabular}{*2{M{l}}}
		2k+1 & \#\blank\#\#\\
		2k & \#\\
		2k-1 & \blank\\
		\vdots\\
		4 & \#\\
		3 & \blank\\
		2 & \#\\
		\text{Spur }1 & a_1a_2\dots a_2
	\end{tabular} 
	
	Springe nach Sim($\qinit$), der Zustand in $M'$, an dem die Simulation des Zustands $q$ aus $M$ beginnt.
	
	\item Simulation eines Rechnerschritts im Zustand Sim($q$):\\
	Kopf auf linker Begrenzung, d.h. linker \# auf Spur $2k+1$
	\begin{itemize}
		\item Durchlauf bis rechter Rand, sammle dabei Symbole unter den Köpfen, speichern in endl. Zustand $\overrightarrow{\gamma} \in \Gamma^k$
		\item Berechne $\delta(q,\overrightarrow{\gamma})=(q',\overrightarrow{\gamma'},\overrightarrow{d})$\\
		neuer Zustand, für jeden Kopf ein neues Symbol $\overrightarrow{\gamma'}$ und Richtung $\overrightarrow{d}$.
		\item Rücklauf nach links, dabei Schreiben um $\overrightarrow{\gamma}'$ und Versetzen der Köpfe gem"a"s $\overrightarrow{d}$.
	\end{itemize}
	Falls eine Kopfbewegung den Rand auf Spur $2k+1$ überschreitet, dann verschiebe Randmarke entsprechend.
	
	Beim Rücklauf: Test auf Haltekonfiguration der $k$-Band \ac{TM}.\\
	Falls ja, dann Sprung in Haltekonf. von $M'$
	
	Weiter im Zustand Sim$(q')$.
	\end{enumerate}
\end{proof}

\begin{Korollar*}
	Beim Erkunden eines Worts der Länge $n$ benötige die $k$-Band Maschine $M\ T(n)$ Schritte und $S(n)$ Zellen auf den Bändern.
	\begin{itemize}
		\item $M'$ benötigt $O(S(n))$ Zellen
		\item $M'$ benötigt $O(S(n\cdot T(n)))$ Schritte $=O(T(n)^2)$
	\end{itemize}
	Weitere \ac{TM}-Booster
	\begin{itemize}
		\item Unbeschränkt großer Speicher
		\begin{itemize}[label=\->]
			\item für jede "`Variable"' ein neues Band
	\end{itemize}
	\item Datenstrukturen
	\begin{itemize}[label={\rotatebox[origin=c]{180}{$\Lsh$}}]
		\item ensprechend kodieren.
	\end{itemize}
	\end{itemize}
\end{Korollar*}

\subsection{Das Gesetz von Church-Turing (Churchsche These)} % 2.6
\begin{Satz}[name={[Intuitiv berechenbare Funktionen sind mit \acs*{TM} berechenbar]}]
	Jede intuitiv berechenbare Funktion ist mit \ac{TM} (in formalem Sinn) berechenbar.
	
	"`Intuitiv berechenbar"' $\equiv$ man kann Algorithmus hinschreiben
	\begin{itemize}
		\item endliche Beschreibung
		\item jeder Schritt effektiv durchführbar
		\item klare Vorschrift
	\end{itemize}
	Status wie Naturgesetz -- nicht beweisbar
	\begin{itemize}[label=\->]
		\item allgemein anerkannt
		\item weitere Versuche Berechenbarkeit zu formulieren, äquivalent zu \ac{TM}en erwiesen.
	\end{itemize}
\end{Satz}

%%% Local Variables:
%%% mode: latex
%%% TeX-master: "Info_3_Skript_WS2016-17"
%%% End:
