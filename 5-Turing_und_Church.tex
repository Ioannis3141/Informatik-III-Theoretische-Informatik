\section{Turing und Church}
%\section{\acf{TM}}
1930er Jahre\\
Suche nach formalem Modell für maschinelle Berechenbarkeit
\begin{description}
	\item[Alaen Turing:] (1912-1954) Turingmaschine 1936
	\item[Church:] Lambdakalkül 1936
	\item[Kleene Sturgis:] partielle rekursive Funktionen
	\item[Chomsky:] Typ-0-Grammatiken 1956
\end{description}
\underline{Alan Turing:}\begin{minipage}[t]{.8\textwidth}
\begin{itemize}[parsep=0pt]
	\item Informatik, Logik
	\item Kryptographie (Enigma Entschlüsselung, Sprachverschlüsselung)
	\item KI (Turing-Test)
\end{itemize}\end{minipage}

außerdem: Turing-Award

\subsection{Turingmaschine \normalfont(informell)}
Ein primitives Rechenmodell:
\begin{figure}[H]\centering
	\begin{tikzpicture}[every node/.style={block}]
		\node (1) {\blank};
		\node (2) [right=of 1] {b};
		\node (3) [right=of 2] {a};
		\node (4) [right=of 3] {n};
		\node (5) [right=of 4] {a};
		\node (6) [right=of 5] {n};
		\node (7) [right=of 6] {e};
		\node (8) [right=of 7] {\blank};
		\node (9) [right=of 8] {\blank};
		\node (10) [right=of 9] {\blank};
		\node (11) [right=of 10] {\blank};
		\node (last) [right=of 11] {\blank};
		
		\node (Kopf) [below=1.5em of 2, draw=none, text height=.5em] {Kopf}
		edge [->, shorten >=.5ex, semithick] (2);
		
		\node (TB) [below=.5em of 9, draw=none, text height=.5em,  anchor=north west] {Turingband};
		
		\draw (8.south) -- ($(8.south)-(0,1em)$) -- ($(TB.north west)-(0,.5em)$);
		
		% Open begin and end.
		\draw (1.north west) -- ++(-1cm,0) (1.south west)
		-- ++ (-1cm,0) (last.north east)
		-- ++ ( 1cm,0) (last.south east)
		-- ++ ( 1cm,0);
	\end{tikzpicture}
	\vspace{-1em}
	\caption{Turingband}
	\framebox{q} = Zustand
\end{figure}
\vspace{-.5em}
\begin{tabu}{>{\bfseries}X[.22]X[.72]}
	Turingband & \vspace{-1em}\begin{itemize}[leftmargin=1em,parsep=0pt,topsep=0pt]
	\item unendliches Band
	\item Jedes Feld enthält ein Symbol aus einem Bandalphabet $\Gamma$
	\item uninitialisiert: Blank \blank ist ein spezielles Symbol $\blank\in\Gamma$
	\end{itemize}
	\\
	Kopf & \vspace{-1em}\begin{itemize}[leftmargin=1em,parsep=0pt,topsep=0pt]
	\item zeigt immer auf ein Feld
	\item nur am Kopf kann die \ac{TM} ein Zeichen lesen und schreiben
	\item kann nach rechts /links bewegt werden
	\item kann verändert werden
	\end{itemize}\\
	Zustand & \vspace{-1em}\begin{itemize}[leftmargin=1em,parsep=0pt,topsep=0pt]
	\item kann verändert werden
	\item kann gelesen werden
	\item es gibt nur endlich viele Zustände
	\end{itemize}\\
	Turingtabelle\newline\normalfont
	\begin{tabular}{|*5{c|}}
		q & a & q' & a' & d \\\hline
		&&&&\\
		&&&&
	\end{tabular}
	& $\sim$ Programm $\sim$ Transitionsfunktion \newline
	$\rightarrow$ Wenn \ac{TM} in Zustand $q$ und Kopf liest gerade Symbol $a\in\Gamma$ dann wechsle in Zustand $q'$. Schreibe $a'$ (über altes $a$) und bewege den Kopf gemäß $d\in\{L,R,N\}$
\end{tabu}\

\begin{samepage}
	\begin{Bsp*}\ 
		\vspace{-2em}
		\begin{figure}[H]\centering
			\begin{tikzpicture}[every node/.style={block}]
				\node (1) {\blank};
				\node (2) [right=of 1] {\blank};
				\node (3) [right=of 2] {\cancel{b}};
				\node (4) [right=of 3] {a};
				\node (5) [right=of 4] {n};
				\node (6) [right=of 5] {a};
				\node (7) [right=of 6] {n};
				\node (8) [right=of 7] {e};
				\node (9) [right=of 8] {\cancel{\blank}};
				\node (10) [right=of 9] {\blank};
				\node (last) [right=of 10] {\blank};
				
				\node (q1) [draw=none, above=-2pt of 3] {\blank};
				\node (q3) [draw=none, above=-2pt of 9] {b};
				
				% Open begin and end.
				\draw (1.north west) -- ++(-1cm,0) (1.south west)
				-- ++ (-1cm,0) (last.north east)
				-- ++ ( 1cm,0) (last.south east)
				-- ++ ( 1cm,0);
			\end{tikzpicture}
			\caption{Bsp.: Turingmaschine}
			\vspace{.5em}
			\begin{tabu}{|*5{M{c}|}l}\tabucline{1-5}
				\everyrow{\tabucline{1-5}}
				q_1 & b & q_2 & \blank & R\\
				q_1 & x\pm b & q_1 & x & N\\
				q_2 & \blank & q_3 & b & L\\
				q_2 & x+\blank & q_2 & x & R\\
				q_3 & x+\blank & q_3 & x & L\\
				q_3 & \blank & q_4 & \blank & L\\
				q_4 & x & q_4 & x & N & \-> Endzustand
			\end{tabu}
		\end{figure}
	\end{Bsp*}
\end{samepage}
%
\datenote{28.10.15}
Was kann die \ac{TM} ausrechnen?
\begin{enumerate}
	\item Die \ac{TM} kann eine Sprache $L\subseteq\Sigma^*$ erkennen.
	\begin{itemize}
		\item Wörter müssen auf Band repräsentierbar sein $\Sigma\subseteq\Gamma\setminus\{\blank\}$
	\end{itemize}
	Ein Wort $w$ wird von einer \ac{TM} erkannt, wenn
	\begin{itemize}
		\item zu Beginn steht nur $w$ auf dem Band, alle anderen Zellen $=\blank$
		\item Kopf auf erstem Zeichen von $w$
		\item Zustand ist Startzustand $q_0$
		\item Abarbeitung der \ac{TT}
		\item Falls \ac{TM} nicht terminert: $w\notin L$
		\item Falls \ac{TM} terminert betrachte den errechneten Zustand $q$.\\
		Falls $q\in F$ (akzeptierter Zustand), dann $w\in L$, anderenfalls $w\notin L$
	\end{itemize}
	
	\begin{Bsp*}
		\begin{flalign*}
			\Sigma &=\{0,1\} &\\
			L &=\left\{w\in\Sigma^* \mid \,w\text{ ist Palindrom}\right\}\\
			Q &= \{q_0,q_1,q_r^0, q_r^1, {q_r^0}', {q_r^1}', q_l^0, q_l^1 \} \quad F=\{q_1\}
		\end{flalign*}
		\begin{tabular}{@{}*6{M{l}}}
			q_0      & \blank & q_1      & \blank & N & q_1 \x q_1\x N\\
			q_0      & 0      & q_r^0    & \blank & R\\
			q_0      & 1      & q_r^1    & \blank & R
			\\ \cmidrule{1-5}
			q_r^0    & \blank & q_1      & \blank & N\\
			q_r^0    & 0      & {q_r^0}' & 0      & R\\
			q_r^0    & 1      & {q_r^0}' & 1      & R\\
			{q_r^0}' & \blank & q_l^0    & \blank & L & q_l\->\text{prüfe $0$, fahre zum linken Rand und weiter mit }q_0\\
			{q_r^0}' & 0      & {q_r^0}' & 0      & R & \multirow{2}{*}{\hspace{-1em}$\begin{rcases}\\[1em]\end{rcases}$ Rechtslauf} \\
			{q_r^0}' & 1      & {q_r^0}' & 1      & R
		\end{tabular}\\[.5em]
		\begin{tabular}{@{}*6{M{l}}|}
			\multicolumn{6}{@{}l|}{Alternative 1:}\\
			\multicolumn{6}{@{}l|}{\ac{TM} hält bei jeder Eingabe an.}\\[.5em]
			q_l^0 & \blank & q_l^0  & \blank & N & \<-\text{Halt}\\
			& 0   & q_l    & \blank & L &\\
			& 1   & q_l^0  & 1      & N & \<-\text{Halt}
		\end{tabular}\quad\begin{tabular}{@{}*5{M{l}}@{ }l}
		\multicolumn{6}{@{}l}{Alternative 2:}\\
		\multicolumn{6}{@{}l}{\ac{TM} hält nur bei Palindrom an.}\\[.5em]
		q_l^0 & \blank & q_l^0 & 1      & N & \multirow{3}{*}{\scalebox{2.9}{\rotatebox[origin=rb]{-90}{$\curvearrowleftright$}}}\\
		q_l^0 & 0      & q_l   & \blank & L\\
		q_l^0 & 1      & q_l^0 & \blank & N
		\end{tabular}
	\end{Bsp*}
	
	\item Die \ac{TM} errechnet Funktion $f: \Sigma^*\-->\Sigma^*$\\
	Die Berechnung von $f(w),\ w\in\Sigma^*$
	\begin{itemize}
		\item $w$ auf leeres Band
		\item Kopf auf erstes Zeichen, Standardzustand $q_0$
		\item Abarbeitung der \ac{TT}
		\item Falls terminiert, dann Kopf zuerst auf erste Symbol von $v\in\Sigma^*$\\
		Dann $f(w)=v$
	\end{itemize}
\end{enumerate}
\begin{alignat*}{3}
	\text{Schreibe}&\quad& A &\-->B &\quad& \text{totale Funktion von $A$ nach $B$}\\
	&& A&\dashrightarrow B && \text{partielle Funktion von $A$ nach $B$}
\end{alignat*}
\begin{Bsp} % 2.1
	$\Sigma=\{0,1\}$\\
	Gesucht die \ac{TM}, die die Nachfolgefunktion auf natürliche Zahlen in Binärdarstellung berechnet.\\
	Ausnahme: niederwertigste Stelle von der Zahl.\medskip\\
	\begin{tabular}{@{}M{l}@{ } *5{M{l}} @{ }l}
		\xrightarrow{\text{Start}} & q_0 & \blank & q_2 & 1 & L \\
		& q_0 & 0      & q_1 & 1 & L \\
		& q_1 & 1      & q_0 & 0 & R \\[.5em]
		& q_1 & \blank & q_1 & \blank & N & \<-Halt\\
		& q_1 & 0      & q_2 & 0 & L & \multirow{2}{*}{$\begin{rcases}\\[1em]\end{rcases}$ Linksmaschine}\\
		& q_1 & 1      & q_2 & 1 & L
	\end{tabular}
\end{Bsp}

\subsection{Formalisierung der \ac{TM}} % 2.2
\begin{Def}[name={[\acs*{TM}]}] % 2.1
	Eine \ac{TM} ist ein 7-Tupel
	\begin{equation*}
		\mathcal{A}=\left(Q,\Sigma,\Gamma,\delta,q_0,\blank,F\right)\\
	\end{equation*}
	\begin{itemize}
		\item $Q$ ist endliche Menge der Zustände
		\item $\Sigma$ ist endliches Alphabet
		\item $\Gamma\supsetneq\Sigma$ ist endliches Bandalphabet
		\item $\delta: Q\times\Gamma\-->Q\times\Gamma\times\{R,L,N\}$
		\item $q_0\in Q$ Standardzustand
		\item $\blank\in\Gamma\setminus\Sigma$ das Blank
		\item $F\subseteq Q$ Menge der akzeptierenden Zustände
	\end{itemize}
\end{Def}
Im Folgenden sei $\A=(Q, \Sigma, \Gamma, \delta, q_0, \blank, F)$ eine \ac{TM}.

\begin{Def}[name={[Konfiguration einer \acs*{TM}]}] % 2.2
	Eine Konfiguration einer \ac{TM}
	ist ein Tupel
	\[ (v,q,w) \in \Konf(\A)=\Gamma^*\times Q\times\Gamma^+  \]
\end{Def}
%
\begin{itemize}
	\item $v$ linke Bandhälfte,
	\item $q$ Zustand,
	\item $w$ rechte Bandhälfte,
	\item Kopfpos auf erstem Symbol von $w$
\end{itemize}

Abk"urzend $ v qw \in \Konf(\mathcal{A}) $ steht für Band:
\begin{figure}[H]\centering
	\begin{tikzpicture}[every node/.style={block}, decoration={brace, amplitude=5pt}]
		\node (A) {$v$};
		\node (B) [right=of A] {$a$};
		\node (C) [right=of B] {$w'$};
		\node (D) [below=1em of B, draw=none] {Kopf; Zustand $q$}
		edge [->, shorten >=.5ex, semithick] (B);
		
		\draw [decorate, semithick] (B.north west) -- (C.north east)
		node [draw=none,above,midway] {$w$};
		\draw (A.north west) -- ++(-1cm,0) (A.south west)
		-- ++ (-1cm,0) (C.north east)
		-- ++ ( 1cm,0) (C.south east)
		-- ++ ( 1cm,0);
	\end{tikzpicture}
	\caption{$vqw$-Band}
\end{figure}
%
\paragraph{Forts.:} Die \underline{Startkonfiguration} bei Eingabe $w$ ist:
$\begin{cases}
q_0\,w &, w\neq\epsilon\\
q_0\,\blank &, w=\epsilon
\end{cases}$\\
Eine \underline{Haltekonfiguration} hat folgende Form: $vqaw$, so dass $\delta(q,a)=(q,a,N)$

\begin{Def}[name={[Rechenschrittrelation]}] % 2.3
	Die \underline{Rechenschrittrelation}
	\[ \vdash\, \subseteq \Konf(\A)\x\Konf(A) \]
	ist definiert durch
	\begin{alignat*}{3}
		1.&\ & v qaw &\vdash v q'a'w &\quad& \text{falls }\delta(q,a)=(q',a',N)\\
		2.&& v qaw &\vdash v a'q'w && \text{falls }\delta(q,a)=(q',a',R), w\neq \epsilon\\
		&&&\phantom{{}\vdash{}} v a'q'\blank && \ruleplaceholder{\widthof{falls $\delta(q,a)=(q',a',R)$}}, w=\epsilon\\
		3.&& qaw &\vdash q'\blank a'w && \delta(q,a)=(q',a',L)\\
		&& vbqaw &\vdash vq'ba'w && \ruleplaceholder{\widthof{$\delta(q,a)=(q',a',L)$}}\  b\in\Gamma
	\end{alignat*}
	$\vdash$ Einzelschritt, gesuchte Relation für endlich viele Schritte \smallskip\\
	$\vdash^* \subseteq \Konf(\A) \x \Konf(\A)$ die reflexive transitive Hülle von $\vdash$
\end{Def}


\datenote{30.10.15}
$\vdash$ Relation auf $\Konf(\A)\hat=$ Berechnungsschritt\\
$\vdash^*$ reflexiv, transitiver Abschluss $\hat=$ endl.\ viele Berechnungsschritte
%
\paragraph{Exkurs:} binäre Relationen\\
$A$ Menge, jedes $R\subseteq A\x A$ ist binäre Relation auf A\\
\begin{Bsp*}\ \\
	\begin{tabular}{M{r}l}
		\varnothing & leere Relation \\
		A\x A     & volle Relation \\
		\1_A =\{(x,x) \mid x\in A\} & Gleichheit auf $A$\\
		\leq\,\subseteq\N\x\N\quad <\, \subseteq \N\x\N\quad & $\mid\, \subseteq\N\x\N$
	\end{tabular}\medskip\\
	Mengenoperationen auf Rel. ok
	\begin{align*}
		R_1 &=\, <\cup\, \1_\N =\, \leq\\
		(x,y) &\in R_1 \<=> x<y\ \lor\ x=y\\
		R_2 &= <\cap\, \1_\N = \varnothing\\
		F_3 &= \{(x,y)\mid y=3x\}\subseteq \N\x\N
		% y durchgestrichen
	\end{align*}
\end{Bsp*}
\refstepcounter{Def}
\begin{subDef}[name={[$R,S\subseteq A\x A$ Relation]}] %2.4.1
	$R,S\subseteq A\x A$ Relation\\
	Die Verkettung (Komposition) von $R$ und $S$
	\[ R\circ S = \{(x,y)\in A\times A \mid \exists z\in A: (x,z)\in R,(z,y)\in S)\} \]
\end{subDef}
\begin{Bsp*}
	\begin{align*}
		&\underbrace{<\circ\1_\N}_{=S_1} =\, <\\
		&(x,y)\in S_1\\
		\<=>\ & \exists z: x<z\ ,z=y\\
		\<=>\ & x<y\\
		&F_3\circ F_3\\
		&(x,y)\in F_3\circ F_3\\
		\<=>\ & \exists z: (x,z)\in F_3\ ,\ (z,y)\in F_3\\
		\<=>\ & \exists z: z=3x\ \land\ y=3z\\
		\<=>\ & y=9x
	\end{align*}
\end{Bsp*}
\begin{subDef}[name={[$R\subseteq A\x A$]}] % 2.4.2
	$R\subseteq A\x A$
	\begin{enumerate}[label={\alph*)}]
		\item R ist \underline{reflexiv}, falls $\1_A\subseteq R$
		\item R ist \underline{transitiv}, falls $R\circ R \subseteq R$
	\end{enumerate}
\end{subDef}
\begin{Bsp*}
	\begin{tabular}[t]{M{l}ll}
		\1_A & refl. & trans.\\
		\leq_\N & refl. & trans.\\
		\varnothing & nicht refl. & trans.\\
		\mid & refl. & trans.\\
	\end{tabular}
\end{Bsp*}
\begin{subDef}[name={[$R\subseteq A\x A$ Relation]}] % 2.4.3
	$R\subseteq A\x A$ Relation\\
	Der reflexiv transitive Abschluss (Hülle) von $R$ ist
	\begin{align*}
		R^* &=\bigcup\limits_{n\in\N} R^n\\
		\shortintertext{mit}
		R^0 &=\1_A \quad,\ R^{n+1}=R\circ R^n
	\end{align*}
\end{subDef}
%
Bem:\quad $R^* =1\cup R\circ R^*$ gilt auch.

Es gilt: Für bel. $R$:
\begin{itemize}
	\item $R^*$ refl. \quad:\quad $\1 = R^0\subseteq\bigcup\limits_{n\in\N}R^n=R^*$
	\item $R^*$ trans.
	\begin{align*}
		R^* &\circ R^* \subseteq R^*\\
		R^* &\circ \bigcup_n R^n\\
		\forall n: R^* &\circ R^n \subseteq R^*\\
		\=> R^* &\circ \bigcup_n R^n \subseteq R^*
	\end{align*}
%	\rlerror*{Wortkorrektur nötig}{Refvim?}
\end{itemize}
${\vdash^2} = {\vdash}\circ\vdash$ zwei Schritte\\
$\vdash^*$ endl. viele Schritte
%
\begin{Def}[Die von \acs*{TM} $\A$ erkannte Sprache] % 2.5
	\begin{align*}
		L(\A)=\{ w\in\Sigma^* \mid{}
		& q_0 w \vdash^*uqv\\
		&uqv \text{ Haltekonfiguration}\\
		&q\in F\}
	\end{align*}
\end{Def}
\underline{Beachte:}
$w\notin L(\A)\begin{casesarrows}
\A\text{ kann anhalten}          \\
\A\text{ kann nicht terminieren}
\end{casesarrows}$

\begin{Def}[Die von \ac{TM} $\A$ berechnete Funktion] % 2.6
	\begin{alignat*}{3}
		&&&f_\A:\Sigma^*- \->\Sigma^*\\
		&&&f_\A(w)=v\\
		&\text{ falls } &&q_0w \vdash^*uqv'&\quad&\text{Haltekonf.}\\
		&\text{ und } &&v=\out( v')\\[.5em]
		&\out:\Gamma^* &&\-> \Sigma^*\\
		&\out(\epsilon) &&= \epsilon\\
		&\out(au) &&= a\cdot\out(u) &\quad& a\in\Sigma\\
		&\out(bu) &&= \epsilon && b\in\Gamma\setminus\Sigma
	\end{alignat*}
\end{Def}
\underline{Beachte:} Falls $q_0 w$ nicht terminiert, dann ist $f_\A(w)$ nicht definiert.

Eine \ac{TM} $\A$ terminiert nicht bei Eingabe $w$, falls für alle $uq'v$, so dass $q_0w\vdash^*uq'v$\\
$uq'v$ ist keine Haltekonfiguration.

\subsection{Techniken zur \ac{TM} Programmierung}
\begin{itemize}
	\item Endlicher Speicher\\
	Zum Abspeichern eines Elements aus endl. Menge $A$ verwende
	\[ Q'=Q\x A \]
	\item Mehrspurmachinen
	\begin{figure}[H]\centering
		{\renewcommand{\arraystretch}{0.8}
		\begin{tabu} to .5\textwidth {X[.35]|X[.65]}
			&\\\hline
			&\\\hline
			&\\\hline
			&\\\hline
			&
		\end{tabu}}
		\caption{Mehrspurmachine}
	\end{figure}
	
	Eine $k$-Spur \ac{TM} kann gleichzeitig $k\geq 1$ Symbole $\<- \Gamma$ unter dem Kopf lesen.\\
	Kann durch Standard \ac{TM} simuliert werden:
	\[ \Gamma' = \Sigma \overset{.}{\cup} \Gamma^k\text{ mit } \blank'=\blank^k \]
	\dots vereinfacht die Programmierung\\
	\begin{Bsp*}
		Schulalg. für binäre Addition, Multiplikation
	\end{Bsp*}
	\item \underline{Mehrbandmachinenen}\\
	Eine $k$-Band \ac{TM} besitzt $k\geq1$ Bänder und $k$ Köpfe, die bei jedem Schritt lesen, schreiben und sich unabhängig voneinander bewegen.
	\[ \delta_K:Q\x\Gamma^k \-> Q\x\Gamma^k\x \{R,L,N\}^k \]
	\item keine herkömmliche \ac{TM} (für $k>1$)
	\item kann durch $2k+1$ Spur \ac{TM} simuliert werden:\\
	\begin{tabular}{lll}
		Spur\\
		1 & \ruleplaceholder[ Band 1 ]{.5\linewidth}\\
		2 & \hspace{.23\linewidth}\# Kopf für Band 1\\
		3 & \ruleplaceholder[ Band 2 ]{.5\linewidth}\\
		4 & \multicolumn1r{\# Kopf\qquad\ }\\
		& \vdots\\
		$2k$ & \hspace{.09\linewidth}\dots\hspace{.09\linewidth} \# Kopf $k$\\
		$2k+1$ & \# &\#\#\\
		& linker Rand & rechter Rand
	\end{tabular}
\end{itemize}
\vspace{1em}

\datenote{04.11.15}
\begin{Satz}[name={[Simulation von $k$-Band \acs*{TM} durch 1-Band \acs*{TM}]}]
	Eine $k$-Band \ac{TM} kann durch eine 1-Band \ac{TM} simuliert werden.\quad $M=(Q\dots)$
\end{Satz}
\begin{proof}
	Zeige: ein Schritt der $k$-Band \ac{TM} wird durch endlich viele Schritte auf einer 1-Band \ac{TM} simuliert.
	\begin{enumerate}
		\item Schritt: Kodierung der Konfiguration der $k$-Band \ac{TM}\\
		Definiere $M'$ als \ac{TM} mit $2k+1$ Spuren und $\Gamma'=\Gamma\cup\{\#\}$
		\begin{itemize}
			\item Die Spuren $1,3,\dots,2k-1$ enthalten das entspr. Band von $M$: Band $i\<->$ Spur $2i-1$
			\item Die Spuren $2,4,\dots,2k$ sind leer bis auf eine Marke \#, die auf Spur $2i$ die Position des Kopfes auf Band $i$ markiert
			\item Spur $2k+1$ enthält\\
			\#\phantom{\#} Marke für linken Rand\\
			\#\# Marke für rechten Rand\\
			Zwischen den beiden Marken befindet sich der bearbeitete Bereich des Bands. D.h. die \ac{TM} arbeitet zwischen der linken und rechten Marke und schiebt die Marken bei Bedarf weiter.
		\end{itemize}
		
		\item Schritt: Herstellen der Start-Konfiguration.\\
		Annahme: Eingabe für $M$ auf Band 1\\
		Jetzt Eingabe (für $M'$) $w = a_1\dots a_n$
		\begin{enumerate}
			\item Kopiere $w$ auf Spur 1
			\item Kopf setzen auf Spur $2,\dots,2k$ an die Position des ersten Symbols von $w$
			\item auf Spur $2k+1$: \verb*!# ##!
	\end{enumerate}
	\begin{tabular}{*2{M{l}}}
		2k+1 & \#\blank\#\#\\
		2k & \#\\
		2k-1 & \blank\\
		\vdots\\
		4 & \#\\
		3 & \blank\\
		2 & \#\\
		\text{Spur }1 & a_1a_2\dots a_2
	\end{tabular} 
	
	Springe nach Sim($q_0$), der Zustand in $M'$, an dem die Simulation des Zustands $q$ aus $M$ beginnt.
	
	\item Simulation eines Rechnerschritts im Zustand Sim($q$):\\
	Kopf auf linker Begrenzung, d.h. linker \# auf Spur $2k+1$
	\begin{itemize}
		\item Durchlauf bis rechter Rand, sammle dabei Symbole unter den Köpfen, speichern in endl. Zustand $\overrightarrow{\gamma} \in \Gamma^k$
		\item Berechne $\delta(q,\overrightarrow{\gamma})=(q',\overrightarrow{\gamma'},\overrightarrow{d})$\\
		neuer Zustand, für jeden Kopf ein neues Symbol $\overrightarrow{\gamma'}$ und Richtung $\overrightarrow{d}$.
		\item Rücklauf nach links, dabei Schreiben um $\overrightarrow{\gamma}'$ und Versetzen der Köpfe gem"a"s $\overrightarrow{d}$.
	\end{itemize}
	Falls eine Kopfbewegung den Rand auf Spur $2k+1$ überschreitet, dann verschiebe Randmarke entsprechend.
	
	Beim Rücklauf: Test auf Haltekonfiguration der $k$-Band \ac{TM}.\\
	Falls ja, dann Sprung in Haltekonf. von $M'$
	
	Weiter im Zustand Sim$(q')$.
	\end{enumerate}
\end{proof}

\begin{Korollar*}
	Beim Erkunden eines Worts der Länge $n$ benötige die $k$-Band Maschine $M\ T(n)$ Schritte und $S(n)$ Zellen auf den Bändern.
	\begin{itemize}
		\item $M'$ benötigt $O(S(n))$ Zellen
		\item $M'$ benötigt $O(S(n\cdot T(n)))$ Schritte $=O(T(n)^2)$
	\end{itemize}
	Weitere \ac{TM}-Booster
	\begin{itemize}
		\item Unbeschränkt großer Speicher
		\begin{itemize}[label=\->]
			\item für jede "`Variable"' ein neues Band
	\end{itemize}
	\item Datenstrukturen
	\begin{itemize}[label={\rotatebox[origin=c]{180}{$\Lsh$}}]
		\item ensprechend kodieren.
	\end{itemize}
	\end{itemize}
\end{Korollar*}

\subsection{Registermaschinen}
\begin{figure}[H]\centering
	\begin{tikzpicture}[every node/.style={block}, decoration={brace, amplitude=5pt}]
		\node (A) {};
		\node (A1) [draw=none, below=of A] {$r_1$};
		\node (B) [right=of A] {};
		\node (B1) [draw=none, below=of B] {$r_2$};
		\node (C) [right=of B] {};
		\node (C1) [draw=none, below=of C] {\dots};
		\node (D) [right=of C] {};
		\node (E) [right=of D] {};
		\draw (E.north east)
		-- ++ ( 1cm,0) (E.south east)
		-- ++ ( 1cm,0);
	\end{tikzpicture}
	\caption{Registermaschine}
\end{figure}
\vspace{-1em}
Register:
\begin{itemize}
	\item unendlich viele Register
	\item jedes enthält natürliche Zahl
\end{itemize}
\framebox{pc} Befehlszähler $\in\N$\\
Programm = \underline{endliche} Tabelle von Instruktionen.

Folgende Instruktionen gibt es:
\begin{itemize}
	\item $\inc(i)$ \ Inkrementiere $r_i$
	\item $\dec(i)$ \ Dekrementiere $r_i$
	\item if0($i$) goto $j$ \ Falls $r_i=0$ springe nach Instruktion $j$
\end{itemize}
\begin{Bsp}
	Addiere $r_1$ und $r_2$ Ergebnis in Register $r_1$.
	
	0: if0(2) goto 4\\
	1: dec(2)\\
	2: inc(1)\\
	3: if0(3) goto 0
	
	Terminert, falls $pc$ nicht mehr auf Instruktion zeigt.
\end{Bsp}
\begin{Def}[\acf{RM}]
	Die \underline{\ac{RM}-Instruktion} sind gegeben durch
	\[ \text{Instr} = \{\inc, \dec\}\x\N\ \cup\ \{\text{if0}\}\x\N\x\N \]
	Ein \underline{\ac{RM}-Programm} ist eine Abbildung\\
	$P:[0\dots m]\->\Instr$
\end{Def}
\begin{Def}[Semantik der \ac{RM} bezüglich eines Programms $P$]\
	
	Der Zustandsraum der \ac{RM} ist
	\begin{align*}
		Z ={} &\N\x\N^\N\\
		&pc \quad \text{Registerinhalte entspr. }\N\->\N
	\end{align*}
	Die Semantik eines \ac{RM}-Befehls:
	\begin{align*}
		\llbracket\cdot\rrbracket &: \Instr \-> Z\-> Z\\
		\llbracket\inc(i)\rrbracket(pc,r) &= (pc+1,r[i\mapsto r(i)+1])\\
		\llbracket\dec(i)\rrbracket(pc,r) &= (pc+1,r[i\mapsto r(i)\overset{\text{\normalsize .}}-1])\\
		\llbracket\text{if0($i$) goto }j\rrbracket(pc,r) &= 
		\begin{cases}
			(pc+1, r) &, r(i)\neq 0\\
			(j,r) &, r(i)=0
		\end{cases}
	\end{align*}
	%
	Semantik eines Programms
	\begin{align*}
		\llbracket\cdot\rrbracket &: \text{Prog} \-> Z- \-> Z\\
		\llbracket P \rrbracket(pc,r) &=
		\begin{cases}
			\llbracket P \rrbracket \left(\llbracket \text{ins} \rrbracket(pc,r)\right) &P(pc)=\text{ins}\\
			(pc,r) &P(pc)\text{ nicht definiert}
		\end{cases}
	\end{align*}
\end{Def}
Dabei wird verwendet: 
\begin{enumerate}
	\item 
	Funktionsupdate $f[a \mapsto b]$ 
	\begin{align*}
		f&: A\->B\\
		f[a\mapsto b] &=: f'\\
		\text{mit } f'(x) &=
		\begin{cases}
			b & x=a \\
			f(x) & x\ne a
		\end{cases}
	\end{align*}
	\item abgeschnittene Subtraktion
	\begin{align*}
		m \overset{\text{\normalsize .}}- n &=
		\begin{cases}
			m-n &,m\geq n\\
			0 &, m<n
		\end{cases}
	\end{align*}
\end{enumerate}


\subsection*{\acl{RM}\datenote{06.11.15}}
\[ P:([0\dots m] \-> \text{Instr}) = \text{Prog} \]
Die Ein- und Ausgabefunktionen seien wie folgt definiert:
\begin{align*}
	\text{in}^{(n)} &:\N^n \-> Z & \text{in}^{(n)}(\overrightarrow{x}) &= \left( 0,r_0[i\->x_i] \right) \quad \forall n: r_0(n)=0\\
	\out^{(m)} &: Z \-> \N^m & \out^{(m)}(pc,r) &= \left( r(1),\dots,r(m) \right)\\
	\shortintertext{Die von $P$ berechnete Funktion}
	f_P: &\N^n \->\N^m & f_P &= \out^{(m)} \circ [[P]]\circ \text{in}^{(n)}
\end{align*}

\subsection{Simulation} % 2.5
\begin{Satz}[name={[Simulation von \acs*{RM} durch \acs*{TM}]}] % 2.2
	Jedes \ac{RM}-Programm kann durch eine \ac{TM} simuliert werden.
\end{Satz}
\begin{proof}
	Das \ac{RM}-Programm benutzt nur Register $[0\dots K]$\\
	Verwende eine $K+2$-Band \ac{TM}:
	\begin{itemize}
		\item Band $1\dots k+1$: Registerinhalt binär kodiert
		\item Band 0: Ein-/Ausgabe
		\item Kodiere $pc\ [0\dots m]$ im Zustand der \ac{TM}
	\end{itemize}
	Initialzustand
	\begin{itemize}
		\item Dekodiere Eing. $v_1\#v_2\dots \#{v_n}$\\
		Kopiere $v_i\->\operatorname{Band}_i$
		\item Lösche Band 0
		\item Initialisiere weitere Bänder 0
		\item Zustand $\curvearrowright\Sim(pc=0)$
	\end{itemize}
	\paragraph{Schritt:}
	\begin{itemize}
		\item Falls $P(pc)=\operatorname{Inc}(i)$ wende \ac{TM} aus Beispiel 2.2 %\ref{bsp:2.2}
		auf Band $i+1$ an
		\item Zustand $\curvearrowright \Sim(pc+1)$
		\item Falls $P(pc)=\operatorname{Dec}(i)$
		\begin{itemize}
			\item wende \ac{TM} für Dec auf Band $i+1$ an
			\item Zustand $\curvearrowright \Sim(pc+1)$
	\end{itemize}
	\item Falls $P(pc) =$ If $O(i)$ goto $j$
	\begin{itemize}
		\item Teste ob Band $i+1=0$:\\
		Zustand $\curvearrowright \begin{cases}
		\Sim(pc=j) & ,\text{Band inf 0}\\
		\Sim(pc+1)
		\end{cases}$
	\end{itemize}
	\item Falls $P(pc)$ nicht definiert ist\\
	Kopiere Ausgaben von Band $1-n$ nach Band 0\\
	$\->$ Haltezustand
	\end{itemize}
\end{proof}

\begin{Satz}[name={[Simulation von \acs*{TM} durch \acs*{RM}]}]
	Jede \ac{TM} kann durch ein \ac{RM}-Programm simuliert werden.
\end{Satz}
\begin{proof}
	Wir nummerieren:
	\begin{itemize}
		\item die Zustände $Q=\{q_0,\dots,q_z\}$
		\item die Elemente des Bandalphabets $\Gamma=\{a_0,\dots,a_s\}$ wobei $a_0=\blank$ (Blank hat Nummer 0)
	\end{itemize}
	Nun können wir Zustände/Elemente des Bandalphabets als Zahlen $[0\dots z]/[0\dots s]$ interpretieren.\\[1em]
	\begin{minipage}{.5\textwidth}
		\begin{itemize}
			\item Register 0 rechter Teil des Bandes
			\item Register 1 linker Teil des Bandes
			\item Register $Z$ enthält Zustand
		\end{itemize}
	\end{minipage}\begin{minipage}{.4\textwidth}\centering
	    \captionsetup{type=figure}
		\begin{tikzpicture}[every node/.style={block}]
			\node (b) {$b_1$};
			\node (l) [draw=none, left=of b] {\dots};
			\node (r) [draw=none, right=of b] {\dots};
			
			\draw [{Bar[]}-{Bar[]},semithick] ($(b.north west)+(0,.5em)$) -- ($(b.north east)+(1cm,.5em)$);
			
			\draw [->,semithick,shorten >=.5ex] ($(b.south)-(0,1.5em)$) -- (b.south);
			
			% Open begin and end.
			\draw (b.north west) -- ++(-1cm,0) (b.south west)
			-- ++ (-1cm,0) (b.north east)
			-- ++ ( 1cm,0) (b.south east)
			-- ++ ( 1cm,0);
		\end{tikzpicture}
		\captionof{figure}{\ac{TM} Simulation durch \ac{RM}}
	\end{minipage}
	
	Allgemein wird Konfiguration $c_m\dotsc\ c_1\ q\ b_1\dots b_n$ repräsentiert als
	\begin{align*}
		r_0 &= b_1\cdot (s+1)^0+b_2\cdot(s+1)^1+ \dots +b_n\cdot(s+1)^{n-1}\\
		r_1 &= c_1\cdot (s+1)^0+c_2\cdot(s+1)^1+ \dots +c_m\cdot(s+1)^{m-1}
	\end{align*}
	\paragraph{Initialisierung:}
	\begin{itemize}[label=\textbullet]
		\item Eingabe von $b_1,\dots,b_n$ kodiert als Zahl in $r_0=b_1+b_2(s+1)+\dots+b_n(s+1)^{n-1}$
		\item $r_1=0$
		\item $r_2=0$ Startzustand $q_0$
	\end{itemize}
	Schritt
	\begin{align*}
		r_3 &:= r_0 \mod(s+1) &&\text{erstes Symbol}\\
		r_0 &:= r_1 \div(s+1) &&"(r_0<<1)"\\
		\delta(q,a) &= (q',a',d) &&\text{Hält, falls }q=q',a=a',d=N\\
		\phantom{\delta(}r_2,r_3\\
		r_2 &:= q'\\
		r_3 &:= a'
	\end{align*}
	\begin{itemize}
		\item If $d=N$
		\[ r_0 := r_3+(s+1)r_0 \]
		\item else if $d=R$
		\[ r_1 := r_3+(s+1)r_1 \quad "(r_1>>1)" \]
		\item else if $d=L$
		\begin{align*}
			r_0 &:= r_3+(s+1)r_0\\
			r_3 &:= r_1\mod(s+1)\\
			r_1 &:= r_1\div(s+1)\\
			r_0 &:\phantom{=} r_3+(s+1)r_0
		\end{align*}
	\end{itemize}
\end{proof}

\subsection{Das Gesetz von Church-Turing (Churchsche These)} % 2.6
\begin{Satz}[name={[Intuitiv berechenbare Funktionen sind mit \acs*{TM} berechenbar]}]
	Jede intuitiv berechenbare Funktion ist mit \ac{TM} (in formalem Sinn) berechenbar.
	
	"`Intuitiv berechenbar"' $\equiv$ man kann Algorithmus hinschreiben
	\begin{itemize}
		\item endliche Beschreibung
		\item jeder Schritt effektiv durchführbar
		\item klare Vorschrift
	\end{itemize}
	Status wie Naturgesetz -- nicht beweisbar
	\begin{itemize}[label=\->]
		\item allgemein anerkannt
		\item weitere Versuche Berechenbarkeit zu formulieren, äquivalent zu \ac{TM}en erwiesen.
	\end{itemize}
\end{Satz}