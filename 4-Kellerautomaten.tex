\section[Kellerautomaten (\acs*{PDA})]{Kellerautomaten \quad\normalfont\normalsize \acf{PDA}}
Kellerautomat $\approx$ Endlicher Automat + Kellerspeicher von unbeschränkter Größe (Stack, push down)
\paragraph*{Neu:}
\begin{itemize}
	\item bei jedem Schritt darf der \ac{PDA} das oberste Kellersymbol inspizieren und durch beliebiges Kellerwort ersetzen (den neuen Kellerpräfix).
	\item der \ac{PDA} darf auf dem Keller rechnen, ohne in der Eingabe weiter zu lesen. ($\Eps$-Transition oder Spontantransition).
\end{itemize}
\begin{Bsp}
\label{bsp:pda-wwr}
	\begin{align}
		\Sigma &= \{0,1\} &\qquad&\text{Eingabealphabet} \notag\\
		\Gamma &= \{0,1,\bot\} &&\text{Kelleralphabet} \notag\\
		Q &= \{q_0,q_1\} &&\bot\,\hat=\,\text{Kellerbodensymbol} \notag\\
		\delta(q_0,a,Z) &= \{(q_0,aZ) \}&& a\in\{0,1\},Z\in\Gamma\\
		\delta(q_0,\Eps,Z) &= \{(q_1,Z)\} \label{eq:5.2}\\
		\delta(q_1,a,a) &= \{(q_1,\Eps)\}\\
		\delta(q_1,\Eps,\bot) &= \{(q_1,\Eps)\}
	\end{align}
  Die Übergangsfunktion $\delta$ bildet den aktuellem Zustand, das aktuelle Eingabesymbol und das aktuell oberste Kellersymbol auf Paare von Folgezustand und neuem Kellerpräfix ab.
  In diesem Beispiel ist die zurückgegebene Paarmenge in allen Fällen einelementig.
  Durch die $\Eps$ Transitionen ist der PDA aber trotzdem nichtdeterministisch.

  Bei der graphischen Darstellung werden die Transitionen mit Tripeln $a;Z;\gamma$ beschriftet, wobei $a \in \Sigma$ das Eingabesymbol, $Z \in \Gamma$ das oberste Kellersymbol und $\gamma \in \Gamma^*$ der neue Kellerpräfix ist:
  \begin{center}
  \begin{tikzpicture}[node distance = 3cm]
    \node[state] (0) at (0,0) {$q_0$};
    \node[node distance = 1cm, left of = 0] (start) {};
    \node[state, right of = 0] (1) {$q_1$};

    \draw[->] (start) to (0);
    \draw[->, loop above] (0) to node{$a;Z;aZ$} (0);
    \draw[->] (0) to node[auto]{$\Eps;Z;Z$} (1);
    \draw[->, loop above] (1) to node[auto]{$a;a;\Eps$} (1);
    \draw[->, loop right] (1) to node[auto]{$\Eps;\bot;\Eps$} (1);
  \end{tikzpicture}
\end{center}
wobei hier $a \in \Sigma$ und $Z \in \Gamma$.

Der Automat beginnt im Startzustand $q_o$ mit einem Kellerspeicher, der nur das Kellerbodensymbol $Z_0$ enthält.
Er akzeptiert ein Wort, wenn er alle Eingabesymbole gelesen und den Keller komplett leeren konnte.
Anders als bei EAs gibt es \emph{keine} finalen Zustände.

Die erkannte Sprache ist hier $L=\{ww^R \mid w\in\{0,1\}^*\}$.
\end{Bsp}


\begin{Def}[name={[NPDA]}]
	Ein \ac{NPDA} ist Tupel $(Q,\Sigma,\Gamma,q_0,Z_0,\delta)$
	\begin{itemize}
		\item $Q$ endliche Zustandsmenge
		\item $\Sigma$ endliches Eingabealphabet
		\item $\Gamma$ endliches Kelleralphabet
		\item $q_0\in Q$ Startzustand
		\item $Z_0\in\Gamma$ Kellerbodensymbol
		\item $\delta: Q\x(\Sigma\cup\{\Eps\})\x\Gamma \-> \mathcal{P}(Q\x\Gamma^*)$ \qedhere
	\end{itemize}
\end{Def}
Im weiteren sei $M=(\dots)$ ein \ac{NPDA}.
\begin{Def}[name={[Menge der Konfigurationen eines \acs*{NPDA}]}]
	Die Menge der \emph{Konfigurationen} von $M$ ist $\Konf(M) = Q\x\Sigma^*\x\Gamma^*$\\
	Die \emph{Schrittrelation} von $M$
  \begin{displaymath}
    \vdash\ \subseteq \Konf(M) \times \Konf(M) 
  \end{displaymath}
  ist definiert durch
	\begin{align*}
		(q,aw,Z\gamma) &\vdash (q',w,\beta\gamma) &&\text{falls }\delta(q,a,z)\ni(q',\beta)\\
		(q,w,Z\gamma) &\vdash (q',w,\beta\gamma) &&\text{falls }\delta(q,\Eps,z)\ni(q',\beta)\\
	\end{align*}

  Wir schreiben $(q,w,\gamma) \vdash^n (q',w',\gamma')$ wenn $M$ in $n \in \mathbb{N}$ Schritten von Konfiguration $(q,w,\gamma)$ nach Konfiguration $(q',w',\gamma')$ gelangt.

  Wir schreiben ${\vdash^*}$ für die reflexive, transitive Hülle von ${\vdash}$.
  Falls $(q,w,\gamma) \vdash^* (q',w',\gamma')$ existiert also $n \in \mathbb{N}$, so dass $(q,w,\gamma) \vdash^n (q',w',\gamma')$.
  
	Die von $M$ \emph{erkannte Sprache} ist
  \begin{displaymath}
		L(M) = \{ w\in\Sigma^* \mid (q_0,w,Z_0) \vdash^{\!\!*} (q',\Eps,\Eps) \} \tag*{\qedhere}
  \end{displaymath}
\end{Def}

\begin{Bsp*}
  Die folgenden Schritte von $M$ aus Beispiel \ref{bsp:pda-wwr} zeigen, dass $w = 0110 \in L(M)$:
  \begin{displaymath}
  \begin{array}{ll}
    (q_0, 0110, \bot) \\
    \vdash (q_0, 110, 0\bot)  &\text{(,,pushen'' des Eingabesymbols $0$)}\\
    \vdash (q_0, 10, 10\bot)  &\text{(,,pushen'' des Eingabesymbols $1$)}\\
    \vdash (q_1, 10, 10\bot)  &\text{($\Eps$-Übergang von $q_0$ nach $q_1$)} \\
    \vdash (q_1, 0, 0\bot)  &\text{(,,poppen'' des Eingabesymbols $1$)}\\
    \vdash (q_1,\Eps, \bot) &\text{(,,poppen'' des Eingabesymbols $0$)}\\
    \vdash (q_1, \Eps, \Eps) &\text{($\Eps$-Übergang zum Entfernen von $\bot$)}
  \end{array}
\end{displaymath}
\end{Bsp*}

\begin{Satz}\label{satz:5.1}
  \begin{align*}
		L \in \ac{CFL}  \text{ gdw } L =L(M) \text{ für einen NPDA $M$}
  \end{align*}
\end{Satz}
\begin{proof}\hfill
	\begin{itemize}
	\item CFG zu NPDA:
		Sei $\mathcal{G} = (N,\Sigma,P,S)$ Grammatik für $L$ in \ac{CNF}.
    
		Definiere \ac{NPDA} $M$ durch
		\begin{itemize}
			\item $Q = \{q_0\}$ 
			\item $\Gamma = \Sigma\uplus N$
			\item $Z_0 = S$
			\item $\delta(q_0,a,a) = \{(q_0,\Eps)\} $ für $ a\in\Sigma$
			\item $\delta(q_0,\Eps,A) = \{(q_0,\alpha)\}$ für $A\to\alpha\in P$
		\end{itemize}
  Wir zeigen nun, dass $L(M) = L(\mathcal{G})$.

  Zum Führen des Beweises benötigen wir folgende Beobachtung, die für alle NPDAs $M = (\ldots),\enspace q,q' \in Q,\enspace w \in \Sigma^*,\enspace Z \in \Gamma$ gilt:
  \begin{displaymath}
    \text{Wenn } (q,w,Z) \vdash^* (q', \Eps, \Eps) \text{ dann } \forall v \in \Sigma^*, \gamma \in \Gamma^*: (q, wv, Z\gamma) \vdash^* (q', v, \gamma)
  \end{displaymath}
  Der Beweis ist per Induktion über die Länge der Ableitung (nicht gezeigt).

  \begin{itemize}
  \item Wir zeigen
    \begin{displaymath}
      \text{wenn } S \stackrel{*}{\Longrightarrow} w \text{ dann } (q_0, w, S) \vdash^* (q_0, \Eps, \Eps)
    \end{displaymath}
    Wir beweisen dazu die stärkere Aussage
    \begin{displaymath}
      \forall A \in N: \text{ wenn } A \stackrel{*}{\Longrightarrow} w\text{ dann } (q_0, w, A) \vdash^* (q_0, \Eps, \Eps)
    \end{displaymath}
    per Induktion über den Ableitungsbaum $\mathcal{A} = \pi(\mathcal{A}_1, \ldots, \mathcal{A}_n)\in \operatorname{Abl}(\mathcal{G}, A)$ mit $Y(\mathcal{A}) = w$.
    \begin{description}
    \item[IV] Für alle $n' \le n$ und $1 \le i \le n',\enspace A_i \in N,\enspace \mathcal{A}_i \in \operatorname{Abl}(\mathcal{G}, A_i)$ gilt:
      \begin{equation}
      \label{eq:pda-step-base}
      (q_0, Y(\mathcal{A}_i), A_i) \vdash^* (q_0, \Eps, \Eps) \tag{$\star$}
    \end{equation}
    \item[IS] Unterscheide die Form der Produktionen der CNF-Grammatik $\mathcal{G}$:
      \begin{itemize}
      \item $\pi = S \to \Eps$.
        Per Konstruktion gilt $(q_0, \Eps) \in \delta(q_0, \Eps, S)$ und somit $(q_0, \Eps, S) \vdash (q_0, \Eps, \Eps)$.
      \item $\pi = A \to a$, $a \in \Sigma^*$.
        Per Konstruktion gilt $(q_0, a) \in \delta(q_0, \Eps, A)$ und $(q_0, \Eps) \in \delta(q_0, \Eps)$.

        Somit gilt $(q_0, a, A) \vdash (q_0, a, a) \vdash (q_0, \Eps, \Eps)$.
      \item $\pi = A \to BC$, $B,C \in N$, $\mathcal{A} = \pi(\mathcal{A}_1, \mathcal{A}_2)$, $\mathcal{A}_1 \in \operatorname{Abl}(\mathcal{G}, B)$, $\mathcal{A}_1 \in \operatorname{Abl}(\mathcal{G}, B)$, $w = Y(\mathcal{A}_1)Y(\mathcal{A}_2)$.

        Per Konstruktion gilt $(q_0, BC) \in \delta(q_0, \Eps, A)$. 

        Ferner gilt per IV, dass $(q_0, Y(\mathcal{A}_1), B) \vdash^* (q_0, \Eps, \Eps)$ und $(q_0, Y(\mathcal{A}_2), C) \vdash^* (q_0, \Eps, \Eps)$.

        Es folgt mit Beobachtung \eqref{eq:pda-step-base}, dass $(q_0, w, A) = (q_0, Y(\mathcal{A}_1)Y(\mathcal{A}_2), A)\vdash (q_0, Y(\mathcal{A}_1)Y(\mathcal{A}_2), BC) \vdash^* (q_0, Y(\mathcal{A}_2), C) \vdash^* (q_0, \Eps, \Eps)$.
      \end{itemize}
    \end{description}
  \item Wir zeigen
    \begin{equation}
      \text{wenn } (q_0, w, S) \vdash^* (q_0, \Eps, \Eps) \text{ dann } S \stackrel{*}{\Longrightarrow} w
    \end{equation}
    Wir beweisen dazu die stärkere Aussage
    \begin{displaymath}
      \forall n \in \mathbb{N}: \forall w \in \Sigma^*, \alpha \in \Gamma^*: \text{ wenn } (q_0, w, \alpha) \vdash^n (q_0, \Eps, \Eps) \text{ dann } \alpha \stackrel{*}{\Longrightarrow} w 
    \end{displaymath}
    per Induktion über die Anzahl der Berechnungsschritte $n$.

    \begin{description}
    \item[IA] $n = 0$.
      $w = \Eps$, $\alpha = \Eps$: es gilt $\Eps \Longrightarrow \Eps$.
    \item[IV] Für alle $n' < n$ und $w \in \Sigma^*, \alpha \in \Gamma^*$ gilt
      \begin{displaymath}
      \text{ wenn } (q_0, w, \alpha) \vdash^{n'} (q_0, \Eps, \Eps) \text{ dann } \alpha \stackrel{*}{\Longrightarrow} w
    \end{displaymath}
  \item[IS] $n > 0$, $\alpha = Z\alpha'$,

    $(q_0,w,Z\alpha') \vdash (q_0, w', \beta\alpha') \vdash^{n-1} (q_0, \Eps, \Eps)$, $\delta(q_0, x, Z) \ni (q_0, \beta)$

    Es gibt zwei Fälle für $Z$:
    \begin{itemize}
    \item $Z = a$, $a \in \Sigma$.

      Es folgt $\beta = \Eps$ und $w = aw'$.

      Per IV gilt $\alpha' \stackrel{*}{\Longrightarrow} w'$ und somit auch $\alpha = a\alpha' \stackrel{*}{\Longrightarrow}aw' = w$.
    \item $Z = A$, $A \to \beta \in P$.

      Es folgt $w = w'$.

      Per IV gilt $\beta\alpha' \stackrel{*}{\Longrightarrow} w'$ und somit auch $A\alpha \Longrightarrow \beta\alpha' \stackrel{*}{\Longrightarrow} w' = w$.
    \end{itemize}
    \end{description}
  \end{itemize}
  \item NPDA zu CFG:

    Zunächst zeigen wir, dass es genügt NPDAs zu betrachten, die bei jeder Transition Wörter der maximalen Länge $2$ auf den Keller schreiben:
\begin{lemma}
	Zu jedem \ac{NPDA} gibt es einen äquivalenten \ac{NPDA}, so dass
	falls $\delta(q,x,Z)\ni (q',\gamma) \quad x\in\Sigma\cup\{\Eps\}$
	dann ist $|\gamma| \le 2$
\end{lemma}
\begin{proof}
	Sei $(q',\gamma)\in\delta(q,x,Z)$ mit $\gamma = Z_n\dots Z_1$ f"ur $n>2$:
	\begin{itemize}
	\item 	neue Zustände $q_2\dots q_{n-1}$
	\item Ersetze $(q',\gamma)$ durch $(q_2, Z_2Z_1)$
	\item Definiere $\delta(q_i, \Eps, Z_i) = \{ (q_{i+1}, Z_{i+1}Z_i) \}$, f"ur $2\le i < n-1$
	\item Definiere $\delta(q_{n-1}, \Eps, Z_{n-1}) = \{ (q', Z_nZ_{n-1}) \}$
	\end{itemize}
	Wiederhole bis alle Transitionen die gew"unschte Form haben. \qedhere
\end{proof}

Sei $M = (Q, \Sigma, \Gamma, q_0, Z_0, \delta)$ nun ein NPDA wobei $|\gamma| \le 2$ für alle $(q', \gamma) \in \delta(q, x, Z), q \in Q, x \in \Sigma \cup \{\Eps\}, Z \in \Gamma$.

    Definiere $\mathcal{G} = (N, \Sigma, S, P)$ mit
    \begin{itemize}
    \item $N = Q \times \Gamma \times Q \cup \{S\}$ 
    \item $P =
      \begin{aligned}[t]
        &\{ (q, Z, q') \to x \mid \delta(q, x, Z) \ni (q', \Eps), x \in \Sigma \cup \{\Eps\} \} \\
        &\cup  \{(q, Z, q') \to x(q'', Z', q') \mid \delta(q, x, Z) \ni (q'', Z'), q'' \in Q, x \in \Sigma \cup \{\Eps\}\} \\
        &\cup  \{(q, Z, q') \to x(q_1,Z_1,q_2)(q_2, Z_2, q') \mid 
        \begin{aligned}[t]
          & \delta(q, x, Z) \ni (q'', Z_1Z_2), \\
          & q_1 \in Q, q_2 \in Q, x \in \Sigma \cup \{\Eps\} \}
        \end{aligned}
      \end{aligned}$
    \end{itemize}

    Es bleibt zu zeigen, dass $L(\mathcal{G}) = L(M)$.
    \begin{itemize}
    \item Wir zeigen, dass 
      \begin{displaymath}
        \text{wenn }(q,Z,q')\stackrel{*}{\Longrightarrow} w \text{ dann } (q,w,Z) \vdash^* (q',\Eps,\Eps)
      \end{displaymath}
      per Induktion über den Ableitungsbaum $\mathcal{A}=\pi(\mathcal{A}_1, \ldots, \mathcal{A}_n) \in \operatorname{Abl}(\mathcal{G}, (q,Z,q'))$ mit $Y(\mathcal{A}) = w$.

      \begin{description}
      \item[IV] Für $1 \le i \le n$ und $Z_i \in \Gamma,\enspace q_i,q_i' \in Q,\enspace \mathcal{A}_i \in \operatorname{Abl}(\mathcal{G}, (q_i, Z_i, q_i'))$ gilt
        \begin{displaymath}
          (q_i,Y(\mathcal{A}_i),Z_i) \vdash^* (q_i',\Eps,\Eps)
        \end{displaymath}
      \item[IS]
        Es gibt $3$ Fälle für $\pi$:
        \begin{itemize}
        \item $(q, Z, q') \to x,\enspace \delta(q, x, Z) \ni (q', \Eps), x \in \Sigma \cup \{\Eps\}$.

          Es folgt $w = x$ und damit: $(q, x, Z) \vdash (q', \Eps, \Eps)$.

        \item $(q, Z, q') \to x(q'', Z', q'), \enspace \delta(q, x, Z) \ni (q'', Z'), q'' \in Q, x \in \Sigma \cup \{\Eps\}$

          Es folgt $w = xY(\mathcal{A}_1)$, $\mathcal{A} = \pi(\mathcal{A}_1)$, $\mathcal{A}_1 \in \operatorname{Abl}(\mathcal{G}, (q'', Z', q'))$.

          Es folgt per IV, dass $(q'', Y(\mathcal{A}_1), (q'', Z', q')) \vdash^* (q_0, \Eps, \Eps)$ und damit auch $(q, xY(\mathcal{A}_1), (q, Z, q')) \vdash (q'', Y(\mathcal{A}_1), (q'', Z', q')) \vdash^* (q_0, \Eps, \Eps)$.

        \item $(q, Z, q') \to x(q_1,Z_1,q_2)(q_2, Z_2, q')$, $\delta(q, x, Z) \ni (q_1, Z_1Z_2)$, $q_1 \in Q, q_2 \in Q, x \in \Sigma \cup \{\Eps\}$ 

          Es folgt $w = xY(\mathcal{A}_1)Y(\mathcal{A}_2)$,\enspace $\mathcal{A} = \pi(\mathcal{A}_1,\mathcal{A}_2)$,\enspace $\mathcal{A}_1 \in \operatorname{Abl}(\mathcal{G}, (q_1, Z_1, q_2))$, $\mathcal{A}_2 \in \operatorname{Abl}(\mathcal{G}, (q_2, Z_2, q'))$.

          Es folgt per IV, dass
          \begin{displaymath}
            (q_1, Y(\mathcal{A}_1), Z_1)) \vdash^* (q_2, \Eps, \Eps)
          \end{displaymath}
          mit Beobachtung \eqref{eq:pda-step-base} auch
          \begin{displaymath}
            (q_1, Y(\mathcal{A}_1)Y(\mathcal{A}_2), Z_1Z_2) \vdash^* (q_2, Y(\mathcal{A}_2), Z_2)
          \end{displaymath}

          Es folgt ferner per IV, dass
          \begin{displaymath}
            (q_2, Y(\mathcal{A}_2), Z_2) \vdash^* (q', \Eps, \Eps)
          \end{displaymath}

          Somit gilt
          \begin{displaymath}
            (q, xY(\mathcal{A}_1)Y(\mathcal{A}_2), Z) \vdash (q_1, Y(\mathcal{A}_1)Y(\mathcal{A}_2), Z_1Z_2)) \vdash^* (q_2, Y(\mathcal{A}_2), Z_2) \vdash^* (q', \Eps, \Eps)
          \end{displaymath}
          
          
        \end{itemize}
      \end{description}

    \item Wir zeigen für alle $n \in \mathbb{N}: \forall m \in \mathbb{N},\enspace Z_1,\ldots,Z_m \in \Gamma,\enspace q,q' \in Q$, dass wenn
      \begin{displaymath}
        (q,w,Z_1\ldots Z_m) \vdash^n (q', \Eps, \Eps)
      \end{displaymath}
      dann existieren $q_1,\ldots,q_{m+1} \in Q$, so dass
      \begin{displaymath}
        (q_1,Z_1,q_2)(q_2,Z_2,q_3)\ldots(q_m,Z_m,q_{m+1}) \stackrel{*}{\Longrightarrow} w
      \end{displaymath}
      mit $q_1 = q$ und $q_{m+1} = q'$ per Induktion über $n$.

      \begin{description}
      \item[IA] $n = 0$.
        $w = \Eps$, $m = 0$.

        Es folgt $\Eps \stackrel{*}{\Longrightarrow} w$.
      \item[IV] Für alle $n' < n$,\enspace $m \in \mathbb{N},\enspace Z_1,\ldots,Z_m\in \Gamma,\enspace q,q' \in Q$ gilt, dass wenn
        \begin{displaymath}
        (q,w,Z_1\ldots Z_m) \vdash^{n'} (q', \Eps, \Eps)
      \end{displaymath}
      dann existieren $q_1,\ldots,q_{m+1} \in Q$, so dass
      \begin{displaymath}
        (q_1,Z_1,q_2)(q_2,Z_2,q_3)\ldots(q_m,Z_m,q_{m+1}) \stackrel{*}{\Longrightarrow} w
      \end{displaymath}

    \item[IS] $n > 0$.
      $(q, w, Z_1\ldots Z_m) \vdash (q'', w', \gamma Z_2\ldots Z_m) \vdash^{n-1} (q',\Eps, \Eps)$.
      $w =xw'$, \enspace $x \in \{\Eps\} \cup \Sigma$.
      \begin{itemize}
      \item $\gamma = \Eps$, \enspace $(q, Z_1, q'') \to x \in P$.

        Per IV gilt: es existieren $q_2,\ldots,q_{m+1} \in Q$, so dass
        \begin{displaymath}
          (q_2, Z_2, q_3)\ldots(q_m,Z_m,q_{m+1}) \stackrel{*}{\Longrightarrow} w' \text{ und } q''=q_2 \text{ und } q'=q_{m+1}.
        \end{displaymath}

        Somit gilt auch
        \begin{displaymath}
          (q, Z_1, q'') (q_2, Z_2, q_3)\ldots(q_m,Z_m,q_{m+1})\Longrightarrow x\stackrel{*}{\Longrightarrow} xw'
        \end{displaymath}
      \item \ldots (andere Fälle ähnlich)
      \end{itemize}
      \end{description}
  \end{itemize}
\end{itemize}
  
% 		per Induktion über Ableitungsbaum von $X\overset{*}{\==>}v$\\
% 		(Höhe von)
% 		\begin{description}
% 			\item[Höhe $0$:]
% 				\begin{align*}
% 					\bullet\quad  a &= X = v\\
% 					(q,a,a) &\vdash (q,\Eps,\Eps)\ \checkmark
% 				\end{align*}
% 			\item[Höhe $>0$:]\
% 			\begin{enumerate}
% 			\item Fall:
% 				\tikz[baseline=-.3em]{
% 					\Tree [.\node [label={right:$A=X$}] {$\bullet$};
% 						\node [label={right:$a$}] {$\bullet$};
% 					]
% 				}
% 				\quad $(q,a,A) \overset{A\-> a}{\vdash} (q,a,a) \vdash (q,\Eps,\Eps)$
% 			\item Fall:
% 				\tikz[baseline, level 1/.style={sibling distance=1em}]{
% 					\Tree [.\node[label={right:$A = X$}] {$\bullet$};
% 						[.\node[label={left:$B_1$}] {$\bullet$}; \edge[roof]; {$v_1 $} ]
% 						[.\node[label={right:$B_2$}] {$\bullet$}; \edge[roof]; {$v_2 $} ]
% 					]
% 				} Regel $A\-> B_1B_2$
% 				\begin{align*}
% 					\text{d.h. }v = v_1v_2 \quad B_1 &\overset{*}{\==>} v_1 \text{ mit kleinerem Abeitungsbaum}\\
% 					B_2 &\overset{*}{\==>} v_2 \text{ \ruleplaceholder{\widthof{mit kleinerem Abeitungsbaum}}}\\
% 					(q,v_1v_2,A) &\vdash (q,v_1v_2,B_1B_2)\\
% 					&\vdash^* (q,v_2,B_2) \quad \text{nach I.V. für }B_1\\
% 					&\vdash^* (q,\Eps,\Eps) \quad \text{nach I.V. für }B_2\\
% 					\text{D.h. } S\overset{*}{\==>} w &\<==> (q,w,S) \vdash^* (q,\Eps,\Eps)\\
% 					&\<==> w\in L(M)
% 				\end{align*}
% 			\item["'\<="'] Gegeben \ac{NPDA} $M=(Q,\dots)$ bei dem alle Transitionen pop, top oder push sind.
% 			\end{enumerate}
% 		\end{description}
% 	\end{itemize}
% 	\begin{figure}[H]\centering\vspace{-1em}
% 		\input{2015-12-22_diagramm.pdf_tex}
% 		{\setlength{\belowdisplayskip}{-.5em}
% 		\begin{align*}
% 			(q,Z,q') &\overset{*}{\Rightarrow} v\\
% 			(q,v,Z) &\vdash^* (q',\Eps,\Eps)
% 		\end{align*}}
% 		\caption{Beweis zu \autoref{satz:5.1}}
% 	\end{figure}\vspace{-2.5em}\qedhere\\*
%     Grammatik $\mathcal{G}'$ leitet gerade $v$ ab, falls $Z$ im Stack.
\end{proof}


% %
% \datenote{22.12.15}
% \vspace{3em}

% %\rlerror{Anfang unvollständig (Aufschrieb fehlt!)}

% $L=L(M)$ für \ac{NPDA} $M$\\
% $\curvearrowright\ L$ ist \ac{CFL}

% %\begin{figure}[H]\centering
% %\vspace{2em}
% %\rlerror*{Grafik fehlt}{<Bild>}% Handschr. Mitschrieb vorhanden.
% %\caption{
% %}
% %\end{figure}

% \begin{itemize}
% \item Def. \ac{CFG} mit $N= Q\x \Gamma\x Q\cup \{S\}$, so dass $(q,Z,q')\overset{*}{\=>} v$ gdw.\ 
% $(q, v, Z) \vdash^* (q',\Eps,\Eps)$
% \\
% 	Transitionen von $M$ haben eine von drei Formen:
% 	\begin{align*}
% 		&(q,\Eps) && (q,Z) && (q,Z_1 Z_2)
% 	\end{align*}
% \item Def. $P$ durch $P \supseteq \{ S \-> (q_0,Z_0,q') \mid q'\in Q\}$ sowie die weiteren Produktion wie folgt
% \end{itemize}
% Produktion für $(q,Z,q')$
% \begin{enumerate}[label={Fall \arabic*:},ref={Fall \arabic*},leftmargin=*]
% \item\label{itm:Fall 1} $\delta(q,x,Z)\ni (q',\Eps)\ , x\in \Sigma\cup \{\Eps\}$\\
% 	$P \supseteq \{ (q,Z,q')\-> x \}$
% \item $\delta(q,x,Z)\ni (q'',Z')\ , x\in \Sigma\cup \{\Eps\}$\\
% 	$P \supseteq \{ (q,Z,q')\-> x(q'',Z',q') \mid q'\in Q\}$
% \item $\delta(q,x,Z)\ni (q_1,Z_1Z_2)\ , x\in \Sigma\cup \{\Eps\}$\\
% 	$P \supseteq \{ (q,Z,q')\-> x(q_1,Z_1,q_2)(q_2,Z_2,q') \mid q_1,q_2\in Q \}$
% \end{enumerate}
% Korrektheit: Zeige $(q,Z,q')\overset{*}{\=>} w\curvearrowleftright (q,w,Z) \vdash^* (q',\Eps,\Eps)$\\
% Induktion über Ableitungsbaum von $(q,Z,q')\overset{*}{\=>} w$ (siehe oben)
% \begin{itemize}[align=left]
% \item[Höhe 1:] \tikz[baseline]{\Tree[.$(q,Z,q')$ $x$ ]} \quad \ref{itm:Fall 1} $\curvearrowleftright (q,x,Z)\vdash (q',\Eps,\Eps)$
% \item[Höhe $>1$:] Zwei Möglichkeiten:
% 	\begin{enumerate}[label={Fall \arabic*:},ref={Fall \arabic*},leftmargin=*,start=2]
% 	\item \ \vspace{-1em}
% 	\begin{align*}
% 		\raisebox{0pt}[0pt][0pt]{\tikz[baseline]{\Tree[.$(q,Z,q')$ $x$ $(q'',Z,q')$ ]}}
% 			&\curvearrowright w = xw'\land (q'',Z',q')\overset{*}{\=>} w'\\
% 			&\begin{aligned}
% 				\curvearrowright (q,xw',Z) &\vdash (q'',w',Z')\\
% 				\text{I.V.}\quad &\vdash^* (q',\Eps,\Eps)
% 			\end{aligned}
% 	\end{align*}
% 	\item \tikz[baseline]{\Tree[.$(q,Z,q')$ $x$ $(q_1,Z_1,q_2)$ $(q_2,Z_2,q')$ ]}
% 		\begin{align*}
% 			w &= xw_1w_2\\
% 			(q_1,Z_1,q_2) &\overset{*}{\=>} w_1 \quad\text{mit kleinerem Ableitungsbaum}\\
% 			(q_2,Z_2,q') &\overset{*}{\=>} w_2 \quad\text{mit kleinerem Ableitungsbaum}\\
% 			\curvearrowright (q,xw_1w_2,Z) &\vdash (q_1,w_1w_2,Z_1Z_2)\\
% 			\text{I.V. für }w_1\text{ + Lemma} &\vdash^* (q_2,w_2,Z_2)\\
% 			\text{I.V. für }w_2\phantom{\text{ + Lemma}} &\vdash^* (q',\Eps,\Eps)
% 		\end{align*}
% 	\end{enumerate}
% \end{itemize}

\begin{Def}[name={[DPDA]}]
	Ein \ac{DPDA} ist ein Tupel $(\underbrace{Q,\Sigma,\Gamma,q_0,Z_0}_{\text{wie gehabt}},\delta,F)$
	\vspace{-1em}
	\begin{itemize}
	\item $F\subseteq Q$ akzeptierende Zustände
	\item $\delta: Q\x (\Sigma\cup\{\Eps\})\x \Gamma \-> \mathcal{P}(Q\x \Gamma^*)$ wobei für alle $q\in Q,a\in\Sigma,Z\in\Gamma$ gelten muss, dass
	$|\delta(q,a,Z)| + |\delta(q,\Eps,Z)| \leq 1$
	\item Die Schrittrelation ,,$\vdash$'' ist definiert wie bei NPDAs.
	\item $L(M) = \{w\in\Sigma^* \mid (q_0,w,Z_0) \vdash^* (q',\Eps,\gamma) \land q'\in F \}$ \qedhere
	\end{itemize}
\end{Def}

\begin{lemma}[name={[\acs*{DPDA}, der gesamte Eingabe verarbeitet]}]
	\label{lem:DPDA ges. Eingabe}
	Zu jedem \ac{DPDA} gibt es einen äquivalenten \ac{DPDA}, der die gesamte Eingabe verarbeitet.
\end{lemma}
% \begin{proof}
%     Erste Möglichkeit: Die Transitionsrelation ist nicht total.

% 	Führe einen neuen Zustand $q_s\notin F$ ein.
% 	\begin{align*}
% 	\shortintertext{Falls $\exists a\in\Sigma, q, Z$}
% 		\delta(q,a,Z) &\cup \delta(q,\Eps,Z) = \varnothing\\
% 	\shortintertext{dann erweitere $\delta$ um}
% 		\delta(q,a,Z) &= \{ (q_s,Z) \}\\
% 	\shortintertext{und $\forall a\in\Sigma, Z\in \Gamma: \delta(q_s,a,Z) = \{(q_s,Z)\}$}
% 	\end{align*}
% 	Weitere Möglichkeit: der Automat bleibt wegen leerem Keller stecken.\\
% 	Abhilfe: Neues Kellerbodensymbol $Z_0'$ und neuer Startzustand $q_0'$.
% 	\begin{align*}
% 		\delta(q_0',\Eps,Z_0') &= \{(q_0,Z_0Z_0')\}\\
% 	\shortintertext{Erweitere $\delta$ für alle originalen zustände $q\in Q$ um}
% 		\delta(q,\Eps,Z_0') &= \{(q_s,Z_0')\}
% 	\end{align*}
% 	"`Falls Keller abgeräumt, Wechsel nach $q_s$"'
% \end{proof}

% \begin{Satz}[name={[Abgeschlossenheit der deterministischen \acs*{CFL}]}]
% 	Die deterministischen \ac{CFL} sind unter Komplement abgeschlossen.
% \end{Satz}
% \begin{proof}
% 	Sei $L=L(M)$ für \ac{DPDA} $M$. Nach \autoref{lem:DPDA ges. Eingabe} liest $M$ die komplette Eingabe.\\
% 	Def. $M'$ mit $Q'= q\x \{0,1,2\}$\\
% 	Zustand $(q,0)$: seit Lesen des letzten Symbols wurde kein akz. Zustand durchlaufen.\\
% 	Zustand $(q,1)$ seit Lesen \dots$\geq 1$ akz. Zustand durchlaufen\\
% 	\phantom{Zustand} $(q,2)$ \dots akz. Zustände d.h. $F'=F\x \{2\}$
% 	\begin{align*}
% 		q_0' &= \begin{cases}
% 			(q_0,0) & q_0\notin F\\
% 			(q_0,1) & q_0\in F
% 		\end{cases}\\
% 		\text{Falls }\delta(q,\Eps,Z) = \{(q',\gamma)\}\text{ dann}&\\
% 		\delta'((q,0),\Eps,Z) &= \begin{cases}
% 				((q',0),\gamma) & q'\notin F\\
% 				((q',1),\gamma) & q'\in F
% 			\end{cases}\\
% 		\delta'((q,1),\Eps,Z) &= ((q',1),\gamma)\\
% 		\text{Falls }\delta(q,a,Z) = \{(q',\gamma)\}& \\
% 		\delta'((q,0), \Eps, Z) &= \{ ((q,2), Z) \} \\
% 		\delta'((q,2), a, Z) &=
% 		    \begin{cases}
% 		        ((q',0), \gamma) & q' \notin F\\
% 		        ((q',1), \gamma) & q' \in F
% 		    \end{cases}\\
% 		\delta'((q,1), a, Z) &=
%             \begin{cases}
%                 ((q',0), \gamma & q' \notin F\\
%                 ((q',1), \gamma & q' \in F
%             \end{cases}
%         \\ \tag*{\qedhere}
% 	\end{align*}
% \end{proof}
% \begin{Satz}
%     Die deterministischen \ac{CFL} sind \textbf{nicht} unter Vereinigung und Durchschnitt abgeschlossen.
% \end{Satz}
% \begin{proof}
%     Betrachte
%     \begin{align*}
%         L_1 &= \{ a^nb^nc^m \mid n, m \ge 1 \} \\
%         L_2 &= \{ a^mb^nc^n \mid n, m \ge 1 \}
%     \end{align*}
%     Sowohl $L_1$ als auch $L_2$ sind DCFL, aber $L_1 \cap L_2 = \{ a^nb^nc^n \mid n \ge 1\}$ ist nicht kontextfrei.
    
%     DCFL ist nicht abgeschlossen unter Vereinigung. Angenommen doch: Seien $U, V$ DCFL. Dann sind auch $\overline{U}$ und $\overline{V}$ DCFL. Bei Abschluss unter Vereinigung wäre $\overline{U} \cup \overline{V}$ eine DCFL und somit auch $\overline{\overline{U} \cup \overline{V}} = U \cap V$, ein Widerspruch gegen den ersten Teil.
% \end{proof}
% \begin{Satz}
%     DCFL ist abgeschlossen unter Schnitt mit REG.
% \end{Satz}
% \begin{proof}
%     Sei $L$ DCFL und $R$ regulär.
%     Bilde das Produkt aus einem DPDA für $L$ und einem DFA für $R$.
%     Offenbar ist das Ergebnis ein DPDA, der $L\cap R$ erkennt.
% \end{proof}
% \begin{Satz}
%     Sei $L$ DCFL und $R$ regulär.
%     Es ist entscheidbar, ob $R=L$, $R\subseteq L$ und $L=\Sigma^*$.
% \end{Satz}
% \begin{proof}
%     Es gilt $R\subseteq L$ gdw.\ $R \cap \overline{L} = \emptyset$.
    
%     Weiter ist $R = L$ gdw.\ $R\subseteq L$ und $L \subseteq R$. Für den zweiten Teil betrachte $L\cap \overline{R}$.
    
%     Für kontextfreie Sprachen ist $L\ne \emptyset$ entscheidbar, also betrachte $L=\Sigma^*$ gdw.\ $\overline{L}=\emptyset$.
% \end{proof}

% \begin{Satz} \textbf{DPDA Äquivalenzproblem}
%     Seien $L_1, L_2$ DCFL. Dann ist $L_1 = L_2$ entscheidbar.
% \end{Satz}

% \begin{proof}
%     Siehe Senizergues (2000) und Stirling (2001).
% \end{proof}
