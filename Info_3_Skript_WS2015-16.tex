% Vorlesungsskript / -mitschrieb zu Informatik III - Theoretische Informatik, gehalten von Prof. Dr. Peter Thiemann im WS 2014/15
%    Copyright (C) 2016 Ralph Lesch
%
%    This program is free software: you can redistribute it and/or modify
%    it under the terms of the GNU General Public License as published by
%    the Free Software Foundation, either version 3 of the License, or
%    (at your option) any later version.
%
%    This program is distributed in the hope that it will be useful,
%    but WITHOUT ANY WARRANTY; without even the implied warranty of
%    MERCHANTABILITY or FITNESS FOR A PARTICULAR PURPOSE.  See the
%    GNU General Public License for more details.
%
%    You should have received a copy of the GNU General Public License
%    along with this program.  If not, see <http://www.gnu.org/licenses/>.

% Compiled with pdflatex -enable-write18 -synctex=1 -interaction=nonstopmode --shell-escape %.tex
%\RequirePackage[l2tabu,orthodox]{nag}
\documentclass[11pt,paper=a4,DIV=12,titlepage,headsepline,ngerman,listof=totoc]{scrartcl}
\usepackage[utf8]{inputenc}
\usepackage[T1]{fontenc}
\usepackage{lmodern}
\usepackage{textcomp}
\usepackage{babel}
%
\usepackage{acronym}
\usepackage{amsmath,amsfonts,amssymb}
\usepackage{array}
\usepackage{booktabs} % Better rules: \toprule, \midrule, \bottomrule
\usepackage{calc}
\usepackage{cancel}
\usepackage[iso]{datetime} % Date format
\usepackage{enumitem}
\usepackage{float} % H - option for figure
\usepackage{graphicx}
\usepackage{listings} % Source printing & syntax highlighting
\usepackage{mathtools}
\usepackage{multirow}
\usepackage{placeins} % \FloatBarrier
\usepackage{stmaryrd} % Symbols like \llbracket for [[
\usepackage{tabu}

% physics package, without redefinitions
\usepackage[log-declarations=false]{xparse} % Fix for invalid error file reference for errors in log file (because of parentheses in d() arguments).
\usepackage[notrig]{physics}
\let\div\divisionsymbol             % \div = tex: \div | physics: \divergence
\let\Real\Re \let\Re\real           % \Re = tex: \Re   | physics: \Re => \Real
\let\Imaginary\Im \let\Im\imaginary % \Im = tex: \Im   | physics \Im => \Imaginary

\usepackage[dvipsnames]{xcolor}
\usepackage{ulem}  % emph = \underline
\newcommand\coloruline[2]{\colorlet{colorsave}{.}{\color{#1}\uline{{\color{colorsave}#2}}}}
%\usepackage{xhfill}

% References
\usepackage[bookmarksnumbered,colorlinks=false,linkbordercolor={0 0 0},pdfborder={0 0 0}]{hyperref}
%\usepackage{hypcap}
\usepackage[hypcap]{caption} % \captionof + hypcap = link at begin of figure etc.
\usepackage{subcaption} % subfigure
%\usepackage{lastpage}

% Custom packages and macros
%\usepackage{import} \subimport{..\}{latexmacros.tex}
% LaTeX Macros, shortcuts, frequently used code
% Version: 2015-11-15
%
%    Copyright (C) 2015 Ralph Lesch
%
%    This program is free software: you can redistribute it and/or modify
%    it under the terms of the GNU General Public License as published by
%    the Free Software Foundation, either version 3 of the License, or
%    (at your option) any later version.
%
%    This program is distributed in the hope that it will be useful,
%    but WITHOUT ANY WARRANTY; without even the implied warranty of
%    MERCHANTABILITY or FITNESS FOR A PARTICULAR PURPOSE.  See the
%    GNU General Public License for more details.
%
%    You should have received a copy of the GNU General Public License
%    along with this program.  If not, see <http://www.gnu.org/licenses/>.
%
% Include with: \usepackage{import} \subimport{../}{latexmacros.tex}

% == Default packages ==
%\usepackage{amsmath,amsfonts,amssymb} % Math packages.
%\usepackage{array} % Extended array and tabular environments.
%\usepackage{booktabs} % Better rules: \toprule, \midrule, \bottomrule
%\usepackage{enumitem} % Extended lists, with key=value options.
%\usepackage{mathtools} % amsmath extension.

% == Optional packages ==
%\usepackage[bookmarksnumbered,colorlinks=false,linkbordercolor={0 0 0},pdfborder={0 0 0}]{hyperref} % Links and bookmarks.
%\usepackage{hypcap} % Link to top of float instead of caption.

%\usepackage{float} % Put float HERE with [H].
%\usepackage{graphicx} % \includegraphics
%\usepackage{placeins} % \FloatBarrier

%\usepackage{stmaryrd} % \lightning for contrapositions or failed proofs

%\usepackage{scrlayer-scrpage} % Header and footer.
%\usepackage{lastpage} % \pageref{LastPage}
%\ofoot{\usekomafont{pagenumber}\pagemark/\pageref{LastPage}} % Site number.

% == Custom packages ==
\RequirePackage{arrowmacros} % Arrow macros, e.g. \=>

% == Number types ==
\RequirePackage{amssymb}
\newcommand*\N{\mathbb{N}} % Natural numbers
\newcommand*\Z{\mathbb{Z}} % Integers
\newcommand*\Q{\mathbb{Q}} % Rational numbers
\newcommand*\R{\mathbb{R}} % Real numbers
\newcommand*\C{\mathbb{C}} % Complex numbers
\DeclareMathAlphabet{\mathbbn}{U}{bbold}{m}{n} % bbold font for numbers.
\newcommand*{\1}{\mathbbn{1}} % Symbol for identity
\newcommand*{\0}{\mathbbn{0}} % Symbol for empty element

% == Column types for tables (tabular/array/tabu) ==
\newcolumntype{M}[1]{>{$}#1<{$}} % Math mode for column, e.g. M{c}
% {RL}: Correct aligned columns for e.g. 1.2 &\pm 0.2
\newcolumntype{R}{>{$}r<{{}$}} % = M{r}<{{}}
\newcolumntype{L}{@{}>{${}}l<{$}} % Empty math symbol {} for correct space.

% == Equation numbering ==
% (single, manually) for math environments - typically followed by \label{eq:}
% In display math mode
\newcommand*\numbereq{\refstepcounter{equation}\tag{\theequation}}
% Inline numbering (for $math mode$), s: * without \hfill
\DeclareDocumentCommand\numberinlineeq{s}{
	\IfBooleanTF#1{}{\hfill}\refstepcounter{equation}(\theequation)%
}

% == Equation labeling ==
% An underbrace upside down (overbrace with under the line), taking no horizontal space
% \underoverbrace{text}{subscript}
\newcommand\underoverbrace[2]{\underset{\mathclap{\overbrace{#1}}}{#2}}
% An overbrace upside down (underbrace over the line), taking no horizontal space
% \overunderbrace{text}{superscript}
\newcommand\overunderbrace[2]{\overset{\mathclap{\underbrace{#1}}}{#2}}

% == Shortcuts ==
\newcommand*\x{\times}
\newcommand*\zz{\ensuremath{\mathrm{Z\kern-.5em\raise-0.5ex\hbox{Z}}}} % "Zu zeigen" symbol

% == Short figure names ==
\RequirePackage{babel}
% Change figure name from Figure to Fig. for english
\addto\captionsenglish{\renewcommand*{\figurename}{Fig.}}%
\addto\extrasenglish{\renewcommand*{\figureautorefname}{Fig.}}%
% Change figure name from Abbildung to Abb. for german
\addto\captionsngerman{\renewcommand*{\figurename}{Abb.}}%
\addto\extrasngerman{\renewcommand*{\figureautorefname}{Abb.}}%

% proof theorem with amsthm qed symbol
% with ntheorem
%\PassOptionsToPackage{amsmath,hyperref,thmmarks}{ntheorem}
%\RequirePackage{ntheorem,thmtools}
%%\usepackage[amsmath,hyperref,thmmarks]{ntheorem}
%%\usepackage{thmtools}
%\newcommand{\openbox}{\leavevmode
%  \hbox to.77778em{%
%  \hfil\vrule
%  \vbox to.675em{\hrule width.6em\vfil\hrule}%
%  \vrule\hfil}}
%
%\AtBeginDocument{%
%\declaretheoremstyle[
%	headfont=\scshape,
%	bodyfont=\normalfont,
%	headpunct={:\ },
%	qed=\openbox
%]{proofstyle}
%\newtheorem*{proof}{\csname proofname\endcsname} % \proofname from babel for babel
%}

% with amsthm
%\RequirePackage{amsthm,thmtools}
%\declaretheoremstyle[
%	headfont=\bfseries,
%	notefont=\normalfont,
%	bodyfont=\normalfont,
%	headpunct={:\ },
%	qed=\openbox,
%	spacebelow=
%]{proofstyle}
%\let\proof=\relax

%\AtBeginDocument{%
%	\declaretheorem[style=proofstyle,name=\proofname,numbered=no]{proof}
%}


% == fixme ==
\usepackage{fixme}
% Register users for user commands
\FXRegisterAuthor{rl}{anrl}{RL} % Ralph Lesch  => \rlnote = \txnote[user=RL]
\FXRegisterAuthor{pt}{anpt}{\bfseries PT} % Peter Thiemann
\FXRegisterAuthor{date}{andate}{\color{black}Vorlesung} % \datenote = Lecture date
%
\fxsetup{
	status=draft,
	multiuser,
	theme=color,
	innerlayout={layout=marginnote}, % fxnotes also in displaymath...
%	targetface=\color{red},
%	marginface=\color{red},
%	inlineface=\color{red}
}
\fxloadlayouts{marginnote}
% For color theme/layout:
\definecolor{fxnote}{named}{Green}
\definecolor{fxwarning}{named}{Orange}
\definecolor{fxerror}{named}{red}
\definecolor{fxfatal}{named}{BrickRed}
\renewcommand*{\marginfont}{\color{red}}
%\renewcommand*{\fxnotename}[1]{\!} % Not named for noncolor theme
%\definecolor{fxtarget}{named}{red} % color for target with color layout/theme
\renewcommand\germanlistfixmename{Anmerkungsverzeichnis}
% Customisation
\makeatletter
% target color = note (type) color
\renewcommand\FXTargetLayoutColor[2]{\@fxuseface{target}\color{fx#1}#2}
% Inline: with braces []
\renewcommand*\FXLayoutInline[3]{%
    \@fxdocolon{#3}{\@fxuseface{inline}\color{fx#1}[\ignorespaces#3\@fxcolon#2]}%
}
% Color theme: marginnote like marginpar
\renewcommand*\FXLayoutMarginNote[3]{%
	\@fxdocolon{#3}%
	\marginnote[%
		\raggedleft\@fxuseface{margin}\color{fx#1}\ignorespaces#3\@fxcolon#2%
	]{%
		\raggedright\@fxuseface{margin}\color{fx#1}\ignorespaces#3\@fxcolon#2%
	}%
}
\makeatother
% ====

% == tikz ==
\usepackage{tikz}
\usepackage{tikz-qtree}
\usetikzlibrary{arrows.meta,automata,calc,decorations.pathmorphing,decorations.pathreplacing,graphs,positioning%
%,external
}
%\tikzexternalize
\tikzset{>=stealth,
	block/.style={% Nodes as blocks with aligned text, e.g. for Turing machines
		draw, rectangle,
		%minimum height=1em,
		minimum width=1.5em,
		outer sep=0pt,
		node distance=0pt,
		text height=2ex,
		text depth=.5ex,
		align=center
	},
	circle/.style={
		draw,
		shape=circle,
		minimum size=0.5cm,
		text=black, 
		text width=0.5cm,
		align=center
	}
}
% ====

% == Header and footer ==
\usepackage{scrlayer-scrpage}
\automark[subsection]{section}
\pagestyle{scrheadings}
\clearscrheadfoot
\setkomafont{pageheadfoot}{\normalfont\sffamily}
\ohead{\pagemark}
\ihead{\rightmark}
%\ofoot{\usekomafont{pagenumber}\pagemark/\pageref{LastPage}} % Site number

% == Style ==
\setkomafont{captionlabel}{\bfseries}
\let\dashlabel=\labelitemii
\setlist[itemize]{label=\dashlabel}
% Paragraph
\setlength{\parindent}{0pt}
\setlength{\parskip}{\medskipamount}

% == Environments ==
\usepackage{amsthm,thmtools}
% Fix for theoream key "restate" with name={[short name]}
\makeatletter
\kv@set@family@handler{restate phase 2}{%
%  \ifthmt@restatethis
%  \@xa\@xa\@xa\g@addto@macro\@xa\@xa\@xa\thmt@storedoptargs\@xa\@xa\@xa{\@xa\@xa\@xa,%
%    \@xa\kv@key\@xa=\kv@value}%
%  \fi
}
\makeatother
% German "continues" text
\renewcommand\thmcontinues[1]{%
	\ifcsname hyperref\endcsname%
		\hyperref[#1]{Fortsetzung}%
	\else%
		Fortsetzung%
	\fi%
	von S.\,\pageref{#1}%
}
% Theorem Styles
\declaretheoremstyle[
	headfont=\bfseries,%\scshape
	notefont=\normalfont,
	bodyfont=\normalfont,
	headpunct={:\ },
	qed={},
	spaceabove=\bigskipamount,
	spacebelow=\parskip
]{basic}
\declaretheoremstyle[
	headfont=\bfseries,%\scshape
	notefont=\normalfont,
	bodyfont=\normalfont,
	headpunct={:\ },
	qed={\ensuremath{\bigoplus}},
	spaceabove=\bigskipamount,
	spacebelow=\parskip
]{definition}
\declaretheoremstyle[
	headfont=\scshape,%\bfseries,
	notefont=\normalfont,
	bodyfont=\normalfont,
	headpunct={:\ },
	qed=\openbox,
	spaceabove=\parskip,
	spacebelow=\parskip
]{proofstyle}

%\theoremstyle{theorem}
\declaretheorem[style=basic,numbered=no,name=Bemerkung]{Bemerkung}
\declaretheorem[style=basic,numbered=no,name=Bem.]{Bem}
\declaretheorem[style=basic,numbered=no,name=Beobachtung]{Beobachtung}
\declaretheorem[style=basic,parent=section,name=Bsp.]{Bsp}
\declaretheorem[style=basic,name=Bsp.,numbered=no]{Bsp*}
\declaretheorem[style=definition,parent=section,name=Def.]{Def}
\declaretheorem[style=definition,name=Def.,numbered=no]{Def*}
\declaretheorem[style=definition,parent=Def,name=Def.]{subDef}
\declaretheorem[style=definition,numbered=no,name=Erinnerung]{Erinnerung}
\declaretheorem[style=basic,parent=section,name=Satz]{Satz}
\declaretheorem[style=basic,name=Satz,numbered=no]{Satz*}
\declaretheorem[style=basic,numbered=no,name=Lemma]{lemma*}
\declaretheorem[style=basic,sibling=Satz,name=Lemma]{lemma}
\declaretheorem[style=definition,sibling=Satz,name=Korollar]{Korollar}
\declaretheorem[style=definition,numbered=no,name=Korollar]{Korollar*}
\let\proof=\relax
\AtBeginDocument{% For \proofname as of babel.
	\declaretheorem[style=proofstyle,name=\proofname,numbered=no]{proof}
}

% cases environment with 2 arrows
% Source: https://tex.stackexchange.com/questions/89250/modify-case-equations-brace/89257#89257
% \splitlines[->] for arrow
\newcommand*{\splitlines}[1][]{
\begin{tikzpicture}[baseline=-0.5ex]
\draw[#1] (0,0) -- (0.4,0.25);
\draw[#1] (0,0) -- (0.4,-0.15);
\end{tikzpicture}
}
\newenvironment{casesarrows}[1][->]%
{\;\splitlines[#1]\;\begin{array}{@{}l@{}}}%
{\end{array}}


% == Custom commands ==

% Source: https://tex.stackexchange.com/questions/164506/how-to-get-a-curved-arrow-pointing-left-and-right/164511#164511
%\newcommand{\curvearrowleftright}{\scalebox{1.2}[2]{$\mathclap{\curvearrowleft}\mkern2.2mu\mathclap{\curvearrowright}$}}
\newcommand{\curvearrowleftright}{\mathrel{\curvearrowleft\mkern-2.7mu\mathllap{\curvearrowright}}}

% \dash of given length
\newcommand*{\xdash}[1]{\rule[0.5ex]{#1}{0.55pt}}

% \ruleplaceholder["]{width} = --"--
\newcommand*{\ruleplaceholder}[2][\text{\texttt{ " }}]{%
    \setlength{\dimen0}{(#2-\widthof{#1})/2}%
    \xdash{\dimen0}\text{#1}\xdash{\dimen0}}

% Math operators and abbreviations
\usepackage{forloop}
% Define \mcX = \mathcal{X}, with X = A-Z
% and \mbX = \mathbb{X}
\newcounter{ct}
\newcommand*\mc[1]{\mathcal{#1}}
\newcommand*\mb[1]{\mathbb{#1}}
\forloop{ct}{1}{\value{ct} < 27}{%
	\expandafter\edef\csname mc\Alph{ct}\endcsname{\noexpand\mathcal{\Alph{ct}}}%
	\expandafter\edef\csname mb\Alph{ct}\endcsname{\noexpand\mathbb{\Alph{ct}}}%
}
%
\newcommand*{\blank}{\raisebox{1pt}{\texttt{\char32}}}
\newcommand*\A{\mathcal{A}}
\newcommand*\dotcup{\mathrel{\dot\cup}}
\DeclareMathOperator\Konf{Konf}
\DeclareMathOperator\out{out}
\DeclareMathOperator\Sim{Sim}
\DeclareMathOperator\Instr{Instr}
\DeclareMathOperator\inc{inc}
\DeclareMathOperator\dec{dec}
\DeclareMathOperator\DTAPE{DTAPE}
\DeclareMathOperator\NTAPE{NTAPE}
\DeclareMathOperator\NTIME{NTIME}
\DeclareMathOperator\DTIME{DTIME}
\DeclareMathOperator\code{code}
\DeclareMathOperator\pos{pos}
\DeclareMathOperator\state{state}
\DeclareMathOperator\tape{tape}
\DeclareMathOperator\CLIQUE{CLIQUE}
\DeclareMathOperator\PREC{PREC} % Def. 8.2
\DeclareMathOperator\add{add} % Bsp. zu 8.2
\DeclareMathOperator\mult{mult} % Bsp. zu 8.2

% Faster compilation: one section / file.
%\includeonly{6-Berechenbarkeit}

%========================%
\begin{document}
\title{Vorlesung Informatik III -- Theoretische Informatik}
\author{Prof. Dr. Peter Thiemann\\\normalsize{\LaTeX\ von Ralph Lesch\thanks{ralph.lesch@neptun.uni-freiburg.de}}}
\date{WS\,2014/15\\2015-10-21 -- 2016-02-22}%\today}

\maketitle
\vspace{\baselineskip}
\tableofcontents

\section[Vorspann: Sprachen]{Vorspann: Sprachen\datenote{23.10.15}}
Zeichen, Symbole: Buchstaben, Ziffern\\
hier: abstrakte Zeichen
\begin{Def}[name={[Alphabet $\Sigma$]}]
	Ein \underline{Alphabet} $\Sigma$ ist eine endl. Menge von Zeichen.
	
	Aus Zeichen \-> Wörter durch Hintereinanderschreiben.
\end{Def} % 1.1
\begin{Bsp*} $\Sigma = \{a,\dots,z\}$\\
	Worte: rambo (5 Zeichen), ist, hungrig, \qquad $\epsilon$ (0 Zeichen) = leeres Wort\\
	Wörter verketten = konkatenieren
\end{Bsp*}
\begin{Bsp*} rambo$\cdot$ist$\cdot$hungrig\\
	"`$\cdot$"' ist Konkat.-Operator
	
	Wörter potenzieren: $\begin{aligned}[t]
	(\text{la})^3 &= \text{la}\cdot\text{la}\cdot\text{la} = \text{lalala}\\
	(\text{rambo})^0 &= \epsilon
	\end{aligned}$
\end{Bsp*}

\begin{Def}[name={[Wort $w$ über $\Sigma$]}]\label{def:1.2}
	Ein Wort $w$ über $\Sigma$ ist eine endliche Folge von Zeichen $w=a_1a_2\dots a_n$ mit $n\in\N$ und $a_i\in\Sigma\ (1\leq i\leq n)$.\\
	Schreibe $\epsilon$ falls $n=0$\\
	$|w|=n$ ist die Länge des Wortes $w$.\\
	$\Sigma^*$ ist die Menge aller Wörter über $\Sigma$.
\end{Def}
\begin{Def}[Konkatenation von Wörtern]
	\begin{align*}
		\text{Sei }u &= a_1\dots a_n\in\Sigma^*\\
		v &= b_1\dots b_m\in\Sigma^*\\
		\shortintertext{dann ist $u\cdot v=c_1\cdots c_{n+m}\ ,\ c_i \in \Sigma$}
		c_i &= \begin{cases}
		a_i & 1\leq i\leq n\\
		b_{i-n} & n+1\leq i\leq n+m
		\end{cases}
	\end{align*}
	Eigenschaften von "`$\cdot$"':
	\begin{itemize}
		\item assoziativ
		\item $\epsilon$ ist neutrales Element
	\end{itemize}
	Die Potenz $v^k$ , $v\in \Sigma^*,k\in\N^*$ ist def. durch
	\[ v^0=\epsilon,\ v^{k+1}=v\cdot v^k \]
\end{Def}
%
Eine Sprache ist Menge von Wörtern über $\Sigma=\{a,\dots,z,"a,"o,"u\}$
\begin{align*}
	L_\text{obst} &= \{\text{banane,aprikose,orange,\dots}\}\\
	L_\text{farbe} &= \{\text{rot,gelb,grün}\}\\
	L_\text{krach} &= \{\text{ra}\cdot(\text{ta})^n\cdot\text{mm} \mid n\in\N \}\\
	L &= \{\} \quad\text{leere Sprache}\\
	L &= \Sigma^*
\end{align*}

\begin{Def}[name={[Sprache über $\Sigma$]}]
	Eine \underline{Sprache über $\Sigma$} ist Menge $L\subseteq\Sigma^*$.
	\[ L_\text{lala} = \{(\text{la})^n \mid n\in\N\} \ni\Sigma
	\qedhere \]
\end{Def}
sämtliche Mengenoperationen sind auch Sprachoperationen, insbesondere:\medskip\\
\begin{tabular}{lllll}
	$L_1\cap L_2$ &,& $L_1\cup L_2$ &,& $\Sigma^*\setminus L$\\
	Schnitt && Verein. && Komplement
\end{tabular}\medskip\\
$\rightsquigarrow$ Weitere Operationen auf Sprachen: Konkatenation
\begin{Bsp*}
	\begin{align*}
		L_\text{farbe}\cdot L_\text{obst} &\subseteq \{\text{rotbanane, rotaprikose, gelbbanane, gelbaprikose,}\\
		&\phantom{{}\subseteq \{}\text{gelborange, grünbanane, grünaprikose, grünorange}\}\\
		L_\text{farbe}\cdot\{\epsilon\}\cdot L_\text{obst} &\subseteq \{ \text{rot\textbf{e}banane, rot\textbf{e}aprikose, rot\textbf{e}orange, gelb\textbf{e}banane, \dots} \}
	\end{align*}
\end{Bsp*}
\begin{Def}[Konkatenation von Sprachen] % 1.5
	Sei $U,V\subseteq \Sigma^*$ dann ist
	\[ U\cdot V = \{uv \mid u\in U, v\in V \} \qedhere \]
\end{Def}
Potenzieren
\begin{align*}
	L_\text{farbe}^2 &= L_\text{farbe}\cdot L_\text{farbe}\\
	&= \{ \text{rotorot, rotgelb, rotgrün, gelbrot, gelbgelb, \dots} \}\\
	L^0 &=\{\epsilon\} \qquad \{\epsilon\}\cdot L = \{\epsilon\cdot w \mid w\in L\} = L \qedhere
\end{align*}
%
\begin{Def}[Potenzierung von Sprachen] Sei $U\subseteq\Sigma^*$
	\begin{align*}
		U^0 &= \{\epsilon\} & U^{n+1}= U \cdot U^n
	\end{align*}
\end{Def}
Gegeben $L$, sind sämtliche Kombinationen von Worten aus $L$ gesucht.
\begin{Bsp*}
	$\{(la)^n \mid n\in\N\}\cdot\{la\}^*$
\end{Bsp*}
\vspace{.5em}
\begin{Def}[Stern-Operator, Abschluss, Kleene-Stern]
	Sei $U\in\Sigma^*$ dann ist
	\begin{align*}
		U^* &= \bigcup_{n\in\N} U^n \quad [\ni\epsilon] && \text{Leeres Wort ist darin }n=0\\
		U^+ &= \bigcup_{n\ge1} U^n && \text{wenn $\epsilon\in U$ \=> auch in $U^*$} \qedhere
	\end{align*}
\end{Def}
\begin{Bemerkung}
	$\Sigma^*=\bigcup_{n\in\N}\Sigma^n \text{ ebenso } \Sigma^+ = \bigcup_{n\ge1}\Sigma^n$
\end{Bemerkung}
\begin{Def}[Alternative Definition von $\Sigma^*$]\label{def:1.8}\ \\
	Die Menge $\Sigma^*$ ist kleinste Menge, so dass
	\begin{enumerate}[label={(\arabic*)}]
		\item $\epsilon\in\Sigma^*$
		\item\label{def:1.8.2} $a\in\Sigma\ ,w\in\Sigma^*\ \=> aw\in\Sigma^*$\qedhere
	\end{enumerate}
\end{Def}
Die Definitionen \ref{def:1.8} und \ref{def:1.2} sind gleichwertig.
\begin{proof}\ 
	\begin{itemize}
		\item "`\autoref{def:1.2}"' \=> "`\autoref{def:1.8}"':\\
		$\forall n\in\N\ w=a_1\dots a_n \quad, a_i\in\Sigma$
		
		Zeige $w\in\Sigma^*$ (gemäß \autoref{def:1.8})
		
		\underline{I.A.: $n=0$} $\=> w = \epsilon \in \Sigma^*$ \ (\autoref{def:1.8})\\
		\underline{$n\->n+1$} $w = a_1\dots a_{n+1}$ nach \autoref{def:1.2} $w'=\overbrace{a_2\dots a_{n+1}}^{\smash{n\text{ Buchstaben}}} \in \Sigma^*$\\
		\-> Nach \autoref{def:1.8} \ref{def:1.8.2} $a_1a_2\dots a_{n+1} \in\Sigma^*$ (\autoref{def:1.8})
		\item "`\autoref{def:1.8}"' \=> "`\autoref{def:1.2}"':\\
		Induktion über $w\in\Sigma^*$ \ (\ref{def:1.8})
		Zeige: f"ur jedes $w\in\Sigma$ gibt es ein $n\in\N$ und $a_1, \dots, a_n \in \Sigma$, so dass $w=a_1\dots a_n$.
		\begin{itemize}
			\item $\epsilon\in\Sigma^*$ Wähle $n=0$ dann $w=\epsilon$.
			\item $aw':\ a\in\Sigma,\ w'\in\Sigma^*$ (\ref{def:1.8})\\
			Nach Induktionsbehauptung $w'\in\Sigma^*$ (\autoref{def:1.2})\\
			Also $\exists n: w'=a_1\dots a_n\in\Sigma^*$ (\autoref{def:1.2})\\
			W"ahle (als neues $n$) $m = n+1$ und $b_1, \dots b_{n+1} \in\Sigma$
			mit $b_1 = a$ und $b_{i+1} = a_i$ f"ur $1\le i\le n$.\\
			Dann ist $w = b_1b_2\dots b_{n+1}\in\Sigma^*$ (\ref{def:1.2}) \qedhere
	\end{itemize}
	\end{itemize}
\end{proof}
Alternative Definition von Konkatenation:
\begin{align*}
	\epsilon\cdot v &= v\\
	(aw)\cdot v &= a(w\cdot v)
\end{align*}
%
\begin{Bsp*} Für Eigenschaft von Sprachen:
	\begin{align*}
		U^* &= \{\epsilon\}\cup U\cdot U^*
	\end{align*}
	Beweis durch Kalkulation:
	$U^*=\{\epsilon\}\cup U\cdot U^*\subseteq U^*\cup U\cdot\bigcup_n U^n = U^*\cup\bigcup_n U^{n+1} \subseteq U^*\cup U^* = U^*$
\end{Bsp*}
Elementarer Beweis:
\begin{enumerate}[label={Zeige (\arabic*)},leftmargin=*,itemindent=*]
	\item $U^*\subseteq \{\epsilon\}\cup U\cdot U^*$\\
	Sei $w\in U^*=\bigcup_{n\in\N} U^n$\\
	$\curvearrowright \exists n: w\subset U^n$
	\begin{itemize}
		\item Falls $n=0: w=\epsilon\subset\{\epsilon\}\cup U\cdot U^*$
		\item Falls $n= n'+1: w\in U\cdot U^n\subseteq U\cdot U^*\cup \{\epsilon\}$
	\end{itemize}
	\item $\{\epsilon\}\cup U\cdot U^*\subseteq U^*$\\
	Sei $w\in\{\epsilon\} \cup U\cdot U^*$
	\begin{itemize}
		\item Falls $w=\epsilon: w=\epsilon\in U^*$
		\item Falls $w\in U\cdot U^n: \exists n\ w\in U\cdot U^n = U^{n+1}\subseteq U^*$
	\end{itemize}
\end{enumerate}
\section{Turing und Church}
%\section{\acf{TM}}
1930er Jahre\\
Suche nach formalem Modell für maschinelle Berechenbarkeit
\begin{description}
	\item[Alaen Turing:] (1912-1954) Turingmaschine 1936
	\item[Church:] Lambdakalkül 1936
	\item[Kleene Sturgis:] partielle rekursive Funktionen
	\item[Chomsky:] Typ-0-Grammatiken 1956
\end{description}
\underline{Alan Turing:}\begin{minipage}[t]{.8\textwidth}
\begin{itemize}[parsep=0pt]
	\item Informatik, Logik
	\item Kryptographie (Enigma Entschlüsselung, Sprachverschlüsselung)
	\item KI (Turing-Test)
\end{itemize}\end{minipage}

außerdem: Turing-Award

\subsection{Turingmaschine \normalfont(informell)}
Ein primitives Rechenmodell:
\begin{figure}[H]\centering
	\begin{tikzpicture}[every node/.style={block}]
		\node (1) {\blank};
		\node (2) [right=of 1] {b};
		\node (3) [right=of 2] {a};
		\node (4) [right=of 3] {n};
		\node (5) [right=of 4] {a};
		\node (6) [right=of 5] {n};
		\node (7) [right=of 6] {e};
		\node (8) [right=of 7] {\blank};
		\node (9) [right=of 8] {\blank};
		\node (10) [right=of 9] {\blank};
		\node (11) [right=of 10] {\blank};
		\node (last) [right=of 11] {\blank};
		
		\node (Kopf) [below=1.5em of 2, draw=none, text height=.5em] {Kopf}
		edge [->, shorten >=.5ex, semithick] (2);
		
		\node (TB) [below=.5em of 9, draw=none, text height=.5em,  anchor=north west] {Turingband};
		
		\draw (8.south) -- ($(8.south)-(0,1em)$) -- ($(TB.north west)-(0,.5em)$);
		
		% Open begin and end.
		\draw (1.north west) -- ++(-1cm,0) (1.south west)
		-- ++ (-1cm,0) (last.north east)
		-- ++ ( 1cm,0) (last.south east)
		-- ++ ( 1cm,0);
	\end{tikzpicture}
	\vspace{-1em}
	\caption{Turingband}
	\framebox{q} = Zustand
\end{figure}
\vspace{-.5em}
\begin{tabu}{>{\bfseries}X[.22]X[.72]}
	Turingband & \vspace{-1em}\begin{itemize}[leftmargin=1em,parsep=0pt,topsep=0pt]
	\item unendliches Band
	\item Jedes Feld enthält ein Symbol aus einem Bandalphabet $\Gamma$
	\item uninitialisiert: Blank \blank ist ein spezielles Symbol $\blank\in\Gamma$
	\end{itemize}
	\\
	Kopf & \vspace{-1em}\begin{itemize}[leftmargin=1em,parsep=0pt,topsep=0pt]
	\item zeigt immer auf ein Feld
	\item nur am Kopf kann die \ac{TM} ein Zeichen lesen und schreiben
	\item kann nach rechts /links bewegt werden
	\item kann verändert werden
	\end{itemize}\\
	Zustand & \vspace{-1em}\begin{itemize}[leftmargin=1em,parsep=0pt,topsep=0pt]
	\item kann verändert werden
	\item kann gelesen werden
	\item es gibt nur endlich viele Zustände
	\end{itemize}\\
	Turingtabelle\newline\normalfont
	\begin{tabular}{|*5{c|}}
		q & a & q' & a' & d \\\hline
		&&&&\\
		&&&&
	\end{tabular}
	& $\sim$ Programm $\sim$ Transitionsfunktion \newline
	$\rightarrow$ Wenn \ac{TM} in Zustand $q$ und Kopf liest gerade Symbol $a\in\Gamma$ dann wechsle in Zustand $q'$. Schreibe $a'$ (über altes $a$) und bewege den Kopf gemäß $d\in\{L,R,N\}$
\end{tabu}\

\begin{samepage}
	\begin{Bsp*}\ 
		\vspace{-2em}
		\begin{figure}[H]\centering
			\begin{tikzpicture}[every node/.style={block}]
				\node (1) {\blank};
				\node (2) [right=of 1] {\blank};
				\node (3) [right=of 2] {\cancel{b}};
				\node (4) [right=of 3] {a};
				\node (5) [right=of 4] {n};
				\node (6) [right=of 5] {a};
				\node (7) [right=of 6] {n};
				\node (8) [right=of 7] {e};
				\node (9) [right=of 8] {\cancel{\blank}};
				\node (10) [right=of 9] {\blank};
				\node (last) [right=of 10] {\blank};
				
				\node (q1) [draw=none, above=-2pt of 3] {\blank};
				\node (q3) [draw=none, above=-2pt of 9] {b};
				
				% Open begin and end.
				\draw (1.north west) -- ++(-1cm,0) (1.south west)
				-- ++ (-1cm,0) (last.north east)
				-- ++ ( 1cm,0) (last.south east)
				-- ++ ( 1cm,0);
			\end{tikzpicture}
			\caption{Bsp.: Turingmaschine}
			\vspace{.5em}
			\begin{tabu}{|*5{M{c}|}l}\tabucline{1-5}
				\everyrow{\tabucline{1-5}}
				q_1 & b & q_2 & \blank & R\\
				q_1 & x\pm b & q_1 & x & N\\
				q_2 & \blank & q_3 & b & L\\
				q_2 & x+\blank & q_2 & x & R\\
				q_3 & x+\blank & q_3 & x & L\\
				q_3 & \blank & q_4 & \blank & L\\
				q_4 & x & q_4 & x & N & \-> Endzustand
			\end{tabu}
		\end{figure}
	\end{Bsp*}
\end{samepage}
%
\datenote{28.10.15}
Was kann die \ac{TM} ausrechnen?
\begin{enumerate}
	\item Die \ac{TM} kann eine Sprache $L\subseteq\Sigma^*$ erkennen.
	\begin{itemize}
		\item Wörter müssen auf Band repräsentierbar sein $\Sigma\subseteq\Gamma\setminus\{\blank\}$
	\end{itemize}
	Ein Wort $w$ wird von einer \ac{TM} erkannt, wenn
	\begin{itemize}
		\item zu Beginn steht nur $w$ auf dem Band, alle anderen Zellen $=\blank$
		\item Kopf auf erstem Zeichen von $w$
		\item Zustand ist Startzustand $q_0$
		\item Abarbeitung der \ac{TT}
		\item Falls \ac{TM} nicht terminert: $w\notin L$
		\item Falls \ac{TM} terminert betrachte den errechneten Zustand $q$.\\
		Falls $q\in F$ (akzeptierter Zustand), dann $w\in L$, anderenfalls $w\notin L$
	\end{itemize}
	
	\begin{Bsp*}
		\begin{flalign*}
			\Sigma &=\{0,1\} &\\
			L &=\left\{w\in\Sigma^* \mid \,w\text{ ist Palindrom}\right\}\\
			Q &= \{q_0,q_1,q_r^0, q_r^1, {q_r^0}', {q_r^1}', q_l^0, q_l^1 \} \quad F=\{q_1\}
		\end{flalign*}
		\begin{tabular}{@{}*6{M{l}}}
			q_0      & \blank & q_1      & \blank & N & q_1 \x q_1\x N\\
			q_0      & 0      & q_r^0    & \blank & R\\
			q_0      & 1      & q_r^1    & \blank & R
			\\ \cmidrule{1-5}
			q_r^0    & \blank & q_1      & \blank & N\\
			q_r^0    & 0      & {q_r^0}' & 0      & R\\
			q_r^0    & 1      & {q_r^0}' & 1      & R\\
			{q_r^0}' & \blank & q_l^0    & \blank & L & q_l\->\text{prüfe $0$, fahre zum linken Rand und weiter mit }q_0\\
			{q_r^0}' & 0      & {q_r^0}' & 0      & R & \multirow{2}{*}{\hspace{-1em}$\begin{rcases}\\[1em]\end{rcases}$ Rechtslauf} \\
			{q_r^0}' & 1      & {q_r^0}' & 1      & R
		\end{tabular}\\[.5em]
		\begin{tabular}{@{}*6{M{l}}|}
			\multicolumn{6}{@{}l|}{Alternative 1:}\\
			\multicolumn{6}{@{}l|}{\ac{TM} hält bei jeder Eingabe an.}\\[.5em]
			q_l^0 & \blank & q_l^0  & \blank & N & \<-\text{Halt}\\
			& 0   & q_l    & \blank & L &\\
			& 1   & q_l^0  & 1      & N & \<-\text{Halt}
		\end{tabular}\quad\begin{tabular}{@{}*5{M{l}}@{ }l}
		\multicolumn{6}{@{}l}{Alternative 2:}\\
		\multicolumn{6}{@{}l}{\ac{TM} hält nur bei Palindrom an.}\\[.5em]
		q_l^0 & \blank & q_l^0 & 1      & N & \multirow{3}{*}{\scalebox{2.9}{\rotatebox[origin=rb]{-90}{$\curvearrowleftright$}}}\\
		q_l^0 & 0      & q_l   & \blank & L\\
		q_l^0 & 1      & q_l^0 & \blank & N
		\end{tabular}
	\end{Bsp*}
	
	\item Die \ac{TM} errechnet Funktion $f: \Sigma^*\-->\Sigma^*$\\
	Die Berechnung von $f(w),\ w\in\Sigma^*$
	\begin{itemize}
		\item $w$ auf leeres Band
		\item Kopf auf erstes Zeichen, Standardzustand $q_0$
		\item Abarbeitung der \ac{TT}
		\item Falls terminiert, dann Kopf zuerst auf erste Symbol von $v\in\Sigma^*$\\
		Dann $f(w)=v$
	\end{itemize}
\end{enumerate}
\begin{alignat*}{3}
	\text{Schreibe}&\quad& A &\-->B &\quad& \text{totale Funktion von $A$ nach $B$}\\
	&& A&\dashrightarrow B && \text{partielle Funktion von $A$ nach $B$}
\end{alignat*}
\begin{Bsp} % 2.1
	$\Sigma=\{0,1\}$\\
	Gesucht die \ac{TM}, die die Nachfolgefunktion auf natürliche Zahlen in Binärdarstellung berechnet.\\
	Ausnahme: niederwertigste Stelle von der Zahl.\medskip\\
	\begin{tabular}{@{}M{l}@{ } *5{M{l}} @{ }l}
		\xrightarrow{\text{Start}} & q_0 & \blank & q_2 & 1 & L \\
		& q_0 & 0      & q_1 & 1 & L \\
		& q_1 & 1      & q_0 & 0 & R \\[.5em]
		& q_1 & \blank & q_1 & \blank & N & \<-Halt\\
		& q_1 & 0      & q_2 & 0 & L & \multirow{2}{*}{$\begin{rcases}\\[1em]\end{rcases}$ Linksmaschine}\\
		& q_1 & 1      & q_2 & 1 & L
	\end{tabular}
\end{Bsp}

\subsection{Formalisierung der \ac{TM}} % 2.2
\begin{Def}[name={[\acs*{TM}]}] % 2.1
	Eine \ac{TM} ist ein 7-Tupel
	\begin{equation*}
		\mathcal{A}=\left(Q,\Sigma,\Gamma,\delta,q_0,\blank,F\right)\\
	\end{equation*}
	\begin{itemize}
		\item $Q$ ist endliche Menge der Zustände
		\item $\Sigma$ ist endliches Alphabet
		\item $\Gamma\supsetneq\Sigma$ ist endliches Bandalphabet
		\item $\delta: Q\times\Gamma\-->Q\times\Gamma\times\{R,L,N\}$
		\item $q_0\in Q$ Standardzustand
		\item $\blank\in\Gamma\setminus\Sigma$ das Blank
		\item $F\subseteq Q$ Menge der akzeptierenden Zustände
	\end{itemize}
\end{Def}
Im Folgenden sei $\A=(Q, \Sigma, \Gamma, \delta, q_0, \blank, F)$ eine \ac{TM}.

\begin{Def}[name={[Konfiguration einer \acs*{TM}]}] % 2.2
	Eine Konfiguration einer \ac{TM}
	ist ein Tupel
	\[ (v,q,w) \in \Konf(\A)=\Gamma^*\times Q\times\Gamma^+  \]
\end{Def}
%
\begin{itemize}
	\item $v$ linke Bandhälfte,
	\item $q$ Zustand,
	\item $w$ rechte Bandhälfte,
	\item Kopfpos auf erstem Symbol von $w$
\end{itemize}

Abk"urzend $ v qw \in \Konf(\mathcal{A}) $ steht für Band:
\begin{figure}[H]\centering
	\begin{tikzpicture}[every node/.style={block}, decoration={brace, amplitude=5pt}]
		\node (A) {$v$};
		\node (B) [right=of A] {$a$};
		\node (C) [right=of B] {$w'$};
		\node (D) [below=1em of B, draw=none] {Kopf; Zustand $q$}
		edge [->, shorten >=.5ex, semithick] (B);
		
		\draw [decorate, semithick] (B.north west) -- (C.north east)
		node [draw=none,above,midway] {$w$};
		\draw (A.north west) -- ++(-1cm,0) (A.south west)
		-- ++ (-1cm,0) (C.north east)
		-- ++ ( 1cm,0) (C.south east)
		-- ++ ( 1cm,0);
	\end{tikzpicture}
	\caption{$vqw$-Band}
\end{figure}
%
\paragraph{Forts.:} Die \underline{Startkonfiguration} bei Eingabe $w$ ist:
$\begin{cases}
q_0\,w &, w\neq\epsilon\\
q_0\,\blank &, w=\epsilon
\end{cases}$\\
Eine \underline{Haltekonfiguration} hat folgende Form: $vqaw$, so dass $\delta(q,a)=(q,a,N)$

\begin{Def}[name={[Rechenschrittrelation]}] % 2.3
	Die \underline{Rechenschrittrelation}
	\[ \vdash\, \subseteq \Konf(\A)\x\Konf(A) \]
	ist definiert durch
	\begin{alignat*}{3}
		1.&\ & v qaw &\vdash v q'a'w &\quad& \text{falls }\delta(q,a)=(q',a',N)\\
		2.&& v qaw &\vdash v a'q'w && \text{falls }\delta(q,a)=(q',a',R), w\neq \epsilon\\
		&&&\phantom{{}\vdash{}} v a'q'\blank && \ruleplaceholder{\widthof{falls $\delta(q,a)=(q',a',R)$}}, w=\epsilon\\
		3.&& qaw &\vdash q'\blank a'w && \delta(q,a)=(q',a',L)\\
		&& vbqaw &\vdash vq'ba'w && \ruleplaceholder{\widthof{$\delta(q,a)=(q',a',L)$}}\  b\in\Gamma
	\end{alignat*}
	$\vdash$ Einzelschritt, gesuchte Relation für endlich viele Schritte \smallskip\\
	$\vdash^* \subseteq \Konf(\A) \x \Konf(\A)$ die reflexive transitive Hülle von $\vdash$
\end{Def}


\datenote{30.10.15}
$\vdash$ Relation auf $\Konf(\A)\hat=$ Berechnungsschritt\\
$\vdash^*$ reflexiv, transitiver Abschluss $\hat=$ endl.\ viele Berechnungsschritte
%
\paragraph{Exkurs:} binäre Relationen\\
$A$ Menge, jedes $R\subseteq A\x A$ ist binäre Relation auf A\\
\begin{Bsp*}\ \\
	\begin{tabular}{M{r}l}
		\varnothing & leere Relation \\
		A\x A     & volle Relation \\
		\1_A =\{(x,x) \mid x\in A\} & Gleichheit auf $A$\\
		\leq\,\subseteq\N\x\N\quad <\, \subseteq \N\x\N\quad & $\mid\, \subseteq\N\x\N$
	\end{tabular}\medskip\\
	Mengenoperationen auf Rel. ok
	\begin{align*}
		R_1 &=\, <\cup\, \1_\N =\, \leq\\
		(x,y) &\in R_1 \<=> x<y\ \lor\ x=y\\
		R_2 &= <\cap\, \1_\N = \varnothing\\
		F_3 &= \{(x,y)\mid y=3x\}\subseteq \N\x\N
		% y durchgestrichen
	\end{align*}
\end{Bsp*}
\refstepcounter{Def}
\begin{subDef}[name={[$R,S\subseteq A\x A$ Relation]}] %2.4.1
	$R,S\subseteq A\x A$ Relation\\
	Die Verkettung (Komposition) von $R$ und $S$
	\[ R\circ S = \{(x,y)\in A\times A \mid \exists z\in A: (x,z)\in R,(z,y)\in S)\} \]
\end{subDef}
\begin{Bsp*}
	\begin{align*}
		&\underbrace{<\circ\1_\N}_{=S_1} =\, <\\
		&(x,y)\in S_1\\
		\<=>\ & \exists z: x<z\ ,z=y\\
		\<=>\ & x<y\\
		&F_3\circ F_3\\
		&(x,y)\in F_3\circ F_3\\
		\<=>\ & \exists z: (x,z)\in F_3\ ,\ (z,y)\in F_3\\
		\<=>\ & \exists z: z=3x\ \land\ y=3z\\
		\<=>\ & y=9x
	\end{align*}
\end{Bsp*}
\begin{subDef}[name={[$R\subseteq A\x A$]}] % 2.4.2
	$R\subseteq A\x A$
	\begin{enumerate}[label={\alph*)}]
		\item R ist \underline{reflexiv}, falls $\1_A\subseteq R$
		\item R ist \underline{transitiv}, falls $R\circ R \subseteq R$
	\end{enumerate}
\end{subDef}
\begin{Bsp*}
	\begin{tabular}[t]{M{l}ll}
		\1_A & refl. & trans.\\
		\leq_\N & refl. & trans.\\
		\varnothing & nicht refl. & trans.\\
		\mid & refl. & trans.\\
	\end{tabular}
\end{Bsp*}
\begin{subDef}[name={[$R\subseteq A\x A$ Relation]}] % 2.4.3
	$R\subseteq A\x A$ Relation\\
	Der reflexiv transitive Abschluss (Hülle) von $R$ ist
	\begin{align*}
		R^* &=\bigcup\limits_{n\in\N} R^n\\
		\shortintertext{mit}
		R^0 &=\1_A \quad,\ R^{n+1}=R\circ R^n
	\end{align*}
\end{subDef}
%
Bem:\quad $R^* =1\cup R\circ R^*$ gilt auch.

Es gilt: Für bel. $R$:
\begin{itemize}
	\item $R^*$ refl. \quad:\quad $\1 = R^0\subseteq\bigcup\limits_{n\in\N}R^n=R^*$
	\item $R^*$ trans.
	\begin{align*}
		R^* &\circ R^* \subseteq R^*\\
		R^* &\circ \bigcup_n R^n\\
		\forall n: R^* &\circ R^n \subseteq R^*\\
		\=> R^* &\circ \bigcup_n R^n \subseteq R^*
	\end{align*}
%	\rlerror*{Wortkorrektur nötig}{Refvim?}
\end{itemize}
${\vdash^2} = {\vdash}\circ\vdash$ zwei Schritte\\
$\vdash^*$ endl. viele Schritte
%
\begin{Def}[Die von \acs*{TM} $\A$ erkannte Sprache] % 2.5
	\begin{align*}
		L(\A)=\{ w\in\Sigma^* \mid{}
		& q_0 w \vdash^*uqv\\
		&uqv \text{ Haltekonfiguration}\\
		&q\in F\}
	\end{align*}
\end{Def}
\underline{Beachte:}
$w\notin L(\A)\begin{casesarrows}
\A\text{ kann anhalten}          \\
\A\text{ kann nicht terminieren}
\end{casesarrows}$

\begin{Def}[Die von \ac{TM} $\A$ berechnete Funktion] % 2.6
	\begin{alignat*}{3}
		&&&f_\A:\Sigma^*- \->\Sigma^*\\
		&&&f_\A(w)=v\\
		&\text{ falls } &&q_0w \vdash^*uqv'&\quad&\text{Haltekonf.}\\
		&\text{ und } &&v=\out( v')\\[.5em]
		&\out:\Gamma^* &&\-> \Sigma^*\\
		&\out(\epsilon) &&= \epsilon\\
		&\out(au) &&= a\cdot\out(u) &\quad& a\in\Sigma\\
		&\out(bu) &&= \epsilon && b\in\Gamma\setminus\Sigma
	\end{alignat*}
\end{Def}
\underline{Beachte:} Falls $q_0 w$ nicht terminiert, dann ist $f_\A(w)$ nicht definiert.

Eine \ac{TM} $\A$ terminiert nicht bei Eingabe $w$, falls für alle $uq'v$, so dass $q_0w\vdash^*uq'v$\\
$uq'v$ ist keine Haltekonfiguration.

\subsection{Techniken zur \ac{TM} Programmierung}
\begin{itemize}
	\item Endlicher Speicher\\
	Zum Abspeichern eines Elements aus endl. Menge $A$ verwende
	\[ Q'=Q\x A \]
	\item Mehrspurmachinen
	\begin{figure}[H]\centering
		{\renewcommand{\arraystretch}{0.8}
		\begin{tabu} to .5\textwidth {X[.35]|X[.65]}
			&\\\hline
			&\\\hline
			&\\\hline
			&\\\hline
			&
		\end{tabu}}
		\caption{Mehrspurmachine}
	\end{figure}
	
	Eine $k$-Spur \ac{TM} kann gleichzeitig $k\geq 1$ Symbole $\<- \Gamma$ unter dem Kopf lesen.\\
	Kann durch Standard \ac{TM} simuliert werden:
	\[ \Gamma' = \Sigma \overset{.}{\cup} \Gamma^k\text{ mit } \blank'=\blank^k \]
	\dots vereinfacht die Programmierung\\
	\begin{Bsp*}
		Schulalg. für binäre Addition, Multiplikation
	\end{Bsp*}
	\item \underline{Mehrbandmachinenen}\\
	Eine $k$-Band \ac{TM} besitzt $k\geq1$ Bänder und $k$ Köpfe, die bei jedem Schritt lesen, schreiben und sich unabhängig voneinander bewegen.
	\[ \delta_K:Q\x\Gamma^k \-> Q\x\Gamma^k\x \{R,L,N\}^k \]
	\item keine herkömmliche \ac{TM} (für $k>1$)
	\item kann durch $2k+1$ Spur \ac{TM} simuliert werden:\\
	\begin{tabular}{lll}
		Spur\\
		1 & \ruleplaceholder[ Band 1 ]{.5\linewidth}\\
		2 & \hspace{.23\linewidth}\# Kopf für Band 1\\
		3 & \ruleplaceholder[ Band 2 ]{.5\linewidth}\\
		4 & \multicolumn1r{\# Kopf\qquad\ }\\
		& \vdots\\
		$2k$ & \hspace{.09\linewidth}\dots\hspace{.09\linewidth} \# Kopf $k$\\
		$2k+1$ & \# &\#\#\\
		& linker Rand & rechter Rand
	\end{tabular}
\end{itemize}
\vspace{1em}

\datenote{04.11.15}
\begin{Satz}[name={[Simulation von $k$-Band \acs*{TM} durch 1-Band \acs*{TM}]}]
	Eine $k$-Band \ac{TM} kann durch eine 1-Band \ac{TM} simuliert werden.\quad $M=(Q\dots)$
\end{Satz}
\begin{proof}
	Zeige: ein Schritt der $k$-Band \ac{TM} wird durch endlich viele Schritte auf einer 1-Band \ac{TM} simuliert.
	\begin{enumerate}
		\item Schritt: Kodierung der Konfiguration der $k$-Band \ac{TM}\\
		Definiere $M'$ als \ac{TM} mit $2k+1$ Spuren und $\Gamma'=\Gamma\cup\{\#\}$
		\begin{itemize}
			\item Die Spuren $1,3,\dots,2k-1$ enthalten das entspr. Band von $M$: Band $i\<->$ Spur $2i-1$
			\item Die Spuren $2,4,\dots,2k$ sind leer bis auf eine Marke \#, die auf Spur $2i$ die Position des Kopfes auf Band $i$ markiert
			\item Spur $2k+1$ enthält\\
			\#\phantom{\#} Marke für linken Rand\\
			\#\# Marke für rechten Rand\\
			Zwischen den beiden Marken befindet sich der bearbeitete Bereich des Bands. D.h. die \ac{TM} arbeitet zwischen der linken und rechten Marke und schiebt die Marken bei Bedarf weiter.
		\end{itemize}
		
		\item Schritt: Herstellen der Start-Konfiguration.\\
		Annahme: Eingabe für $M$ auf Band 1\\
		Jetzt Eingabe (für $M'$) $w = a_1\dots a_n$
		\begin{enumerate}
			\item Kopiere $w$ auf Spur 1
			\item Kopf setzen auf Spur $2,\dots,2k$ an die Position des ersten Symbols von $w$
			\item auf Spur $2k+1$: \verb*!# ##!
	\end{enumerate}
	\begin{tabular}{*2{M{l}}}
		2k+1 & \#\blank\#\#\\
		2k & \#\\
		2k-1 & \blank\\
		\vdots\\
		4 & \#\\
		3 & \blank\\
		2 & \#\\
		\text{Spur }1 & a_1a_2\dots a_2
	\end{tabular} 
	
	Springe nach Sim($q_0$), der Zustand in $M'$, an dem die Simulation des Zustands $q$ aus $M$ beginnt.
	
	\item Simulation eines Rechnerschritts im Zustand Sim($q$):\\
	Kopf auf linker Begrenzung, d.h. linker \# auf Spur $2k+1$
	\begin{itemize}
		\item Durchlauf bis rechter Rand, sammle dabei Symbole unter den Köpfen, speichern in endl. Zustand $\overrightarrow{\gamma} \in \Gamma^k$
		\item Berechne $\delta(q,\overrightarrow{\gamma})=(q',\overrightarrow{\gamma'},\overrightarrow{d})$\\
		neuer Zustand, für jeden Kopf ein neues Symbol $\overrightarrow{\gamma'}$ und Richtung $\overrightarrow{d}$.
		\item Rücklauf nach links, dabei Schreiben um $\overrightarrow{\gamma}'$ und Versetzen der Köpfe gem"a"s $\overrightarrow{d}$.
	\end{itemize}
	Falls eine Kopfbewegung den Rand auf Spur $2k+1$ überschreitet, dann verschiebe Randmarke entsprechend.
	
	Beim Rücklauf: Test auf Haltekonfiguration der $k$-Band \ac{TM}.\\
	Falls ja, dann Sprung in Haltekonf. von $M'$
	
	Weiter im Zustand Sim$(q')$.
	\end{enumerate}
\end{proof}

\begin{Korollar*}
	Beim Erkunden eines Worts der Länge $n$ benötige die $k$-Band Maschine $M\ T(n)$ Schritte und $S(n)$ Zellen auf den Bändern.
	\begin{itemize}
		\item $M'$ benötigt $O(S(n))$ Zellen
		\item $M'$ benötigt $O(S(n\cdot T(n)))$ Schritte $=O(T(n)^2)$
	\end{itemize}
	Weitere \ac{TM}-Booster
	\begin{itemize}
		\item Unbeschränkt großer Speicher
		\begin{itemize}[label=\->]
			\item für jede "`Variable"' ein neues Band
	\end{itemize}
	\item Datenstrukturen
	\begin{itemize}[label={\rotatebox[origin=c]{180}{$\Lsh$}}]
		\item ensprechend kodieren.
	\end{itemize}
	\end{itemize}
\end{Korollar*}

\subsection{Registermaschinen}
\begin{figure}[H]\centering
	\begin{tikzpicture}[every node/.style={block}, decoration={brace, amplitude=5pt}]
		\node (A) {};
		\node (A1) [draw=none, below=of A] {$r_1$};
		\node (B) [right=of A] {};
		\node (B1) [draw=none, below=of B] {$r_2$};
		\node (C) [right=of B] {};
		\node (C1) [draw=none, below=of C] {\dots};
		\node (D) [right=of C] {};
		\node (E) [right=of D] {};
		\draw (E.north east)
		-- ++ ( 1cm,0) (E.south east)
		-- ++ ( 1cm,0);
	\end{tikzpicture}
	\caption{Registermaschine}
\end{figure}
\vspace{-1em}
Register:
\begin{itemize}
	\item unendlich viele Register
	\item jedes enthält natürliche Zahl
\end{itemize}
\framebox{pc} Befehlszähler $\in\N$\\
Programm = \underline{endliche} Tabelle von Instruktionen.

Folgende Instruktionen gibt es:
\begin{itemize}
	\item $\inc(i)$ \ Inkrementiere $r_i$
	\item $\dec(i)$ \ Dekrementiere $r_i$
	\item if0($i$) goto $j$ \ Falls $r_i=0$ springe nach Instruktion $j$
\end{itemize}
\begin{Bsp}
	Addiere $r_1$ und $r_2$ Ergebnis in Register $r_1$.
	
	0: if0(2) goto 4\\
	1: dec(2)\\
	2: inc(1)\\
	3: if0(3) goto 0
	
	Terminert, falls $pc$ nicht mehr auf Instruktion zeigt.
\end{Bsp}
\begin{Def}[\acf{RM}]
	Die \underline{\ac{RM}-Instruktion} sind gegeben durch
	\[ \text{Instr} = \{\inc, \dec\}\x\N\ \cup\ \{\text{if0}\}\x\N\x\N \]
	Ein \underline{\ac{RM}-Programm} ist eine Abbildung\\
	$P:[0\dots m]\->\Instr$
\end{Def}
\begin{Def}[Semantik der \ac{RM} bezüglich eines Programms $P$]\
	
	Der Zustandsraum der \ac{RM} ist
	\begin{align*}
		Z ={} &\N\x\N^\N\\
		&pc \quad \text{Registerinhalte entspr. }\N\->\N
	\end{align*}
	Die Semantik eines \ac{RM}-Befehls:
	\begin{align*}
		\llbracket\cdot\rrbracket &: \Instr \-> Z\-> Z\\
		\llbracket\inc(i)\rrbracket(pc,r) &= (pc+1,r[i\mapsto r(i)+1])\\
		\llbracket\dec(i)\rrbracket(pc,r) &= (pc+1,r[i\mapsto r(i)\overset{\text{\normalsize .}}-1])\\
		\llbracket\text{if0($i$) goto }j\rrbracket(pc,r) &= 
		\begin{cases}
			(pc+1, r) &, r(i)\neq 0\\
			(j,r) &, r(i)=0
		\end{cases}
	\end{align*}
	%
	Semantik eines Programms
	\begin{align*}
		\llbracket\cdot\rrbracket &: \text{Prog} \-> Z- \-> Z\\
		\llbracket P \rrbracket(pc,r) &=
		\begin{cases}
			\llbracket P \rrbracket \left(\llbracket \text{ins} \rrbracket(pc,r)\right) &P(pc)=\text{ins}\\
			(pc,r) &P(pc)\text{ nicht definiert}
		\end{cases}
	\end{align*}
\end{Def}
Dabei wird verwendet: 
\begin{enumerate}
	\item 
	Funktionsupdate $f[a \mapsto b]$ 
	\begin{align*}
		f&: A\->B\\
		f[a\mapsto b] &=: f'\\
		\text{mit } f'(x) &=
		\begin{cases}
			b & x=a \\
			f(x) & x\ne a
		\end{cases}
	\end{align*}
	\item abgeschnittene Subtraktion
	\begin{align*}
		m \overset{\text{\normalsize .}}- n &=
		\begin{cases}
			m-n &,m\geq n\\
			0 &, m<n
		\end{cases}
	\end{align*}
\end{enumerate}


\subsection*{\acl{RM}\datenote{06.11.15}}
\[ P:([0\dots m] \-> \text{Instr}) = \text{Prog} \]
Die Ein- und Ausgabefunktionen seien wie folgt definiert:
\begin{align*}
	\text{in}^{(n)} &:\N^n \-> Z & \text{in}^{(n)}(\overrightarrow{x}) &= \left( 0,r_0[i\->x_i] \right) \quad \forall n: r_0(n)=0\\
	\out^{(m)} &: Z \-> \N^m & \out^{(m)}(pc,r) &= \left( r(1),\dots,r(m) \right)\\
	\shortintertext{Die von $P$ berechnete Funktion}
	f_P: &\N^n \->\N^m & f_P &= \out^{(m)} \circ [[P]]\circ \text{in}^{(n)}
\end{align*}

\subsection{Simulation} % 2.5
\begin{Satz}[name={[Simulation von \acs*{RM} durch \acs*{TM}]}] % 2.2
	Jedes \ac{RM}-Programm kann durch eine \ac{TM} simuliert werden.
\end{Satz}
\begin{proof}
	Das \ac{RM}-Programm benutzt nur Register $[0\dots K]$\\
	Verwende eine $K+2$-Band \ac{TM}:
	\begin{itemize}
		\item Band $1\dots k+1$: Registerinhalt binär kodiert
		\item Band 0: Ein-/Ausgabe
		\item Kodiere $pc\ [0\dots m]$ im Zustand der \ac{TM}
	\end{itemize}
	Initialzustand
	\begin{itemize}
		\item Dekodiere Eing. $v_1\#v_2\dots \#{v_n}$\\
		Kopiere $v_i\->\operatorname{Band}_i$
		\item Lösche Band 0
		\item Initialisiere weitere Bänder 0
		\item Zustand $\curvearrowright\Sim(pc=0)$
	\end{itemize}
	\paragraph{Schritt:}
	\begin{itemize}
		\item Falls $P(pc)=\operatorname{Inc}(i)$ wende \ac{TM} aus Beispiel 2.2 %\ref{bsp:2.2}
		auf Band $i+1$ an
		\item Zustand $\curvearrowright \Sim(pc+1)$
		\item Falls $P(pc)=\operatorname{Dec}(i)$
		\begin{itemize}
			\item wende \ac{TM} für Dec auf Band $i+1$ an
			\item Zustand $\curvearrowright \Sim(pc+1)$
	\end{itemize}
	\item Falls $P(pc) =$ If $O(i)$ goto $j$
	\begin{itemize}
		\item Teste ob Band $i+1=0$:\\
		Zustand $\curvearrowright \begin{cases}
		\Sim(pc=j) & ,\text{Band inf 0}\\
		\Sim(pc+1)
		\end{cases}$
	\end{itemize}
	\item Falls $P(pc)$ nicht definiert ist\\
	Kopiere Ausgaben von Band $1-n$ nach Band 0\\
	$\->$ Haltezustand
	\end{itemize}
\end{proof}

\begin{Satz}[name={[Simulation von \acs*{TM} durch \acs*{RM}]}]
	Jede \ac{TM} kann durch ein \ac{RM}-Programm simuliert werden.
\end{Satz}
\begin{proof}
	Wir nummerieren:
	\begin{itemize}
		\item die Zustände $Q=\{q_0,\dots,q_z\}$
		\item die Elemente des Bandalphabets $\Gamma=\{a_0,\dots,a_s\}$ wobei $a_0=\blank$ (Blank hat Nummer 0)
	\end{itemize}
	Nun können wir Zustände/Elemente des Bandalphabets als Zahlen $[0\dots z]/[0\dots s]$ interpretieren.\\[1em]
	\begin{minipage}{.5\textwidth}
		\begin{itemize}
			\item Register 0 rechter Teil des Bandes
			\item Register 1 linker Teil des Bandes
			\item Register $Z$ enthält Zustand
		\end{itemize}
	\end{minipage}\begin{minipage}{.4\textwidth}\centering
	    \captionsetup{type=figure}
		\begin{tikzpicture}[every node/.style={block}]
			\node (b) {$b_1$};
			\node (l) [draw=none, left=of b] {\dots};
			\node (r) [draw=none, right=of b] {\dots};
			
			\draw [{Bar[]}-{Bar[]},semithick] ($(b.north west)+(0,.5em)$) -- ($(b.north east)+(1cm,.5em)$);
			
			\draw [->,semithick,shorten >=.5ex] ($(b.south)-(0,1.5em)$) -- (b.south);
			
			% Open begin and end.
			\draw (b.north west) -- ++(-1cm,0) (b.south west)
			-- ++ (-1cm,0) (b.north east)
			-- ++ ( 1cm,0) (b.south east)
			-- ++ ( 1cm,0);
		\end{tikzpicture}
		\captionof{figure}{\ac{TM} Simulation durch \ac{RM}}
	\end{minipage}
	
	Allgemein wird Konfiguration $c_m\dotsc\ c_1\ q\ b_1\dots b_n$ repräsentiert als
	\begin{align*}
		r_0 &= b_1\cdot (s+1)^0+b_2\cdot(s+1)^1+ \dots +b_n\cdot(s+1)^{n-1}\\
		r_1 &= c_1\cdot (s+1)^0+c_2\cdot(s+1)^1+ \dots +c_m\cdot(s+1)^{m-1}
	\end{align*}
	\paragraph{Initialisierung:}
	\begin{itemize}[label=\textbullet]
		\item Eingabe von $b_1,\dots,b_n$ kodiert als Zahl in $r_0=b_1+b_2(s+1)+\dots+b_n(s+1)^{n-1}$
		\item $r_1=0$
		\item $r_2=0$ Startzustand $q_0$
	\end{itemize}
	Schritt
	\begin{align*}
		r_3 &:= r_0 \mod(s+1) &&\text{erstes Symbol}\\
		r_0 &:= r_1 \div(s+1) &&"(r_0<<1)"\\
		\delta(q,a) &= (q',a',d) &&\text{Hält, falls }q=q',a=a',d=N\\
		\phantom{\delta(}r_2,r_3\\
		r_2 &:= q'\\
		r_3 &:= a'
	\end{align*}
	\begin{itemize}
		\item If $d=N$
		\[ r_0 := r_3+(s+1)r_0 \]
		\item else if $d=R$
		\[ r_1 := r_3+(s+1)r_1 \quad "(r_1>>1)" \]
		\item else if $d=L$
		\begin{align*}
			r_0 &:= r_3+(s+1)r_0\\
			r_3 &:= r_1\mod(s+1)\\
			r_1 &:= r_1\div(s+1)\\
			r_0 &:\phantom{=} r_3+(s+1)r_0
		\end{align*}
	\end{itemize}
\end{proof}

\subsection{Das Gesetz von Church-Turing (Churchsche These)} % 2.6
\begin{Satz}[name={[Intuitiv berechenbare Funktionen sind mit \acs*{TM} berechenbar]}]
	Jede intuitiv berechenbare Funktion ist mit \ac{TM} (in formalem Sinn) berechenbar.
	
	"`Intuitiv berechenbar"' $\equiv$ man kann Algorithmus hinschreiben
	\begin{itemize}
		\item endliche Beschreibung
		\item jeder Schritt effektiv durchführbar
		\item klare Vorschrift
	\end{itemize}
	Status wie Naturgesetz -- nicht beweisbar
	\begin{itemize}[label=\->]
		\item allgemein anerkannt
		\item weitere Versuche Berechenbarkeit zu formulieren, äquivalent zu \ac{TM}en erwiesen.
	\end{itemize}
\end{Satz}
\section[Reguläre Sprachen und endliche Automaten]{Reguläre Sprachen und endliche Automaten\datenote{11.11.15}}
\begin{itemize}
	\item \ac{TM} mächtig
	\item Wir werden sehen, dass Fragen, wie
	\begin{itemize}
		\item $w\in L(\ac{TM})$ ? (Wortproblem),
		\item $L(\ac{TM})\neq\varnothing$ ? (Leerheitsproblem)
\end{itemize}
f"ur Turingmaschinen im Allgemeinen nicht beantwortbar sind. Daher beschneiden wir die F"ahigkeiten der \ac{TM} und betrachten ein sehr einfaches Maschinenmodell, den endlichen Automaten, wo all diese Fragen entscheidbar sind.
D.h. f"ur jede dieser Fragen existiert ein Algorithmus.
\end{itemize}

\subsection{Endliche Automaten}
\underline{Endliches Band} (read-only, jede Zelle enth"alt ein $a_i\in\Sigma$)

\begin{figure}[H]\centering
	\begin{tikzpicture}[every node/.style={block}, decoration={brace, amplitude=5pt}]
		\node (A) {$a_0$};
		\node (B) [right=of A] {$\dots$};
		\node (C) [right=of B] {$a_n$};
	\end{tikzpicture}
	\caption{Endliches Band}
\end{figure}
\vspace{-1em}
\underline{Lesekopf}
\begin{itemize}
	\item ein Zeichen
	\item nur Bewegung nach rechts
	\item kann auch hinter Eingabe zeigen: Eingabeende = Maschine stoppt.
\end{itemize}
\underline{Eingabeband}
\begin{itemize}
	\item read-only
	\item start mit $w\in\Sigma^*$
\end{itemize}
\framebox{q} Zustand aus endl. Zustandsmenge

\underline{Startzustand:} $q_0$\\
Ein Teil $F$ davon sind \underline{akzeptierende} Zustände.

Transitionsfunktion: Im Zustand $q$ beim Lesen von $a$ gehe nach Zustand $q'$ \medskip\\
\begin{tabular}{|*3{M{l}|}} \hline
	q & a & q'\\ \hline
	&&\\
	&&
\end{tabular}

Der endliche Automat akzeptiert eine Eingabe, falls er in einem akzeptierenden Zustand stoppt.

\begin{Bsp}
	$L=\{w\in\{0,1\}^* \mid w \text{ enthält gerade Anzahl von 0 und gerade Anzahl von 1}\}$

	\begin{minipage}[t]{.4\textwidth}\centering\vspace{0pt}
	    \captionsetup{type=figure}
		\begin{tikzpicture}[circle/.style={
			shape=circle,
			minimum size=0.5cm,
			text=black, draw,
			text width=0.5cm,
			align=center}]
			\node (v1) at (-3.5,3.5) {};
			\node [circle,double] (v2) at (-2.5,3) {$q_{00}$};
			\node [circle] (v3) at (0.5,3) {$q_{01}$};
			\node [circle] (v4) at (0.5,0.5) {$q_{10}$};
			\node [circle] (v5) at (-2.5,0.5) {$q_{11}$};
			\draw [->] (v1) edge (v2);
			\draw [->] (v2) edge [bend left=15] node[auto] {1} (v3);
			\draw [->] (v3) edge [bend left=15] node[auto] {1} (v2);
			\draw [->] (v3) edge [bend left=15] node[auto] {0} (v4);
			\draw [->] (v4) edge [bend left=15] node[auto] {0} (v3);
			\draw [->] (v2) edge [bend left=15] node[auto] {0} (v5);
			\draw [->] (v5) edge [bend left=15] node[auto] {0} (v2);
			\draw [->] (v5) edge [bend left=15] node[auto] {1} (v4);
			\draw [->] (v4) edge [bend left=15] node[auto] {1} (v5);
		\end{tikzpicture}
		\captionof{figure}{Automat zu $L$}
	\end{minipage}\begin{minipage}[t]{.55\textwidth}\vspace{0pt}
	Graphische Darstellung $\hat=$ gerichteter Graph mit Knoten $Q$ und markierten Kanten gemäß $\delta$.\\
	$Q=\{q_{00},q_{01},q_{10},q_{11}\}$\\
	$q_{00}$ einziger akzeptierender Zustand ($F=\{q_{00}\}$)
	\end{minipage}
	
	\begin{tabular}{M{l}|M{l}|M{l}l @{\quad}l}
		& 0 & 1 &\\ \cline{1-3}
		q_{00} & q_{10} & q_{01} && gerade Anzahl von 0 und 1 gesehen\\
		q_{01} & q_{11} & q_{00} && gerade \ruleplaceholder{\widthof{Anzahl von 0}}, ungerade Anzahl von 1 gesehen\\
		q_{10} & q_{00} & q_{11} && ungerade \ruleplaceholder{\widthof{Anzahl von 0}}, gerade \ruleplaceholder{\widthof{Anzahl von 1 gesehen}} \\
		q_{11} & q_{01} & q_{10} && ungerade \ruleplaceholder{\widthof{Anzahl von 0}}, ungerade \ruleplaceholder{\widthof{Anzahl von 1 gesehen}}
	\end{tabular}
\end{Bsp}
\
\begin{Def}[\acs*{DEA}]
	Ein \underline{\acf{DEA}}, (\acsu{DFA} $\hat=$ \acl{DFA}) ist ein 5-Tupel
	\[ M= (Q,\Sigma,\delta,q_0,F) \]
	\begin{itemize}
		\item $Q$ \underline{endliche} Zustandsmenge
		\item $\Sigma$ \underline{endl.} Alphabet
		\item $\delta:Q\x\Sigma\->Q$ Transitionsfunktion
		\item $q_0\in Q$ Startzustand
		\item $F\subseteq Q$ akzeptierende Zustande
	\end{itemize}
\end{Def}

\begin{Bsp}
    Sei $M=(Q,\Sigma,\delta,q_0,F)$ ein DEA.
    \begin{itemize}
    \item Wenn $F=Q$, dann ist $L(M)=\Sigma^*$.
    \item Wenn $F=\emptyset$, dann ist $L(M)=\emptyset$.
    \end{itemize}
\end{Bsp}
\begin{Def}[name={[Erweiterung von $\delta$ auf Worte]}]
	Die Erweiterung von $\delta:Q\x\Sigma\->Q$ auf Worte
	\begin{align*}
		\hat{\delta}: Q\x\Sigma^*&\->Q\ \rlap{ist induktiv def. durch:}\\
		\hat\delta(q,\epsilon) &=q & \hat\delta(q,aw)=\hat\delta(\delta(q,a),w)\\
		(\text{Wortende\ }&\text{erreicht}) & (\text{Rest im Folgezustand verarbeiten})
	\end{align*}
\end{Def}
\begin{Def}[name={[Die durch einen \acs*{DEA} erkannte Sprache]}]
	Sei $M=(Q,\Sigma,\delta,q_0,F)$\\
	Die \underline{von $M$ erkannte Sprache} ist
	\[ L(M) = \{ w\in\Sigma^* \mid \hat\delta(q_0,w)\in F \} \]
	Eine durch einen \ac{DEA} erkannte Sprache heißt \underline{regulär}.
\end{Def}
\begin{Bsp} für reg. \underline{Sprachen}
	\begin{equation*}
		L=\{ w\in\{0,1\}^* \mid w\text{ enthält höchstens eine 1} \} \tag{3.1}\label{bsp:3.1}
	\end{equation*}
	\begin{figure}[H]\centering
		\begin{tikzpicture}
			\node (start) at (-4,1.5) {};
			\node (q0) [circle,double] at (-3,1) {$q_0$};
			\node (q1) [circle,double] at (-0.5,1) {$q_1$};
			\node (q2) [circle] at (2,1) {$q_2$};
			\draw [->] (start) edge (q0);
			\draw [->] (q0) edge[loop above] node {0} (q0);
			\draw [->] (q1) edge[loop above] node {0} (q1);
			\draw [->] (q2) edge[loop above] node {0,1} (q2);
			\draw [->] (q0) edge node [auto] {1} (q1);
			\draw [->] (q1) edge node [auto] {1} (q2);
		\end{tikzpicture}
		\caption{Endlicher Automat, der die reguläre Sprache $L$ erkennt. Sobald mehr als eine $1$ gelesen wurde, wird in den Zustand $q_2$, eine sogenannte \emph{Senke}, gewechselt, von der alle ausgehenden Kanten in sich selbst enden.}
	\end{figure}
	
	Sei $A>=0$ nat. Zahl, $\Sigma=\{0,1,\dots,A\}$
	\begin{equation*}
		L = \{ a_1\dots a_n \mid \exists J\subseteq \{1,\dots,n \},\ \sum_{i\in J} a_i = A \} \subseteq \Sigma^* \tag{3.2}\label{bsp:3.2}
	\end{equation*}
	D.h. gegeben eine Liste von Zahlen $\in\Sigma$.
	Akzeptiere diejenigen Listen, für die eine Teilliste existiert, deren Summe genau $A$ ist.
	\begin{align*}
		Q &=\mathbb{P}\{0,1,\dots,A\} \ \text{Menge der möglichen Summen}\\
		\delta(q,a) &= q \cup \{ x\in \{0,\dots,A\} \mid x-a \in q \} \\
		q_0 &=\{0\} \\
		F &= \{ q\in Q \mid A \in q \}
	\end{align*}
	
	Beispiel f"ur eine nicht-regul"are Sprache.
	\begin{equation*}
		L = \{ 0^n1^n \mid n\in\N \} \tag{3.3}\label{bsp:3.3}
	\end{equation*}
	erkennbar durch \ac{TM} die immer anhält, \underline{aber nicht} von einem \ac{DEA} [\underline{nicht} regulär] akzeptiert werden kann.
	\begin{proof}
		Angenommen $L=L(M)$ für \ac{DEA} $M=(Q,\Sigma,q_0,\delta,F)$
		
		Beobachtung: $\exists m\neq n$, sodass $\hat\delta(q_0,0^m)=\hat\delta(q_0,0^n)=q'$ weil $Q$ endlich.
		\begin{itemize}
			\item Falls nun $\hat\delta(q',1^m)\in F$, dann ist auch $\hat\delta(q_0,0^n1^m)\in F$ und somit $0^n1^m\in L(M)$ mit $n\neq m\ \lightning$
			\item Falls $\hat\delta(q',1^m)\notin F$, dann gilt auch $\hat\delta(q_0,0^m1^m)\notin F$ und somit $0^m1^m \notin L$ $\lightning$
		\end{itemize}
		Also kann $M$ nicht existieren!
	\end{proof}
\end{Bsp}

\datenote{13.11.15}
\underline{Beobachtung:} Auch ein Automat mit lauter erreichbaren Zuständen muss nicht minimal sein

\begin{Bsp}\label{Bsp:3.4}\
	\begin{figure}[H]\centering
		\begin{tikzpicture}
			\node (start) at (-4,1.5) {};
			\node (q0) [circle,double] at (-3,1) {$q_0$};
			\node (q1) [circle,double] at (-0.5,1) {$q_1$};
			\node (q2) [circle] at (2,1) {$q_2$};
			\node (q3) [circle] at (3.5,1) {$q_3$};
			\draw [->] (start) edge (q0);
			\draw [->] (q0) edge[loop above] node {0} (q0);
			\draw [->] (q0) edge node [auto] {1} (q1);
			\draw [->] (q1) edge[loop above] node {0} (q1);
			\draw [->] (q1) edge node [auto] {1} (q2);
			\draw [->] (q2) edge[loop above] node {1} (q2);
			\draw [->] (q2) edge[bend left] node [auto] {0} (q3);
			\draw [->] (q3) edge[bend left] node [auto] {1,0} (q2);
		\end{tikzpicture}\\
		höchstens eine "`1"'
		\caption{Automat zu Bsp. \ref{Bsp:3.4}}
	\end{figure}
	Erkennt die gleiche Sprache wie in \eqref{bsp:3.1}, hat nur erreichbare Zustände, aber mehr Zustände als in \eqref{bsp:3.1}.
	
	\underline{Beobachtung:} $q_2$ und $q_3$ verhalten sich gleich in dem Sinn, dass
	\[ \forall w: \hat\delta(q_2,w) \notin F\text{ und }\hat\delta(q_3,w)\notin F \]
\end{Bsp}
%
%\stepcounter{Def}
%
\begin{Def}[name={[Äquivalenz von \acs*{DFA}-Zuständen]}] %\rlnote{Def.-Num. überprüfen}
	Zwei Zustände $q,p\in Q$ eines \ac{DFA} sind \underline{äquivalent}, geschrieben $p\equiv q$, falls $\forall w\in\Sigma^*$, $\hat\delta(p,w)\in F \<=> \hat\delta(q,w)\in F$
\end{Def}
%\stepcounter{lemma}

\begin{lemma}[name={[$\equiv$ ist Äquivalenzrelation]}] %\rlnote{Satz = Lemma-Nummer: 3.2 statt 3.1?}
	$\equiv$ ist Äquivalenzrelation\\
	\framebox{\parbox{.96\linewidth}{Eine Relation ist genau dann eine Äquivalenzrelation, wenn sie reflexiv, transitiv und symmetrisch
	ist.}}
\end{lemma}
\begin{proof} $\equiv$ ist offensichtlich reflexiv.
	
	$\begin{rcases}
	\text{transitiv}\\
	\text{symmetrisch}
	\end{rcases}$ wegen \<=>
	
	Also $q_2,q_3$ aus \autoref{Bsp:3.4} sind äquivalent.
\end{proof}
\begin{Erinnerung}
Hauptlemma "uber "Aquivalenzrelationen
\begin{align*}
	[q] &= \{p\in Q \mid p \equiv q\} &&[q] \text{ ist Äquivalenzklasse von }q
\end{align*}
"Aquivalenzklassen sind paarweise disjunkt:

F"ur alle $ p,q\in Q$ gilt entweder $[p]=[q]$ oder $[p]\cap[q] = \emptyset$ (folgt aus Transitivität).

D.h. $Q$ wird in disjunkte Äquivalenzklassen aufgeteilt. Anzahl der Äquivalenzklassen ist der \textbf{Index}.
\end{Erinnerung}

\eqref{bsp:3.1}:
\begin{minipage}{.5\textwidth}
    \captionsetup{type=figure}
	\begin{tikzpicture}
		\node (start) at (-4,1.5) {};
		\node (q0) [circle,double] at (-3,1) {$q_0$};
		\node (q1) [circle,double] at (-0.5,1) {$q_1$};
		\node (q2) [circle] at (2,1) {$q_2$};
		\draw [->] (start) edge (q0);
		\draw [->] (q0) edge[loop above] node {0} (q0);
		\draw [->] (q1) edge[loop above] node {0} (q1);
		\draw [->] (q2) edge[loop above] node {0,1} (q2);
		\draw [->] (q0) edge node [auto] {1} (q1);
		\draw [->] (q1) edge node [auto] {1} (q2);
	\end{tikzpicture}
	\captionof{figure}{Automat zu \eqref{bsp:3.1}}
\end{minipage}

Allgemein gilt f"ur alle $p,q\in Q$:
\begin{align}
\label{eqn:delta-wohldefiniert}
	p &\equiv q \=> \forall a\in\Sigma:\ \delta(p,a)\equiv\delta(q,a)
\end{align}
Denn
\begin{align*}
	p \equiv q  
	& \<=> \forall w\in\Sigma^*: \hat\delta(p,w)\in F \<=> \hat\delta(q,w) \in F\\
	& \<=> (p\in F \<=> q \in F) \land \forall a\in \Sigma: \forall w\in\Sigma^*:
	\hat\delta(p,aw)\in F \<=> \hat\delta(q,aw)\in F\\
	&\=>  \forall a\in\Sigma: \forall w\in\Sigma^*: \hat\delta(\delta(p,a),w)\in F \<=> \hat\delta(\delta(q,a),w)\in F\\
	& \<=>\forall a\in\Sigma: \delta(p,a)\equiv\delta(q,a)
\end{align*}
Also können wir äquivalente Zustände zusammenfassen und Transitionen verschmelzen, wie in der folgenden Definition formalisiert.
\begin{Def}[name={[Äquivalenzklassenautomat]}]
	Der Äquivalenzklassenautomat $M'=(Q',\Sigma,\delta',q_0',F')$ zu $M$ ist bestimmt durch:
	\begin{align*}
		Q' &= \{[q]\mid q\in Q\} & \delta'([q],a) &= [\delta(q,a)]\\
		q_0' &= [q_0] & F'&=\{[q]\mid q\in F \}
	\end{align*}
\end{Def}
Dabei ist $[q]=\{p\in Q \mid p\equiv q\}$
\begin{Satz}[name={[Äquivalenzklassenautomat ist wohldefiniert]}]
	Der Äquivalenzklassenautomat ist wohldefiniert und $L(M)=L(M')$.
\end{Satz}
\begin{proof}\ 
	\begin{enumerate}
		\item Wohldefiniert: zu zeigen $\delta'([q],a) =[\delta(q,a)]$ ist nicht abhängig von der Wahl des Repräsentanten $q\in [q]$. Das folgt direkt aus \eqref{eqn:delta-wohldefiniert} gezeigt.
		\item $L(M)=L(M')$ zeige für alle $w\in\Sigma^*$ und alle $q$: $\hat\delta(q,w)\in F \<=> \hat\delta'([q],w)\in F'$\\
		Induktion über $w$:\\
		I.A. $w=\epsilon$: $\hat\delta(q,\epsilon)=q\in F \<=> \hat\delta'([q],\epsilon)=[q]\in F'$ nach Definition.\\
		I.V.: $\forall w'\in\Sigma$, $\forall q\in Q$, $\hat\delta(q,w')\in F \<=> \hat\delta'([q],w')\in F'$\\
		I.S.: \begin{align*}
		\hat\delta(q,aw')\in F &\<==> \hat\delta(\delta(q,a),w')\in F\\ &\xLeftrightarrow{I.V.} \hat\delta'([\delta(q,a)],w')\in F'\\
		&\<==> \hat\delta'(\delta'([q],a),w')\in F'\\
		&\<==> \hat\delta'([q],a w')\in F'
		\end{align*}
		Also f"ur $q=q_0$: $\forall w\in\Sigma^*$, $w\in L(M)\<==> \hat\delta(q_0,w) \in F \<=> \hat\delta'([q_0],w) \in F' \<==> w\in L(M')$
	\end{enumerate}
\end{proof}
\underline{Bem:} Die Konstruktion von $M'$ kann in $O(|Q||\Sigma|\log|Q|)$ passieren.

Warum ist nun der Äquivalenzklassenautomat minimal?\\
\-> Satz von Myhill-Nerode
\begin{Def}[name={[Rechtsinvariante Äquivalenzrelation]}]
	Eine Äquivalenzrelation $R\subseteq\Sigma^*\x\Sigma^*$ heißt rechtsinvariant, falls
	\[ (u,v)\in R \=> \forall w\in\Sigma^*,(u\cdot w,v\cdot w) \in R \]
\end{Def}
%\setcounter{Bsp}{5}
\begin{Bsp} %\rlnote{Bsp.-Num. überprüfen (3.6)}
	Für einen \ac{DEA} $M$ definiere
	\[ R_M = \{(u,v) \mid \hat\delta(q_0,u)=\hat\delta(q_0,v)\} \]
	\begin{itemize}
		\item ist Äquivalenzrelation
		\item ist rechtsinvariant
		\item Anzahl der Äquivalenzklassen(Index von $R_M$)\\
		= Anzahl der "`nützlichen"' Zustände, die von $q_0$ erreichbar sind.
	\end{itemize}
\end{Bsp}
\begin{Bsp}
	Für eine Sprache $L\subseteq \Sigma^*$ definiere die Nerode Relation
	\[ R_L = \{(u,v) \mid \forall w\in\Sigma^*: uw\in L \<=> vw\in L \} \]
	\begin{itemize}
		\item ist Äquivalenzrel.
		\item ist rechtsinvariant. Sei $(u,v)\in R_L$\\
		Zeige $\forall w\in\Sigma^*\ (uw,vw)\in R_L$\\
		Induktion:\\
		I.A. $w=\epsilon$: $ (u\epsilon,v\epsilon)=(u,v)\in R_L$\\
		I.S. $w=w'a$:\\
		I.V.: $(uw', vw') \in R_L $
		\begin{align*}
			(uw',vw')\in R_L 
			&\<=> \forall z\in\Sigma^*, \quad uw'z\in L \<=> vw'z\in L\\
			&\=> \forall a\in \Sigma, z'\in\Sigma^*: uw'az'\in L \<=> vw'az'\in L\\
			& \<=>\ (uw'a,vw'a)\in R_L
		\end{align*}
	\end{itemize}
\end{Bsp}

\begin{alignat*}{2}
	&\begin{rcases}
	L=\{\epsilon\} & [\epsilon]\equiv[\epsilon]\\
	w,v\in\Sigma^*,\ w,v\ne\epsilon & [w]=[v]
	\end{rcases} &\ &\text{Index}= 2\\
	&L = \varnothing, L= \Sigma^* &&\text{Index}= 1
\end{alignat*}

\begin{Satz}[Nerode] % 3.4
	Die folgende Aussagen sind äquivalent:
	\begin{enumerate}
		\item\label{itm:Nerode1} $L\subseteq \Sigma^*$ wird von \ac{DEA} akzeptiert.
		\item\label{itm:Nerode2} $L$ ist Vereinigung von Äquivalenzklassen einer rechtsinvarianten Äquivalenzrel. mit \underline{endl.} Index.
		\item\label{itm:Nerode3} Die Nerode Relation $R_L$ hat \underline{endl.} Index
	\end{enumerate}
\end{Satz}

\begin{proof}
	(1) \=> (2): Sei $M$ ein \ac{DEA} mit $L(M)=\{w \mid \hat\delta(q_0,w)\in F \} = \bigcup\limits_{q\in F}[q]_M$
	wenn $[q]_M =\{ w \mid \hat\delta(q_0,w)=q \}$\\
	Äquivalenzklassen und \#Klassen$\leq|Q|$ endl.
	
	(2) \=> (3): Sei $R$ rechtsinv. Äquivrel. mit endl. Index sodass $L=\bigcup R$-Äquivalenzklassen
	
	Sei $(u,v)\in R$. Zeige $(u,v)\in R_L $.\\
	Wegen $R$ rechtsinv. $\forall w\in\Sigma^*\ (uw,vw)\in R$
	\begin{alignat*}{2}
		&&&\forall w\in\Sigma^*: (uw,vw)\in R\\
		&\<=>&\quad&\forall w\in\Sigma^*\quad uw\text{ und $vw$ in gleicher $R$-Klasse}\\
		&\=>&&\forall w\in\Sigma^* \quad uw\in L\<=>vw\in L\\
		&\<=>&& (u,v)\in R_L\\
		&\=>&&\text{\#Klassen($R_L$)$\leq$\#Klassen($R$)}<\infty
	\end{alignat*}

\datenote{18.11.15}

    (3) \=> (1) Gegeben $R_L$\\
		Konstruiere $\A'=(Q,\Sigma,\delta',q_0',F')$
		\begin{alignat*}{3}
			&&Q' &= \{ [w]_{R_L} \mid w\in \Sigma^* \} &\quad& \text{endlich, weil index($R_L$) endl.}\\
			&&\delta'([w],a) &= [wa] && \text{wohldefiniert, da $R_L$ rechtsinvariant}\\
			&&q_0' &= [\epsilon]\\
			&&F' &= \{ [w] \mid w\in L \}\\
			\shortintertext{Zeige $L(\A')=L$, d.h. }
			&\forall w\in \Sigma^* &: w\in L(\A') &\<=> \hat\delta([\epsilon],w)\in F'\\
			&&&\overset{???}{\<=>} [w]\in F' \\
			&&&\<=> w\in L
		\end{alignat*}
		Zeige nun noch, dass $???$ gilt: $\hat\delta([\epsilon], w) = [w]$. 
		Daf"ur m"ussen wir wie folgt verallgemeinern um eine funktionierende Induktionsvoraussetzung zu erhalten.
		\begin{alignat*}{3}
			&&\forall w\in\Sigma^* &:
			\forall v\in\Sigma^* : \hat\delta'([v],w) = [v\cdot w]\\
			\shortintertext{Induktion über $w$}
			&\text{I.A.: }& \epsilon &: \hat\delta'([v],\epsilon) = [v]=[v\cdot\epsilon]\\
			&\text{I.S.: }& \hat\delta'([v],aw') &= \hat\delta'(\delta'([v],a),w')\\
			&&&= \hat\delta'([v\cdot a],w')\\
			&&&\overset{\mathrlap{\text{I.V.}}}{=} [va\cdot w']\\
			&&&= [v\cdot \underbrace{aw'}_{=w}]
		\end{alignat*}
		Das gewünschte Ergebnis \framebox{???} ergibt sich für $v=\epsilon$. \qedhere
\end{proof}
%
\setcounter{Korollar}{4}
\begin{Korollar}
	Der im Beweisschritt (3) \=> (1) konstruierte Automat $\A'$ ist minimaler Automat für $L$.
\end{Korollar}
\begin{proof}
	$L$ regulär. Sei $\A$ \underline{beliebiger \ac{DFA} mit $L=L(\A)$}\\
	$\begin{rcases}
	\A\text{ induziert }R_\A\text{ mit}\\
	|Q|\geq\text{index}(R_\A)
	\end{rcases}$ $\ref{itm:Nerode1} \overset{vgl.}{\<=>} \ref{itm:Nerode2}$
	
	In \ref{itm:Nerode2} \=> \ref{itm:Nerode3}: $R_\A\subseteq R_L$,\ index($R_\A$) $\geq$ index($R_L$)\\
	In \ref{itm:Nerode3} \=> \ref{itm:Nerode1} $A'$ mit $|Q'|=\text{index}(R_L) \leq \text{index}(R_\A)\leq |Q|$
	
	Minimalität von $A'$ folgt aus freier Wahl von $\A$
\end{proof}

Angenommen $\exists\,\A$ mit $L=L(\A)$ und $|\A|<|\A'|$\\
Nach Folgerung gilt aber
\[ |\A'| \leq |\A|\ \lightning \]

\begin{minipage}{\textwidth} % prevent pagebreak
%
%\setcounter{subsection}{2}
\subsection{\acf{PL} für reguläre Sprachen} %\rlnote{subsection \# (3.3)?}
Suche: Notwendiges Kriterium für Regularität
	\begin{Bsp*}
		$L = \{ w\in \{0,1\}^* \mid \text{bin}(w)=0\mod 3 \}$ ist regulär, dabei ist "`bin"' die Dekodierung von einem Bitstring in eine nat"urliche Zahl.
	\end{Bsp*}
	\begin{figure}[H]\centering
		\begin{tikzpicture}
			\node (start) at (-4,1.5) {};
			\node (q0) [circle, double] at (-3,1) {$q_0$};
			\node (q1) [circle] at (-0.5,1) {$q_1$};
			\node (q2) [circle] at (2,1) {$q_2$};
			\draw [->] (start) edge (q0);
			\draw [->] (q0) edge [loop above] node {0} (q0);
			\draw [->] (q0) edge node [auto] {1} (q1);
			\draw [->] (q1) edge [bend left] node [auto] {1} (q0);
			\draw [->] (q1) edge node [auto] {0} (q2);
			\draw [->] (q2) edge [bend left] node [auto] {0} (q1);
			\draw [->] (q2) edge [loop above] node {1} (q2);
		\end{tikzpicture}
		\caption{Automat: \acl{PL}}
	\end{figure}
\end{minipage}
%
\begin{align*}
	1 \underline{0 0} 1 &\in L\\
	\hat\delta(\overset{\rlap{Schleife}}{q_1,00}) &= q_1\\
	11 &\in L\\
	100001 &\in L\\
	\forall i\in\N: 1(00)^i1 &\in L
\end{align*}

\begin{lemma}[Pumping Lemma]\label{lem:pumping}
	Sei $L$ regulär:
	\begin{alignat*}{2}
		&\exists n\in\N,\ n>0\quad \forall z\in L,\ |z|\geq n:\\
		&\exists u,v,w\in\Sigma^* :\\
		&z = uvw,\ |uv| \leq n,\ |v| \geq 1\\
		\text{sodass }& \forall i\in\N:\ uv^iw\in L
	\end{alignat*}
\end{lemma}
\vspace{-1em}
\begin{proof}
	Sei $\A=(Q,\Sigma,\delta,q_0,F)$ ein \ac{DFA} für $L$.\\
	Wähle $n=|Q|$ und $z\in L$ mit $|z|\geq n$.
	
	Beim Erkunden von $z$ durchläuft $\A\ \underbrace{|z|+1}_{\geq n+1}$ Zustände.\\
	\-> $\exists\, q$, das mehrmals besucht wird.
	
	Wähle das $q$, dessen zweiter Besuch zuerst passiert.
	\begin{alignat*}{3}
		\text{D.h.}:&\quad& \hat\delta(q_0,u)&=q &\qquad& u\text{ Präfix von }z\\
		\exists v:&& \hat\delta(q,v)&=q && uv\text{ Präfix von }z\\
		\exists w:&& \hat\delta(q,w)&\in F && uvw=z\\
		&& |v| &\geq 1\\
		&& |uv| &\leq n && \text{ergibt sich aus Wahl von }q
	\end{alignat*}
	\begin{alignat*}{2}
		\text{jetzt:}\quad &\hat\delta(q_0,uv^iw) &\quad& i\in\N\\
		&= \hat\delta(q,v^iw)\\
		&= \hat\delta(q,w) && \text{denn }\forall i: \hat\delta(q,v^i)=q\\
		&\in F \tag*{\qedhere}
	\end{alignat*}
\end{proof}
%
\begin{Bsp*}
	$L=\{0^n1^n \mid n\in\N\}$ ist nicht regulär.\\
	Sei $n$ die Konstante aus dem \ac{PL}.
	
	Wähle $z=0^n1^n \quad |z|=2n\geq n$\\
	Laut PL existieren $u$, $v$, $w$, sodass $z=uvw$ mit $|v|\geq 1, |uv|\leq n$ und $\forall i \in \N$ $uv^iw \in L$. Nach Wahl von $z$ gilt nun
	{\setlength\belowdisplayskip{0pt}
	\begin{alignat*}{2}
		uv &= 0^m &\quad m&\leq n\\
		v &= 0^k & k&\geq 1\\
		\text{Betr. } uv^2w &= 0^{n+k}1^n \notin L\\
		\=> L\text{ nicht regulär}
	\end{alignat*}
	\begin{gather*}
		\underbrace{0\ \dots\dots\ 0}_{n} \ \underbrace{1\ \dots\dots\ 1}_{n}\\
		|\!\ruleplaceholder[u]{\widthof{0\ \dots }} \!|\! \ruleplaceholder[v]{\widthof{\dots 0}}\!|% 
		\ruleplaceholder[w]{\widthof{\ \ \ $1\ \dots\dots\ 1$}}\!|
	\end{gather*}
	}
\end{Bsp*}
\begin{Bsp*}
$L_2 = \{0^p \mid p\text{ ist Primzahl}\}$ ist nicht regulär.

Sei $n$ Konst. aus dem \ac{PL}, $p$ Primzahl mit $p \geq n$.\\
Wähle $z=0^p \in L_2$
\begin{align*}
	\ac{PL}:\ &z=uvw \text{ mit } |uv|\leq n &&,|v| \geq 1\\
	&\curvearrowright |z|= p=a+b &&, a = |uw| \quad, b= |v|\\
	&\curvearrowright |uv^iw| = a + ib &&, \text{w"ahle }i=p+1\\
	&\curvearrowright |uv^{p+1}w| = a + (p+1)b & =& a + pb + b = p+pb \text{ keine Primzahl} \\
	\text{Also } &uv^{p+1}w \notin L_2\\
	&\curvearrowright L_2\text{ nicht regulär.}
\end{align*}
\end{Bsp*}

\subsection[\acf{NEA}]{\acf{NEA}\datenote{25.11.15}}
\begin{Bsp*} Mustererkennung\\
	kommt $\underset{\text{konsistent}}{\text{\underline{ein String}}}$ in einem anderen vor?
	
	Gegeben: festes Wort $w$.\\
	Gesucht: Sprache aller Worte, in denen $w$ als Teilwort vorkommt.
	\begin{align*}
		L &= \{ v\in\Sigma^* \mid \exists u,x\in\Sigma^*, v=uwx \}\\
		\Sigma &= \{a,b,c\}\\
		& \text{konkretes Beispiel:}\\
		w &= abac
	\end{align*}
	\begin{figure}[tp]
	\centering
		\begin{tikzpicture}[>=stealth, shorten >=1pt,
				node distance=2cm, on grid, initial text=,
				every state/.style={minimum size=0pt,inner sep=0pt}
			]
			\node[state,initial] (q0) {};
			\node[state] (q1) [right of=q0] {};
			\node[state] (q2) [right of=q1] {};
			\node[state] (q3) [right of=q2] {};
			\node[state,accepting] (q4) [right of=q3] {};
			\path[->]
				(q0) edge [loop above]    node [auto]  {$b,c$}    ()
				     edge                 node [auto]  {$a$}      (q1)
				(q1) edge [loop above]    node [auto]  {$a$}      ()
				     edge [bend left]     node [auto]  {$c$}      (q0)
				     edge                 node [auto]  {$b$}      (q2)
				(q2) edge [bend left=50]  node [auto]  {$b,c$}    (q0)
				     edge                 node [auto]  {$a$}      (q3)
				(q3) edge                 node [auto]  {$c$}      (q4)
				     edge [bend right=70] node [above] {$a$}      (q1)
				     edge [bend right=40] node [above] {$b$}      (q2)
				(q4) edge [loop right]    node [auto]  {$\Sigma$} ()
			;
		\end{tikzpicture}
	\caption{DFA für $L$}
	\label{fig:dfa-teilwort}
	\end{figure}
	\hyperref[fig:dfa-teilwort]{Abbildung~\ref*{fig:dfa-teilwort}} enthält einen \ac{DFA} für die Sprache $L$. Beobachtung: nicht-trivial zu konstruieren.
	\begin{figure}[tp]\centering
		\begin{tikzpicture}[>=stealth,shorten >=1pt,
				node distance=2cm,on grid,
				initial text=
			]
			\node[state,initial] (q0) {$q_0$};
			\node[state] (q1) [right of=q0] {$q_1$};
			\node[state] (q2) [right of=q1] {$q_2$};
			\node[state] (q3) [right of=q2] {$q_3$};
			\node[state,accepting] (q4) [right of=q3] {$q_4$};
			\path[->]
				(q0) edge [loop above] node [auto] {$\Sigma$} ()
				     edge              node [auto] {$a$} (q1)
				(q1) edge              node [auto] {$b$} (q2)
				(q2) edge              node [auto] {$a$} (q3)
				(q3) edge              node [auto] {$c$} (q4)
				(q4) edge [loop above] node [auto] {$\Sigma$} ()
			;
		\end{tikzpicture}
		\caption{Bsp.: Mustererkennung}
		\label{fig:nfa-teilwort}
	\end{figure}
	
	\hyperref[fig:nfa-teilwort]{Abbildung~\ref*{fig:nfa-teilwort}} enthält einen nicht-deterministischen endlichen Automat für die Sprache $L$. Idee: Ein Wort $w$ wird akzeptiert, falls es einen mit $w$ markierten Pfad von $q_0$ zu einen akzeptierenden Zustand gibt.
	\begin{figure}[tp]
	\centering
		\begin{tikzpicture}[>=stealth,shorten >=1pt,
				node distance=2cm,on grid
			]
			\node (q0) {$\{0\}$};
			\node (q1) [right of=q0] {$\{0,1\}$};
			\node (q2) [right of=q1] {$\{0,2\}$};
			\node (q3) [right of=q2] {$\{0,1,3\}$};
			\path[->]
				(q0) edge [loop below] node [auto] {$b,c$} ()
				     edge              node [auto] {$a$} (q1)
				(q1) edge [loop below] node [auto] {$a$} ()
				(q1) edge              node [auto] {$b$} (q2)
				(q2) edge              node [auto] {$a$} (q3)
			;
		\end{tikzpicture}
	\caption{Potenzmengenkonstruktion auf dem NFA}
	\label{fig:nfa-teilwort-powerset}
	\end{figure}
	
	\hyperref[fig:nfa-teilwort-powerset]{Abbildung~\ref{fig:nfa-teilwort-powerset}} zeigt (einen Ausschnitt) aus dem deterministischen Automaten, der schematisch aus dem \acsu{NFA} in \autoref{fig:nfa-teilwort} konstruiert werden kann. Idee: bei Schritt mit Symbol $a$ ist der \ac{NFA} gleichzeitig in allen Zuständen, die durch $a$ von (der Menge der) aktuellen Zustände erreichbar sind.
	
	Variante: erkenne \textbf{Subwort} $w=a_1,\dots,a_n$
	\[ L' = \{ v\in\Sigma^* \mid \exists x_0,\dots,x_n\in\Sigma^*, v=x_0a_1x_1a_2\dots a_nx_n \} \]
	Nicht det. Automat für $L'$ mit $(w=abac)$ ist sehr einfach. Der entsprechende deterministische Automat ist deutlich komplizierter. (selbst)
	\begin{figure}[H]\centering
		\begin{tikzpicture}[>=stealth, shorten >=1pt,
				node distance=2cm, on grid, initial text=,
				every state/.style={minimum size=0pt,inner sep=0pt}
			]
			\node[state,initial] (q0) {};
			\node[state] (q1) [right of=q0] {};
			\node[state] (q2) [right of=q1] {};
			\node[state] (q3) [right of=q2] {};
			\node[state,accepting] (q4) [right of=q3] {};
			\path[->]
				(q0) edge [loop above] node [auto] {$\Sigma$} ()
				     edge              node [auto] {$a$}      (q1)
				(q1) edge [loop below] node [auto] {$\Sigma$} ()
				     edge              node [auto] {$b$}      (q2)
				(q2) edge [loop below] node [auto] {$\Sigma$} ()
				     edge              node [auto] {$a$}      (q3)
				(q3) edge [loop below] node [auto] {$\Sigma$} ()
				     edge              node [auto] {$c$}      (q4)
				(q4) edge [loop right] node [auto] {$\Sigma$} ()
			;
		\end{tikzpicture}
		\caption{Nichtdet. Automat für $L'$}
	\end{figure}
	
	Weiteres Beispiel, bei dem der deterministische Automat beweisbar exponentiell größer ist.
	\begin{align*}
		L_n &= \{ w\in\{0,1^* \mid \text{das $n$-letzte Symbol von $w$ ist 1} \}
	\end{align*}
	\begin{figure}[H]\centering
		\begin{tikzpicture}[>=stealth, shorten >=1pt,
				node distance=1.5cm, on grid, initial text=,
				every state/.style={minimum size=0pt,inner sep=0pt}
			]
			\node[state,initial] (q0) {};
			\node[state] (q1) [right of=q0] {};
			\node        (q2) [right of=q1] {\dots};
			\node[state,accepting] (q3) [right of=q2] {};
			\path[->]
				(q0) edge [loop above] node [auto] {$\Sigma$} ()
				     edge              node [auto] {1}        (q1)
				(q1) edge              node [auto] {$\Sigma$} (q2)
				(q2) edge              node [auto] {$\Sigma$} (q3)
			;
			\draw [thick, decoration={brace, mirror, raise=.3cm, amplitude=10pt}, decorate]
			    (q1.west) -- (q3.east)
			    node [pos=0.5,anchor=north,yshift=-0.65cm] {n};
		\end{tikzpicture}
		\caption{Nichtdet. Automat für $L_n$}
	\end{figure}
	
	deterministischer Automat für $L_n$ hat $\sim 2^n$ Zustände.
\end{Bsp*}

\begin{Def}[name={[NEA]}]
	Ein \ac{NEA} (\acsu{NFA} = \acl{NFA}) $\A = (Q,\Sigma,\delta,q_0,F)$ mit
	\begin{itemize}
		\item $Q$ endliche Zustandsmenge
		\item $\Sigma$ endl. Alphabet
		\item $\delta:Q\x\Sigma\->\mathcal{P}(Q)$ Transitionsfunktion
		\item $q_0\in Q$ Startzustand
		\item $F\subseteq Q$ akzeptierende Zust"ande
	\end{itemize}
\end{Def}
\begin{Def}[name={[Lauf eines Automaten]}]
	Ein \underline{Lauf des Automaten $\A$ auf $w=a_1\dots a_n$} ist eine Folge $q_0q_1\dots q_n$ mit $q_i\in Q$, $q_0$ Startzustand,\\
	$\forall 1\leq i\leq n,\ q_i\in\delta(q_{i-1},a_i)$\\
	Ein Lauf heißt \underline{akzeptierend}, falls $q_n\in F$.
\end{Def}
\begin{Def}[name={[NFA zu DFA]}]
	$L(\A)=\{ w\in\Sigma^* \mid \exists\text{ akzeptierender Lauf von $\A$ auf }w \}$
\end{Def}
\begin{Satz}[Rabin]
	Zu jedem \ac{NFA} $\A$ mit $n$ Zuständen gibt es einen \ac{DFA} $\A'$ mit $2^n$ Zuständen, so dass $L(\A)=L(\A')$.
\end{Satz}
\begin{proof}[Potenzmengenkonstruktion]
	Definiere $\A'$ durch
	\begin{align*}
		Q' &= \mathcal{P}(Q)\\
		\delta'(q',a) &= \bigcup_{q\in q'} \delta(q,a)\\
		q_0' &= \{q_0\}\\
		F' &= \{ q'\in Q' \mid q'\cap F\neq \varnothing \}
	\end{align*}
	Zeige $L(\A)=L(\A')$
	\begin{align*}
		\text{Es gilt } w\in L(\A') &\<=> \hat\delta'(q_0',w)\in F'\\
		&\<=> \hat\delta'(q_0',w)\cap F\neq \varnothing\\
		w\in L(\A) &\<=> \exists \text{ akzeptierender Lauf von $\A$ auf $w$}.
		\intertext{Zeige $\forall w\in\Sigma^*$}
		\forall q'\in Q' &\phantom{\<=>} \hat\delta'(q',w)\cap F\neq \varnothing\\
		&\<=> \exists\text{ akzeptierender Lauf von $\A$ \underline{ab $q'$} auf }w\\
		&\<=> \exists \underbrace{p_0p_1\dots p_n}_{\text{Lauf}}\in Q \quad p_0=q',\ p_n\in F,\ n=|w|
	\end{align*}
	Induktion nach $w$
	\begin{description}
	\item[I.A.]
		\begin{alignat*}{2}
			\epsilon:\\
			&&\forall q'\in Q'\ \hat\delta(&q',\epsilon)\cap F\neq \varnothing\\
			\<=>&& &q'\cap F\neq \varnothing
		\end{alignat*}
		Wähle einen beliebigen Zustand $p_0=p_n\in q'\cap F$
	\item[I.S.]
	\begin{align*}
		aw'&\\
		&\hat\delta'(q',aw')\cap F\neq \varnothing\\
		\<=>\quad& \hat\delta'(\underbrace{\delta'(q',a)}_{\in Q'},w')\cap F\neq\varnothing\\
		\xLeftrightarrow{\text{I.V.}}\quad& \exists \text{Lauf\ } p_1\dots p_{n+1}\in Q : p_0\in\hat\delta'(q', a),\ p_{n+1}\in F
	\end{align*}
	Suche $p_0\in q'$ mit $p_1\in\delta(p_0,a)$\\
	existiert, denn
	\begin{align*}
		& p_1\in \delta'(q', a) =  \bigcup_{q\in q'} \delta(q,a)\\
		\<=>\quad & \exists p_0\in q' : \delta(p_0,a) \ni p_1\\
		\shortintertext{Gesuchter Lauf ist}
		p_0p_1\dots p_{n+1}
	\end{align*}
	\end{description}
\end{proof}
Also: Eine Sprache $L$ ist regulär, falls
\begin{itemize}
\item $L = L(\A)$ für einen \ac{DFA}\\
	oder
\item $L = L(\A)$ für \ac{NFA}
\end{itemize}

\subsection{Abschlusseigenschaften}
\begin{Def}[name={[Abgeschlossenheit von $\mathcal{L}$]}]
	Eine Menge $\mathcal{L}\subseteq \mathcal{P}(\Sigma^*)$ von Sprachen heißt \emph{abgeschlossen} unter Operation \\
	$f:\mathcal{P}(\Sigma^*)^n \-> \mathcal{P}(\Sigma^*)$ falls $\forall L_1,\dots, L_n\in \mathcal{L} : f(L_1,\dots, L_n)\in \mathcal{L}$.
\end{Def}
\begin{Satz}[name={[Abgeschlossenheit von $REG$]}]\label{satz:3.8}
	Die Menge $REG$ der regulären Sprachen ist abgeschlossen unter $\cup$ (Vereinigung), $\cap$ (Durchschnitt), $\overline{\phantom{X}}$ (Komplement), Produkt (Konkatenation), Stern.
\end{Satz}
\begin{proof}
	Sei $\A_i:=(Q_i,\Sigma,\delta_i,q_{0i},F_i)\quad i=1,2$ \acs{NFA}s
	\begin{itemize}
	\item $\cup:$ Def $\A$ durch (vgl.\ Abb.~\ref{fig:reg-closure-union})
		\begin{align*}
			Q &= Q_1\overset.\cup Q_2\overset.\cup\{q_0\}\\
			\delta(q,a) &=
                \begin{cases}
                    \delta_1(q,a) & q\in Q_1\\
                    \delta_2(q,a) & q\in Q_2\\
                    \delta_1(q,a)\cup\delta_2(q,a) & q=q_0
    			\end{cases}\\
			F &= F_1\overset.\cup F_2\overset.\cup (q_{01}\in F_1\lor q_{02}\in F_2) \rhd \{q_0\}
		\end{align*}
		Zeige $L(\A)=L(\A_1)\cup L(\A_2)$ (selbst: betrachte die Läufe).
	\item $\cap:$ 
	Annahme: $A_1$ und $\A_1$ deterministisch.\\
	Def. $\A$ durch den \emph{Produktautomaten}\\
		\begin{figure}[tp]\centering
    		\begin{tikzpicture}[>=stealth]
                \node (q0) at (0,0) {$q_0$};
                
                \node (q01) at (1.5,1.5) {$q_{01}$};
                \node (y) at (2.5,2) {$\bullet$};
                \node (x1) at (2.5,1) {$\bullet$};
                \node (qf1) at (4,1.5) {$q_{f_1}$};
                
                \node (q02) at (1.5,-1.5) {$q_{02}$};
                \node (x2) at (2.5,-1) {$\bullet$};
                \node (l) at (2.5,-2) {$\bullet$};
                \node (qf2) at (4,-1.5) {$q_{f_2}$};
                
                \path [->] (q0) edge [bend left=45] (y)
                                edge (x1)
                                edge (x2)
                                edge [bend right=45] (l)
                          (q01) edge (y)
                                edge (x1)
                          (q02) edge (x2)
                                edge (l)
%                ;
%                \path (q01) edge [bend left=80] (qf1)
%                                    edge [bend right=80] (qf1)
                ;\draw (2.75,-1.5) ellipse (1.75 and 1.35);
                \node at (5,2) {$A_1$}
                ;
                \draw (2.75,1.5) ellipse (1.75 and 1.35);
                \node at (5,-1) {$A_2$};
            \end{tikzpicture}
            \caption{\acs{NFA} f"ur Vereinigung}
            \label{fig:reg-closure-union}
        \end{figure}
		\underline{DFA :}
		\begin{align*}
			Q &= Q_1\x Q_2\\
			\delta((q_1,q_2),a) &= (\delta_1(q_1,a),\delta_2(q_2,a))\\
			q_0 &= (q_{01},q_{02})\\
			F &= F_1\x F_2\\
			\text{Zeige }L(\A) &= &L(\A_1)\cap L(\A_2)
		\end{align*}
	\item Komplement: Ang. $\A_1$ ist \ac{DFA}.\\
		Ersetze $F_1$ durch $Q_1\setminus F_1$.
%
\datenote{27.11.15}
%
	\item Produkt: Seien $L_1$, $L_2$ regulär.\\
		Zeige $L_1\cdot L_2$ regulär.
		\begin{align*}
			Q &= Q_1 \overset.\cup Q_2\\
			\delta(q,a) &=
				\begin{cases}
					\delta_1(q,a) & q\in Q_1\setminus F_1\\
					\delta_1(q,a)\cup\delta_2(q_{02},a) & q\in F_1\\
					\delta_2(q,a) & q\in Q_2
				\end{cases}\\
		q_0& = q_{01}\\
		F &= F_2\cup(q_{02}\in F_2) \rhd F_1
		\end{align*}
		Zeige $L(\A) = L(\A_1)\cdot L(A_2)$
	\item Stern
	\begin{align*}
		Q &= Q_1\overset.\cup \{q_0\}\\
		\delta(q,a) &=
			\begin{cases}
				\delta_1(q,a) & q\in Q_1\setminus F_1\\
				\delta_1(q,a)\cup\delta_1(q_{01},a) & q\in F_1\\
				\delta_1(q_{01},a) & q=q_0
			\end{cases}\\
		F &= \{q_0\}\cup F_1\\
		\dots\ L(\A) &= L(\A_1)^*
	\end{align*}
	\end{itemize}
\end{proof}
%
\subsection{Reguläre Ausdrücke}
\begin{Def}[name={[RE($\Sigma$)]}]
	Die Menge $RE(\Sigma)$ der \emph{regulären Ausdrücke über $\Sigma$} ist induktiv definiert durch:
	\begin{itemize}
	\item $\0\in RE(\Sigma)$
	\item $\1\in RE(\Sigma)$
	\item $\forall a\in\Sigma$, $a\in RE(\Sigma)$
	\item falls $r,s\in RE(\Sigma)$
		\begin{itemize}[label=\textbullet]
		\item $r+s\in RE(\Sigma)$
		\item $r\cdot s\in RE(\Sigma)$
		\item $r^*\in RE(\Sigma)$
		\end{itemize}
	\end{itemize}
\end{Def}
\begin{Def}[name={[Semantik eines regulären Ausdrucks]}]
	Die Semantik eines regulären Ausdrucks
	\begin{align*}
		\llbracket\cdot \rrbracket &: RE(\Sigma) \-> \mathcal{P}(\Sigma^*)\text{ ist induktiv def. durch}\\
		\llbracket \0 \rrbracket &= \varnothing\\
		\llbracket \1 \rrbracket &= \{\epsilon\}\\
		\llbracket a \rrbracket &= \{a\} \quad a\in\Sigma\\
		\llbracket r+s \rrbracket &= \llbracket r\rrbracket \cup \llbracket s\rrbracket\\
		\llbracket r\cdot s \rrbracket &= \llbracket r\rrbracket \cdot \llbracket s\rrbracket\\
		\llbracket r^* \rrbracket &= \llbracket r\rrbracket^* \qedhere
	\end{align*}
\end{Def}
$+,\cdot ,*$ reguläre Operatoren.\\
%
\begin{minipage}[t]{.5\textwidth}
    \begin{Bsp*} Mustererkennung
    	\begin{itemize}
    	\item $\underset{\vphantom{\big(}\mathrlap{\hspace{-4pt}\rotatebox[origin=c]{90}{$\Rsh$}\ =(a_1+a_2+\dots) \text{ alle $a_i\in\Sigma$ aufgez"ahlt}}}{\Sigma^*}\ abac\ \Sigma^*$
    	\item $n$-letztes Symbol = 1 ($\Sigma=\{0,1\}$)
    	\begin{gather*}
    	(0+1)^*1\underbrace{(0+1)\dots(0+1)}_{n-1}\\
    	\xcancel{0^n1^n}\notin RE(\Sigma)
    	\end{gather*}
    	\item Binärdarstellung modulo $3=0$
    	\[ (0+1(01^*0)^*1)^* \]
    	\end{itemize}
    \end{Bsp*}
\end{minipage}%
\begin{minipage}[t]{.5\textwidth}
    \vspace{0pt}
    \captionsetup{type=figure}
    \begin{tikzpicture}[>=stealth, shorten >=1pt, on grid, node distance=2cm, initial text=]
        \node[state, initial, accepting] (q0) {$q_0$};
        \node[state] (q1) [right=of q0] {$q_1$};
        \node[state] (q2) [right=of q1] {$q_2$};
        \path [->]
            (q0) edge[loop below] node[auto] {0} ()
                 edge[bend left]  node[auto] {1} (q1)
            (q1) edge[bend left]  node[auto] {0} (q2)
                 edge[bend left]  node[auto] {1} (q0)
            (q2) edge[loop right] node[auto] {1} ()
                 edge[bend left]  node[auto] {0} (q1)
        ;
        
        \draw [->,decorate,decoration=snake] ($(q1.south) - (0,.5cm)$) -- ++(0,-1cm);
        
        \node [state, initial, accepting] (q0) [below=3cm of q0] {$q_0$};
        \node [state] (q1) [right=of q0] {$q_1$};
        \path [->]
            (q0) edge[loop below] node[auto] {0} ()
                 edge[bend left]  node[auto] {1} (q1)
            (q1) edge[loop right] node[auto] {$01^*0$} ()
                 edge[bend left]  node[auto] {1} (q0)
        ;
        
        \node [state, initial, accepting] (q0) [below=2.5cm of q0] {$q_0$};
        \path [->]
            (q0) edge[loop below] node[auto] {0} ()
                 edge[loop right] node[auto] {$1(01^*0)^*1$} ()
        ;
    \end{tikzpicture}
    \captionof{figure}{Informell vom Automaten zum regul"aren Ausdruck f"ur mod 3}
\end{minipage}

\begin{Satz}[Kleene]
$L$ ist regulär\\
\<=> $L$ ist Sprache eines regulären Ausdrucks.
\end{Satz}
\begin{proof}\
	\begin{description}[labelwidth=\widthof{\<=},leftmargin=!]
	\item["`\<="'] Sei $L=\llbracket r \rrbracket$ für $r\in RE(\Sigma)$\\
		Zeige per Induktion über $r: \forall r\in RE(\Sigma) : \llbracket r \rrbracket$ regulär.
    	\begin{description}[font=\normalfont]
    		\setlength{\abovedisplayskip}{-1em}
    		\item[I.A.:]
    		\begin{align*}
    			\0 &: \varnothing\text{ ist reg.}\\
    			\1 &: \{\epsilon\}\text{ ist reg.}\\
    			a &: \{a\} \tikz[>=stealth, shorten >=1pt, initial text=,
    					on grid, baseline=-.6ex,
    					every state/.style={minimum size=0pt,inner sep=0pt}
    				]{
    				\node [state,initial] (a) {}; \node [state,accepting] (b) [right=of a] {};
    				\path [->] (a) edge node [auto] {a} (b);
    			}
    			 \quad\acs{NFA}
    		\end{align*}
    		\item[I.S.:]
    		{\settowidth{\dimen1}{reg. nach \autoref{satz:3.8}}
    		\begin{alignat*}{3}
    			r+s &:{}& \llbracket r+s \rrbracket &= \llbracket r \rrbracket \cup \llbracket s \rrbracket &\quad &\text{reg. nach \autoref{satz:3.8}}\\
    			r\cdot s &:& \llbracket r\cdot s \rrbracket &= \llbracket r \rrbracket\cdot \llbracket s \rrbracket && \ruleplaceholder{\dimen1}\\
    			r^* &:& \llbracket r* \rrbracket &= \llbracket r \rrbracket^* && \ruleplaceholder{\dimen1}
    		\end{alignat*}}
		\end{description}
	\item["`\=>"'] Sei $L=L(\A)$ für einen \ac{DFA} $\A=(\underoverbrace{=\{q_0,q_1,\dots,q_n\}}{Q},\Sigma,\delta,q_0,F)$.\\[.5em]
		Def. $L_i=\{ w \mid \hat\delta(q_i,w)\in F \}$\quad(Also $L(\A)=L_0$)
		\begin{alignat*}{2}
			&\text{Betrachte }\quad & \delta(q_i,a) &=q_j\\
			&\quad\curvearrowright & L_i &\supseteq a_i\cdot L_j\\
			&\text{Betrachte} &  & \text{alle Transitionen ab }q_i \\
			&\quad\curvearrowright & L_i &= N(q_i)+\sum_{0 \le j \le n} A_{ij}\cdot L_j\\
			&& A_{ij} &= \sum_{a \in \Sigma, \delta(q_i,a) = q_j} a \\
			&& N(q_i) &= 
				\begin{cases}
					\1 &, q_i\in F\\
					\0 &, q_i\notin F
				\end{cases}
		\end{alignat*}
		Also: $\A$ berechnet die Lösung eines Gleichungssystems
		\[ L_i=N(q_i) + \sum_j A_{ij}L_j \quad\text{mit }\epsilon\notin A_{ij} \]
		Zur L"osung dieses Gleichungssystems verwenden wir das Lemma von Arden:
	\end{description}
\end{proof}
\begin{lemma}[Arden's Lemma]\label{lem:arden}\ \\
	Sei $X=A\cdot X+B$ für $\underset{\vphantom{\big(}\ \mathrlap{\hspace{-4pt}\rotatebox[origin=c]{90}{$\Rsh$}\ \text{Unbekannte}}}{X} ,A,B\subseteq \Sigma^*$\\
	dann ist $X=A^*B$ falls $\epsilon\notin A$.
%	\begin{align*}
%		&& B &\subseteq X\\
%		&& B+AB &\subseteq X\\
%		X&=\epsilon\cdot X+B & AAB &\subseteq X\\
%		&& \forall n : A^nB &\subseteq X
%	\end{align*}
\end{lemma}
\begin{proof}[Fortsetzung]
    Verfahren zur L"osung des Gleichungssystems:
    
	Eliminiere sukzessive die Gleichung für $L_n$ (bis nur noch eine Gleichung f"ur $L_0$ "ubrig ist)
	\begin{enumerate}[label=(\arabic*)]
		\item Falls Gleichung für $L_n$ rekursiv
			\begin{align*}
				\text{dann hat sie die Form}:\quad L_n &= \underbrace{\Big( N(q_n)+\sum_{j\neq n} A_{nj}\cdot L_j \Big)}_{=: B} + \underbrace{A_{nn}}_{=: A \not\ni \epsilon}\cdot L_n\\
				\text{Nach Ardens Lemma}:\quad L_n &= A_{nn}^*\cdot \Big( N(q_n)+\sum_{j\neq n} \underbrace{A_{nj}}_{\not\ni\epsilon} \cdot L_j \Big)\\
				&= A_{nn}^*\cdot N(q_n) + \sum_{j\neq n} \underbrace{A_{nn}^*\cdot A_{nj}}_{\not\ni\epsilon} \cdot L_j
			\end{align*}
			Gleichungssystem der gleichen Form: linear in den $L_j$ mit Koeffizienten, die nicht $\epsilon$ enthalten.
		\item Gleichung für $L_n$ nicht rekursiv:\\
		Setze rechte Seite für $L_n$ in die Gleichungen $L_0, L_1, \dots, L_{n-1}$ ein. \--> Gleichungssystem der gleichen Form.
	\end{enumerate}
    Nach $n+1$ Iterationen erhalten wir ein Gleichungssytem der Form $L_0=r$. \hfill$\bigoplus$
\end{proof}
\begin{proof}(Arden's Lemma)
	\begin{alignat*}{3}
		\text{Sei}&\quad& X&= AX+B\text{ mit }\epsilon\notin A.\\
		\text{Zeige}&& A^*B&\subseteq X.\\
		&& A^*B &= (\1+AA^*)B = B+A(A^*B) \quad\checkmark\\
		\shortintertext{Angenommen $A^*B \subsetneq X$, d.h. $\exists w\in X$ mit $w\notin A^*B$, davon sei $w$ das kürzeste.}
		\exists n\geq 1: && X &= \underbrace{A^nX}_{\ni w} + \underbrace{A^{n-1}B+\dots +AB+B}_{\not\ni w}\\
		\curvearrowright && w &= u_1\dots u_n w'\text{ mit } u_1,\dots,u_n\in A\text{ und } w'\in X\\
		\curvearrowright && |w'| &< |w|\\
		\text{Falls}&& w'&\in A^*B \curvearrowright w\in A^nA^*B\subseteq A^*B \quad \lightning\\
		\text{Also} && w'&\notin A^*B \quad\lightning\text{ gegen Minimalität von }w\\
		\curvearrowright && X&\subseteq A^*B\\
		\-> && X &= A^*B \tag*{\qedhere}
	\end{alignat*}
\end{proof}

\datenote{02.12.15}
\begin{Bsp*} für Konv. \ac{DFA}\-> $RE$
	\begin{figure}[H]\centering
		\begin{tikzpicture}[>=stealth, shorten >=1pt, on grid, node distance=2cm, initial text=]
			\node[state, initial, accepting] (q0) {$q_0$};
			\node[state] (q1) [right=of q0] {$q_1$};
			\node[state] (q2) [right=of q1] {$q_2$};
			\path [->]
			    (q0) edge[loop above] node[auto] {0} ()
			         edge[bend left]  node[auto] {1} (q1)
			    (q1) edge[bend left]  node[auto] {0} (q2)
			         edge[bend left]  node[auto] {1} (q0)
			    (q2) edge[loop right] node[auto] {1} ()
			         edge[bend left]  node[auto] {0} (q1)
			;
		\end{tikzpicture}
		\caption{\ac{DFA} "`modulo 3"'}
	\end{figure}
	lineares Gleichungssystem mit 3 Unbekannten.
	\begin{align*}
		L_0 &= \1 + 0\cdot L_0 + 1\cdot L_1\\
		L_1 &= 1\cdot L_0 + 0\cdot L_2\\
		L_2 &= \underbrace{0\cdot L_1}_B + \underbrace{1}_A\cdot L_2\\
		\shortintertext{\nameref{lem:arden} auf $q_2$:}
		L_2 &= 1^*\cdot 0\cdot L_1\\
		\shortintertext{Einsetzen in $q_1$}
		L_1 &= \underbrace{1\cdot L_0}_B+\underbrace{0\cdot 1^*\cdot 0}_A\cdot L_1\\
		\shortintertext{\nameref{lem:arden} auf $q_1$:}
		L_1 &= (01^*0)^*\cdot 1\cdot L_0\\
		\shortintertext{Einsetzen:}
		L_0 &= \1 + 0\cdot L_0+1\cdot (01^*0)^*\cdot 1\cdot L_0\\
		&= \1+(0+1\cdot (01^*0)^*\cdot 1)\cdot L_0\\
		\shortintertext{\nameref{lem:arden} auf $q_0$:}
		L_0 &= (0+1\cdot (01^*0)^*\cdot 1)^*
	\end{align*}
\end{Bsp*}
%
\subsection{Entscheidungsprobleme}
\begin{Satz}[name={[Wortproblem]}]\label{satz:wortproblem}
	Das Wortproblem ist für reguläre Sprachen entscheidbar.
	
	D.h. Falls $L$ reg. Sprache und $w\in\Sigma^*$, dann ex. Algorithmus, der entscheidet, ob $w\in L$.
\end{Satz}
\begin{proof}
	$L$ sei durch \ac{DFA} gegeben.\\
	Berechnung von $\hat\delta(q_0,w)$ entspricht Durchlauf durch Graph des \ac{DFA} + Test ob erreichter Zustand $\in F$ in Zeit $O(n)\ ,\ n=|w|$.
\end{proof}

\begin{Satz}[name={[Leerheitsproblem]}]\label{satz:leerheitsproblem}
	Das \underline{Leerheitsproblem} ist für reg. Sprachen entscheidbar.
	Falls $L$ reg. Sprache, dann existiert ein Algorithmus, der entscheidet, ob $L=\varnothing$.
\end{Satz}
\begin{proof}
	Sei $\A$ \ac{DFA} für $L$.\\
	Setze Tiefensuche auf den Graphen von $\A$ an. Start bei $q_0$.\\
	Falls die Suche einem akzeptierenden Zustand findet: Nein.\\
	Ansonsten: Ja: $L=\varnothing$\\
	Zeit: $O(|\Sigma||Q|)$
	\rlwarning{Beweis vollständig?}
\end{proof}

\begin{Satz}[name={[Endlichkeitsproblem]}]\label{satz:endlichkeitsproblem}
	Das Endlichkeitsproblem für reg. Sprachen ist entscheidbar.
\end{Satz}
\begin{proof}
	Falls $L$ durch $r\in RE(\Sigma)$ gegeben.\\
	$r$ enthält keinen $^* \=> \llbracket r \rrbracket$ endlich.\\
	Zeit: $O(|r|)$
	
	[Reicht nicht, liefert nur eine Richtung]
	
	Falls $L$ durch \ac{DFA} $\A$ gegeben.
	
	\begin{tikzpicture}[>=stealth]
		\node (q0) {$q_0$};
		\node (q1)  [right=.75cm of q0]{$\cdot$};
		\node (q2) [right=.48cm of q1] {$\times$};
		\node (q3) [below=.4cm of q2, xshift=.4cm, inner sep=0pt] {$\times$};
		\draw[->] (q0) edge (q1)
			(q2) ++(-.1cm,-.3cm) -- (q3);
		\draw[->] (q1.south) arc (-155:155:.5cm);
	\end{tikzpicture}
	
	\paragraph*{Oder:} mit \nameref{lem:pumping}.\\
	Sei $L$ regulär und $n$ die Konstante aus dem \ac{PL}.\\
	$L$ unendlich \<=> $\exists w\in L : n\leq |w| <2n$
	\begin{description}[font=\normalfont,labelwidth=\widthof{"'\<="':},leftmargin=!]
	\item["'\<="':] $w$ erfüllt Voraussetzung des \ac{PL}, also $w=uvx$ mit $|uv|\leq n$ und $|v|\geq 1$.\\
		Nach \ac{PL}: $\forall i\in\N$, $uv^ix\in L$, also $L$ unendlich.
	\item["'\=>"'] \underline{Angenommen} $L$ unendlich, aber $\forall w\in L : |w|<n$ oder $|w|\geq 2 n$\\
	Sei $w\in L$ minimal gewählt, so dass $|w|\geq 2n$.
	
	$w$ erfüllt Voraussetzung vom \ac{PL}, also $w=xyz$ mit $|xy|\leq n$ und $|y|\geq 1$\\
	also $\forall i\in\N: xy^iz\in L$ insbes. $i=0: xz\in L$ mit $|xz|<|w|$.
	
	Zwei Möglichkeiten:
	\begin{enumerate}[label=(\alph*)]
	\item $|xz|\geq 2n\ \lightning$ Minimalität von $w$
	\item $|xz|<2n$
		\begin{align*}
			|xz|+|y| &= |w|\geq 2n\text{ mit } 1\leq|y|\leq n\\
			\curvearrowright |xz| &= |w|-|y|\geq 2n-n=n \quad\lightning\text{ zur Annahme}
		\end{align*}
		Also $\exists w\in L$ mit $n\leq|w|<2n$ \qedhere
	\end{enumerate}
	\end{description}
\end{proof}

\begin{Satz}[name={[Schnittproblem]}]\label{satz:schnittproblem}
	Das \underline{Schnittproblem} ist für REG entscheidbar.\\
	D.h. $L_1,L_2$ reguläre Sprachen. Ist $L_1\cap L_2 = \varnothing$?
\end{Satz}
\begin{proof}
	Nach Satz \ref{satz:schnittproblem} ist $L_1\cap L_2$ regulär. $L_1\cap L_2=\varnothing$ entscheidbar nach \autoref{satz:leerheitsproblem}.
\end{proof}

\begin{Satz}[name={[Äquivalenzproblem]}]\label{satz:äquivalenzproblem}
	Das \underline{Äquivalenzproblem} ist für REG entscheidbar.\\
	D.h. gegeben \ac{DFA}s für $L_1$ und $L_2$, $\A_1$ und $\A_2$
	\[ L_1 = L(\A_1) = L(\A_2) = L_2\ ? \qquad \framebox{Inklusionsproblem}\]
\end{Satz}
\vspace{-2em}
\begin{proof}
	\begin{alignat*}{3}
		L_1\cap \overline{L}_2 &= \varnothing &\quad&\<=>\quad & L_1 &\subseteq L_2\\
		(L_1\cap\overline{L}_2)\cup(L_2\cap \overline{L}_1) &= \varnothing &&\<=> & L_1 &= L_2 \tag*{\qedhere}
	\end{alignat*}
\end{proof}

\begin{Satz}[name={[Inklusionsproblem]}] Äquivalenzproblem \<=> Inklusionsproblem (für REG)
\end{Satz}
\begin{proof}
\begin{itemize}
\item  $=$ entspricht $\subseteq\land\supseteq$
\item $L_1 \subseteq L_2$ genau dann, wenn $L_1 \cup L_2 = L_2$; REG ist abgeschlossen unter Vereinigung
\end{itemize}
\end{proof}


\datenote{04.12.15}
Anwendungsbeispiel f"ur regul"are Sprachen.

$N$ -- liest vom Netz\\
$R$ -- liest lokalen Speicher (ggf. vertrauliche Info)\\
$W$ -- postet auf FB

Programm:

\begin{tabular}{M{l}@{}M{l}@{}M{l}}
	p = N | R | W &| \text{ if }&* \text{ then } p_1\\
	&&\phantom{*}\text{ else }p_2\\
	&|\text{ while }&*\text{ do }p\\
	\mathrlap{
		\begin{rcases}
			\text{while }*\text{ do }N;\\
			R;\\
			\text{if }*\text{ then }N\text{ else }W
		\end{rcases} N^*R\cdot (N+W)
	}
\end{tabular}

Sicherheitspolitik: nach Lesen von lokalem Speicher kein Posten auf FB  $\overline{\Sigma^*R\Sigma^*W\Sigma^*}$

Programm erfüllt Sicherheitspolitik nicht, denn
\[
	\underset{\underset{\displaystyle N^*RW}{\rotatebox[origin=c]{-90}{$\supseteq$}}}{N^*\cdot R(N+W)} \not\subseteq \overline{\Sigma^*R\Sigma^*W\Sigma^*}
\]
\section{Grammatiken und kontextfreie Sprachen}
Wechsel des Standpunkts Spracherkennung \-> Spracherzeugung\\
Werkzeug: Phasenstrukturgrammatiken\\
Ansatz: Neben Alphabet gibt es weitere Symbole (Variable) und ein Regelsystem mit dem Worte, die Variable enthalten, geändert werden können.
\begin{Bsp}
	\begin{align*}
		\Sigma &= \{ a,(,),*,+ \}\\
		N &= \{E\}\\
		\shortintertext{Regeln:}
		\{\ E &\-> a,\\
		E &\-> (E*E),\\
		E &\-> (E+E)\\
		\}\ \phantom{ E }\\
		\shortintertext{Startsymbol $E$}
		E &\=> a\\
		E &\=> (E*E) \=> (a*E)\\
		&\=> (a*(E+E)) \=> \dots \=> (a*(a+a))
	\end{align*}
\end{Bsp}

\begin{Def}[Chomsky]
	Eine \underline{Grammatik} ist ein 4-Tupel $(N,\Sigma,P,S)$
	\begin{itemize}
	\item $N$ ist endliche Menge von \underline{Nichtterminalsymbolen}
	\item $\Sigma$ ist Alphabet von \underline{Terminalsymbolen} (Variablen)
	\item $P\subset (N\cup\Sigma)^*N(N\cup\Sigma)^* \x (N\cup\Sigma)^*$ endliche Menge von Regeln bzw. \underline{Produktionen}
	\item $S\in N$ das \underline{Startsymbol}
	\end{itemize}
\end{Def}
\begin{Def}[name={[Ableitungsrelation]}]
	Die \underline{Ableitungsrelation} zur Grammatik
	%\rlwarning{Ab hier: Symbol $G$ oder $\mathcal{G}$?}
	%\ptwarning{ich m"ochte $\mathcal{G}$ f"ur Grammatiken verwenden}
	% RL: Korrigiert!
	\begin{align*}
		\mathcal{G} &=(N,\Sigma,P,S) \text{ ist Relation}\\
		\==>_{\mathcal{G}}& \subseteq (N\cup\Sigma)^*\x(N\cup\Sigma)^*\text{ mit}\\
		v&\=>_{\mathcal{G}} w\ ,\ 
		\begin{aligned}[t]
			&\text{falls ex. Produktion}\\
			&l \-> r\in P\text{ und }\\
			&v = v_1lv_2\\
			&w=v_1rv_2
		\end{aligned}\\
		\=>^*& \text{ refl. trans. Hülle von }\=>
	\end{align*}
	\underline{Sprache von $\mathcal{G}$ erzeugt}.
	\[ L(\mathcal{G}) = \{w\in\Sigma^* \mid S\xRightarrow*_{\mathcal{G}} w \}\]
	
	Jede Folge $\alpha_0,\alpha_1,\dots,\alpha_n$ mit $\alpha_i \in (N\cup \Sigma)^*$ mit $\alpha_0=S$ und $\alpha_n\in\Sigma^*$ und $\forall 1\leq i\leq n : \alpha_{i-1} \=>_\mathcal{G} \alpha_i$ heißt \underline{Ableitung von $\alpha_n$}
	
	[\,Jedes solche $\alpha_i$ heißt \underline{Satzform von $\mathcal{G}$}\,]
\end{Def}
$n$ ist \underline{Länge der Ableitung}.
\vspace{.5em}
\begin{Bsp} $\mathcal{G}=(N,\Sigma,P,S)$ mit
	\begin{align*}
		N &= \{S,B,C\}\\
		\Sigma &= \{a,b,c\}\\8
		P &= 
		\begin{aligned}[t]
			 \{ &S \-> aSBC,\ S\->aBC,\ CB\-> BC,\ aB\-> ab,\\
			 &bB\-> bb, bC\-> bc, cC\-> cc \} & \quad\text{Startsymbol }S
		\end{aligned}\\
		L(\mathcal{G}) &=  \{ a^nb^nc^n \mid n\geq 1 \}\\
		S &\=> aBC \=> abC \=> abc \qquad S\=>a\underline{S}BC \=> aa\underline{S}BCBC\\
		S &\=> aSBC \xRightarrow{n-2} n \=> a^{n-1}S(BC)^{n-1} \=> a^n(BC)^n = a^nBCBC \dots \=> a^nBBCCBC\\
		&\=> a^nB^nC^n \xRightarrow* a^nb^nc^n
	\end{align*}
\end{Bsp}
Die Chomsky-Hierarchie teilt die Grammatiken in vier Typen unterschiedlicher Mächtigkeit ein.
\begin{Def}[Chomsky Hierarchie]\
	\begin{itemize}
	\item Jede Grammatik ist eine \underline{Typ-0 Grammatik}.
	\item Eine Grammatik ist \underline{Typ-1} oder \underline{kontextsensitiv}, falls alle Regeln expansiv sind, d.h. $\alpha\->\beta\in P$, dann ist $|\alpha|\leq |\beta|$ mit einer Ausnahme: falls $S$ nicht in einer rechten Regelseite auftritt, dann ist $S\->\epsilon$ erlaubt.
	\item Eine Grammatik heißt \underline{Typ-2} oder \underline{kontextfrei}, falls alle Regeln die Form $A\->\alpha$ mit $A\in N$ und $\alpha\in(N\cup\Sigma)^*$ haben.
	\item Eine Grammatik heißt \uline{Typ-3} oder \uline{regulär}, falls alle Regeln die Form
	\begin{alignat*}{2}
		\text{Form}\quad &A\->w &\quad w&\in\Sigma^*\\
		\text{oder}\quad &A\->aB& a&\in\Sigma,\ B\in N
	\end{alignat*}
	\end{itemize}
	Eine Sprache heißt Typ-$i$ Sprache, falls $\exists$ Typ-$i$ Grammatik für sie.
\end{Def}

\begin{Beobachtung}
	Jede Typ $i+1$ Sprache ist nicht Typ-$i$ Sprache.
\end{Beobachtung}
Jede Typ-3 Grammatik ist Typ-2 Grammatik.\\
Jede Typ-2 Grammatik kann in äquivalente $\epsilon$-freie Typ-2 Grammatik transformiert werden. \-> Typ-1 Grammatik. (Vgl.\ Elmination von $\epsilon$-Produktionen)

Ziel: Hierarchie-Satz (Chomsky)\\
Sei $\mathcal{L}_i$ die Menge der Typ-$i$ Sprachen.\\
Es gilt $\mathcal{L}_3 \subsetneq \mathcal{L}_2 \subsetneq\mathcal{L}_1 \subsetneq\mathcal{L}_0$

\begin{Satz}[name={[Typ-3 Sprache ist regulär]}]
	$L$ ist regulär \<=> $L$ ist Type-3 Sprache
\end{Satz}
\begin{proof}
	\begin{description}[font=\normalfont,labelwidth=\widthof{"'\=>"':},leftmargin=!]
	\item["`\=>"'] Sei $\A =(Q,\Sigma,\delta,q_0,F)$ \ac{DFA} für $L$.\\
		Konstruiere Grammatik $\mathcal{G}=(N,\Sigma,P,S)$
		\begin{alignat*}{2}
			N &=Q &\qquad S&=q_0\\
			q&\in Q\ \mathrlap{\forall a: \delta(q,a)=q'\ \curvearrowright\ q\->aq'\in P}\\
			q&\in F & q&\->\epsilon\in P
		\end{alignat*}
		Zeige noch $L(\mathcal{G})=L(\A)$
	\item["`\<="'] Sei $\mathcal{G}=(N,\Sigma,P,S)$ Typ-3 Grammatik
	Def. $\A=(Q,\Sigma,\delta,q_0,F)$ \ac{NFA}
	\begin{alignat*}{2}
		Q &=N\cup\dots\\
		q_0 &=S\\
		A &\->aB\in P & \delta(A,a)&\ni B\\
		A &\->a,\dots a_n &\qquad n=0\=>A &\in F\\
		&\mathclap{\tikz[shorten >=1pt,on grid,node distance=1.5cm, every state/.style={minimum size=0pt,inner sep=0pt}]{
			\node[state,inner sep=3pt] (q0) {$A$};
			\node [above=0pt of q0.north] {$n>0$: $n$ weitere Zust"ande erforderlich, letzter Endzustand};
			\node[state] (q1) [right=of q0] {};
			\node (dots) [right=of q1] {\dots};
			\node[state,accepting] (qa) [right=of dots] {};
			\path[-stealth] (q0) edge node[auto] {$a_1$} (q1)
				(q1)   edge node[auto] {$a_2$} (dots)
				(dots) edge node[auto] {$a_n$} (qa);
		}}\\
		\text{Zeige noch }\quad
		L(\mathcal{G})&=L(\A) \tag*{\qedhere}
	\end{alignat*}
	\end{description}
\end{proof}

\datenote{09.12.15}

\begin{lemma}
	$\mathcal{L}_3 \subsetneq \mathcal{L}_2$
\end{lemma}
\begin{proof}
	Betrachte $L=\{a^nb^n \mid n\in\N \}$\\
	bekannt: $L$ nicht regulär.\\
	Aber $\exists$ Typ-2 Grammatik für $L$:
	\[ \mathcal{G} = (\{S\}, \{a,b\}, \{S\->\epsilon, S\->aSb \}, S) \]
	Zeige (selbst) $L=L(\mathcal{G})$
\end{proof}

\subsection{Kontextfreie Sprachen}
\begin{itemize}
\item Arithmetische Ausdrücke\\
	$E\->a,E\->(E+E),E\->(E*E)$
\item Syntax von Programmiersprachen%\\[\abovedisplayskip]
	\begin{center}
		\begin{tabular}[t]{r@{ }l}
			<Stmt> &\-> <Var> = <Exp>\\
			&| <Stmt>\,;\,<Stmt>\\
			&| if\,(<Exp>) <Stmt>\\
			&\phantom{|} else <Stmt>\\
			&| while\,(<Exp>) <Stmt>
		\end{tabular}
	\end{center}%\\[\belowdisplayskip]
	Hier: Nichtterminal$\hat=$ Wort in spitzen Klammern <Stmt>;
	Terminalsymbol --- alles andere.
\item Palindrome über $\{a,b\}$
	\[ S\-> aSa \mid bSb \mid a \mid b \mid \epsilon \]
\item Sprache der Werte, die gleich viele Nullen wie Einsen haben $=L$
\end{itemize}
\begin{align*}
	\mathcal{G}: S &\-> 1S0S \mid 0S1S \mid \epsilon\\
	L(\mathcal{G}) &\subseteq L \quad \text{klar}\\
	\uline{\text{Def.:}}\quad d:\Sigma^* &\-> \Z\\
	d(\epsilon) &= 0\\
	d(1w) &= d(w) + 1\\
	d(0w) &= d(w) - 1\\
	\text{Es gilt} & \text{(Beweis per Induktion "uber $v$)} \\
	d(v\cdot w) &= d(v)+d(w)\\[.5em]
	L &= \{ w \mid d(w)=0 \}\\
	\text{Zeige:}\quad L &\subseteq L(\mathcal{G})\\
	\forall w\in L : w&\in L(\mathcal{G})
\end{align*}
Induktion über Länge von $w$
\begin{alignat*}{2}
	&\text{Länge }0:\quad& \epsilon &\in L
		\quad,\quad \epsilon \in L(\mathcal{G})
		\quad,\quad S \-> \epsilon\\
	&\text{Länge}>0:\quad& \text{Ang. } 0w&\in L\\
	&&\curvearrowright d(0w) &= 0\\
	&&\curvearrowright d(w) &= 1 \\
	&&\curvearrowright \exists \text{ k"urzestes Suffix $1w_2$ von $w$ mit $d(1w_2)=1$ }\\
	&&\curvearrowright \exists w_1 : w=w_1 1 w_2\\
	&&\curvearrowright d(w_1)=0 \wedge d(w_2)=0
\end{alignat*}
Per Induktion sind sowohl $S \=>^*w_1$ als auch $S\=>^*w_2$; die erste Produktion ist
$S \to 0S1S$.

Der Fall $1w$ geht analog.

%Abk.: \aclu{CFL}: \acs{CFL}\\
%\phantom{Abk.:} \aclu{CFG}: \acs{CFG}

\begin{Def}[name={[Ableitungsbaum]}] Sei $\mathcal{G}$ \ac{CFG}.\\
	Ein \uline{Ableitungsbaum} zu $w\in L(\mathcal{G})$ ist ein geordneter, markierter Baum $B$.
	\begin{itemize}
	\item Die Wurzel von $B$ ist mit $S$ markiert.
	\item Die inneren Knoten von $B$ sind mit Nichtterminalen markiert.
	\item Jedes Blatt ist mit $x\in\Sigma\cup\{\epsilon\}$ markiert.
	\item Falls Knoten $k_0$ mit $A\in N$ markiert ist und die  Nachfolgerknoten von $k_0$ sind $k_1,\dots, k_n$, dann gibt Produktion $A\-> X_1\dots X_n$ mit $X_i\in\Sigma\cup N$ und $k_i$ ist mit $X_i$ markiert.
	
	\textbf{Oder:} $n=1$, die Produktion ist $A\->\epsilon$ und $k_1$ ist mit $\epsilon$ markiert.
	\end{itemize}
	Die Markierungen der Blätter von links nach rechts gelesen ergeben $w$.
\end{Def}

\begin{Bsp}\label{bsp:Ableitungsbaum} Siehe \autoref{fig:Ableitungsbäume}.
	\begin{figure}[H]\centering
		\begin{subfigure}[t]{.25\linewidth}\centering
			\tikz{\Tree [.$E$ $a$ ]}
			\caption{$E \Rightarrow a$}
		\end{subfigure}
		\begin{subfigure}[t]{.4\linewidth}\centering
			\begin{tikzpicture}[every tree node/.style={execute at begin node=$, execute at end node=$}]
				\Tree [.E (
			          [.E a ]
			          +
			          [.E ( [.E a ] * [.E a ] ) ]
			          )
			        ]
			\end{tikzpicture}
			\caption[$E \xRightarrow{*} (a+(a*a))$]{$E \xRightarrow{*} (a+(a*a)) \newline \longrightarrow E \Rightarrow (E+E)\Rightarrow(E+(E*E))\xRightarrow{*}$}
		\end{subfigure}
		\begin{subfigure}[t]{.3\linewidth}\centering
			\begin{tikzpicture}[every tree node/.style={execute at begin node=$, execute at end node=$}]
				\Tree [.A X_1 X_2 \edge[draw=none]; {\dots} X_n ]
			\end{tikzpicture}
			\caption{$A\rightarrow X_1\dots X_n$}
		\end{subfigure}
		\caption{Ableitungsbäume zu \autoref{bsp:Ableitungsbaum}}\label{fig:Ableitungsbäume}
	\end{figure}
	$\exists$ Ableitungsbaum für $w$ genau dann, wenn $S \xRightarrow{*} w$
\end{Bsp}
\begin{Def}[name={[Eindeutigkeit von \acs*{CFG} und \acs*{CFL}]}]\
	\begin{itemize}
	\item $\mathcal{G}\in$ \ac{CFG} heißt \uline{eindeutig}, falls es für jedes Wort $\in L(\mathcal{G})$ genau einen Ableitungsbaum gibt.
	\item Eine kontextfreie Sprache heißt \uline{eindeutig}, falls $\exists$ eindeutige \ac{CFG} für sie.
	\end{itemize}
\end{Def}
\begin{Bsp}
	\begin{align*}
		L &= \{ a^ib^jc^k \mid i=j\text{ oder }j=k \}\\
		S &\-> AC \mid DB\\
		A &\-> aAb \mid \epsilon & D &\->aD \mid \epsilon\\
		C &\->cC \mid \epsilon & B &\-> bBc \mid \epsilon \quad[\text{für $L$ gibt es keine eindeutige \ac{CFG}}]
	\end{align*}
	Werte der Form $a^nb^nc^n$ haben zwei Ableitungen $\curvearrowright$ Grammatik ist nicht eindeutig.
\end{Bsp}

\subsection{Die Chomsky Normalform für \ac*{CFG}}
Notw. für \ac{PL} + \nameref{satz:wortproblem}.
\begin{Def}[name={[\acs*{CFG} in \acs*{CNF}]}]
	Eine \ac{CFG} ist in \ac{CNF}, falls jede Produktion die Form $A\->a$ oder $A\->BC$ hat, wobei $A,B,C\in N$, $a\in\Sigma$.
\end{Def}
\begin{Beobachtung}
	$\mathcal{G}$ in \ac{CNF} $\curvearrowright \epsilon \notin L(\mathcal{G})$
\end{Beobachtung}
	
\paragraph*{Transformation} einer Grammatik $\mathcal{G}$ mit $\epsilon\notin L(\mathcal{G})$ nach \ac{CNF}.

\paragraph*{1. Schritt:} Separierte Grammatik\\
d.h. jede Regel hat die Form
\begin{alignat*}{2}
	A&\-> A_1\dots A_n &\qquad A_i&\in N,n\geq 0\\
	\text{oder}\quad A &\-> a & a&\in\Sigma
\end{alignat*}
\uline{Verfahren:}
\begin{itemize}
\item Erweitere $N$ um neue \acs{NT} $\{Y_a \mid a\in\Sigma\}$
\item Zusätzliche neue Regeln $Y_a\-> a$
\item Ersetze in den "`alten"' Regeln alle Terminale $a$ durch $Y_a$
\item[$\curvearrowright$] Grammatik separiert, Sprache ist unverändert.
\end{itemize}

\paragraph*{2. Schritt:} Eliminiere alle Regeln der Form $A\->\epsilon$ ($\epsilon$-Produktion)\\
Ziel: $\epsilon$-freie Grammatik.\\
Gesucht: $M=\{ A \mid A \xRightarrow{*}\epsilon \} \subseteq N$
\begin{alignat*}{2}
    \text{Def.:}&\quad& M_0 &= \{ A \mid A\-> \epsilon \}\\
    &\forall i\in\N& M_{i+1} &= M_i\cup \{ A \mid A\->\alpha, \alpha \in M_i^* \}\\
    \text{Es gilt }&\forall i\in\N& M_i &\subseteq N \text{ endlich}\\
    && \curvearrowright\quad \exists n\in\N &: M_n = M_{n+1} = \bigcup_{i\in\N} M_i\\[.5em]
    \text{Beh.:}&& M&=\bigcup\limits_{i\in\N} M_i\\
    \forall i\in\N :&& M_i &\subseteq M\\
    i=0 :&& \quad M_0 &= \{ A \mid A\-> \epsilon \in P \} \subseteq M\\
    i>0 :&& M_{i+1} &= M_i\cup \{ A \mid A\-> \alpha,\alpha\in M_i^* \}
\end{alignat*}
\begin{alignat*}{2}
    &\text{Falls}\quad& A &\->A_1\dots A_n\in P,\ A_j\in M_i\\
    &\text{dann} & A &\=> A_1\dots A_n\\
    &&&\xrightarrow{\text{I.V.}} A_2\dots A_n\\
    &&&\ \vdots\\
    &&&\xrightarrow{\text{I.V.}} A_n\\
    &&&\xrightarrow{\text{I.V.}} \epsilon\\
    &\curvearrowright & A &\in M\\
    &\-> & \bigcup_{i\in\N} M_i &\subseteq M
\end{alignat*}
Zeige noch $M\subseteq \bigcup\limits_{i\in\N} M_i$

Sei $A\in M : A\xRightarrow{*} \epsilon$ mit Ableitungsbaum der Höhe $h+1$, dann: $A\in M_h$

\uline{h=0:} Ableitungsbaum \quad \tikz[baseline=(a.base)]{\Tree [.\node (a) {$A$}; $\epsilon$ ]} \quad $A\->\epsilon \curvearrowright A\in M_0$

\uline{h>0:}
\tikz[baseline=(a.base)]{
    \Tree [.\node (a) {$A$}; [.$A_1$ \edge[roof]; {$\epsilon$} ]
                             [.$A_2$ \edge[roof]; {$\epsilon$} ]
                             [.$\dots$ ]
                             [.$A_n$ \edge[roof]; {$\epsilon$} ]
          ]
}

so dass $A_j \=>^* \epsilon$ mit Ableitungsbaum der Höhe $<h+1$

Nach I.V. $A_j\in M_{h-1}$ und $A \-> A_1\dots A_n \in P$, so dass $ A \in M_h$.


%
\datenote{11.12.15}
%
Transformation im \ac{CNF} (Fortsetzung)
\begin{enumerate}
\item Schritt: Separierte Grammatik: $\A\->A_1\dots A_n$ oder $A\-> a\ ,\ A_i\in N$
\item Schritt: $\epsilon$-freie Grammatik, bereits bestimmt:
	\[ M=\{A \mid A\=> \epsilon \} \]
	inkl. Algorithmus dazu: berechne $M_i$ f"ur $i=0, 1, \dots$ bis keine neuen NT hinzukommen. 
	
	Anpassen der Produktionen
	\begin{enumerate}
		\item\label{itm:Prod-b} Falls Regel $A\-> A_1\dots A_n\in P$ und $\exists j:A_j\in M$, dann füge neue Regel\\
			$A\->A_1\dots A_{j-1} A_{j+1}\dots A_n$ zu $P$ hinzu.
		\item Wiederhole \nameref{itm:Prod-b} bis keine neuen Regeln erzeugt werden können.
		\item Entferne alle $\epsilon$-Regeln.
	\end{enumerate}
	Ergebnis: Grammatik, die $\epsilon$-frei ist und die gleiche Sprache erzeugt (o.B.).

\item Schritt: Beseitige Kettenregeln der Form $A\-> B$
	\begin{itemize}
		\item Betrachte den gerichteten Graphen mit Knotenmenge $N$ und den Kettenregeln als Kanten.
		\item Suche Zyklus mit Tiefensuche $A_1\->A_2\->\dots\->A_r\->A_i$. Dann ersetzt Vorkommen von $A_j\ (j>1)$ durch $A_1$ (auf linker und rechter Seite aller Produktionen).
		\item Wiederhole bis alle Zyklen eliminiert.
		\item Betrachte topologische Sortierung des verbleibenden \acsu{DAG}: $B_1\dots B_m$, so dass $B_m$ keine Kettenregel besitzt.
		\item Starte mit $B_m$ mit Produktion $B_m\->\alpha_i \mid \dots \mid \alpha_k$\\
			Ersetze alle Vorkommen von $B_m$ auf der rechten Seite von Kettenproduktionen durch $\alpha_1 \mid \dots \mid \alpha_k$
		\item Wiederhole f"ur $B_{m+1}\dots B_1$\\
			$\curvearrowright$ alle Kettenproduktionen eliminiert.
	\end{itemize}
\item Schritt: jetzt haben alle Produktionen die Form $A\-> A_1\dots A_n$, $n\geq 2$ oder $A\-> a$.
	\begin{itemize}
		\item Falls $A\-> A_1 A_2\dots A_n \in P$ mit $n>2$ dann ersetze durch zwei neue Produktionen
		\begin{align*}
			A &\-> A_1 B & \text{mit $B$ neues \acs{NT}}\\
			B &\-> A_2\dots A_n 
		\end{align*}
		\item Wdhl. bis alle rechten Seiten eine L"ange $\le2$ haben.
	\end{itemize}
	$\curvearrowright$ \ac{CNF} erreicht.
\end{enumerate}
\begin{Bemerkung} Sei $|\mathcal{G}|=\sum\limits_{A\->\alpha\in P}(|\alpha|+1)$ die Gr"o"se einer CFG.\\
Die Transformation nach \ac{CNF} benötigt Zeit $O(|\mathcal{G}|^2)$. Die Größe der \ac{CNF}-Grammatik ist $O(|\mathcal{G}|^2)$.
\end{Bemerkung}

\subsection{Pumping Lemma für \acs*{CFL}}
\begin{Satz}[Pumping lemma für \acs*{CFL}, uvwxy Lemma]
\label{satz:PL für CFL}
%\rlnote{Satz \# 4.2?}
	Sei $L\in {CFL}$. Dann $\exists n>0$, so dass $\forall z\in L$ mit $|z|\geq n\ \exists u,v,w,x,y:$ mit
	\begin{itemize}
	\item $z=uvwxy$
	\item $|vwx|\leq n$
	\item $|vx|\geq 1$
	\item $\forall i \in \N : uv^iwx^iy\in L$
	\end{itemize}
\end{Satz}
\begin{proof}
	Sei $\mathcal{G}= (N,\Sigma,P,S)$ in \ac{CNF} mit
    $L(\mathcal{G}) = L\setminus\{\epsilon\}$.
    
	Sei $k = |N|$ und wähle $n=2^k$
	
	Betrachte den Ableitungsbaum $B$ von $z$ mit $|z|\geq n$.
	\begin{itemize}[label=$\curvearrowright$]
		\item $B$ ist (im wesentlichen) ein Binärbaum mit $|z|\geq n = 2^k$ Blättern.
		\item in einem solchen Bin"arbaum $B$ $\exists$ Pfad  der Länge $\geq k$.
	\end{itemize}
	Auf diesem Pfad liegen $\geq k+1$ \acfp{NT}\\
	Also: mindestens ein \ac{NT} $A$ muss mehrmals auf dem Pfad vorkommen. Suche so ein $A$ beginnend vom Blatt dieses Pfads, dann ist das zweite Vorkommen von $A\leq k$ Schritte vom Blatt entfernt.
	
	Aufgrund der Wahl von $A$ hat der Ableitungsbaum für $A \=>^* vAx \xRightarrow{*} vwx$ Höhe $\leq k$.
	\begin{itemize}[label=$\curvearrowright$]
		\item hat $\leq 2^k=n$ Blätter
		\item $|vwx|\leq 2^k =n$
	\end{itemize}
	Es gilt $|vx|\geq 1$, weil
	\begin{itemize}
		\item Ableitungsbaum ist Binärbaum, mit inneren Knoten $A \-> BC$
		\item Pfad belegt nur eine Abzweigung
		\item Wegen \ac{CNF} $\not\exists C, C\xRightarrow{*} \epsilon$
	\end{itemize}
		\begin{figure}[H]\centering
		\begin{subfigure}[b]{.29\linewidth}\centering
			\begin{tikzpicture}[node distance=.75cm, on grid,
					every node/.style={
						execute at begin node=$,
						execute at end node=$
					}
				]
				\node (0) {};
				\node (A1) [below=of 0] {A};
				\node (A2) [below=of A1] {A};
				\node (A3) [below=of A2] {A};
				
				\newlength\Offset \setlength\Offset{.75cm}
				\coordinate(p1) at ($(A2.south) - (0,\Offset)$);
				%
				\coordinate (A2l) at ($(p1) - (\Offset,0)$);
				\coordinate (A1l) at ($(A2l) - (\Offset,0)$);
				\coordinate (0l) at ($(A1l) - (\Offset,0)$);
				%
				\coordinate (A2r) at ($(p1) + (\Offset,0)$);
				\coordinate (A1r) at ($(A2r) + (\Offset,0)$);
				\coordinate (0r) at ($(A1r) + (\Offset,0)$) ;
				
				\coordinate (A3l) at ($(A3.south) - (\Offset,\Offset)$);
				\coordinate (A2l2) at ($(A3l) - (\Offset,0)$);
				%
				\coordinate (A3r) at ($(A3.south) + (\Offset,-\Offset)$);
				\coordinate (A2r2) at ($(A3r) + (\Offset,0)$);
				
				\node (dots) [below=\Offset of A3] {\vdots};
				
				\node (A4) [below=\Offset of dots] {A};
				\coordinate (A4l) at ($(A3l) - (0,2*\Offset)$);
				\coordinate (A4r) at ($(A3r) - (0,2*\Offset)$);
				
				\draw (0.south) edge (0l.center) edge (0r.center)
					(A1.south) edge (A1l.center) edge (A1r.center)
					(A2.south) edge (A2l2.center) edge (A2r2.center)
					(A3.south) edge (A3l.center) edge (A3r.center)
					(A4.south) edge (A4l.center) edge (A4r.center)
				;
				\draw (0l.center) edge node[below] {u} (A1l.center)
					(A1l.center)  edge node[below] {v} (A2l.center)
					(A2r.center)  edge node[below] {x} (A1r.center)
					(A1r.center)  edge node[below] {y} (0r.center)
					(A2l2.center) edge node[below] {v} (A3l.center)
					(A3r.center)  edge node[below] {x} (A2r2.center)
					(A4l.center)  edge node[below] {w} (A4r.center)
				;
			\end{tikzpicture}
			\caption{$uv^iwx^iy\in L$}
		\end{subfigure}
		\begin{subfigure}[b]{.28\linewidth}\centering
			\begin{tikzpicture}[node distance=.5cm, on grid,
					every node/.style={
						execute at begin node=$,
						execute at end node=$
					},
					widebox/.style={draw, minimum width=.75cm, minimum height=.35cm, inner sep=0pt}
				]
				\node (S) {S};
				\node[inner sep=2pt] (A) [below=of S.south, xshift=.2cm] {A};
				\coordinate (Ap) at (A.south west);
				\node[widebox] (w) [below=of Ap] {w};
				\node[widebox] (u) at ($(Ap)!2!(w.north west)+.5*(-.75,-.3)$) {u};
				\node[widebox] (y) at ($(Ap)!2!(w.north east)+.5*(.75,-.3)$) {y};
				
				\draw (S.south) edge (u.north west) edge (y.north east)
					(Ap) edge (u.north east) edge (y.north west) edge (w.north)
					($(S.south) - (.25,.6)$) edge (S.south) edge (Ap)
				;
			\end{tikzpicture}
			\caption{$\curvearrowright uwy\in L$}
		\end{subfigure}
		\begin{subfigure}[b]{.33\linewidth}\centering
			\begin{tikzpicture}[>=stealth,
				widebox/.style={draw, minimum width=.75cm, minimum height=.35cm, inner sep=0pt}
				,dot/.style={inner sep=0pt,outer sep=0pt,label={center:\scalebox{.75}{\textbullet}}}
				]
				
				\node[dot,label={above right:$S$}] (v1) at (0,-1) {};
				\node[dot] (v2) at (-0.25,-1.6) {};
				\node[dot,label={above right:$A$}] (v3) at (0,-2) {};
				\node[dot] (v4) at (-0.5,-2.5) {};
				\node[dot,label={above right:$A$}] (v5) at (0,-3) {};
				\node[widebox] (v6) at (-1.5,-3.5) {$u$};
				\node[widebox] (v7) at (-.75,-3.5) {$v$};
				\node[widebox] (v8) at (0,-3.5) {$w$};
				\node[widebox] (v9) at (.75,-3.5) {$x$};
				\node[widebox] (v10) at (1.5,-3.5) {$y$};
				
				\draw  (v1) edge (v2);
				\draw  (v2) edge (v3);
				\draw  (v3) edge (v4);
				\draw  (v4) edge (v7.north west);
				\draw  (v4) edge (v5);
				\draw  (v3) edge (v10.north west);
				\draw  (v5) edge (v8.north west);
				\draw  (v5) edge (v8.north east);
				\draw  (v1) edge (v6.north west);
				\draw  (v1) edge (v10.north east);
				\node[align=left, text height=3em,inner sep=2pt] at (2.1,-1.5) {$A\rightarrow BC$\\[-3pt]\small{ist Binärbaum}} edge[->,shorten >=.1cm] (v3);
				\node[inner sep=0pt] at (2.75,-3.5) {$A\rightarrow a$} edge[->] (v10);
				\node (v11) at (0,-4) {$z$};
				\draw[semithick]
				let \p1 = (v6.west), \p2 = (v11), \p3 = (v10.east) in 
				 (\x1,\y2) edge[{Bar[]<}-] (v11) (v11) edge[-{>Bar[]}] (\x3,\y2);
			\end{tikzpicture}
			\caption{$\exists$ Pfad mit Länge $\geq k$}
		\end{subfigure}
		\caption{Schema zu \autoref{satz:PL für CFL}}
	\end{figure}\vspace{-2em}\qedhere
\end{proof}

\begin{lemma} %\rlnote{Lemma \# 4.3?}
	$\mathcal{L}_2 \subsetneq \mathcal{L}_1$
\end{lemma}
\begin{proof}
	Sei $L=\{L=\{a^nb^nc^n \mid n\geq 1\}$.\\
	$L$ ist nicht kontextfrei. Verwende \ac{PL}. Angenommen $L\in {CFL}$.\\
	Sei $n$ die Konstante aus dem \ac{PL}.
	
	Wähle $z=a^nb^nc^n$
	\begin{align*}
		\text{Also: } \exists &u,v,w,x,y : z = uvwxy\\
		& |vwx| \leq n
	\end{align*}
	$\curvearrowright vwx$  kann nicht gleichzeitig $a,b$ und $c$ enthalten.\\
	Angenommen $vwx$ enthält kein $c$. Dann muss $uwy \in L$
	\begin{itemize}
		\item $y$ endet mit $c^n$
		\item $|vx|\geq 1$, also $|uwy|+1\leq |z|$ also "`fehlt"' mindestens ein $a$ oder $b$
	\end{itemize}
	$\curvearrowright uvw \notin L$
\end{proof}

\subsection{Entscheidungsprobleme für \acs*{CFL}}
\begin{Satz}[name={[Wortproblem für \acs*{CFL} entscheidbar]}]
	Das Wortproblem "`$w\in L?$"' ist für \ac{CFL} entscheidbar. Falls $|w|=n$, benötigt der Algorithmus $O(n^3)$ Schritte und $O(n^2)$ Platz.
\end{Satz}
\begin{proof}
	Algorithmus \ac{CYK}.
\end{proof}
\begin{Bsp}
	$L=\{a^nb^nc^m \mid n,m\geq 1\}$ mit Grammatik (\acs{CFG})
	\begin{align*}
		S &\-> XY\\
		X &\-> ab \mid aXb\\
		Y &\-> c \mid cY\\
		\shortintertext{Umformen in \ac{CNF}}
		S &\-> XY & A &\->a\\
		X &\-> AB \mid AZ & B&\->b\\
		Y &\->c \mid CY & C &\->c\\
		Z &\-> XB
	\end{align*}
\end{Bsp}

\datenote{18.12.15}
	$\begin{aligned}[t]
		S &\-> XY\\
		X &\-> AB\\
		X &\-> AZ\\
		Z &\-> XB\\
		Y &\-> c\\
		Y &\-> CY\\
		A &\-> a\\
		B &\-> b\\
		C &\-> c
	\end{aligned}$\quad
\begin{proof}
		\ac{CYK}-Algorithmus
		\begin{description}
	    \item[Eingabe:]
		\begin{itemize}
			\item \ac{CFG} $\mathcal{G}$ in \ac{CNF}
			\item $w = a_1\dots a_n\in\Sigma^*, |w|=n$
		\end{itemize}
		\item[Ausgabe:] "`$w\in L(\mathcal{G})$"'
		\item[Idee:] Berechne eine $(n\x n)$ Matrix $M$ mit Einträgen in $\mathcal{P}(N)$ mit folgender Spezifikation:
		\end{description}
	\begin{align*}
		M_{ij} &= \{ A \mid A\xRightarrow{*} a_i\dots a_j \} && 1\leq i\leq j\leq n \quad\text{nur diese }M_{ij}\neq \varnothing\\
		& \text{falls $i=j$:}\\
		M_{ii} &= \{ A \mid A\xRightarrow{*} a_i\}\\
		&= \{A \mid A\-> a_i \} \\
		&\text{falls }1\leq i<j\leq n:\\
		M_{ij} &= \{ A \mid A\xRightarrow{*} a_i\dots a_j \}\\
		&= \{ A \mid A\=> BC \xRightarrow{*} a_i\dots a_j \}\\
		&= \{ A \mid A\-> BC \land BC \xRightarrow{*} a_i\dots a_j \}\\
		&= \{ A \mid A\-> BC \land \exists k: i\leq k<j\quad 
			\begin{aligned}[t]
				B &\xRightarrow{*} a_i\dots a_k \land{}\\
				C &\xRightarrow{*} a_{k+1}\dots a_j \}
			\end{aligned}\\
		&= \bigcup_{k=i}^{j-1} \{ A \mid A \-> BC \land 
			\begin{aligned}[t]
				B &\xRightarrow{*} a_i\dots a_k \land{}\\
				C &\xRightarrow{*} a_{k+1}\dots a_j \}
			\end{aligned}\\
		&= \bigcup_{k=i}^{j-1} \{ A \mid A \-> BC \land B \in M_{i,k} \land C \in M_{k+1,j} \}
	\end{align*}
	Damit: $w\in L$ genau dann, wenn $S \=>^* w$ genau dann, wenn $S \in M_{1,n}$.
	
	\begin{table}[H]\centering
		\caption[$w=aaa\ bbb\ cc$]{$w=aaa\ bbb\ cc$ \protect\footnotemark}
		\begin{tabular}[t]{M{c}|*8{M{c}|}}
		\cline{2-9}
			M = & A               & \bullet & \bullet & \bullet & \bullet & \color{Green}X & \coloruline{Green}{S} & S       \\
		\cline{2-9}
			\multicolumn{2}{c|}{} & \color{Blue}A       & \bullet & \bullet & \color{Red}\coloruline{Blue}{X} & \coloruline{Red}{Z} & \bullet & \bullet \\
		\cline{3-9}
			     \multicolumn{3}{c|}{}      & A       & X       & \color{Blue}Z       & \bullet & \bullet & \bullet \\
		\cline{4-9}
			          \multicolumn{4}{c|}{}           & B       & \bullet & \bullet & \bullet & \bullet \\
		\cline{5-9}
			               \multicolumn{5}{c|}{}                & B       & \bullet & \bullet & \bullet \\
		\cline{6-9}
			                    \multicolumn{6}{c|}{}                     & \color{Red}B & \bullet & \bullet \\
		\cline{7-9}
			                         \multicolumn{7}{c|}{}                          & C,\color{Green}Y     & Y       \\
		\cline{8-9}
			                              \multicolumn{8}{c|}{}                               & C,Y     \\
		\cline{9-9}
		\end{tabular} $M_{1n}$
	\end{table}\footnotetext{Zur Veranschaulichung: das Unterstrichene wird aus den Elementen der entsprechenden Farbe gebildet.}
\end{proof}

$\acs{CYK}(\mathcal{G},a_1,\dots,a_n)$\\
$M\ n\x n$ Matrix mit $M_{ij}=\varnothing$
\begin{lstlisting}[mathescape,morekeywords={for,do,return},morecomment={[l]{//}}]
  for i=1 .. n do  // $O(|\mathcal{G}|)\cdot O(n)$
    $M_{ii}=\{A \mid A \-> a_i \}$
  for i=n-1 ..  1 do
    for j=i+1 .. n do
      for k=i .. j-1 do  // $O(|\mathcal{G}|)\cdot O(n^3)$
        $M_{ij}=M_{ij}\cup \{ A \mid A \-> BC, B\in M_{ik}, C\in M_{k+1,j} \}$
  return $S\in M_{1n}$
\end{lstlisting}

\begin{Satz}[name={[Entscheidbarkeit des Leerheitsproblems für \acs*{CFL}]}] % 4.7
    Das Leerheitsproblem ist für \ac{CFL} entscheidbar.
\end{Satz}

\begin{proof}
    Sei $\mathcal{G} = (N, \Sigma, P, S)$ eine CFG für $L$.
    Gefragt ist, ob $L(\mathcal{G}) = \{ w\in\Sigma^* \mid S\overset{*}{\=>} w \} \overset?= \varnothing
$. 
    Betrachte dazu die Menge $M$ der nützlichen Nichtterminalsymbole, aus denen ein Terminalwort herleitbar ist.
	\begin{align*}
		M &= \{ A \mid A \overset{*}{\=>} w, w\in \Sigma^* \}\\
		M_0 &= \{ A \mid A \-> w\in P, w\in \Sigma^* \}\\
		M_{i+1} &= M_i\cup \{ A \mid A \-> \alpha\in P, \alpha\in(\Sigma\cup M_i)^* \}\\
		\exists n : M_n &= M_{n+1} \overset!= M\\
		L =\varnothing &\<=> S\notin M \qedhere
	\end{align*}
	Offenbar ist $M_0 \subseteq M$ eine Approximation von $M$, $M \supseteq M_{i+1} \supseteq M_i$ verbessert $M_i$ und das jeweils nächste $M_i$ ist in Linearzeit berechenbar. Da $M \subseteq N$ endlich ist, muss es ein $n$ geben, sodass $M_n = M_{n+1}$. Für dieses $n$ kann man zeigen, dass $M = M_n$. 
\end{proof}
\begin{Bem}
	$M$ ist die Menge der nützlichen Nichterminale. Nicht nützliche Nichtterminale können entfernt werden, ohne dass $L(\mathcal{G})$ sich ändert.
\end{Bem}
\begin{Satz}[name={[Entscheidbarkeit des Endlichkeitsproblem für \acs*{CFL}]}]
	Das Endlichkeitsproblem für \ac{CFL} ist entscheidbar.
\end{Satz}
\begin{proof}
	Mit \ac{PL} analog zum Endlichkeitsproblem für reguläre Sprachen.
\end{proof}

\subsection{Abschlusseingenschaften für \acs*{CFL}}


\begin{Satz} % 4.7
	Die Menge \acs{CFL} ist abgeschlossen unter $\cup, \cdot, ^*$, jedoch \underline{nicht} unter $\cap, \overline{\phantom{A}}$.
\end{Satz}
\begin{proof}
	\begin{align*}
		\mathcal{G}_i :&= (N_i, \Sigma,P_i,S_i) \quad i=1,2\\
		\text{"`}\cup\text{"'}: N &= N_1\dotcup N_2\dotcup\{S\}\\
		P &= \{S\->S_1, S\->S_2 \} \cup P_1\cup P_2\\
		\text{"`}\cdot\text{"'}:: N &= N_1\dotcup N_2\dotcup\{S\}\\
		P &= \{ S\->S_1S_2 \} \cup P_1\cup P_2\\
		\text{"`}^*\text{"'}: N &= N_1 \dotcup \{S\}\\
		P &= \{ S\->\epsilon, S\-> S_1S \} \dotcup P_1
	\end{align*}
	\begin{itemize}
	\item für $n\geq 1$
		\[ \underbrace{\{a^nb^nc^n\}}_{\notin \acs{CFL}} = \underbrace{\{a^nb^nc^m \mid n,m\geq 1 \}}_{\in \acs{CFL}} \cap \underbrace{\{a^mb^nc^n \mid m,n\geq 1 \}}_{\in \acs{CFL}} \]
		Also \ac{CFL} nicht abgeschlossen unter $\cap$.
	\item Angenommen \ac{CFG} abgeschlossen unter $\overline{\phantom{X}}$\\
		Falls $L_1,L_2\in \acs*{CFL}$, dann ist $\overline{L}_1,\overline{L}_2\in\acs*{CFL}$ nach Annahme.\\
		$\curvearrowright \overline{L}_1\cup \overline{L}_2\in\acs*{CFL}$ wegen Teil "'$\cup$"'.\\
		$\curvearrowright \overline{\overline{L}_1\cup \overline{L}_2} = L_1\cap L_2\in\acs*{CFL}\ \lightning$ zu Teil "'$\cap$"'. \qedhere
	\end{itemize}
\end{proof}

\begin{Satz} % 4.8
	Die Menge \ac{CFG} ist abgeschlossen unter $\cap$ mit reg. Sprachen. "`$\acs*{CFL} \cap \text{REG} = \acs*{CFL}$"'.
\end{Satz}
\begin{proof}
	\begin{align*}
			\text{Sei }L &= L(\mathcal{G})\quad\acs{CFL}\\
			R &= L(M)\quad\acs{REG}\\
			\mathcal{G} &= (N,\Sigma,P,S)\\
			M &= (Q,\Sigma,\delta,q_0,F)\quad\acs{DFA}
	\end{align*}
	Zeige $L\cap R$ \acs{CFL}\\
	Konstruiere \ac{CFG} für $L\cap R$\\
	Ang. $\mathcal{G}$ ist in \ac{CNF}
	\begin{alignat*}{2}
		A &\-> a &\quad& A \-> BC\\
		\tikz[baseline]{\Tree [.$qAq'$ $a$ ]}
			&= \delta(q,a) &&\qquad \mathclap{%
				\tikz[baseline]{\Tree [.$q_1Aq_3$ $q_1Bq_2$ $q_2Cq_3$ ]}
			}\\
		N' &= Q\x N\x Q\cup \{S'\}\\
		S' &\-> (q_0,S,q')\ && \forall q'\in F\\
		(q,A,q') &\-> a && \text{falls } \delta(q,a) = q'\text{ und }A\-> a\in P\\
		(q_1, A, q_3) &\-> (q_1,B,q_2)(q_2,C,q_3) &&\text{falls } A\-> BC\in P \forall q_1,q_2,q_3\in Q
	\end{alignat*}
	Zeige $(p,A,q)\overset*{\=>} w \<=> \exists$ Lauf $p\dots q$ von $M$ auf $w$ und $A \Rightarrow^* w$\\
	per Induktion über Höhe $h$ des Ableitungsbaums:
	\begin{itemize}[labelwidth=\widthof{$h=1:$},leftmargin=!]
	\item[$h=1:$] $(p,A,q) \-> a\ \exists$ Lauf denn $\delta(p,a)=q$; offenbar $A\Rightarrow a$
	\item[$h>1:$] $(p,A,q) \-> (p,\underset{\Downarrow*}{B},r)(r,\underset{\Downarrow*}{C},q)$\\
	\hspace{-2em}mit $w\phantom{A}\,=\phantom{(p,\,} w_1 \phantom{,r)(r,\ }w_2$
	\end{itemize}
	und Höhen $<h:$\\
	Nach IV $\exists$ Lauf $p\dots r$ auf $w_1$ und $B \Rightarrow^* w_1$\\
	\phantom{Nach IV }$\exists$ Lauf $r\dots q$ auf $w_2$ und $C \Rightarrow^* w_2$\\
	$\curvearrowright\ \exists$ Lauf $p\dots r\dots q$ auf $w_1 w_2=w$ und $A \Rightarrow BC \Rightarrow^* w$
	
	Nun gilt 
	\begin{align*}
	    w \in L(\mathcal{G}') & \<=> S \=>^* w \\
	    & \<=> \exists q'\in F: (q_0, S, q') \=>^* w \\
	    & \<=> \exists q'\in F: \exists \text{ Lauf von $q_0$ nach $q'$ auf $w$} \land S \=>^* w \\
	    & \<=> w \in R = L(M) \land w \in L = L(\mathcal{G}) \\
	    & \<=> w \in R \cap L
	\end{align*}
\end{proof}
%\rlerror{Vorlesung 18.12.15 unvollständig (Aufschrieb vorhanden)}
%
\datenote{21.12.15}
\vspace{-.5em}
%
%Zuletzt: $\acsu{CFL}\land\acsu{REG}=\ac{REG}$
\begin{Satz}[name={[$L\subseteq R$ entscheidbar]}] %\rlwarning{Satz 4.9}
	Falls $L\in\ac{CFL}$ und $R\in\textrm{REG}$, dann ist "`$L\subseteq R$?"' entscheidbar.
\end{Satz}
\begin{proof}
	$L\subseteq R\curvearrowleftright \underbrace{L \cap\overline{R}}_{\ac{CFL}} = \varnothing$ entscheidbar.
\end{proof}
\section[Kellerautomaten (\acs*{PDA})]{Kellerautomaten \quad\normalfont\normalsize \acf{PDA}}
Kellerautomat $\approx$ Endlicher Automat + Kellerspeicher von unbeschränkter Größe (Stack, push down)
\paragraph*{Neu:}
\begin{itemize}
	\item bei jedem Schritt darf der \ac{PDA} das oberste Kellersymbol inspizieren und durch beliebiges Kellerwort ersetzen.
	\item der \ac{PDA} darf auf dem Keller rechnen, ohne in der Eingabe weiter zu lesen. ($\epsilon$-Transition oder Spontantransition).
\end{itemize}
\begin{Bsp}
	\begin{align}
		\Sigma &= \{a,b,\#\} &\qquad&\text{Eingabealphabet} \notag\\
		\Gamma &= \{a,b,\bot\} &&\text{Kelleralphabet} \notag\\
		Q &= \{q_0,q_1\} &&\bot\,\hat=\,\text{Kellerbodensymbol} \notag\\
		\delta: (q_0,x,Z) &\mapsto (q_0,xZ) && x\in\{a,b\},Z\in\Gamma\\
		(q_0,\#,Z) &\mapsto (q_1,Z) \label{eq:5.2}\\
		(q_1,x,x) &\mapsto (q_1,\epsilon)\\
		(q_1,\epsilon,\bot) &\mapsto (q_1,\epsilon)\\
		\text{Konfiguration: } (q_0,&w,\bot) &&\text{Start}\\
		&\downarrow\\
		(q',&\,\epsilon,\epsilon) &&\text{akzeptiert!}
	\end{align}
	Erkannte Sprache $L=\{w \# w^R \mid w\in\{a,b\}^*\}$
	\begin{align*}
		&(q_0,abb\#bba,\bot)\\
		\vdash\quad &(q_0, bb\#bba, a\bot)\\
		\vdash\quad &(q_0, b\#bba, ba\bot) \vdash (q_0, \#bba, bba\bot)\\
		\vdash\quad &(q_1, bba, bba\bot)\\
		\vdash\quad &(q_1, ba, ba\bot) \vdash (q_1, a, a\bot)\\
		\vdash\quad &(q_1, \epsilon, \bot) \vdash (q_1,\epsilon,\epsilon) \qquad\text{akzeptiert!}
	\end{align*}
	deterministisch.
\end{Bsp}
Nicht deterministisch: $L'=\{ww^R \mid w\in \{a,b\}^*\}$\\
Ersetze \eqref{eq:5.2} durch $(q_0,x,Z) \mapsto (q_1,Z) \quad Z\in\Gamma$

\underline{Alle Palindrome} zusätzlich $(q_0,x,Z)\mapsto(q_1,Z) \ Z\in\Gamma,\ x\in\{a,b\}$ auch nicht deterministisch.

\begin{Def}[name={[NPDA]}]
	Ein \ac{NPDA} ist Tupel $(Q,\Sigma,\Gamma,q_0,Z_0,\delta)$
	\begin{itemize}
		\item $Q$ endl. Zustandsmenge
		\item $\Sigma$ endl. Eingabealphabet
		\item $\Gamma$ endl. Kelleralphabet
		\item $q_0\in Q$ Startzustand
		\item $Z_0\in\Gamma$ Kellerbodensymbol
		\item $\delta: Q\x(\Sigma\cup\{\epsilon\})\x\Gamma \-> \mathcal{P}(Q\x\Gamma^*)$ \qedhere
	\end{itemize}
\end{Def}
Im weiteren sei $M=(\dots)$ ein \ac{NPDA}.
\begin{Def}[name={[Menge der Konfigurationen eines \acs*{NPDA}]}]
	Die Menge der Konfigurationen von $M$ ist $\Konf(M) = Q\x\Sigma^*\x\Gamma^*$\\
	Die Schrittrelation von $M$ ist:
	\begin{align*}
		&\vdash\ \subseteq \Konf(M)^2 &&\text{definiert durch}\\
		(q,aw,Z\gamma) &\vdash (q',w,\beta\gamma) &&\text{falls }\delta(q,a,z)\ni(q',\beta)\\
		(q,w,Z\gamma) &\vdash (q',w,\beta\gamma) &&\text{falls }\delta(q,\epsilon,z)\ni(q',\beta)\\
	\shortintertext{Die von $M$ erkannte Sprache}
		L(M) = \{ w\in\Sigma^* \mid (q_0,w,Z_0) &\vdash^{\!\!*} (q',\epsilon,\epsilon) \} \tag*{\qedhere}
	\end{align*}
\end{Def}
\begin{Satz}\label{satz:5.1}
	\[
		L \in \ac{CFL} \xLeftrightarrow{\text{gdw.}}
		L =L(M) \quad\text{für \ac{NPDA} $M$.}
	\]
\end{Satz}
\begin{proof}
	\begin{itemize}
	\item["'\=>"'] Sei $L\in\ac{CFL}$\\
		Sei $\mathcal{G} = (N,\Sigma,P,S)$ Grammatik für $L\setminus \epsilon$ in \ac{CNF}. \\
		Def. \ac{NPDA} M durch
		\begin{alignat*}{2}
			Q &= \{q\} &\quad(=&\text{ Startzustand})\\
			\Gamma &= \Sigma\dotcup N\\
			Z_0 &= S\\
			\delta(q,a,a) &= \{(q,\epsilon)\} & a&\in\Sigma\\
			\delta(q,\epsilon,A) &= \{(q,\alpha)\} & A&\->\alpha\in P
		\end{alignat*}
		\begin{alignat*}{2}
		\text{f"ur alle } X&\in\Sigma\cup N \ ,\ v\in\Sigma^*\\ 
			\text{Zeige }&X\overset{*}{\==>} v &
			\curvearrowleftright\quad& \mathrlap{(q,v,X) \vdash^{\!\!*} (q,\epsilon,\epsilon)}
		\end{alignat*}
		per Induktion über Ableitungsbaum von $X\overset{*}{\==>}v$\\
		(Höhe von)
		\begin{description}
			\item[Höhe $0$:]
				\begin{align*}
					\bullet\quad  a &= X = v\\
					(q,a,a) &\vdash (q,\epsilon,\epsilon)\ \checkmark
				\end{align*}
			\item[Höhe $>0$:]\
			\begin{enumerate}
			\item Fall:
				\tikz[baseline=-.3em]{
					\Tree [.\node [label={right:$A=X$}] {$\bullet$};
						\node [label={right:$a$}] {$\bullet$};
					]
				}
				\quad $(q,a,A) \overset{A\-> a}{\vdash} (q,a,a) \vdash (q,\epsilon,\epsilon)$
			\item Fall:
				\tikz[baseline, level 1/.style={sibling distance=1em}]{
					\Tree [.\node[label={right:$A = X$}] {$\bullet$};
						[.\node[label={left:$B_1$}] {$\bullet$}; \edge[roof]; {$v_1 $} ]
						[.\node[label={right:$B_2$}] {$\bullet$}; \edge[roof]; {$v_2 $} ]
					]
				} Regel $A\-> B_1B_2$
				\begin{align*}
					\text{d.h. }v = v_1v_2 \quad B_1 &\overset{*}{\==>} v_1 \text{ mit kleinerem Abeitungsbaum}\\
					B_2 &\overset{*}{\==>} v_2 \text{ \ruleplaceholder{\widthof{mit kleinerem Abeitungsbaum}}}\\
					(q,v_1v_2,A) &\vdash (q,v_1v_2,B_1B_2)\\
					&\vdash^* (q,v_2,B_2) \quad \text{nach I.V. für }B_1\\
					&\vdash^* (q,\epsilon,\epsilon) \quad \text{nach I.V. für }B_2\\
					\text{D.h. } S\overset{*}{\==>} w &\<==> (q,w,S) \vdash^* (q,\epsilon,\epsilon)\\
					&\<==> w\in L(M)
				\end{align*}
			\item["'\<="'] Gegeben \ac{NPDA} $M=(Q,\dots)$ bei dem alle Transitionen pop, top oder push sind.
			\end{enumerate}
		\end{description}
	\end{itemize}
	\begin{figure}[H]\centering\vspace{-1em}
		\input{2015-12-22_diagramm.pdf_tex}
		{\setlength{\belowdisplayskip}{-.5em}
		\begin{align*}
			(q,Z,q') &\overset{*}{\Rightarrow} v\\
			(q,v,Z) &\vdash^* (q',\epsilon,\epsilon)
		\end{align*}}
		\caption{Beweis zu \autoref{satz:5.1}}
	\end{figure}\vspace{-2.5em}\qedhere\\*
    Grammatik $\mathcal{G}'$ leitet gerade $v$ ab, falls $Z$ im Stack.
\end{proof}

\begin{lemma}
	Zu jedem \ac{NPDA} gibt es einen äquivalenten \ac{NPDA}, so dass
	falls $\delta(q,x,Z)\ni (q',\alpha) \quad x\in\Sigma\cup\{\epsilon\}$
	dann ist entweder
	\begin{itemize}
		\item $\alpha = \epsilon$
		\item $\alpha = Z'$
		\item $\alpha = Z'Z''$
	\end{itemize}
	\vspace{-1.5em}
\end{lemma}
\begin{proof}
	Sei $(q',\alpha)\in\delta(q,x,Z)$ mit $\alpha = Z_n\dots Z_1$ f"ur $n>2$:
	\begin{itemize}
	\item 	neue Zustände $q_2\dots q_{n-1}$
	\item Ersetze $(q',\alpha)$ durch $(q_2, Z_2Z_1)$
	\item Definiere $\delta(q_i, \epsilon, Z_i) = \{ (q_{i+1}, Z_{i+1}Z_i) \}$, f"ur $2\le i < n-1$
	\item Definiere $\delta(q_{n-1}, \epsilon, Z_{n-1}) = \{ (q', Z_nZ_{n-1}) \}$
	\end{itemize}
	Wiederhole bis alle Transitionen die gew"unschte Form haben. \qedhere
\end{proof}

%
\datenote{22.12.15}
\vspace{3em}

%\rlerror{Anfang unvollständig (Aufschrieb fehlt!)}

$L=L(M)$ für \ac{NPDA} $M$\\
$\curvearrowright\ L$ ist \ac{CFL}

%\begin{figure}[H]\centering
%\vspace{2em}
%\rlerror*{Grafik fehlt}{<Bild>}% Handschr. Mitschrieb vorhanden.
%\caption{
%}
%\end{figure}

\begin{itemize}
\item Def. \ac{CFG} mit $N= Q\x \Gamma\x Q\cup \{S\}$, so dass $(q,Z,q')\overset{*}{\=>} v$ gdw.\ 
$(q, v, Z) \vdash^* (q',\epsilon,\epsilon)$
\\
	Transitionen von $M$ haben eine von drei Formen:
	\begin{align*}
		&(q,\epsilon) && (q,Z) && (q,Z_1 Z_2)
	\end{align*}
\item Def. $P$ durch $P \supseteq \{ S \-> (q_0,Z_0,q') \mid q'\in Q\}$ sowie die weiteren Produktion wie folgt
\end{itemize}
Produktion für $(q,Z,q')$
\begin{enumerate}[label={Fall \arabic*:},ref={Fall \arabic*},leftmargin=*]
\item\label{itm:Fall 1} $\delta(q,x,Z)\ni (q',\epsilon)\ , x\in \Sigma\cup \{\epsilon\}$\\
	$P \supseteq \{ (q,Z,q')\-> x \}$
\item $\delta(q,x,Z)\ni (q'',Z')\ , x\in \Sigma\cup \{\epsilon\}$\\
	$P \supseteq \{ (q,Z,q')\-> x(q'',Z',q') \mid q'\in Q\}$
\item $\delta(q,x,Z)\ni (q_1,Z_1Z_2)\ , x\in \Sigma\cup \{\epsilon\}$\\
	$P \supseteq \{ (q,Z,q')\-> x(q_1,Z_1,q_2)(q_2,Z_2,q') \mid q_1,q_2\in Q \}$
\end{enumerate}
Korrektheit: Zeige $(q,Z,q')\overset{*}{\=>} w\curvearrowleftright (q,w,Z) \vdash^* (q',\epsilon,\epsilon)$\\
Induktion über Ableitungsbaum von $(q,Z,q')\overset{*}{\=>} w$ (siehe oben)
\begin{itemize}[align=left]
\item[Höhe 1:] \tikz[baseline]{\Tree[.$(q,Z,q')$ $x$ ]} \quad \ref{itm:Fall 1} $\curvearrowleftright (q,x,Z)\vdash (q',\epsilon,\epsilon)$
\item[Höhe $>1$:] Zwei Möglichkeiten:
	\begin{enumerate}[label={Fall \arabic*:},ref={Fall \arabic*},leftmargin=*,start=2]
	\item \ \vspace{-1em}
	\begin{align*}
		\raisebox{0pt}[0pt][0pt]{\tikz[baseline]{\Tree[.$(q,Z,q')$ $x$ $(q'',Z,q')$ ]}}
			&\curvearrowright w = xw'\land (q'',Z',q')\overset{*}{\=>} w'\\
			&\begin{aligned}
				\curvearrowright (q,xw',Z) &\vdash (q'',w',Z')\\
				\text{I.V.}\quad &\vdash^* (q',\epsilon,\epsilon)
			\end{aligned}
	\end{align*}
	\item \tikz[baseline]{\Tree[.$(q,Z,q')$ $x$ $(q_1,Z_1,q_2)$ $(q_2,Z_2,q')$ ]}
		\begin{align*}
			w &= xw_1w_2\\
			(q_1,Z_1,q_2) &\overset{*}{\=>} w_1 \quad\text{mit kleinerem Ableitungsbaum}\\
			(q_2,Z_2,q') &\overset{*}{\=>} w_2 \quad\text{mit kleinerem Ableitungsbaum}\\
			\curvearrowright (q,xw_1w_2,Z) &\vdash (q_1,w_1w_2,Z_1Z_2)\\
			\text{I.V. für }w_1\text{ + Lemma} &\vdash^* (q_2,w_2,Z_2)\\
			\text{I.V. für }w_2\phantom{\text{ + Lemma}} &\vdash^* (q',\epsilon,\epsilon)
		\end{align*}
	\end{enumerate}
\end{itemize}

\begin{Def}[name={[DPDA]}]
	Ein \ac{DPDA} ist ein Tupel $(\underbrace{Q,\Sigma,\Gamma,q_0,Z_0}_{\text{wie gehabt}},\delta,F)$
	\vspace{-1em}
	\begin{itemize}
	\item $F\subseteq Q$ akzeptierende Zustände
	\item $\delta: Q\x (\Sigma\cup\{\epsilon\})\x \Gamma \-> \mathcal{P}(Q\x \Gamma^*)$\\
	$|\delta(q,a,Z)| + |\delta(q,\epsilon,Z)| \leq 1 \quad \forall q\in Q,a\in\Sigma,Z\in\Gamma$
	\item $\vdash$ wie gehabt
	\item $L(M) = \{w\in\Sigma^* \mid (q_0,w,Z_0) \vdash^* (q',\epsilon,\gamma) \land q'\in F \}$ \qedhere
	\end{itemize}
\end{Def}

\begin{lemma}[name={[\acs*{DPDA}, der gesamte Eingabe verarbeitet]}]
	\label{lem:DPDA ges. Eingabe}
	Zu jedem \ac{DPDA} gibt es einen äquivalenten \ac{DPDA}, der die gesamte Eingabe verarbeitet.
\end{lemma}
\begin{proof}
    Erste Möglichkeit: Die Transitionsrelation ist nicht total.

	Führe einen neuen Zustand $q_s\notin F$ ein.
	\begin{align*}
	\shortintertext{Falls $\exists a\in\Sigma, q, Z$}
		\delta(q,a,Z) &\cup \delta(q,\epsilon,Z) = \varnothing\\
	\shortintertext{dann erweitere $\delta$ um}
		\delta(q,a,Z) &= \{ (q_s,Z) \}\\
	\shortintertext{und $\forall a\in\Sigma, Z\in \Gamma: \delta(q_s,a,Z) = \{(q_s,Z)\}$}
	\end{align*}
	Weitere Möglichkeit: der Automat bleibt wegen leerem Keller stecken.\\
	Abhilfe: Neues Kellerbodensymbol $Z_0'$ und neuer Startzustand $q_0'$.
	\begin{align*}
		\delta(q_0',\epsilon,Z_0') &= \{(q_0,Z_0Z_0')\}\\
	\shortintertext{Erweitere $\delta$ für alle originalen zustände $q\in Q$ um}
		\delta(q,\epsilon,Z_0') &= \{(q_s,Z_0')\}
	\end{align*}
	"`Falls Keller abgeräumt, Wechsel nach $q_s$"'
\end{proof}

\begin{Satz}[name={[Abgeschlossenheit der deterministischen \acs*{CFL}]}]
	Die deterministischen \ac{CFL} sind unter Komplement abgeschlossen.
\end{Satz}
\begin{proof}
	Sei $L=L(M)$ für \ac{DPDA} $M$. Nach \autoref{lem:DPDA ges. Eingabe} liest $M$ die komplette Eingabe.\\
	Def. $M'$ mit $Q'= q\x \{0,1,2\}$\\
	Zustand $(q,0)$: seit Lesen des letzten Symbols wurde kein akz. Zustand durchlaufen.\\
	Zustand $(q,1)$ seit Lesen \dots$\geq 1$ akz. Zustand durchlaufen\\
	\phantom{Zustand} $(q,2)$ \dots akz. Zustände d.h. $F'=F\x \{2\}$
	\begin{align*}
		q_0' &= \begin{cases}
			(q_0,0) & q_0\notin F\\
			(q_0,1) & q_0\in F
		\end{cases}\\
		\text{Falls }\delta(q,\epsilon,Z) = \{(q',\gamma)\}\text{ dann}&\\
		\delta'((q,0),\epsilon,Z) &= \begin{cases}
				((q',0),\gamma) & q'\notin F\\
				((q',1),\gamma) & q'\in F
			\end{cases}\\
		\delta'((q,1),\epsilon,Z) &= ((q',1),\gamma)\\
		\text{Falls }\delta(q,a,Z) = \{(q',\gamma)\}& \\
		\delta'((q,0), \epsilon, Z) &= \{ ((q,2), Z) \} \\
		\delta'((q,2), a, Z) &=
		    \begin{cases}
		        ((q',0), \gamma) & q' \notin F\\
		        ((q',1), \gamma) & q' \in F
		    \end{cases}\\
		\delta'((q,1), a, Z) &=
            \begin{cases}
                ((q',0), \gamma & q' \notin F\\
                ((q',1), \gamma & q' \in F
            \end{cases}
        \\ \tag*{\qedhere}
	\end{align*}
\end{proof}
\begin{Satz}
    Die deterministischen \ac{CFL} sind \textbf{nicht} unter Vereinigung und Durchschnitt abgeschlossen.
\end{Satz}
\begin{proof}
    Betrachte
    \begin{align*}
        L_1 &= \{ a^nb^nc^m \mid n, m \ge 1 \} \\
        L_2 &= \{ a^mb^nc^n \mid n, m \ge 1 \}
    \end{align*}
    Sowohl $L_1$ als auch $L_2$ sind DCFL, aber $L_1 \cap L_2 = \{ a^nb^nc^n \mid n \ge 1\}$ ist nicht kontextfrei.
    
    DCFL ist nicht abgeschlossen unter Vereinigung. Angenommen doch: Seien $U, V$ DCFL. Dann sind auch $\overline{U}$ und $\overline{V}$ DCFL. Bei Abschluss unter Vereinigung wäre $\overline{U} \cup \overline{V}$ eine DCFL und somit auch $\overline{\overline{U} \cup \overline{V}} = U \cap V$, ein Widerspruch gegen den ersten Teil.
\end{proof}
\begin{Satz}
    DCFL ist abgeschlossen unter Schnitt mit REG.
\end{Satz}
\begin{proof}
    Sei $L$ DCFL und $R$ regulär.
    Bilde das Produkt aus einem DPDA für $L$ und einem DFA für $R$.
    Offenbar ist das Ergebnis ein DPDA, der $L\cap R$ erkennt.
\end{proof}
\begin{Satz}
    Sei $L$ DCFL und $R$ regulär.
    Es ist entscheidbar, ob $R=L$, $R\subseteq L$ und $L=\Sigma^*$.
\end{Satz}
\begin{proof}
    Es gilt $R\subseteq L$ gdw.\ $R \cap \overline{L} = \emptyset$.
    
    Weiter ist $R = L$ gdw.\ $R\subseteq L$ und $L \subseteq R$. Für den zweiten Teil betrachte $L\cap \overline{R}$.
    
    Für kontextfreie Sprachen ist $L\ne \emptyset$ entscheidbar, also betrachte $L=\Sigma^*$ gdw.\ $\overline{L}=\emptyset$.
\end{proof}

\begin{Satz} \textbf{DPDA Äquivalenzproblem}
    Seien $L_1, L_2$ DCFL. Dann ist $L_1 = L_2$ entscheidbar.
\end{Satz}

\begin{proof}
    Siehe Senizergues (2000) und Stirling (2001).
\end{proof}

\section[Berechenbarkeit]{Berechenbarkeit} \draftnote{18.01.17}
\subsection{Typ-0 und Typ-1 Sprachen}
%\ptnote{checkpoint}
% \begin{Def}[name={[NTM]}]
% 	\acf{NTM} ist ein Tupel $(Q,\Sigma,\Gamma,\delta,q_0,\blank,F)$, wobei alles wie bei einer \ac{DTM} außer $\delta: Q\x\Gamma\-> \wp(Q\x\Gamma\x\{N,L,R\})$.\\
% 	Konfiguration und Berechnungsrelation wie gehabt.
% \end{Def}

\begin{Def}[name={[Akzeptanz, Entscheidbarkeit, Semi-Entscheidbarkeit]}]
	Sei $M$ \ac{TM}.
	\begin{itemize}
	\item $M$  \emph{akzeptiert} $w\in\Sigma^*$, falls $q_0w
          \vdash^* uq'v$ Haltekonfiguration und $q'\in F$
	\item $M$ \emph{akzeptiert} $L\subseteq\Sigma^*$, falls $M$ akzeptiert $w \<-> w\in L$
	\item $M$ \emph{entscheidet} $L\subseteq\Sigma^*$, falls $M$ akzeptiert $L$ und $M$ hält für jede Eingabe an.
	\item $L\subseteq\Sigma^*$ ist \emph{semi-entscheidbar} (rekursiv aufzählbar), falls $\exists M$, die $L$ akzeptiert.
	\item $L\subseteq\Sigma^*$ ist \emph{entscheidbar} (rekursiv), falls $\exists M$, die $L$ entscheidet.
	\end{itemize}
\end{Def}
\begin{Def}[name={[Laufzeit und Platzbedarf einer \acs*{TM}]}]
	Laufzeit und Platzbedarf einer \ac{TM} $M:$
	
	Laufzeit : $T_M(w) =
	\begin{cases}
		\parbox{.74\textwidth}{\raggedleft Anzahl der Schritte einer kürzesten Berechnung, die zur Akz. von $w$ führt (falls $\exists$)}\\[-1em]
		\text{1, sonst}
	\end{cases}
	$\\
	Platzbedarf : $S_M(w) =
	\begin{cases}
		\parbox{.7\textwidth}{\raggedleft geringster Platzbedarf (Länge einer Konf.) einer akz. Berechnung von $w$ (falls $\exists$)}\\[-1em]
		\text{1, sonst}
	\end{cases}
	$
\textbf{Zeitbeschränkt mit $t(n)$}: $\forall w\in\Sigma^*: |w|\leq n \=> T_M(w)\leq t(n)$,\\
platzbeschränkt analog.
\end{Def}
\begin{Satz}[name={[Zu jeder NTM gibt es \acs*{DTM}]}]\label{satz:6.1}
	Zu jeder NTM gibt es eine \ac{DTM} $M'$, so dass
	\begin{itemize}
	\item $M'$ akzeptiert $L(M)$
	\item $M'$ terminiert gdw. $M$ terminiert
	\item Falls $M$ zeit- und platzbeschränkt ist mit $t(n)$
          bzw. $s(n)$ ($n=$ Länge der Eingabe), dann ist $M'$
          zeitbeschränkt mit $2^{O(t(n))}$ und platzbeschränkt mit
          $O(s(n)\cdot t(n))$. 
	\end{itemize}
\end{Satz}\vspace{-2em}
\begin{proof}
	Die Konfigurationen von $M$ bilden einen Baum, dessen Kanten durch $\vdash$ gegeben sind. Er ist endlich verzweigt, hat aber ggf. unendlich lange Äste.
	
	Definiere eine (Mehrband-)\ac{DTM}, die den Konfigurationsbaum
        systematisch durchläuft und akzeptiert, sobald eine
        Haltekonfiguration erreicht ist, in der $M$ akzeptiert. 
	
	Die \ac{DTM} terminiert ebenfalls, wenn alle Blätter des Baumes besucht worden sind, ohne dass eine akzeptierende Konfiguration gefunden wurde.
	
	Baumsuche mit Kontrollinformation und bereits besuchten Konf. auf ein Extraband.
	\begin{itemize}
	\item Tiefensuche? Nicht geeignet, sie könnte in unendlichen Ast laufen.
	\item Breitensuche? OK, aber Platzbedarf $O(2^{t(n)}\cdot s(n))$
	\item iterative deepending\,: Tiefensuche mit vorgegebener Schranke, bei erfolgloser Suche Neustart mit erhöhter Schranke. \qedhere
	\end{itemize}
\end{proof}
Nächstes Ziel: Charakterisierung von Typ-1 Sprachen.
\begin{Def}[name={[$\DTAPE$ und $\NTAPE$]}]\
\begin{itemize}
\item $\DTAPE(s(n)):$ Menge der Sprachen, die von einer \ac{DTM} in Platz $s(n)$ akzeptiert werden können.
\item $\NTAPE(s(n)):$ Wie für $\DTAPE$, aber mit \ac{NTM}. \qedhere
\end{itemize}
\end{Def}
\begin{Bemerkung}\
	\newcommand{\underarrowset}[2]{%
		\underset{%
			\mathclap{%
				\overset{\displaystyle\uparrow}{\mathclap{#1}}%
			}%
		}{#2}%
	}
	\begin{enumerate}
	\item Für "`$s(n)\leq n$"' betrachte 2-Band \ac{TM}, bei denen die Eingabe read-only ist und nur das zweite Arbeitsband der Platzschranke unterliegt (so ist $s(n)$ sublinear möglich).
	\item Jede Platzbeschränkung impliziert Laufzeitschranke.\\
	Angenommen Platzschranke $s(n)$\\
	$\curvearrowright$ \ac{TM} hat nur endlich viele Konfigurationen
	\[ N := \underarrowset{%
			\parbox{\widthof{\scriptsize Eingabeband}}{\raggedright\scriptsize Kopfpos. im Eingabeband}\hspace{1cm}
		}{n\vphantom{|}}
		|Q| \quad\cdot\quad
		\underarrowset{\parbox{2.2cm}{\scriptsize\centering mögliche Inhalte des Arbeitsbands}}{|\Gamma|}^{s(n)}
		\ \cdot\
		\underarrowset{\hspace{1.7cm}\parbox{2cm}{\scriptsize Kopfpos. auf\\ Arbeitsband}}{s(n)}
		\in 2^{O(\log n + s(n))}
	\]
	\item \ac{DTM} mit Platzschranke\,: $M$ entscheidet,\\
	falls sie akzeptiert, dann in weniger als $N$ Schritten,\\
	falls nach $N$ Schritten keine Termination erfolgt\\
	\quad$\curvearrowright$ Endlosschleife -- Abbruch
	\item \ac{NTM}\,: nutze den \ac{ND} optimistisch aus\,:\\
	falls eine akzeptierende Berechnung existiert, dann muss es eine Berechnung ohne wiederholte Konfiguration geben.
	\end{enumerate}
\end{Bemerkung}
\begin{Satz}[name={[$L\in\DTAPE(n),\ L\in\NTAPE(n)$]}]\label{satz:6.2}\
	\begin{itemize}
	\item $L\in\DTAPE(n) \curvearrowright\ \exists$ \ac{DTM}, die $L$ in Zeit $2^{O(n)}$ entscheidet.
	\item $L\in\NTAPE(n)$ analog.
	\end{itemize}
\end{Satz}\vspace{-2em}
\begin{proof}
	siehe oben.
\end{proof}
\begin{Bemerkung}
	Die Klasse $\NTAPE (n)$ heißt auch \ac{LBA}.
\end{Bemerkung}
%
\begin{Satz}\label{satz:6.3}
	$\mathcal L_1=\mathrm{NTAPE}(n)$
\end{Satz}
\begin{proof}\
\begin{itemize}
	\item["`\=>"'\,:] Sei $G=(N,\Sigma,P,S)$ Typ-1 Grammatik für $L$.\\
		Konstruiere \ac{NTM} $M$ mit $L=L(M),\ \Gamma = \Sigma\cup N\cup\{\blank\}$
		\begin{enumerate}
		\item $M$ rät nicht deterministisch eine Position auf
                  dem Band und eine Produktion $\alpha\-> \beta$. Falls $\beta$ gefunden wird, ersetze durch $\alpha$, weiter bei 1.
		\item Falls Bandinhalt $=S$ \quad stop, akzeptiert.
		\end{enumerate}
		Dieses Verfahren terminiert.
	\item["`\<="'\,:] %
	Gegeben: \ac{NTM} $M$ linear beschränkt.\\
	Gesucht: Typ-1 Grammatik $\mathcal{G}$ mit $L(\mathcal{G})=L(M)$\\
	Idee: 
	\begin{align*}
		a_1\cdots a_n &\-->
			\pmqty{a_1\\a_1} \pmqty{a_2\\a_2} \pmqty{(q,a)\\a_3} \pmqty{a_4\\a_4} \pmqty{a_n\\a_n}
			&&\begin{matrix}\text{Spur 1}\\\text{Spur 2}\end{matrix}\\
	\shortintertext{ad\footnotemark\ Spur 1: Alphabet $\Gamma\cup(Q\x\Gamma) = \triangle$}
		P' &\begin{cases}
			(q,a) &\-> (q',a')   \quad q\in Q,a\in\Gamma\\
			(q,a)b &\-> a'(q',b) \quad b\in\Gamma\\
			b(q,a) &\-> (q',b)a'
		\end{cases}
		& &\begin{aligned}
			\delta(q,a) &\ni (q',a',N)\\
			\delta(q,a) &\ni (q',a',R)\\
			\delta(q,a) &\ni (q',a',L)
		\end{aligned}
	\end{align*}\footnotetext{ad $\approx$ zur}%
	Def. $\widetilde{uqav} = u(q,a)v\ ,\ u,v\in\Gamma^*,a\in\Gamma$\\
	Es gilt: $uqav\vdash^* k' \curvearrowleftright \widetilde{uqav} \overset{*}{\=>} \widetilde{k'}$ mit Produktion $P'$.
	
	Def. $\mathcal{G}$ durch $N = \{S\}\dotcup\triangle\x\Sigma$
	\begin{align*}
		\text{mit } P &= \\
		S &\-> \pmqty{(q_0,a)\\a} &&\forall a\in\Sigma\\
		S &\-> S\pmqty{a\\a} &&\forall a\in\Sigma\\
		\pmqty{\alpha\\a}
			&\-> \pmqty{\beta\\a}
			&&\begin{aligned}
				\forall \alpha\->\beta &\in P'\\
				\alpha,\beta &\in\triangle
			\end{aligned}\\
		\pmqty{\alpha_1\\a_1}\pmqty{\alpha_2\\a_2}
			&\-> \pmqty{\beta_1\\a_1}\pmqty{\beta_2\\a_2}
			&&\begin{aligned}
				\forall \alpha_1\alpha_2\->\beta_1\beta_2 &\in P'\\
				\alpha_i,\beta_i &\in\triangle
			\end{aligned}\\
		\pmqty{x\\a} &\-> a
			&&\begin{aligned}
				x&\in\Gamma\\
				a&\in\Sigma
			\end{aligned}\\
		\pmqty{(q',x)\\a} &\-> a
			&&\begin{aligned}
				x&\in\Gamma, q'\in F, \delta (q',x)=\emptyset\\
				a&\in\Sigma
			\end{aligned}
	\end{align*}
	\begin{align*}
		S &\xRightarrow{*} \pmqty{(q_0,a_1)\\a_1}\pmqty{a_2\\a_2}\dots\pmqty{a_n\\a_n}\\
		&\phantom{{}\xRightarrow{*}{}}\ \acs*{TM}\ \,\dots\\
		&\xRightarrow{*} \pmqty{x_1\\a_1}\dots\pmqty{(q',x_i)\\a_i}\dots\pmqty{x_n\\a_n}\\
		&\xRightarrow{*} a_1\dots a_i\dots a_n
	\end{align*}
	Damit gesehen $L(\mathcal{G})\subseteq L(M)$\\
	Rückrichtung: selbst \qedhere
	\end{itemize}
\end{proof}

\begin{Satz}
	Die Typ-1 Sprachen sind abgeschlossen unter {\thinmuskip=6mu$\cup,\cap,\cdot,{}^*$} und Komplement.
\end{Satz}
\begin{proof}
	Zu $\cup$ und $\cap$ betrachte \ac{NTM}.\\
	Für $\cdot$ und $^*$ konstruiere Grammatik.\\
	ad Komplement "`2. \acsu{LBA}-Problem\footnote{\acs*{LBA} = \acl*{LBA} -- 1964 Kuroda}"' bis 1987, dann gelöst durch Immerman und Szelepcsényi.
	
	1. \ac{LBA}-Problem (1964): Ist $\mathrm{NTAPE}(n) = \mathrm{DTAPE}(n)$? Bisher ungelöst.
\end{proof}
\begin{Satz}
	Das Wortproblem für Typ-1 Sprachen ist entscheidbar.
\end{Satz}
\begin{proof}
	\begin{align*}
		L\in\mathcal{L}_1 &\curvearrowleftright L\in\mathrm{NTAPE}(n)\\
		&\curvearrowright \text{\autoref{satz:6.2} $L$ entscheidbar}
	\end{align*}
	Nach \autoref{satz:6.1} sogar mit \ac{DTM}.
\end{proof}
Die Rückrichtung "`L entscheidbar. $\xcancel{\curvearrowright}\ L$ ist Typ-1 Sprache"' gilt nicht!

\begin{Satz}\label{satz:6.6}
	$\mathcal{L}_0 = \ac{NTM}$
\end{Satz}
\begin{proof}
	\begin{itemize}
	\item["'\=>"'] Kontruktion einer \ac{NTM} $M$ wie in \autoref{satz:6.3}, aber ohne Platzbeschränkung.
	\item["'\<="'] Konstruktion analog zu \autoref{satz:6.3} + Startsymbol $S'$
	\begin{align*}
		S' &\-> \pmqty{\blank\\\Eps} S' \pmqty{\blank\\\Eps} &&\text{Schaffe Platz für Berechnung von }M\\
		S' &\-> S\\
	\shortintertext{Erweitere $N$}
		&= \{S',S\}\cup\triangle\x(\Sigma\cup\{\Eps\})\\
	\shortintertext{Neue Löschregeln:}
		\pmqty{x\\ \Eps } &\-> \Eps &&\forall x\in\Gamma\\
		\rotatebox{90}{$\Rsh$}\ &\mathrlap{\text{die einzigen
                                          Regeln, die Typ-1 Bedingung verletzen.}} \tag*{\qedhere}
	\end{align*}
	\end{itemize}
\end{proof}

\begin{Satz}[name={[Abgeschlossenheit von Typ-0 Sprachen]}]\label{satz:Typ-0-abgeschl}
	Die Typ-0 Sprachen sind unter $\thinmuskip=6mu\cup,\cap,\cdot,{}^*$ abgeschlossen.
\end{Satz}
\begin{proof}
	Konstruiere \ac{NTM} für $\thinmuskip=6mu\cup,\cap$ ; Typ-0-Grammatiken für $\cdot$ und $^*$.
\end{proof}

\begin{Bem}
	Typ-0 Sprachen sind \emph{nicht} unter Komplement abgeschlossen!
\end{Bem}

\subsection{Universelle \acs*{TM} und das Halteproblem}
Ziel: Universelle \ac{TM} (eine \ac{TM}, die \ac{TM}s interpretiert) ist eine \ac{TM} $U$, die zwei Eingaben nimmt:
\begin{enumerate}
\item Kodierung einer (anderen) \ac{TM} $M_0$
\item Eingabe $w$ für $M_0$
\end{enumerate}
so dass
\begin{align*}
	w\in L(M_0) &\curvearrowleftright (M_0,w) \in L(U)\\
	\text{$M_0$ terminiert bei Eingabe $w$} &\curvearrowleftright
                                                  \text{$U$ terminiert
                                                  bei Eingabe $(M_0,w)$}
\end{align*}
Zur Vereinfachung:
\begin{align*}
	\Sigma &= \{0,1\}\\
	\Gamma &=\{\blank,0,1\}\\
\shortintertext{Die Kodierung von $M_0$}
	&= (Q,\dots,q_1,\delta,\dots,\{q_2\})
\end{align*}
mit $Q=\{q_1,q_2,\dots,q_t\}$ ist im wesentlichen die Kodierung von $\delta$. Dazu zwei Hilfsfunktionen:
\begin{tabular}{M{c} | M{c}}
	x\in\Gamma & f(x) \\ \hline
	0          & 1    \\
	1          & 2    \\
	\blank     & 3    \\
\end{tabular}
\quad
\begin{tabular}{M{c} | M{c}}
	D & g(D) \\
\hline
	L & 1    \\
	N & 2    \\
	R & 3
\end{tabular}

Kodiere $\delta$ Zeilenweise:
\begin{align*}
	\delta(q_i,X) &= (q_k,Y,D) \ (= \text{Zeile} z)\\
\shortintertext{durch}
	\mathrm{code}(z) &= 0^i10^{f(x)}10^k10^{f(y)}10^{g(D)}\\
\shortintertext{Wenn $\delta$ durch $s$ Zeilen $z_1\dots z_s$ gegeben,
  dann definiere die \emph{Gödelnummer} von $M_0$ durch}
	\ulcorner M_0 \urcorner &= 111 \mathrm{code}(z_1) 11 \mathrm{code}(z_2) 11 \dots 11 \mathrm{code}(z_s) 111\\
\end{align*}
Definiere $U$ als 3-Band Maschine mit
\begin{itemize}
\item[$B_1:$] Eingabe + Arbeitsband (für $M_0$)
\item[$B_2:$] $\ulcorner M_0 \urcorner$
\item[$B_3:$] $0^k$ für Zustand $q_k$
\end{itemize}
1. Schritt: Transformierte Eingabe
$\ulcorner M_0 \urcorner$
\begin{itemize}
\item Beginnt die Eingabe mit gültiger Gödelnummer?
\item Gleichzeitig: Verschiebe $\ulcorner M_0 \urcorner$ auf $B_2$
\item Wenn Ende von $\ulcorner M_0 \urcorner$ erreicht, schiebe 0 auf $B_3$.
\end{itemize}
Jetzt:
\begin{itemize}
\item[$B_1:$] $w$
\item[$B_2:$] $\ulcorner M_0 \urcorner$
\item[$B_3:$] $0' \sim Z,q$
\end{itemize}


 % Danke @ Lars Nitzke
\vspace{1em}

\begin{Satz}
	Es gibt eine universelle \ac{TM} $U$ mit $L(U) = \{ \ulcorner M \urcorner w\mid w \in L(M)\}$
\end{Satz}
\begin{proof}
	Initialisierung:
	\begin{itemize}
		\item[$B_1:$] $w$ Eingabewort/Arbeitsband
		\item[$B_2:$] $\ulcorner M \urcorner$ Gödelnummer
		\item[$B_3:$] 0 Zustand
	\end{itemize}
	Hauptschleife von U:
	\begin{itemize}
		\item Test auf Haltekonfiguration.
		\item Falls ja: 
		\begin{tabular}[t]{@{}r@{\,}l}
			Falls in $qz$ &: akzeptiert.\\
			sonst&: nicht
		\end{tabular}
		\item Ausführung des nächsten Schritts:\\
		Suche Zeile in $\ulcorner M \urcorner$ gemäß Zustand und aktuellem Symbol auf Arbeitsband.
		Ändere $B_1$ und $B_3$ gemäß $\delta$.\\
		$B_2, B_3$: Kopf zurück zum Anfang.
	\end{itemize}
\end{proof}
Schreibe ab jetzt $M_w$ für die Maschine mit Gödelnummer $w$. Falls
$w$ kein gültiger Code für eine TM, dann sei $M_w$ eine beliebige fest
TM $M$ mit $L(M) = \varnothing$. 
\begin{Def}[name={[Spezielles Halteproblem]}]
  Das spezielle Halteproblem besteht aus allen Codes von Maschinen, die
  anhalten, falls sie auf den eigenen Code angesetzt werden.
  \begin{align*}
    K &= \{ w \in \{0,1\}^* \mid M_w \text{ angesetzt auf }w\text{
        terminiert} \}
  \end{align*}
\end{Def}
\begin{Satz}
  Das spezielle Halteproblem ist unentscheidbar.
\end{Satz}
\begin{proof}
  Angenommen $M$ ist eine TM, die $K$ entscheidet.

  Konstruiere Maschine $M'$, die zunächst $M$ auf ihre Eingabe
  anwendet. Falls $M$ akzeptiert, dann geht $M'$ in eine
  Endlosschleife. Falls $M$ nicht akzeptiert, dann hält $M'$ an.

  Sei $w' = \ulcorner M' \urcorner$ der Code von $M'$ und setze $M$
  auf $w'$ an. Es gilt:

  $M$ akzeptiert $w'$

  gdw.(nach Def von $K$) $M'$ angesetzt auf $w'$ terminiert

  gdw. $M$ akzeptiert $w'$ nicht.

  Ein Widerspruch. $\qquad\lightning$
\end{proof}
\begin{Korollar}\label{kor:6.11}
  $\overline{K} = \{w \in\{0,1\}^* \mid M_w\text{ h"alt nicht bei
    Eingabe }w \}$,   das Komplement von $K$, 
  ist nicht entscheidbar.
\end{Korollar}
\begin{proof}
	Angenommen $\overline{K}$ sei entscheidbar durch $M$. Dann
        entscheidet $M'$  $K$. $M'$ führt zuerst $M$ aus und negiert das Ergebnis. $\lightning$ \autoref{satz:6.10}
\end{proof}
\begin{lemma}[name={[$K$ ist semi-entscheidbar]}]
	$K$ ist semi-entscheidbar.
\end{lemma}
\begin{proof}
  Die Maschine $M$ kopiert die Eingabe $w$ und f''uhrt die universelle
  TM f''ur $M_w$ aus. Falls diese Simulation stoppt, geht $M$ in einen
  akzeptierenden Zustand und terminiert.
\end{proof}
% \begin{Def}[name={[length-lexicographic order]}]
% 	Die \emph{length-lexicographic order} auf $\{0, 1\}^*$ (mit $0<1$) ist definiert durch
% 	\begin{align*}
% 		v \leq w \<=> &\ |v| < |w|\\
% 		\vee &\ v = w\\
% 		\vee &\ |v| = |w| \text{ und } \exists u\in \{0, 1\}^*\\
% 		& \text{mit } v = u0v' \text{ und } w = u1w'
% 	\end{align*}
% \end{Def}
% \begin{Satz}[name={[$\leq:$ totale Ordnung]}]\
% 	\begin{itemize}
% 		\item $\leq$ ist totale Ordnung auf $\{0, 1\}^*$
% 		\item $\exists$ bijektive Abbildung $w: \N \to \{0, 1\}^*$ mit $i \leq j \to w(i) \leq w(j)$
% 	\end{itemize}
% \end{Satz}
% \begin{Def}[name={[Diagonalsprache]}]
% 	Sei $M_i$ die Turingmaschine mit Gödelnummer $\ulcorner M_i\urcorner = w(i)$.
% 	(falls $w(i)$ kein gültiger Code, dann sei $M_i$ eine beliebige \ac{TM} mit $L(M_i) = \varnothing$). Die \emph{Diagonalsprache}
% 	\[D = \{w(i) \mid w(i) \notin L(M_i)\}\]
% 	das heißt $M_i$ akzeptiert $w(i)$ nicht.
% \end{Def}
% \begin{Satz}[name={[$D$ ist nicht entscheidbar]}]\label{satz:6.10}
% 	$D$ ist nicht entscheidbar.
% \end{Satz}
% \begin{proof}
% 	Angenommen $\exists M$, die $D$ entscheidet.\\
% 	$M$ muss in Aufzählung vorkommen, das heißt $\exists j\in \N$, sodass $M = M_j$.
% 	Wende $M_j$ auf $w(j)$ an:
% 	\begin{itemize}
% 		\item $M_j$ stoppt/ja: $w(j) \in L(M_j) = D \qquad \lightning$ Def. von $D$
% 		\item $M_j$ stoppt/nein: $w(j) \notin L(M_j) = D \quad \lightning$ Def. von $D$ \qedhere
% 	\end{itemize}
% \end{proof}
% \begin{Korollar}\label{kor:6.11}
% 	$\overline{D} = \{w(i) \mid M_i\text{ akzepztiert }w(i) \}$ ist nicht entscheidbar.
% \end{Korollar}
% \begin{proof}
% 	Angenommen $\overline{D}$ sei entscheidbar durch $M$. Dann entscheidet $M'\ D$. $M'$ führt 
% 	zuerst $M$ aus und negiert das Ergebnis. $\lightning$ \autoref{satz:6.10}
% \end{proof}
% \begin{lemma}[name={[$\overline{D}$ ist semi-entscheidbar]}]
% 	$\overline{D}$ ist semi-entscheidbar.
% \end{lemma}
% \begin{proof}
% 	Bei Eingabe $w$
% 	\begin{itemize}
% 		\item Falls $w$ kein gültiger Code: stop mit Ergebnis nein.
% 		\item Falls $w$ gültiger Code, dann $\exists i$, sodass $ww = \left\ulcorner M_j
% 		\right\urcorner w(i) \qquad w = w(i)$
% 	\end{itemize}
% 	Wende $U$ auf $ww$ an.\\
% 	Insgesamt: \ac{TM}, die $\overline{D}$ akzeptiert.
% \end{proof}
\begin{Satz}[name={[{$L,\overline{L}$ semi-entscheidbar $\=> L$ entscheidbar}]},restate={[name=Wiederholung]repeatSatz613}]\label{satz:6.13}
	Falls $L$ semi-entscheidbar und $\overline{L}$ semi-entscheidbar, dann ist $L$ entscheidbar.
\end{Satz}
\begin{proof}
	Sei $M$ die \ac{TM} für $L$, $\overline{M}$ die \ac{TM} für $\overline{L}$.\\
	Führe $M$ und $\overline{M}$ "`parallel"' mit der gleichen Eingabe aus.\\
	Falls $M$ akzeptiert $\Rightarrow$ Ja\\
	Falls $\overline{M}$ akzeptiert $\Rightarrow$ Nein.\\
	Eine der Maschinen muss anhalten, wegen Voraussetzung.
\end{proof}
$K$ \quad nicht entscheidbar\\
$\overline{K}$ \quad nicht entscheidbar\\
$K$ \quad semi-entscheidbar (Typ-0)\\
$\overline{K}$ \quad nicht-semientscheidbar (keine Typ-0)
\begin{Korollar}
	$\mathcal{L}_0 \supsetneqq \mathcal{L}_1$
\end{Korollar}
\begin{proof}
	$K$ ist unentscheidbar (also $\notin \mathcal{L}_1$), aber semi-entscheidbar (also $\in \mathcal{L}_0$).
\end{proof}
\emph{\textbf{Fragen:}}
\begin{enumerate}
	\item Ist $\mathcal{L}_1$ = Menge der entscheidbaren Sprachen?
	\item[--] Nein:\\
	\begin{minipage}[t]{.7\linewidth}
	Konstruiere eine Aufzählung aller Typ-1 Grammatiken. \medskip\\
	$D_1 = \{w(i) \mid w(i)\notin L(G_i)\}$ ist entscheidbar, weil das Wortproblem für Typ-1 
	Sprachen entscheidbar, aber es $\nexists j$, sodass $L(G_j) = D_1$s
	\end{minipage}\quad
	\begin{tabular}[t]{M{c} | *4{M{c}@{ }}}
		\ &w_1&w_2&w_3&\cdots\\\hline
		G_1 &\\
		G_2 &\\
		G_3 &\\
		\vdots&
	\end{tabular}
	\item Ist $\mathcal{L}_0$ = Menge aller Sprachen?
	\item[--] Nein: $\overline{K} \notin \mathcal{L}_0$
\end{enumerate}
\begin{Def}[Reduktion]\ \\
  Seien $U, V \subseteq \Sigma^*$ Sprachen.\\
  \emph{$U$ ist auf $V$ reduzierbar ($U \preceq V$)}, falls eine totale berechenbare Funktion
  $f:\Sigma^* \to \Sigma^*$ existiert, so dass $\forall x \in \Sigma^*:x \in U \iff f(x) \in V$.
\end{Def}
\begin{lemma}
  Falls $U \preceq V$ und $V$ \mbox{(semi-)entscheidbar}, dann ist auch $U$
  \mbox{(semi-)entscheidbar}.
\end{lemma}\vspace{-1.5em}
\begin{proof}
  Wenn $M$ ein (Semi-)Entscheidungsverfahren für $V$ ist, dann konstruiere $M'$ wie folgt
  \begin{itemize}
  \item wende erst $f$ auf die Eingabe $x$ an ($f$ ist berechenbare
    Funktion gemäß Reduktion und kann daher programmiert werden)
  \item führe $M$ auf dem Ergebnis $f(x)$ aus
  \end{itemize}
  $\curvearrowright$ $M'$ ist (Semi-)Entscheidungsverfahren für
  $U$. (Weil $f$ total ist, terminiert der Code  f''ur $f$ immer und daher "andert das Terminationsverhalten nicht.)
\end{proof}
\emph{Anwendung:} $U \preceq V$ und $U$
unentscheidbar $\curvearrowright$ $V$ unentscheidbar.

\begin{Def}[name={[Halteproblem]}]
	Das \emph{Halteproblem} ist definiert durch
	\[H = \{\ulcorner M\urcorner\# w \mid M \text{ hält bei Eingabe } w \text{ an}\} \qedhere\]
\end{Def}
\begin{Satz}[name={[$H$ ist unentscheidbar]}]\label{satz:H ist unentscheidbar}
	$H$ ist unentscheidbar.
\end{Satz}
\begin{proof}
  Die Funktion $f (w) = w\#w$ ist total berechenbar und liefert eine
  Reduktion  $K \preceq H$.

  Denn: $w \in K$ gdw. $M_w$ hält bei Eingabe $w$ an gdw. $w\#w \in H$.
\end{proof}
% \begin{proof}
% 	Angenommen $M_0$ entscheidet $H$.\\
% 	Konstruiere $M'$ wie folgt:\\
% 	Bei Eingabe $w$ bestimme $i$, sodass $w = w(i)$\\
% 	Verwende $M_0$ um festzustellen, ob $\ulcorner M_i\urcorner w$ anhält.\\
% 	Antwort von $M_0:$\\
% 	nein: $\begin{aligned}[t]
% 		&\curvearrowright w(i) \notin L(M_i)\\
% 		&\curvearrowright M'\text{ akzeptiert }w\text{ nicht}.
% 	\end{aligned}$\\
% 	ja: Führe $\ulcorner M_i\urcorner w$ mit $U$ aus (muss ja terminieren) und akzeptiere entsprechend das Ergebnis von $U$.\\
% 	Insgesamt: $M'$ entscheidet $\overline{D} \qquad \lightning$ \autoref{kor:6.11}\\
% 	$\curvearrowright M_0$ existiert nicht.
% \end{proof}
\begin{Satz}[name={[$H$ ist semi-entscheidbar]}]
	$H$ ist semi-entscheidbar.
\end{Satz}
\begin{proof}
	Modifiziere $U$, sodass sie jede Eingabe akzeptiert, bei der sie anhält.
\end{proof}
\begin{Def}[name={[Halteproblem auf leerem Band $H_\varepsilon$]}]
  Das Halteproblem auf leerem Band $H_\varepsilon = \{\ulcorner M\urcorner \mid M 
	\text{ terminiert auf leeren Band}\}$
\end{Def}
\begin{Satz}[name={[$H_\varepsilon$ ist unentscheidbar]}]
	$H_\varepsilon$ ist unentscheidbar.
\end{Satz}
\begin{proof}
  Konstruiere eine Reduktion $H \preceq H_\varepsilon$ mit Hilfe der
  Funktion $f(w\#x) = w'$, wobei $w'$ der Code einer TM ist, die
  \begin{itemize}
  \item zuerst $x$ aufs leere Band schreibt und dann
  \item $M_w$ auf diese Eingabe anwendet.
  \end{itemize}
  Offenbar gilt $w\#x\in H$ gdw. $f (w\#x) \in H_\varepsilon$.
\end{proof}
% \begin{proof}
% 	Angenommen $M_\varepsilon$ entscheidet $H_\varepsilon$.\\
% 	Konstruiere $M'$ (mit Hilfe von $M_\varepsilon$), sodass $M'$ entscheidet $H$.\\
% 	\begin{tabular}{r@{ }l}
% 	$M':$ & Bei Eingabe $\ulcorner M\urcorner w$.\\
% 	& Konstruiere $M^*$, $M^*$ schreibt zuerst $w$ aufs (leere) Band und 
% 	startet dann $M$ auf $w$.\\
% 	& Wende $M_\varepsilon$ auf $\ulcorner M^*\urcorner$ an.
% 	\end{tabular}\\
% 	$\curvearrowright M'$ entscheidet $H \qquad \lightning$ \autoref{satz:H ist unentscheidbar}
% \end{proof}

 % Latex-Qelle von M. Geffken
Nun betrachten wir \ac{TM}s vom Blickwinkel der von ihnen berechneten
(partiellen) Funktionen. Sei $R$ die Menge der von \ac{TM}s berechneten
Funktionen.

\begin{Satz}[Satz von Rice]\ \\
  Sei $R$ die Menge aller partiellen \ac{TM}-berechenbaren Funktionen und
  $\varnothing \neq S \subsetneq R$ eine nichttriviale (nicht-leere,
  echte) Teilmenge davon.\\
  Dann ist $L(S)=\{\ulcorner M \urcorner \mid M \text{ berechnet Funkt. aus
  }S\}$ unentscheidbar.
\end{Satz}
\begin{proof}
  Angenommen $M_S$ entscheidet $L(S)$.\\
  Sei $\Omega \in R$ die überall
  undefinierte Funktion. Wir nehmen an, dass $\Omega \in S$ (anderenfalls betrachten wir $\overline{L(S)}$).
  
  Da $R \setminus S \neq \varnothing$ gibt es eine berechenbare Funktion $ f \in R \setminus
  S$ und $f$ werde von \ac{TM} $M_f$ berechnet.
  
  Definiere $M'=M'_{(M, f)}$ wie folgt: $M'$ führt zunächst $M$ (beliebige \ac{TM}) auf leerer Eingabe aus. Falls $M$ anhält, wendet $M'$ dann $M_f$ auf die tatsächliche Eingabe an.\\
  Die von $M'$ berechnete Funktion ist also
  $f_{M'}=\begin{cases}
    f & \text{falls } M \text{ auf leerem Band hält,}\\
    \Omega & \text{sonst.}
  \end{cases}$
  
  Definiere nun $M''$ wie folgt:
  \begin{itemize}
  \item Bei Eingabe $\ulcorner M \urcorner$ berechne die Gödelnummer von $M'$.
  \item Wende nun $M_S$ auf $\ulcorner M' \urcorner$ an.
    \begin{align*}
      M_s\text{ akzeptiert }\ulcorner M' \urcorner
      &\iff M' \text{ berechnet Funktion in }S\\
      &\iff M' \text{ berechnet }\Omega \text{ ($f_{M'} \in \{\Omega, f\}, \Omega \in S, f \notin S$ )}\\
      &\iff M \text{ hält \emph{nicht} auf leerem Band an}
    \end{align*}
  \end{itemize}
  Also entscheidet $M''$ $H_\varepsilon$.$\qquad\lightning$
\end{proof}

\subsection{Eigenschaften von entscheidbaren und semi-entscheidbaren Sprachen}
\begin{Satz}[name={[Eigenschaften von Entscheidbarkeit]}]
\label{thm:eigensch-von-entsch}
  Seien $L_1$ und $L_2$ entscheidbar. Dann sind $\overline{L_1}$, $\overline{L_2}$, $L_1 \cup L_2$ und $L_1 \cap L_2$ entscheidbar.
\end{Satz}\vspace{-1.5em}
\begin{proof}
  Übung oder selbst.
\end{proof}
\begin{Satz}[name={[Eigenschaften von Semi-Entscheidbarkeit]}]
  Seien $L_1$ und $L_2$ semi-entscheidbar. Dann sind $L_1 \cup L_2$ und $L_1 \cap L_2$ semi-entscheidbar.
\end{Satz}
\begin{proof}
  vgl. \autoref{satz:Typ-0-abgeschl}.
\end{proof}
\csname repeatSatz613\endcsname*
\begin{Satz}
  Die Menge der semi-entscheidbaren Sprachen ist \emph{nicht}
  unter Komplement abgeschlossen.
\end{Satz}
\begin{proof}
  Laut \autoref{satz:6.10} und \autoref{kor:6.11} sind das spezielle Halteproblem $K$ und $\overline{K}$ nicht entscheidbar.
  
  $K$ ist semi-entscheidbar, aber nicht $\overline{K}$.
\end{proof}


\subsection{Weitere unentscheidbare Probleme}
\paragraph[\acf*{PCP}]{\acf{PCP}}\ \\
\emph{Gegeben:}\\
Endliche Folge von Wortpaaren $K=((x_1, y_1), \dots, (x_k, y_k))$ mit $x_i, y_i \in \Sigma^+$

\emph{Gesucht:}\\
Indexfolge $i_1, \dots, i_n \in \{1, \dots, k\}\ (n \geq 1)$, so dass $x_{i_1} \cdots x_{i_n}=y_{i_1} \cdots y_{i_n}$

Die Folge $i_1, \dots, i_n$ (falls diese existiert) heißt \emph{Lösung} des Korrespondenzproblems $K$.
\begin{Bsp*}
\begin{align*}
	K &=((\underbrace{1,101}_{x_1, y_1}), (\underbrace{10, 00}_{x_2, y_2}), (\underbrace{011, 11}_{x_3, y_3}))\\
\shortintertext{besitzt die Lösung $(1,3,2,3)$, denn}
	x_1 x_3 x_2 x_3 &= \underbrace{1 \cdot 01}_{y_1}\underbrace{1 \cdot 1}_{y_3}\underbrace{0 \cdot 0}_{y_2}\underbrace{11}_{y_3} = y_1 y_3 y_2 y_3
\end{align*}
\end{Bsp*}\vspace{-1em}
\emph{Frage:}
\begin{align*}
  x_1&=001 & x_2&=01 & x_3&=01 & x_2&=10\\
  y_1&=0  & y_2&=011 & y_3&=101  & y_2&=001\\
\end{align*}
Besitzt dieses \ac{PCP} eine Lösung? [Schöning, S.124]
\begin{Bemerkung}\ \\
  Offensichtlich ist das \ac{PCP} semi-entscheidbar: Systematisches
  Ausprobieren von Indexfolgen findet Lösung nach endlicher Zeit,
  \emph{sofern es eine gibt}.
\end{Bemerkung}
Ziel: \ac{PCP} ist unentscheidbar. Vorbereitung:
Es interessiert uns ab hier nur, ob das Problem eine Lösung hat oder nicht.
\paragraph[\acf*{MPCP}]{\acf{MPCP}}\ \\
\emph{Gegeben:} wie bei \ac{PCP}

\emph{Gesucht:} Lösung des \acsu{CP} mit $i_1=1$
\begin{lemma}[name={[MPCP $\preceq$ PCP]}]
    \ac{MPCP} $\preceq$ \ac{PCP}
\end{lemma}
\begin{proof}
	Betrachte \ac{MPCP} $K=((x_1,y_1),\dots(x_k,y_k))$ über $\Sigma$.\\
	Sei $\Sigma'=\Sigma\cup\{\#,\$\}$ mit neuen Symbolen.\\
	Für ein Wort $w=a_1\dots a_n\in\Sigma^+$ sei
	\begin{align*}
		\bar w &= \#a_1\#a_2\#\dots\#a_n\#\\
		\grave w &= \#a_1\#a_2\#\dots\#a_n &&\text{(am Ende kein \#)}\\
		\acute w &= \phantom{\#}a_1\#a_2\#\dots\#a_n\# &&(\text{am Anfang kein }\#)
	\end{align*}
	Definiere nun
	\[ f(K) = (\underbrace{(\bar x_1,\grave y_1)}_1, \underbrace{(\acute x_1, \grave y_1)}_2, \underbrace{(\acute x_2,\grave y_2)}_{2+1}, \dots, \underbrace{(\acute x_k,\grave y_k)}_{k+1}, \underbrace{(\$,\#\$)}_{k+2}) \]
	eine totale berechenbare Funktion.
	
	Zeige $K \in \ac{MPCP} \<==> f(K) \in \ac{PCP}$:
	
	"`$\==>$"': $1, i_2, \dots, i_n$ Lösung für $K$\\
	$\curvearrowright\ 1, i_{2}+1, \dots, i_{n}+1, k+2$ Lösung für $f(K)$
	
	"`$\<==$"': $i_1, \dots, i_n$ Lösung für $f(K)$\\
	$\curvearrowright i_1=1,\ i_n=k+2 \qquad,\ 1 < j < n: i_j \in \{2, \dots, k+1\}$\\
	$\curvearrowright 1, i_{2}-1, \dots, i_{n-1}-1$ Lösung für $K$
\end{proof}
Es reicht nun zu zeigen, dass \ac{MPCP} unentscheidbar ist!

\begin{lemma}[name={[H $\preceq$ \ac{MPCP}]}] 
	H $\preceq$ \ac{MPCP}
\end{lemma}
\begin{proof}
	\ac{TM} $M=(Q,\Sigma,\Gamma,\delta,q_0,\blank,F)$ und Eingabewort $w\in\Sigma^*$.
	
	Gesucht: totale berechenbare Funktion, die $(\ulcorner M \urcorner, w) \mapsto \underbrace{(x_1,y_1),\dots,(x_k,y_k)}_k$, sodass\\
	$\ulcorner M \urcorner w\in H \<=> K$ eine Lösung als \ac{MPCP} besitzt.
	
	Idee: Definiere $K$ so, dass die Berechnung von $M$ simuliert wird.\\
	Alphabet für $K: \triangle = \Gamma\cup Q\cup\{\#\}$\\
	$(x_1,y_1) = (\#,\#q_0w\#)$
	\begin{enumerate}
	\item Kopieren
		\[ (a,a)\quad, a\in\Gamma\cup\{\#\} \]
	\item Transition
		\begin{alignat*}{2}
			(qa,q'a') &\quad& \forall q,a:\delta(q,a) &\ni(q',a',N)\\
			(qa,a'q') && \dots\quad &\ni(q',a',R)\\
			(bqa,q'ba') &&  &\ni(q',a',L),b\in\Gamma\\
			(q\#,q'a'\#) && \forall q: \delta(q,\blank)&\ni(q',a',N)\\
			(q\#,a'q'\#) && &\ni(q',a',R)\\
			(bq\#,q'ba'\#) && &\ni(q',a',L),b\in\Gamma\\
			(\#qa,\#q'\blank a')&& \forall q,a:\delta(q,a) &\ni(q',a',L)\\
			(\#q\#,\#q'\blank a'\#)&& \forall q:\delta(q,\blank) &\ni(q',a',L)
		\end{alignat*}
	\item Löschen
		\begin{align*}
			\forall q\in \Gamma\\
			\forall q\in F &:(aq,a)\\
			&\phantom{{}:{}}(qa,a)
		\end{align*}
	\item Abschluss
		\[ \forall q\in F: (q\#\#,\#) \]
	\end{enumerate}
	$\ulcorner M \urcorner w\in H \<==>$ Folge von $\Konf$ von $M,\ k_0\dots k_t$ mit $k_0 =q_0w$ und $k_t = uq'v$ mit $q'\in F$\\
	mit $h_{i-1}\vdash k_i\quad \forall 1\leq i\leq t$
	\<==> Die Instanz $K$ von \ac{MPCP} besitzt Lösung und ein Lösungswort der Form
	\[ \#k_0\#k_1\# \dots \#h_t\#k_t^1\#k_t^2\# \dots \#q'\#\# \]
	wobei die $k_t^j$ durch Streichen eines Bandsymbols rechts oder links von $q'$ aus ihrem Vorgänger entsteht.
\end{proof}
\begin{Satz}[name={[\ac{PCP} ist unentscheidbar.]}]
	\ac{PCP} ist unentscheidbar.
\end{Satz}
\begin{proof}
	$\text{H}\leq\ac{MPCP}$ und $\ac{MPCP}\leq\ac{PCP}$
\end{proof}


\begin{Satz}
	Das Schnittproblem "`$L(\mathcal{G}_1)\cap L(\mathcal{G}_2)\neq \varnothing$?"' für \ac{CFL} ist unentscheidbar.
\end{Satz}
\begin{proof} Durch Reduktion $\ac{PCP}\leq{}$Schnittproblem.\\
	Sei $K=\{(x_i,y_i) \mid 1\leq i\leq k\}$ Instanz von \ac{PCP} über $\Sigma$.\\
	Berechne aus $K$ zwei \ac{CFG} $\mathcal{G}_1$ und $\mathcal{G}_2$, so dass $K$ eine Lösung hat $\<==> L(\mathcal{G}_1)\cap L(\mathcal{G}_2)\neq \varnothing$
	\begin{align*}
		\mathcal{G}_1: S_1\-> &1x_1 |\dots |kx_k &&\text{Alphabet}: \Sigma\cup\{1\dots k\}\\
		& |1S_1x_1|\dots|kS_1x_k\\
		\mathcal{G}_2: S_2\-> &1y_1 |\dots |ky_k\\
		&|1S_2x_1|\dots|kS_2y_k
	\end{align*}
	\begin{alignat*}{2}
		&&w&\in L(\mathcal{G}_1)\cap L(\mathcal{G}_2)\\
		\<==>&\ &  w &= k_n\dots k_1,xk_1\dots xk_n\\
		&&&= k_n\dots k_1,yk_1\dots yk_n\\
		\<==>&& \mathrlap{\hspace{-1ex}(k_1\dots k_n)\text{ ist Indexfolge zur Lösung von \ac{PCP} } k} \tag*{\qedhere}
	\end{alignat*}
\end{proof}
\paragraph{Folgerung :}Schnittproblem für Typ 1 und Typ 0 Sprachen ist ebenfalls unentscheidbar.

\begin{Korollar}
	Das Schnittproblem ist auch für \ac{DCFL} unentscheidbar.
\end{Korollar}
\begin{proof}
	$L(G_1)$ ist auch \ac{DPDA} erkennbar.
\end{proof}

\begin{Satz}
	Das Äquivalenzproblem für \ac{CFL} ist unentscheidbar.
\end{Satz}
\begin{proof}
	Sei $A=\{\mcG_1,\mcG_2 \mid L(\mcG_1)=L(\mcG_2)\}$\\
	Angenommen $\mcG_1,\mcG_2$ sind Typ 2 Grammatiken für \ac{DCFL}.\\
	Dann ist $(\mcG_1,\mcG_2)\in{}$Schnittproblem.
	\begin{align*}
		\<==> L(\mcG_1) &\cap L(\mcG_2)=\varnothing\\
		\<==> L(\mcG_1) &\subseteq \overline{L(\mcG_2)}
	\end{align*}
	Da $\mcG_2$ eine \ac{DCFG} $\exists\,\mcG_2$ mit $L(\mcG'_2)=\overline{L(\mcG_2)}$ (Abschluss unter Komplement).
	\begin{align*}
		\<==>\quad & L(\mcG_1) \subseteq L(\mcG'_2) \quad\leadsto\text{Inklusionsproblem} \tag{$*$}\label{eq:Inklusionsproblem}\\
		\<==>\quad & L(\mcG_1) \cup L(\mcG'_2) = L(\mcG'_2)\\
	\shortintertext{Wegen Abschluss unter $\cup:\exists\,\mcG_3\in\ac{CFG}$ mit $L(\mcG_3)=L(\mcG_1)\cup L(\mcG'_2)$}
		\<==>\quad & L(\mcG_3)=L(\mcG'_2)\\
		\<==>\quad & (\mcG_3,\mcG'_2)\in A
	\end{align*}
	$\curvearrowright$ Äquivalenzproblem ist unentscheidbar.\\
	\eqref{eq:Inklusionsproblem} \-> (Inklusionsproblem ist ebenfalls unentscheidbar.)
\end{proof}

\begin{Satz}
	Das Leerheitsproblem für Typ 1 Sprachen ist unentscheidbar.
\end{Satz}
\begin{proof}
	Reduktion auf Schnittproblem für \ac{CFL}.\\
	Sei $(\mcG_1,\mcG_2)\in{}$Schnittproblem (Typ 1).\\
	Insbesondere $\mcG_1,\mcG_2$ Typ 1 Grammatiken.\\
	Typ 1 Sprachen sind unter $\cap$ abgeschlossen, also $\exists\, \mcG$ Typ 1 Gramatik mit $L(\mcG)=L(\mcG_1)\cap L(\mcG_2)$\\
	Also "`$L(\mcG)=\varnothing$"' unentscheidbar.
\end{proof}


%%% Local Variables:
%%% mode: latex
%%% TeX-master: "Info_3_Skript_WS2016-17"
%%% End:

\section[Komplexitätstheorie]{Komplexitätstheorie}
\subsection{Komplexitätsklassen und \ac{P}/\ac{NP}}

{\color{red} TODO: Dieser rote Teil des Skripts wir erst im Lauf des Wochenendes in einen akzeptablen Zustand gebracht.

Aus Info2 kennen Sie: Worst-case Laufzeit eines Algorithmus bestimmen.

Hier: Betrachte Problemstellung und bestime was Wort-case Laufzeit von bestmöglichem Algorithmus ist.

}


\begin{Def}
Sei $f:\N\rightarrow\N$ eine Funktion und $\M$ eine \ac{NTM} mit Eingabealphabet $\Sigma$.
\begin{itemize}
 \item $\M$ hat \emph{Zeitkomplexität} $f(n)$, falls $\forall w\in\Sigma^*$ mit Länge $n$ gilt: $\M$ hält auf Eingabe $w$ für jede Berechnung in höchstens $f(n)$ Schritten.
 \item $\M$ hat \emph{Platzkomplexität} $f(n)$, falls $\forall w\in\Sigma^*$ mit Länge $n$ gilt: Wenn $\M$ auf $w$ angesetzt wird, benutzt jede Berechnung höchstens $f(n)$ Bandzellen.
 \qedhere
\end{itemize}
\end{Def}


\begin{Def}[name={[$\NTIME$ Klasse]}]
	Sei $f:\N\->\N$ eine Funktion.
	\begin{align*}
	\DTIME(f(n)) = \{L \subseteq\Sigma^*\mid & \text{ Es gibt det. Mehrband-\ac{TM} $\M$, sodass $L(\M)=L$ und }\\
	                & \text{$\M$ hat Zeitkomplexität $f(n)$} \}\\
%     \end{align*}
%     \begin{align*}
    \NTIME(f(n)) = \{L \subseteq\Sigma^*\mid & \text{ Es gibt nichtdet. Mehrband-\ac{TM} $\M$, sodass $L(\M)=L$ und }\\
                    & \text{$\M$ hat Zeitkomplexität $f(n)$} \}\\           
%     \end{align*}
% 	\begin{align*}
	\DSPACE(f(n)) = \{L \subseteq\Sigma^*\mid & \text{ Es gibt det. Mehrband-\ac{TM} $\M$, sodass $L(\M)=L$ und }\\
	                & \text{$\M$ hat Platzkomplexität $f(n)$} \}\\
%     \end{align*}
%     \begin{align*}
    \NSPACE(f(n)) = \{L \subseteq\Sigma^*\mid & \text{ Es gibt nichtdet. Mehrband-\ac{TM} $\M$, sodass $L(\M)=L$ und }\\
                    & \text{$\M$ hat Platzkomplexität $f(n)$} \}\qedhere
    \end{align*}	

% 	Die Klasse $\NTIME(f(n))$ besteht aus allen Sprachen, die von einer (Mehrkanal-)\ac{TM} $M$ in $T_M(w)\leq f(|w|)$ akzeptiert werden.\\
% 	Dabei $T_M(w) =
% 	\begin{cases}
% 		\mathrlap{\text{Anzahl der Schritte einer kürzesten akzeptierenden Berechnung von $M$ auf }w}\\
% 		1 & \text{falls }\nexists
% 	\end{cases}$\\
\end{Def}

{\color{red}

Diagramm: Raute mit Inklusionen

Erklärung dazu.

% \bigskip
% 
% Probleme als Sprachen.
% 
% Bsp: Erfüllbarkeitsproblem der Aussagenlogik.
% 
% \begin{Def}[name={[\ac{SAT}: Erfüllbarkeitsproblem der \acs*{AL}]}]
% 	\ac{SAT}, das Erfüllbarkeitsproblem der \acf{AL} ist  definiert durch
% 	\begin{description}
% 	\item[Eingaben:] Formal $F$ der \acl{AL}.
% 	\item[Frage:] Ist $F$ erfüllbar, d.h. existiert eine Belegung $\beta$ der Variablen mit $\{0,1\}$, sodass $F[\beta]=1$ ist.
% 	\end{description}
% 	$\ac{SAT}=\{\code(F) \mid F\text{ist erfüllbare Formel der \acs{AL}}\}$
% \end{Def}

}

Wir wollen im Folgenden Probleme als Sprachen darstellen und beginnen zunächst mit dem Erfüllbarkeitsproblem der Aussagenlogik.

\goodbreak

\textbf{Aussagenlogische Formeln als Sprachen}

Wir gehen davon aus, dass die Leser des Skripts mit \ac{AL}\footnote{%
Eine Einführung in Aussagenlogik finden Sie im Skript zur Vorlesung \emph{Logik für Studierende der Informatik} aus dem WS 2017/18.
Wir werden in dieser Vorlesung eine sehr ähnliche Notation verwenden.
\url{http://home.mathematik.uni-freiburg.de/junker/ws17/logik-info.html}
}
vertraut sind, stellen die von uns verwendete Syntax vor und bitten den Leser, sich die entsprechende Semantik zu erschließen.

Im Folgenden möchten wir Mengen von \ac{AL}-Formeln als Sprachen beschreiben.
Dafür müssen wir zunächst ein geeignetes Alphabet wählen.
Dabei nehmen wir ``0'' und ``1'' für die Konstanten,
``$\neg$'', ``$\land$'' und ``$\lor$ für die Junktoren
und runde Klammern ''$($`` und ''$)$``.
Bei der Darstellung der Aussagenvariablen stehen wir nun vor der folgenden Herausforderung:
Es gibt unendliche viele Aussagenvariablen $A_0,A_1,A_2,A_3,\ldots$, aber unser Alphabet kann nur endlich viele Zeichen haben.
Unsere Lösung dafür ist sehr ähnlich wie die, die wir auch auf Papier verwenden.
Wir stellen eine Aussagenvariable durch ein ''$A$`` dar, das von einer Ziffernfolge gefolgt wird.
Die Aussagenvariable $A_{1337}$ wir so durch das Wort $A1337$ der Länge 5 repräsentiert.
Wir müssen unser Alphabet also noch um ''$A$``, ''$2$``, $\ldots$ ''$9$`` ergänzen und erhalten
$$\Sigma_\mathsf{AL}=\{0,1,2,3,4,5,6,7,8,9,A,\neg,\land,\lor,(,)\}.$$

\begin{Def}\label{def:gal}
Die Menge der \ac{AL}-Formeln ist die Sprache der kontextfreien Grammatik
  $\mathcal{G}_\mathsf{AL} = (\Sigma_\mathsf{AL}, N, P, S)$ mit
	\begin{align*}
		N &= \{S,Z\}\\
		P &= \begin{aligned}[t]
      \{ S \to\ & 0\mid 1\mid AZ\mid \neg S\mid (S\land S)\mid (S\lor S)\\
        Z \to\ & 0Z\mid 1Z\mid 2Z\mid 3Z\mid 4Z\mid 5Z\mid 6Z\mid 7Z\mid 8Z\mid 9Z \mid \\
        &0 \mid 1 \mid 2 \mid 3 \mid 4 \mid 5 \mid 6 \mid 7 \mid 8 \mid 9\}.
        \end{aligned}
      \qedherefixaligned
	\end{align*}
\end{Def}

Wir nehmen nun unsere Definition von Formeln, um ein bekanntes Problem mit Hilfe einer Sprache zu beschreiben.
\begin{Def}
Wir nennen die Menge der erfüllbaren \ac{AL}-Formeln \acsu{SAT}.
\[ \ac{SAT}=\{F\mid F\in L(\mathcal{G}_\mathsf{AL}) \text{ und $F$ erfüllbar}\}\qedhere \]
\end{Def}


Wir haben nun also die Frage, ob die Formel $F$ erfüllbar ist, als Wortproblem ''$F\in \ac{SAT}?$`` dargestellt.
Wir wollen im Folgenden eine \ac{TM} konstruieren, die \ac{SAT} entscheidet, betrachten aber als Vorstufe hierfür zunächst eine einfachere Sprache.


Wir definieren uns $\mathcal{G}_\mathsf{AL0}$ als die obige Grammatik, in der die Regel $S\to AZ$ fehlt.
Offensichtlich enthält $L(\mathcal{G}_\mathsf{AL0})$ genau die \ac{AL}-Formeln, die keine Aussagenvariablen enthalten.

Weiter definieren wir 
$$\text{SAT}_0=\{F\mid F\in L(\mathcal{G}_\mathsf{AL0}) \text{ und $F$ erfüllbar}\}$$
und konstruieren eine \ac{DTM}, die $\text{SAT}_0$ entscheidet.

{\color{red}
TODO: 

\begin{itemize}
 \item beschreibe Konstruktion
 \item Frage: Welche Laufzeitkomplexität hat diese \ac{DTM}?
 \item Antwort: Wir können es nicht genau sagen, die informelle Beschreibung ist zu unpräzise um eine genaue Funktion zu nennen.
\end{itemize}

}

{\color{red}


Exkurs
\begin{Def} Sei $g:\N\rightarrow\N$
\[ \mathcal{O}(g(n)) = \{f:\N\rightarrow\N \mid \exists n_0, k\in\N \forall n\geq n_0: f(n)\leq k\cdot g(n)\} \qedhere \]
\end{Def}

Anschaulich: $f(n)\in \mathcal{O}(g(n))$ $f$ wächst garantiert nicht viel stärker als $g$.


TODO: 

\begin{itemize}
 \item beschreibe Konstruktion von \ac{NTM} für \ac{SAT}
 \item Frage: Welche Laufzeitkomplexität hat diese \ac{NTM}?
 \item \ac{DTM}s?
\end{itemize}


\begin{Def}[name={[Polynom]}]
	Ein Polynom ist eine Funktion $p:\N\->\N$ mit $\exists k\in \N\ a_0,\dots,a_k\in\N$ und \mbox{$p(n)=\sum_i^k a_in^k$}
\end{Def}
\begin{Def}
 \begin{align*}
  \ac{P} &= \bigcup_{p\text{ Polynom}} \DTIME(p(n)) \\
  \ac{NP} &= \bigcup_{p\text{ Polynom}} \NTIME(p(n))
  \qedhere
 \end{align*}
\end{Def}



% \begin{Def}[name={[NP-Klasse]}]
% 	Die Klasse \ac{NP} besteht aus allen Sprachen, die von \ac{NTM} in polynomieller Zeit akzeptiert werden können.
% 	\[ \ac{NP} = \cup_{p\text{ Polynom}} \NTIME(p(n)) \]
% \end{Def}
% Analog für deterministische TM:
% \begin{Def}[name={[$\DTIME$ Klasse]}]
% 	Sei $f:\N\->\N$ Funktion\\
% 	$\DTIME(f(n)) =$ Klasse der Sprachen, die von \ac{DTM} in $T_M(w)\leq f(|w|)$ Schritten akzeptiert wird.
% 	\[ \ac{P} = \cup_{p\text{ Polynom}} DTIME(P(n)) \qedhere \]
% \end{Def}
Offenbar gilt $P\subseteq \ac{NP}$. Seit 1970 weiß man nicht, ob $P=\ac{NP}$ oder $P\neq \ac{NP}$ gilt.
\begin{description}
\item[Praktische Relevanz:] Es existieren wichtige Probleme, die offensichtlich in \ac{NP} liegen, aber die besten bekannten Algorithmen sind exponentiell.\\
	Beispiel: Traveling Salesman ($O(2^n)$), Erfüllbarkeit der Aussagenlogik.
\item[Struktur:] Viele der \ac{NP}-Probleme haben sich als gleichwertig erwiesen, in dem Sinn, dass eine \ac{P}-Lösung für alle anderen liefert.\\
	$\leadsto$ \ac{NP}-Vollständigkeit.
\end{description}
}

\begin{Def}[Polynomielle Reduktion]\label{def:PolyReduktion}\ \\
  Seien $U, V \subseteq \Sigma^*$ Sprachen.
  \emph{$U$ ist auf $V$ polynomiell reduzierbar (Notation: $U \preceq_p V$)}, falls eine totale, berechenbare Funktion
  $f:\Sigma^* \to \Sigma^*$ existiert, sodass
  \begin{itemize}
   \item $\forall w \in \Sigma^*:w \in U \iff f(w) \in V$
   \item $f$ wird von einer \ac{DTM} berechnet, deren Laufzeitkomplexität ein Polynom ist.
  \end{itemize}
  Wir nennen $f$ \emph{Reduktionsfunktion}.
\end{Def}

\datenote{02.02.2018}
Eine aussagenlogische Formel $F$ ist in \emph{konjunktiver Normalform} (CNF)\footnote{%
Verwechslungsgefahr: \ac{CNF} stand bisher immer für \acl{CNF}.
In diesem Kapitel werden wir diese aber nicht benötigen.
}, wenn $F$ eine Konjunktion von Disjunktionen von Literalen ist.
Wir verwenden \emph{CNF} als Notation für die Menge aller AL-Formeln in CNF.
Analog schreiben wir \emph{3CNF} für die Menge aller Formeln in CNF, bei denen jeder Konjunkt aus höchstens drei Disjunkten besteht.



\begin{Def}[name={[3SAT]}]
Das Problem \acsu{3SAT} ist wie folgt definiert.\footnote{
Typischerweise nennt man Sprachen, die im Kontext der Komplexitätstheorie definiert werden, \emph{Probleme}.
Es ist üblich, von einem konkreten Alphabet $\Sigma$ und einer konkreten Codierung der Objekte (hier: Formeln) zu abstrahieren und das Problem als (Gegeben, Frage)-Paar zu formulieren.
Es bleibt dem Leser überlassen, sich selbst geeignete Alphabete und Codierungen zu überlegen.
Für den Fall von aussagenlogische Formeln haben wir dies in {\color{red} TODO} noch einmal gemacht, werden aber in Zukunft darauf verzichten.}
\begin{center}
\framebox[\textwidth]{\parbox{.95\textwidth}{
\smallskip
\textit{Gegeben:} Eine aussagenlogische Formel $F\in$ \emph{3CNF}

\medskip

\textit{Frage:} Ist $F$ erfüllbar?
}}
\end{center}
	
\end{Def}
Alternativ könnten wir die Definition von \ac{3SAT} auch wie folgt aufschreiben.
$$\ac{3SAT} = \{F\in L(\mathcal{G}_\mathsf{AL})\mid F \text{ ist in 3CNF}\}$$

Offensichtlich gilt $\ac{3SAT} \preceq_p \ac{SAT}$.\footnote{
In der Vorlesung vom 02.02. wurde (fälschlicherweise) gesagt, die Identität sei eine geeignete Reduktionsfunktion.
Dies ist nicht korrekt, da z.B. 
$(A_1\lor A_2\lor A_3\lor A_4)\notin \ac{3SAT}$
(weil nicht in 3CNF), aber $(A_1\lor A_2\lor A_3\lor A_4)\in \ac{SAT}$.
} Wir zeigen als Nächstes auch die umgekehrte Richtung.

\begin{lemma}\label{lem:sat3sat}
	$\ac{SAT} \preceq_p \ac{3SAT}$
\end{lemma}


\begin{proof}
Unser Ziel ist es nun, eine Transformation anzugeben, die sich in polynomieller Zeit berechnen lässt und jede aussagenlogische Formel $F_\mathsf{AL}$ in eine 3CNF-Formel $F_\mathsf{3CNF}$ überführt, sodass
$F_\mathsf{AL}\in \ac{SAT} \<==> F_\mathsf{3CNF}\in \ac{3SAT}$
gilt.

Sei $F_\mathsf{AL}$ eine beliebige Formel aus $L(\mathcal{G}_\mathsf{AL})$.

\begin{enumerate}
 \item Erzeuge aus $F_\mathsf{AL}$ eine äquivalente Formel $F_\mathsf{NNF}$ in Negationsnormalform\footnote{Eine Formel ist in Negationsnormalform, wenn der Negationsoperator immer nur direkt vor einer Variable vorkommt.}.
 Wir können jede Formel in Negationsnormalform bringen, indem wir mit Hilfe der De Morganschen Regeln die Negationen "`nach innen"' ziehen.
 
 Beispiel: $\neg(\neg (A_1\lor \neg A_3)\lor A_2) \quad\rightsquigarrow\quad ((A_1\lor \neg A_3)\land \neg A_2)$
 
 Ideen zur Implementierung: Erstelle auf einem zusätzlichen Band eine Kopie der Formel.
 Lasse vor jedem nicht negierten Literal ein Feld Platz, um später ggf.\ ein Negationssymbol zu platzieren.
 Wende De Morgans Regel von außen nach innen an.
 Laufe für jede Anwendung einmal über die Formel.
 
 
 \item Erzeuge aus $F_\mathsf{NNF}$ eine Formel $F_{\alpha\gamma}$ mit Biimplikationszeichen "`$\<->$"' mit Hilfe der folgenden induktiv definierten Abbildungen $\alpha$ und $\gamma$.\footnote{%
 Die hier beschriebene Transformation ist auch als Tseytin-Transformation bekannt.}
 
 Idee: Die Abbildung $\gamma$ liefert für jede $\land$-Teilformel und jede $\lor$-Teilformel eine neue Hilfsvariable.
 Diese Hilfsvariable hat in der resultierenden Formel den Wahrheitswert, den die entsprechende Teilformel in der Eingabeformel hätte.
 Die Abbildung $\alpha$ konstruiert für jede Teilformel die entsprechenden Bedingungen für die Hilfsvariable.
 Erzeugt $\gamma$ keine Hilfsvariable, so erzeugt $\alpha$ die (nicht einschränkende) Bedingung~$1$.
 
 $$\gamma(F)=
 \begin{cases}
   0 & \text{ falls } F=0\\
   1 & \text{ falls } F=1\\
   A & \text{ falls } F=A\\
   \neg F_1 & \text{ falls } F=\neg F_1\\
   B_F & \text{ falls } F=F_1\land F_2\\
   B_F & \text{ falls } F=F_1\lor F_2
  \end{cases}$$
 
 
 
 $$\alpha(F)=
 \begin{cases}
   1 & \text{ falls } F=0\\
   1 & \text{ falls } F=1\\
   1 & \text{ falls } F=A\\
   1 & \text{ falls } F=\neg F_1\\
   \big(\gamma(F)\<-> \gamma(F_1) \land \gamma(F_2)\big)\land\alpha(F_1)\land\alpha(F_2) & \text{ falls } F=F_1\land F_2\\
   \big(\gamma(F)\<-> \gamma(F_1) \lor \gamma(F_2)\big)\land\alpha(F_1)\land\alpha(F_2) & \text{ falls } F=F_1\lor F_2
  \end{cases}$$
  
  Wir definieren das Resultat $F_{\alpha\gamma}:=\gamma(F_\mathsf{NNF})\land \alpha(F_\mathsf{NNF})$.
  
  Beispiel:
  \begin{align*}
   \alpha((A_1\lor \neg A_3)\land \neg A_2)) = & \;
  \big(B_{((A_1\lor \neg A_3)\land \neg A_2)} \leftrightarrow (B_{(A_1\lor \neg A_3)} \land \neg A_2)\big)\\
  & \land (B_{(A_1\lor \neg A_3)} \leftrightarrow (A_1\lor \neg A_3)) \land 1 \land 1\\
  & \land 1
  \end{align*}

  
  Ideen zur Implementierung:
  
  Wähle ein Bandalphabet, sodass das Zeichen $B$ enthalten ist.
  Verwende außerdem eine zusätzliche Art von Klammern (z.B. eckige Klammern), um den Subskriptanteil der B-Variablen vom restlichen Bandinhalt zu unterscheiden.
  Laufe für jede $\{\land,\lor\}$-Teilformel einmal über die Eingabe.
  Verwende ein zusätzliches Band zum Schreiben des Resultats.
  Verwende noch ein zusätzliches Band für die aktuell bearbeitete $\{\land,\lor\}$-Teilformel, da diese immer zweimal benötigt wird (Variablenname und $\alpha$).
  Bearbeite äußere Teilformeln vor inneren.
  Lösche Teile, die nicht mehr benötigt werden.
\item Ersetze die Formelteile mit Biimplikationszeichen in $F_{\alpha\gamma}$ wie folgt durch logisch äquivalente Formeln in 3CNF.
	\begin{itemize}
	\item $F_1\<-> (F_2\land F_3)
% 		= \big(F_1\land (F_2\land F_3)\big)\lor \big(\neg F_1\land \neg(F_2\land F_3)\big)
% 		= \big(F_1\lor \neg(F_2\land F_3)\big)\land \big(\neg F_1\lor (F_2\land F_3)\big)
		\quad\rightsquigarrow\quad \big(F_1\lor \neg F_2\lor \neg F_3\big)\land \big(\neg F_1\lor F_2\big)\land \big(\neg F_1\lor F_3\big)
		$
	\item $F_1\<-> (F_2\lor F_3)
% 		= \big(F_1\lor \neg(F_2\lor F_3)\big)\land \big(\neg F_1\lor (F_2\lor F_3)\big)
		\quad\rightsquigarrow\quad \big(F_1\lor \neg F_2\big)\land\big(F_1\lor \neg F_3\big)\land \big(\neg F_1\lor F_2\lor F_3\big)
$
	\end{itemize}
	
	Ideen zur Implementierung:
	
	Verwende ein zusätzliches Band zum Schreiben des Resultats.
	Verwende noch ein zusätzliches Band für die Operanden der Biimplikation, da diese immer mehrfach benötigt werden.
  
\item Ersetze alle aussagenlogischen Variablen der Form $B_F$ durch aussagenlogische Variablen der Form $A_i$.\footnote{%
Dieser Schritt ist nur nötig, damit das Resultat in dem von uns definierten Alphabet $\Sigma_{AL}$ dargestellt werden kann.
Alternativ hätten wir auch zu Beginn ein reichhaltigeres Alphabet wählen können.
}

	Ideen zur Implementierung:
	
    Finde zunächst den höchsten Index $i_{max}$ von $A_i$-Variablen (speichere aktuelles Maximum auf zusätzlichem Band).
    Verwende anschließend $i_{max}+1, i_{max}+2, \ldots$ für neue Variablen.
    Schreibe zunächst eine Übersetzungsvorschrift (z.B.\ $A_4:=B_{((A_1\lor \neg A_3)\land \neg A_2)}, A_5:=B_{(A_1\lor \neg A_3)}$) auf ein zusätzliches Band und ersetze erst dann.

\end{enumerate}

Sei $f:\Sigma_{AL}\rightarrow\Sigma_{AL}$ eine Funktion, die alle Formeln aus $L(\mathcal{G}_\mathsf{AL})$ entsprechend obiger Konstruktion abbildet und alle anderen Wörter unverändert lässt. Dann gilt:
\begin{itemize}
 \item $F_{AL}\in \ac{SAT} \<==> f(F_{AL})\in \ac{3SAT}$ (hier ohne Beweis)
 \item $f$ ist total und lässt sich in polynomieller Zeit berechnen (hier ohne Beweis).
 \qedhere
\end{itemize}
\end{proof}



\begin{lemma}\label{lem:A<B + BinP => AinP}
	Falls $A\preceq_p B$ und $B\in \ac{P}$ (bzw.\ $B\in \ac{NP}$), dann gilt auch $A\in \ac{P}$ (bzw.\ $A\in \ac{NP}$).
\end{lemma}
\begin{proof}
	$B\in \ac{P}$: Nach Annahme gibt es eine \ac{TM} $\M$, die $B$ in $p(n)$ Schritten akzeptiert.\\
	Es gibt außerdem eine \ac{TM} $\M_f$, welche die Reduktion $A\preceq_p B$ implementiert.
	Die Laufzeit von $\M_f$ sei durch das Polynom $q$ beschränkt.\\
	Betrachte $\M'$ = "`erst $\M_f$, dann $\M$ auf dem Ergebnis"'.
	$\M'$ akzeptiert $A$.\\
	Sei $w\in A$.
	$\M_f(w)$ liefert $f(w)$ in höchstens $q(|w|)$ Schritten mit $|f(w)|\leq q(|w|) + |w|$.\\
	$\M$ angesetzt auf $f(w)$ benötigt höchstens $p(|f(w)|)\leq p(q(|w|) + |w|)$ Schritte zum Akzeptieren.\\
	Insgesamt gilt also $A\in \DTIME(q(|w|) + |w|+p(q(|w|) + |w|)\subseteq \ac{P}$.
\end{proof}


% \draftnote{8.2.17}

\begin{Def}[name={[\ac{NP}-schwierig und \ac{NP}-vollständig]}]\
	\begin{itemize}
	\item Eine Sprache $U$ heißt \emph{\ac{NP}-schwierig}, falls $\forall L\in \ac{NP}$ gilt: $L\preceq_p U$.
	\item Eine Sprache $U$ heißt \emph{\ac{NP}-vollständig}, falls $U$ \ac{NP}-schwierig ist und $U\in \ac{NP}$ gilt. \qedhere
	\end{itemize}
\end{Def}

{\color{red}
TODO: Ausformulieren
\begin{itemize}
 \item \ac{NP}-schwierig, sehr starke Forderung dass an alle(!) \ac{NP}-Probleme reduzieren kann
 \item zunächst unklar ob es überhaupt ein \ac{NP}-schwieriges Problem gibt
 \item bedeutet: wenn ich das \ac{NP}-vollst. Problem gefunden habe dass ich effizient Lösen kann, kann ich alle \ac{NP}-Probleme effizient lösen (wie folgender Satz zeigt).
\end{itemize}
}
\begin{Satz}
	Wenn eine Sprache $A$ \ac{NP}-vollständig ist, dann gilt die folgende Äquivalenz.
	\[ A\in \ac{P} \<==> \ac{P} =\ac{NP} \qedhere \]
\end{Satz}
\begin{proof}\ \\
	"`\<=="' trivial.\\
	"`\==>"' Es gilt $A\in \ac{P} \subseteq \ac{NP}$. Da $A$ \ac{NP}-vollständig ist, gilt $\forall L\in \ac{NP}: L\preceq_p A$. Dann folgt mit \autoref{lem:A<B + BinP => AinP} auch $L\in \ac{P}$.
\end{proof}

\begin{lemma}[name={[$\preceq_p$ ist reflexiv und transitiv]}]
	$\preceq_p$ ist reflexiv und transitiv.
\end{lemma}
\begin{proof}
Übungsblatt~14, Aufgabe~1.
% 	Identität; ähnlich wie Beweis von \autoref{lem:A<B + BinP => AinP}.
\end{proof}

\begin{Bem}
    Aus der Transitivität folgt:
	Sobald ein \ac{NP}-schwieriges Problem $U$ bekannt ist, reicht es $U\preceq_p V$ zu zeigen, um zu beweisen, dass $V$ ebenfalls \ac{NP}-schwierig ist.
\end{Bem}
% Ein erstes \ac{NP}-vollständiges Problem.

\begin{Satz}[Cook]\label{satz:cook}
	\ac{SAT} ist \ac{NP}-vollständig.
\end{Satz}
Wir werden nur für \ac{SAT} die \ac{NP}-Schwierigkeit direkt beweisen und für alle anderen \ac{NP}-vollständigen Probleme die \ac{NP}-Schwierigkeit wie in obiger Bemerkung angedeutet mit Hilfe einer polynomiellen Reduktion zeigen.
Da der Beweis von \autoref{satz:cook} sehr umfangreich ist, wollen wir diesen etwas aufschieben und zunächst die \ac{NP}-Vollständigkeit weiterer Probleme zeigen.


\begin{Satz}[name={[3SAT ist \ac{NP}-vollständig]}]
	\ac{3SAT} ist \ac{NP}-vollständig.
\end{Satz}
\begin{proof}
 \
 \begin{itemize}
  \item $\ac{3SAT} \in \ac{NP}$
  
  Wir können dies auf zwei Arten zeigen. 
  Entweder wir zeigen direkt, dass es eine \ac{NTM} gibt, die \ac{3SAT} in polynomieller Zeit entscheidet, oder
  wir zeigen dies indirekt mit Hilfe einer polynomiellen Reduktion auf ein Problem in \ac{NP}.
  Wir wählen letzteres Vorgehen, da zum Beweis nur nochmal erwähnt werden muss, dass $\ac{SAT} \in \ac{NP}$ und (offensichtlich) $\ac{3SAT} \preceq_p \ac{SAT}$ gilt.
  
  \item \ac{3SAT} ist \ac{NP}-schwierig
  
  Folgt aus \autoref{lem:sat3sat}.\qedhere
 \end{itemize}
\end{proof}


Wir gehen davon aus, dass die Leser des Skripts mit Graphen vertraut sind, machen aber zum Fixieren von Notation und Terminologie die folgende Definition.
\begin{Def}
 Ein \emph{gerichteter Graph} ist ein Paar $\mathcal{G}=(V,E)$, bei dem
 \begin{itemize}
  \item $V$ eine Menge ist, deren Elemente wir \emph{Knoten} nennen und
  \item $E\subseteq V\times V$ eine Menge von geordneten Paaren über $V$ ist. 
  Wir nennen diese geordneten Paare \emph{Kanten}.
 \end{itemize}
 Ein \emph{ungerichteter Graph} ist ein gerichteter Graph, bei dem $E$ symmetrisch ist.\footnote{%
 Für die Zwecke dieser Vorlesung ist es komfortabel, den ungerichteten Graphen als Spezialfall des gerichteten Graphen zu definieren.
 Eine häufig verwendete Alternative ist, die Knotenmenge als Menge von zweielementigen Mengen zu definieren.}
\end{Def}



\begin{Def}[$\CLIQUE$]
    Das Problem $\CLIQUE$ ist wie folgt definiert.
    \begin{center}
    \framebox[\textwidth]{\parbox{.95\textwidth}{
    \smallskip
    \textit{Gegeben:} Ein ungerichteter Graph $\mathcal{G}=(V,E)$ und eine Zahl $k\in\N$.

    \medskip

    \textit{Frage:} Hat $\mathcal{G}$ eine $k$-Clique?
    
    Eine $k$-Clique ist eine $k$-elementige Menge von Knoten, die paarweise durch Kanten verbunden sind,
    d.h., eine Menge $C \subseteq V$, sodass $|C|=k$ und $\forall u, v\in C: u\neq v\rightarrow (u,v)\in E$.
    }}
    \end{center}
\end{Def}
\begin{Satz}[name={[$\CLIQUE$ ist \ac{NP}-vollständig]}]
	$\CLIQUE$ ist \ac{NP}-vollständig.
\end{Satz}
\begin{Bemerkung}
 Der folgende Beweis ist für ein Vorlesungsskript sehr knapp gehalten.
 Der Beweis enthält aber genau die Informationen, die wir in Übungsaufgaben und Klausur zum Erreichen der maximalen Punktzahl erwarten würde.
 
 Beweise der \ac{NP}-Vollständigkeit folgen typischerweise dem hier verwendeten Schema.
 Der schwierige (da Kreativität erfordernde) Teil ist dabei, eine geeignete Konstruktion für die Reduktionsfunktion zu finden.
 Für den folgenden Beweis wird die Idee der Konstruktion in \autoref{fig:3sat-clique} mit Hilfe eines Beispiels illustriert.
 
 In unseren Übungs- und Klausuraufgaben wird die Aufgabenstellung manchmal solch ein Beispiel enthalten.
 Damit soll ein Hinweis auf eine geeignete Konstruktion gegeben werden.
 Eine Besonderheit des folgenden Beweises ist, dass die Konstruktion aus zwei Schritten besteht (Erweiterung der Eingabeformel, Konstruktion des Graphen). Beide Schritte werden in \autoref{fig:3sat-clique} illustriert.
\end{Bemerkung}

\begin{proof}
    \
    \begin{itemize}
     \item Zeige $\CLIQUE\in \ac{NP}$.
     
     Verfahren: 
     Wähle nichtdeterministisch eine $k$-elementige Knotenmenge $C$.
     Prüfe, ob $C$ eine Clique ist.
     Dies ist in polynomieller Laufzeit möglich, da wir für jedes Knotenpaar höchstens einmal die Menge der Kanten durchlaufen müssen.
     
     \item Zeige, dass $\CLIQUE$ \ac{NP}-schwierig ist (mit Hilfe der Reduktion  $\ac{3SAT} \preceq_p \CLIQUE$).
     
     Ziel: Konstruiere für eine gegebene 3CNF-Formel $F$ einen Graphen $\mathcal{G}=(V,E)$, sodass $F$ genau dann erfüllbar ist, wenn $\mathcal{G}$ eine $k$-Clique hat.
     
     Unsere Konstruktion besteht aus zwei Schritten.
     Gegeben sei eine 3CNF-Formel $F$ mit $m$ Konjunkten.
     
     \begin{itemize}
      \item Schritt 1.
      
      Erweitere $F$, sodass jeder Konjunkt aus genau drei Disjunkten besteht.
      Wähle hierfür eine Äquivalenzumformung, die einfach nur existierende Disjunkte wiederholt.
      Die resultierende Formel $F'$ hat also die folgende Form.
      
      $F' = \bigwedge\limits_{i=1}^m (z_{i,1}\lor z_{i,2}\lor z_{i,3})$, wobei\ $z_{i,j}\in \{A_1,\dots,A_n\}\cup\{\neg A_1,\dots,\neg A_n\}$
      \item Schritt 2.
       Definiere zu einer Formel $F$ mit $m$ Konjunkten den Graphen $\mathcal{G} = (V,E)$ und $k$ wie folgt:
	\begin{align*}
		V &= \{ (i,j) \mid 1\leq i\leq m, j\in\{1,2,3\} \}\\
		E &= \{\big((i,j),(p,q)\big) \mid i\neq p, z_{i,j}\neq\neg z_{p,q}\}\\
		k &= m
	\end{align*}
     \end{itemize}

    Es gibt offensichtlich eine totale Reduktionsfunktion, welche diese Konstruktion in polynomieller Zeit berechnet.
    Nun gilt:
    %
    \begin{align*}
    F \text{ ist erfüllbar } \<==> \;\; & \text{Es gibt eine Folge $z_{1,j_1},\dots,z_{m,j_m}$, sodass $F$ unter der}\\
    & \text{Belegung, die jedem Folgenglied $1$ zuordnet, erfüllt ist.}\\
    \<==> \;\;   & \text{Es gibt eine Folge $z_{1,j_1},\dots,z_{m,j_m}$, sodass in jedem Konjunkt}\\
    & \text{ein $z_{i,j_i}$ vorkommt und $\forall i\neq p: z_{i,j_i}\neq \neg z_{p,j_p}$ gilt.}\\
    \<==> \;\;   & \text{$\mathcal{G}$ hat eine Menge von Knoten $\{(1,j_1),\dots,(m,j_m)\}$,}\\
    & \text{die paarweise durch Kanten verbunden sind.}\\
    \<==> \;\;   & \text{$\mathcal{G}$ hat eine $k$-Clique für $k=m$.}
    \end{align*}
    %
    Damit haben wir $\ac{3SAT} \preceq_p \CLIQUE$ gezeigt.
    \qedhere
    \end{itemize}
\end{proof}



 \begin{figure}[H]
 \framebox{
 \begin{minipage}{0.96\textwidth}
 Die Formel 
 $$ (X\lor\neg Y)\land (Y\lor \neg X)\land (X\lor Y) $$
 ist in 3CNF. Die folgende Formel ist äquivalent und jeder Konjunkt besteht aus drei Disjunkten.
 $$ \underbrace{(X\lor\neg Y\lor \neg Y)}_{1}\land \underbrace{(Y\lor \neg X\lor\neg X)}_{2}\land \underbrace{(X\lor X\lor Y)}_{3} $$
 Die Belegung $\beta$ definiert durch $\beta(X)=1, \beta(Y)=1$ ist eine erfüllende Belegung für diese Formel,
 da z.B. der erste Disjunkt im ersten Konjunkt, der erste Disjunkt im zweiten Konjunkt und der dritte Disjunkt im dritten Konjunkt auf 1 gesetzt sind.
 
 Im folgenden Graphen bilden die Knoten $(1,1)$, $(2,1)$ und $(3,3)$ eine $3$-Clique.
 
 \centering
\begin{tikzpicture}[
% node distance=1mm
]
\node (11) at (-2,-1.5) {$(1,1)$};
\node (12) at (0,-1.5) {$(1,2)$};
\node (13) at (2,-1.5) {$(1,3)$};

\node (21) at (-2,3) {$(2,1)$};
\node (22) at (-3,2) {$(2,2)$};
\node (23) at (-4,1) {$(2,3)$};

\node (31) at (2,3) {$(3,1)$};
\node (32) at (3,2) {$(3,2)$};
\node (33) at (4,1) {$(3,3)$};

\node (11v) [node distance=1mm, below =of 11] {$X$};
\node (12v) [node distance=1mm, below =of 12] {$\neg Y$};
\node (13v) [node distance=1mm, below =of 13] {$\neg Y$};

\node (21v) [node distance=1mm, above left =of 21] {$Y$};
\node (22v) [node distance=1mm, above left =of 22] {$\neg X$};
\node (23v) [node distance=1mm, above left =of 23] {$\neg X$};

\node (31v) [node distance=1mm, above right =of 31] {$X$};
\node (32v) [node distance=1mm, above right =of 32] {$X$};
\node (33v) [node distance=1mm, above right =of 33] {$Y$};

\draw[-,very thick] (11) to (21);
\draw[-] (11) to (31);
\draw[-] (11) to (32);
\draw[-,very thick] (11) to (33);

\draw[-] (12) to (22);
\draw[-] (12) to (23);
\draw[-] (12) to (31);
\draw[-] (12) to (32);

\draw[-] (13) to (22);
\draw[-] (13) to (23);
\draw[-] (13) to (31);
\draw[-] (13) to (32);

\draw[-] (21) to (31);
\draw[-] (21) to (32);
\draw[-,very thick] (21) to (33);

\draw[-] (22) to (33);
\draw[-] (23) to (33);
\end{tikzpicture}
\end{minipage}
}
	\caption{Beispiel zu $\ac{3SAT} \preceq_p \CLIQUE$}
	\label{fig:3sat-clique}
\end{figure}






% \begin{Bsp*}
% 	\begin{align*}
% 	F &= (\underbrace{x\lor y\lor \overline{y}}_1) \land (\underbrace{z\lor \overline{y} \lor \overline x}_2)\\
% 	\mathcal{G} &: \tikz[baseline=(11.base),label distance=-.5em]\graph[math nodes, chain shift={(0,-1)}, group shift={(1,0)}]{
% 		{x[label={[name=11](1,1)}], y[label={(1,2)}], "\overline y"[label={(1,3)}]}
% 		--[complete bipartite] 
% 		{z[label={below:(2,1)}], 22/\overline y[label={below:(2,2)}], "\overline x"[label={below:(2,3)}]}
% 	};
% 	\end{align*}\rlnote{Grafik überprüfen}
% \end{Bsp*}

An dieser Stelle wollen wir nun den Beweis von \autoref{satz:cook} nachholen.
\draftnote{07.02.2018}
Wir benötigen dafür zunächst das folgende Lemma.
\begin{lemma}
	Für jedes $k\in\N$ existiert eine Formel $G$, sodass $G(x_1,\dots,x_k)=1$ gdw. $\exists j: x_j = 1$ und $\forall i \neq j: x_i = 0$. Es gilt $|G| \in O(k^2)$.
\end{lemma}
\begin{proof}
	\[ G(x_1,\dots,x_k) = \bigvee_{i=1}^k x_i\land \bigwedge_{i\neq j}\neg (x_i\land x_j) \]
	$\M=(Q,\Sigma,\Gamma,\delta,q_0,\blank,F)$ akzeptiert $L$ in $\NTIME(p),\ p$ Polynom.
\end{proof}

\begin{proof}[von \autoref{satz:cook}]\
	\begin{enumerate}
	\item $\ac{SAT} \in \ac{NP}$\\
		Rate nichtdeterministisch eine Belegung $\beta$.\\
		Werte anschließend $F[\beta]$ in polynomieller Zeit aus.
% 		$\curvearrowright$ in $\NTIME(n)$, polynomiell
	\item \ac{SAT} ist \ac{NP}-schwierig.\\
		Zeige: $\forall L\in \ac{NP}: L \preceq_p \ac{SAT}$\\
		Sei $L\in \ac{NP}$, d.h., es gibt ein Polynom $p$ und eine \ac{NTM} $\M$ mit $L=L(\M)$ mit Zeitkomplexität $p(n)$ für $n = |w|$.
		
		Sei $w = x_1\dots x_n\in\Sigma^*$ eine beliebige Eingabe für $\M$ der Länge~$n$.\\
		Ziel: Definiere $F$, sodass $F$ erfüllbar $\<==> \M$ akzeptiert $w$.
		
		Seien $Q$ die Zustandsmenge von $\M$ mit $\{q_1,\dots,q_k\}=Q$ und $\Gamma$ das Bandalphabet von $\M$ mit $\{a_1,\dots,a_l\} = \Gamma$.\\
		Seien $t\in\{0,1,\dots,p(n)\}$, $i\in\{-p(n),\dots,-1,0,1,\dots,p(n)\}$, $q \in Q$ und $a\in\Gamma$. \\
		Definiere damit folgende aussagenlogische Variablen zur Verwendung in $F$.
		\begin{itemize}
		\item $\mathrm{state}(t,q) = 1$ gdw.\ $\M$ nach $t$ Schritten im Zustand $q$ ist
		\item $\mathrm{pos}(t,i) = 1$ gdw.\ der Kopf von $\M$ nach $t$ Schritten auf Position $i$ steht.
		\item $\mathrm{tape}(t,i,a) = 1$ gdw.\ nach $t$ Schritten an Position $i$ ein $a$ steht.
		\end{itemize}
	\end{enumerate}

	\medskip

	Wir setzen $F = R \land A\land T_1 \land T_2 \land E$, wobei die Teilformeln folgendermaßen definiert sind.
	\begin{enumerate}
	\item Randbedingungen
		\begin{align*}
			R\ =&\phantom{\land}\, \bigwedge_t G(\state(t,q_1),\dots,\state(t,q_k))\\
			& \land \bigwedge_t G(\pos(t,-p(n)),\dots,\pos(t,0),\dots,\pos(t,p(n)))\\
			& \land \bigwedge_{t,i} G(\tape(t,i,a_1),\dots,\tape(t,i,a_l))
		\end{align*}
	\item Anfangskonfiguration
		\begin{align*}
			A\ =&\phantom{\land}\; \state(0,q_1)\land \pos(0,1)\\
			&\land\tape(0,1,x_1)\land\dots\land\tape(0,n,x_n)\\
			&\land\bigwedge_{i< 1 \lor i > n} \tape(0,i,\blank)
		\end{align*}
	\item Transitionsschritte
	\begin{align*}
		T_1\ =& \bigwedge_{\substack{t < p(n),\\i,q}} \state(t,q)\land \pos(t,i) \land \tape(t,i,a)\\
		&\qquad \-> \bigvee_{1 \leq m \leq |\delta(q,a)|} \state(t+1,q_m')\land \pos(t+1,i+d_m) \land \tape(t+1,i,a_m')\\
		&\qquad \text{für jede Transition } (q_m',a_m',d_m) \in \delta(q,a) \text{ mit } d_m\in\{-1,0,1\}\\
		T_2\ =& \bigwedge_{\substack{t < p(n),\\i,q}} \neg \pos(t,i)\land \tape(t,i,a) \-> \tape(t+1,i,a)
	\end{align*}
	\item Endkonfiguration%
% 	\footnote{Wir gehen hier davon aus, dass $\M$ erst im allerletzten Schritt $p(n)$ anhält.}
		\[ E = \bigvee_{q\in F} \state(p(n),q) \]
		$|F|$ ist polynomiell beschränkt in $|\M|+|w|$, also gilt $L\preceq_p \ac{SAT}$.%
		\footnote{$|\M|$ dürfen wir als zu $L$ gehörende Konstante betrachten.}\\
		Es ist klar, dass gilt: $F$ erfüllbar $\<==> \M$ akzeptiert $w$.\\
		Damit haben wir "`$\ac{SAT}$ ist \ac{NP}-vollständig"' gezeigt.
		\qedhere
	\end{enumerate}
\end{proof}

% \subsection{Weitere \ac{NP}-vollständige Probleme}










\newpage

\section{Einordnung von Sprachen in Chomsky-Hierarchie und Abschlusseigenschaften}



\begin{Satz}\label{satz:6.3}
	$\ch{1}=\NSPACE(n)$
\end{Satz}
\begin{proof}\
\begin{itemize}
	\item["`\=>"'\,:] Sei $G=(N,\Sigma,P,S)$ Typ-1 Grammatik für $L$.\\
		Konstruiere \ac{NTM} $M$ mit $L=L(M),\ \Gamma = \Sigma\cup N\cup\{\blank\}$
		\begin{enumerate}
		\item $M$ rät nicht deterministisch eine Position auf
                  dem Band und eine Produktion $\alpha\-> \beta$. Falls $\beta$ gefunden wird, ersetze durch $\alpha$, weiter bei 1.
		\item Falls Bandinhalt $=S$ \quad stop, akzeptiert.
		\end{enumerate}
		Dieses Verfahren terminiert.
	\item["`\<="'\,:] %
	Gegeben: \ac{NTM} $\M$ 
	Gesucht: Typ-1 Grammatik $\mathcal{G}$ mit $L(\mathcal{G})=L(M)$\\
	Wir wollen den Beweis hier nicht formal ausformulieren und beschreiben nur die zugrundliegenden Ideen.
	\begin{itemize}
	\item Idee 1:
	Ahme Berechnungsschritte der \ac{TM} mit Ableitungsschritten der Grammatik nach.
	Die Manipulierten Satzformen repräsentieren dabei Konfigurationen der TM.
	Wir wählen als (vorläufige) Variablenmenge $V'=\Gamma\cup(Q\x\Gamma)$.
	Eine Konfiguration $uqav$ mit $u,v\in\Gamma^*,a\in\Gamma$ wird durch die Satzform $u(q,a)v$ repräsentiert.
	Die Regeln sind wie folgt definiert.
	\begin{align*}
	 P' = \quad &\; \{ (q,a) \-> (q',a') \mid (q',a',N)\in \delta(q,a)\}\\
	 \cup &\; \{ (q,a)b \-> a'(q',b) \mid b\in\Gamma, (q',a',R)\in \delta(q,a) \}\\
	 \cup &\; \{ (q,a) \-> (q',a') \mid (q',a',L)\in \delta(q,a)\}
	\end{align*}
    Somit gilt, dass die \ac{TM} eine Konfiguration in eine andere überführen kann $uqav \vdash^{*} u'q'a'v'$
    genau dann wenn wir mit Regeln aus $P'$ die Ableitung $u(q,a)v \vdash^{*} u'(q',a')v'$ machen können.

	
	\item Idee 2:
	Die erste Idee hat noch das folgende Problem und muss deshalb ergänzt werden:
	Während eine \ac{TM} mit der Eingabe startet und diese in eine ''akzeptiert``/''akzeptiert nicht`` Antwort transformiert,
	arbeitet eine Grammatik umgekehrt. 
	Sie startet mit dem Startsymbol und am Ende erhalten wir das resultierende Wort.
	
	Wir Unterteilen die Satzformen der Grammatik in zwei Spuren, eine obere Spur und eine untere Spur.
	Die untere Spur enthält das Eingabewort, die untere Spur simuliert das Band der Turingmaschine.
	Formal erreichen wir dies in dem wir als Variablenmenge das Karthesische Produkt aus vorläufiger Variablenmenge und Alphabet nehmen:
	$N = \{S\}\dotcup N' \x\Sigma$
	
		\begin{align*}
		a_1\cdots a_n &\-->
			\pmqty{a_1\\a_1} \pmqty{a_2\\a_2} \pmqty{(q,a)\\a_3} \pmqty{a_4\\a_4} \pmqty{a_n\\a_n}
			&&\begin{matrix}\text{Spur 1}\\\text{Spur 2}\end{matrix}\\
	\end{align*}
	
	Eine Ableitung besteht dann aus drei Phasen.
	\begin{itemize}
	\item Phase 1: 
	Wir wählen nichtdeterministisch das Wort, dass wir erzeugen wollen (untere Spur) und bringen das Band der \ac{TM} (obere Spur) in die entsprechende Startkonfiguration.
	\item Phase 2:
	Wir simulieren auf der oberen Spur die Berechnungen der \ac{TM} bis eine Haltekonfiguration erreicht ist.
	\item Phase 3:
	Wenn die \ac{TM} in einem akzeptierenden Zustand ist ersetzen wir alle ''Zweispurzeichen`` durch das ensprechende Zeichen der unteren Spur und erhalten das Eingabewort.
	\end{itemize}
	
	
	
	\end{itemize}
	\begin{align*}
		\text{mit } P &= \\
		S &\-> \pmqty{(q_0,a)\\a} &&\forall a\in\Sigma\\
		S &\-> S\pmqty{a\\a} &&\forall a\in\Sigma\\
		\pmqty{\alpha\\a}
			&\-> \pmqty{\beta\\a}
			&&\begin{aligned}
				\forall \alpha\->\beta &\in P'\\
				\alpha,\beta &\in\triangle
			\end{aligned}\\
		\pmqty{\alpha_1\\a_1}\pmqty{\alpha_2\\a_2}
			&\-> \pmqty{\beta_1\\a_1}\pmqty{\beta_2\\a_2}
			&&\begin{aligned}
				\forall \alpha_1\alpha_2\->\beta_1\beta_2 &\in P'\\
				\alpha_i,\beta_i &\in\triangle
			\end{aligned}\\
		\pmqty{x\\a} &\-> a
			&&\begin{aligned}
				x&\in\Gamma\\
				a&\in\Sigma
			\end{aligned}\\
		\pmqty{(q',x)\\a} &\-> a
			&&\begin{aligned}
				x&\in\Gamma, q'\in F, \delta (q',x)=\emptyset\\
				a&\in\Sigma
			\end{aligned}
	\end{align*}
	\begin{align*}
		S &\xRightarrow{*} \pmqty{(q_0,a_1)\\a_1}\pmqty{a_2\\a_2}\dots\pmqty{a_n\\a_n}\\
		&\phantom{{}\xRightarrow{*}{}}\ \acs*{TM}\ \,\dots\\
		&\xRightarrow{*} \pmqty{x_1\\a_1}\dots\pmqty{(q',x_i)\\a_i}\dots\pmqty{x_n\\a_n}\\
		&\xRightarrow{*} a_1\dots a_i\dots a_n
	\end{align*}
	Damit gesehen $L(\mathcal{G})\subseteq L(M)$\\
	Rückrichtung: selbst \qedhere
	\end{itemize}
\end{proof}


\begin{Satz}
$L\in\ch{1} \==> L$ entscheidbar.
\end{Satz}

\begin{Satz}
 \ch{1} ist nicht die Menge der entscheidbaren Sprachen.
\end{Satz}
\begin{proof}
 	\begin{minipage}[t]{.7\linewidth}
	Konstruiere eine Kodierung von Typ-1 Grammatiken als Worte
        $w\in\{0,1\}^*$. Die Grammatik zum Wort $w$ sei $G_w$; falls
        $w$ kein sinnvoller Kode ist, setze $G_w = (\{S\}, \{0,1\},
        \{\}, S)$ die leere Grammatik.
        
	Die \emph{Diagonalsprache} $D = \{w \in\{\0,1\}^* \mid w\notin
        L(G_w)\}$ ist entscheidbar, weil das Wortproblem für Typ-1  
	Sprachen entscheidbar, aber es $\nexists w$, sodass $L(G_w) =
        D$. Beweis durch Widerspruch.
	\end{minipage}\quad
	\begin{tabular}[t]{M{c} | *4{M{c}@{ }}}
		\ &w_1&w_2&w_3&\cdots\\\hline
		G_1 &\\
		G_2 &\\
		G_3 &\\
		\vdots&
	\end{tabular}
\end{proof}






\begin{Bemerkung}\
	\newcommand{\underarrowset}[2]{%
		\underset{%
			\mathclap{%
				\overset{\displaystyle\uparrow}{\mathclap{#1}}%
			}%
		}{#2}%
	}
	\begin{enumerate}
	\item Für "`$s(n)\leq n$"' betrachte 2-Band \ac{TM}, bei denen die Eingabe read-only ist und nur das zweite Arbeitsband der Platzschranke unterliegt (so ist $s(n)$ sublinear möglich).
	\item Jede Platzbeschränkung impliziert Laufzeitschranke.\\
	Angenommen Platzschranke $s(n)$\\
	$\curvearrowright$ \ac{TM} hat nur endlich viele Konfigurationen
	\[ N := \underarrowset{%
			\parbox{\widthof{\scriptsize Eingabeband}}{\raggedright\scriptsize Kopfpos. im Eingabeband}\hspace{1cm}
		}{n\vphantom{|}}
		|Q| \quad\cdot\quad
		\underarrowset{\parbox{2.2cm}{\scriptsize\centering mögliche Inhalte des Arbeitsbands}}{|\Gamma|}^{s(n)}
		\ \cdot\
		\underarrowset{\hspace{1.7cm}\parbox{2cm}{\scriptsize Kopfpos. auf\\ Arbeitsband}}{s(n)}
		\in 2^{O(\log n + s(n))}
	\]
	\item \ac{DTM} mit Platzschranke\,: $M$ entscheidet,\\
	falls sie akzeptiert, dann in weniger als $N$ Schritten,\\
	falls nach $N$ Schritten keine Terminierung erfolgt\\
	\quad$\curvearrowright$ Endlosschleife -- Abbruch
	\item \ac{NTM}\,: nutze den \ac{ND} optimistisch aus\,:\\
	falls eine akzeptierende Berechnung existiert, dann muss es eine Berechnung ohne wiederholte Konfiguration geben.
	\end{enumerate}
\end{Bemerkung}
\begin{Satz}[name={[$L\in\DTAPE(n),\ L\in\NTAPE(n)$]}]\label{satz:6.2}\
	\begin{itemize}
	\item $L\in\DTAPE(n) \curvearrowright\ \exists$ \ac{DTM}, die $L$ in Zeit $2^{O(n)}$ entscheidet.
	\item $L\in\NTAPE(n)$ analog.
	\end{itemize}
\end{Satz}\vspace{-2em}
\begin{proof}
	siehe oben.
\end{proof}
\begin{Bemerkung}
	Die Klasse $\NTAPE (n)$ heißt auch \ac{LBA}.
\end{Bemerkung}
%


\begin{Satz}
	Die Typ-1 Sprachen sind abgeschlossen unter {\thinmuskip=6mu$\cup,\cap,\cdot,{}^*$} und Komplement.
\end{Satz}
\begin{proof}
	Zu $\cup$ und $\cap$ betrachte \ac{NTM}.\\
	Für $\cdot$ und $^*$ konstruiere Grammatik.\\
	ad Komplement "`2. \acsu{LBA}-Problem\footnote{\acs*{LBA} = \acl*{LBA} -- 1964 Kuroda}"' bis 1987, dann gelöst durch Immerman und Szelepcsényi.
\end{proof}
\begin{Bemerkung}
  1. \ac{LBA}-Problem (1964): Ist $\mathrm{NTAPE}(n) = \mathrm{DTAPE}(n)$? Bisher ungelöst.
\end{Bemerkung}
\begin{Satz}
	Das Wortproblem für Typ-1 Sprachen ist entscheidbar.
\end{Satz}
\begin{proof}
	\begin{align*}
		L\in\mathcal{L}_1 &\curvearrowleftright L\in\mathrm{NTAPE}(n)\\
		&\curvearrowright \text{nach \autoref{satz:6.2}: $L$ entscheidbar}
	\end{align*}
	Nach \autoref{satz:6.1} sogar mit \ac{DTM}.
\end{proof}
Die Rückrichtung "`L entscheidbar. $\xcancel{\curvearrowright}\ L$ ist Typ-1 Sprache"' gilt nicht!

\subsection{Typ-0 Sprachen}

\begin{Satz}\label{satz:6.6}
	$\mathcal{L}_0 = \ac{NTM}$
\end{Satz}
\begin{proof}
	\begin{itemize}
	\item["'\=>"'] Konstruktion einer \ac{NTM} $M$ wie in \autoref{satz:6.3}, aber ohne Platzbeschränkung.
	\item["'\<="'] Konstruktion analog zu \autoref{satz:6.3} + Startsymbol $S'$
	\begin{align*}
		S' &\-> \pmqty{\blank\\\Eps} S' \pmqty{\blank\\\Eps} &&\text{Schaffe Platz für Berechnung von }M\\
		S' &\-> S\\
	\shortintertext{Erweitere $N$}
		&= \{S',S\}\cup\triangle\x(\Sigma\cup\{\Eps\})\\
	\shortintertext{Neue Löschregeln:}
		\pmqty{x\\ \Eps } &\-> \Eps &&\forall x\in\Gamma\\
		\rotatebox{90}{$\Rsh$}\ &\mathrlap{\text{die einzigen
                                          Regeln, die Typ-1 Bedingung verletzen.}} \tag*{\qedhere}
	\end{align*}
	\end{itemize}
\end{proof}

\begin{Satz}[name={[Abgeschlossenheit von Typ-0 Sprachen]}]\label{satz:Typ-0-abgeschl}
	Die Typ-0 Sprachen sind unter $\thinmuskip=6mu\cup,\cap,\cdot,{}^*$ abgeschlossen.
\end{Satz}
\begin{proof}
	Konstruiere \ac{NTM} für $\thinmuskip=6mu\cup,\cap$ ; Typ-0-Grammatiken für $\cdot$ und $^*$.
\end{proof}

\begin{Bem}
	Typ-0 Sprachen sind \emph{nicht} unter Komplement abgeschlossen!
\end{Bem}


\begin{Korollar}
	$\mathcal{L}_0 \supsetneqq \mathcal{L}_1$
\end{Korollar}
\begin{proof}
	$K$ ist unentscheidbar (also $\notin \mathcal{L}_1$), aber semi-entscheidbar (also $\in \mathcal{L}_0$).
\end{proof}

\section[Rekursive Funktionen]{Rekursive Funktionen\datenote{10.02.16}}
\begin{Def}[Schema der primitiven Rekursion]
	\begin{align*}
		\text{Sei } \mathcal{G}&: \N^k \-> \N \qquad,\ k\geq 0\\
		h&: \N^{k+R}\->\N\\
		\text{Sei } f&: \N^{k+1}\->\N\text{ eine Funktion, die folgende Gleichungen erfüllt:}\\
		f(0,\overline{x}) &= g(\overline{x}) \qquad,\ \overline{x}\in\N^k\\
		\begin{split}
			f(n+1,\overline{x}) &= h(f(n,\overline{x}),n,\overline{x})
		\end{split} \numbereq\label{eq:rekursives f}
	\end{align*}
	Dann sei
	\[ f =: PR(g,h) \]
	aus $g$ und $h$ durch primitive Rekursion definiert.
\end{Def}
\begin{lemma}[name={[$PR(g,h)$ ist wohldefiniert.]}]
	$PR(g,h)$ ist wohldefiniert.
\end{lemma}
\begin{proof}
	\begin{align*}
		\text{Angenommen } f&=PR(g,h)\\
		\text{und } f' &= PR(g,h)\\
	\shortintertext{d.h. $f$ und $f'$ erfüllen \eqref{eq:rekursives f}.}
		\text{Zeige } \forall n\in\N,&\ \forall \overline{x}\in\N^k:\\
		f(n,\overline{x}) &= f'(n,\overline{x})\\
		\text{I.A. }n=0: f(0,\overline{x}) &\overset{\eqref{eq:rekursives f}}= g(\overline{x}) \overset{\eqref{eq:rekursives f}}= f'(0,\overline{x})\\
		\text{I.S. }n\->n+1: f(n+1,\overline{x}) &\overset{\eqref{eq:rekursives f}}= h(f(n,\overline{x}),n,\overline{x})\\
		&\overset{\text{I.V.}}= h(f'(n,\overline{x}),n,\overline{x})\\
		&\overset{\eqref{eq:rekursives f}}= f'(n+1,\overline{x}) \tag*{\qedhere}
	\end{align*}
\end{proof}
\begin{Def}[name={[$\PREC$ primitiv rekursiven Funktionen über $\N$]}]\label{def:PREC}
	Die $\PREC$ der \emph{primitiv rekursiven Funktionen} über $\N$, ist induktiv definiert:
	\begin{enumerate}
	\item\label{itm:Nullfunktion} $\forall k\in\N:
		\begin{aligned}[t]
			&\mathcal{O}^k : \N^k\->\N\\
			&\mathcal{O}^k(\overline{x}) = 0\\
		\end{aligned}$\\
		$\mathcal{O}^k\in\PREC$ (die Nullfunktion)
	\item\label{itm:Projektion} $\forall k\geq1: \forall 1\leq j\leq k:
		\begin{aligned}[t]
			&\pi^k_j : \N^k\->\N\\
			\pi^k_j(x_1,\dots,x_k) = x_j
		\end{aligned}$\\
		$\pi^k_j\in\PREC$ (Projektionen)
	\item\label{itm:Nachfolgerfunktion} $S:\N\->\N\ \in\PREC$ (Nachfolgerfunktion)
	\pagebreak[3]
	\item\label{itm:Komposition} Feld $g:\N^k\->\N\ \PREC$
		\begin{align*}
			\forall 1\leq i\leq k: h_i:\N^m\->\N \quad&\in\PREC\\
			\text{dann ist }g\circ[h_1,\dots,h_k]:\N^m\->\N \quad&\in\PREC\\
			(g\circ[h_1,\dots,h_k])(\overline{x}) = g(h_1(\overline{x}),\dots,h_k(\overline{x})) \quad\x&\in\N^m
		\end{align*}
		(Funktionskomposition)
	\item\label{itm:Rekursion} Falls $g:\N^k\->\N$ und $h:\N^{k+2}\->\N\quad\in\PREC$\\
		dann ist auch $PR(g,h)\in\PREC$ \quad(Rekursion) \qedhere
	\end{enumerate}
\end{Def}
\begin{Bsp*}\
	\begin{enumerate}
	\item Addition $\in\PREC$
		\begin{align*}
			\add(0,x) &=x\\
			\add(n+1,x) &= \add(n,x)+1\\
			&= h(\add(n,x),n,x)\\
			\add &= PR(g,h)\\
			\text{mit }g(x) &= x \quad ; g=\pi'_1\\
			h(y,n,x) &= y+1 \quad ; h= S\circ \pi^S_1\\
			\add &= PR(\pi'_1,S\circ\pi^3_1)
		\end{align*}
	\item Multiplikation:
		\begin{alignat*}{2}
			&&\mult&: \N^2\->\N\\
			&&\mult(0,x) &=0\\
			&&\mult(n+1,x) &= \add(x,\mult(n,x))\\
			&&\mult &= PR(g,h)\\
			\text{mit}&\ & g&= \mathcal{O}^1\\
			&&h(y,n,x) &= \add(x,y)\\
			\text{d.h.}&& h &=\add\circ[\pi^3_3,\pi^3_1]
		\end{alignat*}
	\end{enumerate}
\end{Bsp*}
\begin{Beobachtung}
	Alle primitiv rekursiven Funktionen sind total.\\
	Aber: $\exists$ totale Funktion, die nicht primitiv rekursiv ist.
\end{Beobachtung}
\begin{Def}[Minimierung]\label{def:Minimierung}
	Sei $f:\N^{k+1}-\->\N$\\
	dann ist $\mu f:= g\in\N^k-\->\N$\\
	definiert durch:
	\begin{align*}
		g(\overline{x}) &= \min\{n \mid f(n,\overline{x})=0\text{ und }\forall j<n: f(j,\overline{x})\text{ def. und }\neq 0\} \tag*{\qedhere}\\[.5em]
		\text{Falls }f(n,\overline{x}) &\neq 0\ \forall n: g(\overline{x})\text{ undef.}\\
		\text{Falls }f(n,\overline{x}) &=0\text{ aber }\exists j< n\ f(j,\overline{x})\text{ undef.}\\
		&\curvearrowright g(\overline{x})\text{ undef.}
	\end{align*}
\end{Def}
\begin{Def}[name={[Klasse der $\mu$-rekursiven Funktionen]}]
	Die Klasse der \emph{$\mu$-rekursiven Funktionen} über $\N$ ist die kleinste Klasse von Funktionen, die die Basisfunktionen (1,2,3 aus der \autoref{def:PREC}) enthalten und abgeschlossen ist unter \hyperref[itm:Komposition]{Komposition \ref*{itm:Komposition}} \hyperref[itm:Rekursion]{Rekursion \ref*{itm:Rekursion}} und \hyperref[def:Minimierung]{Minimierung}
\end{Def}
\begin{Satz}[name={[$\mu$-Funktionen-Klasse $\hat=$ Klasse der \acs*{TM}-berechenbaren Fkt.]}]
	Die Klasse der $\mu$-Funktionen stimmt mit der Klasse der \ac{TM}-berechenbaren Funktionen über $\N$ überein.
\end{Satz}
\begin{proof}\
	\begin{itemize}
	\item simuliere \rlerror*{URM = unbeschränkte Registermaschine?}{URM} mit $\mu$-Rekursion.
	\item simuliere $\mu$-Rekursion mit $\ac{TM}$ (analog zur universellen \ac{TM}) \qedhere
	\end{itemize}
\end{proof}
\begin{Satz}[Kleene]
	Für jede $\mu$-rekursive Funktion $F: \N^k-\->\N$ gibt es zwei primitiv rekursive Funktionen $p,q:\N^{k+1}\->\N$, sodass $f(\overline{x})=p(\overline{x},\mu q(\overline{x}))$
\end{Satz}
$\curvearrowright$ Jede berechenbare Funktion über $\N$ kann mit einer WHILE-Schleife programmiert werden.
\begin{Def}[name={[Ackermannfunktion]}]
	Die \emph{Ackermannfunktion}
	\begin{align*}
		&A:\N^2\->\N\\
		&A(0,y)=y+1\\
		x>0:\ &A(x,y)=A(x-1,1)\\
		y>0:\ &A(x,y)=A(x-1,A(x.y-1)) \tag*{\qedhere}
	\end{align*}
\end{Def}
\begin{Satz}
	Sei $\A$ die Menge der Funktionen, die durch $A$ majorisiert werden.
	\[ \text{Es gilt: }\PREC\subseteq \A \]\vspace{-2em}
\end{Satz}
\begin{proof}
	$A=\{f:\N^k\->\N \mid \exists n \forall\overline{x}:f(\overline{x})< A(m,\max\overline{x})\}$.\\
	Induktion über $\PREC$:
	\begin{enumerate}
	\item $\mathcal{O}(\overline{x})=0 < A(0,\max\overline{x})=\max \overline{x}+1,\quad \mathcal{O}\in\A$
	\item $S(x)=\A+1 < x+2 = A(\uuline{1},x)$, also $S\in \mathcal A$
	\item $\pi^k_j(x_1,\dots,x_k) = x_j\leq \max\overline{x} < \max \overline{x} +1 = A(\uuline{0},\max \overline{x})$, also $\pi^k_j\in\A$
	\item\ \vspace{-2em}
		\begin{align*}
			&\left.\begin{array}{rr<{{}}@{}l}
				\text{Sei } & g_1,\dots,g_m : &\N^k\;\->\N\\
				& h : &\N^m\->\N
			\end{array}\right\rbrace \in\A\\
			&\begin{array}{rr@{}>{{}}l@{\qquad}r@{\;}l}
				\text{also } &g_i(\overline{x}) &< A(r_i,\max\overline{x}) & r_i &\exists \text{ nach I.V.}\\
				&h(\overline{x}) &< A(s,\max\overline{y}) &s &\exists \text{ nach I.V.}
			\end{array}
		\end{align*}
		\newpage
		Betrachte $f= h\circ [g_i,\dots,g_m]$\\
		Wähle $j$ sodass $r_j=\max r_i$\\
		$\curvearrowright\ g_i(x) < A(r_j,\max\overline{x})$
		\begin{align*}
			\text{Dann }f(\overline{x}) &= h(g_1(\overline{x}),\dots,g_m(\overline{x})\\
			&< A(s,\max(g_1(\overline{x}),\dots,g_m(\overline{x}))\\
			&< A(s,\max(A(r_1,\max\overline{x}), \dots, A(r_m,\max\overline{x}))\\
			&= A(s,A(r_j,\max\overline{x}))\\
			&\overset!< A(\uuline{s+r_j+2},\max\overline{x})\\
			\text{Also }f\in\A
		\end{align*}
	\item Sei $g:\N^k\->\N,\ h:\N^[k+2]\->\N\qquad \in\A$
		\begin{align*}
			\curvearrowright \exists r,s :{}& g(\overline{x}< A(r,\max\overline{x}))\\
			& h(\overline{y}< A(s,\max\overline{y}))
		\end{align*}
		Sei $f= PR(g,h)$. Zeige $f\in\A$\\
		Zunächst $\forall n\in\N: f(n,\overline{x}) < A(q,n+\max\overline x)$, wobei $q$ nicht von $n$ oder $\overline{x}$ abhängt.\\
		Wähle $q=\max(r,s)+1$.
		
		Induktion über $n$:
		\begin{description}
		\item[$n=0:$]\qquad $f(0,\overline{x}) = g(\overline{x}) < A(r,\max\overline{x}) < A(g,\max x)$
		\item[$n\->n+1:$]\ \vspace{-2em}
			\begin{align*}
				f(n+1,\overline{x}) &= h(f(n,\overline{x}),n,\overline{x}) < A(s,z) &&,\ z=\max(f(n,\overline{x}),n,\overline{x})\\
				\text{nach I.V.\,:}\quad f(n,\overline{x}) &< A(g,n+\max\overline{x})\\
				\max(n,\overline{x}) \leq n+\max\overline{x} &< A(q,n+\max\overline{x})\\
				\curvearrowright z &< A(q,n+\max\overline{x})\\
				f(n+1,\overline{x}) &< A(s,z)\\
				&< A(s,A(q,n+\max\overline{x}))\\
				&\leq A(q-1,A(q,n+\max\overline{x}))\\
				&= A(q,A(q,n+1+\max\overline{x})) \tag*{\qedsymbol\ Ind. $n$}
			\end{align*}
		\end{description}
		\begin{align*}
			\text{Nun setze }w &=\max(n,\overline{x})\\
			f(n,\overline{x}) < A(q,n+\max\overline{x})\\
			&\leq A(q,2w)\\
			&\overset!< A(\uuline{q+2},w)\\
			\text{Also }f\in\A \tag*{\qedhere}
		\end{align*}
	\end{enumerate}
\end{proof}
\begin{Korollar}
	Die Ackermannfunktion $A$ ist nicht primitiv rekursiv.
\end{Korollar}
\begin{proof}
	$A\notin\A$
\end{proof}

%\renewcommand{\listtheoremname}{Liste der Sätze und Definitionen}
\renewcommand{\listtheoremname}{Liste der Definitionen}
\listoftheorems[ignoreall,show={Def,Def*,subDef},numwidth=2.8em]
\renewcommand{\listtheoremname}{Liste der Sätze}
\listoftheorems[ignoreall,show={Satz,Satz*,lemma,lemma*,Korollar,Korollar*},numwidth=2.8em]
%\renewcommand{\listtheoremname}{Liste der Beispiele}
%\listoftheorems[ignoreall,show={Bsp,Bsp*},numwidth=2.8em]
\listoffigures
\addsec{Abkürzungsverzeichnis}
\begin{acronym}[MPCP]
	\acro{AL}{Aussagenlogik}
	\acro{CFL}{Menge der kontextfreien Sprachen}
	\acro{CFG}{Menge der kontextfreien Grammatiken}
	\acro{CNF}{Chomsky Normalform}
	\acro{CP}{Korrespondenzproblem}
	\acro{CYK}{Cocke, Younger, Kasami}
	\acro{DAG}{gerichteter azyklischer Graph}
	\acro{DCFG}{deterministische \acs*{CFG}}
	\acro{DCFL}{deterministische \acs*{CFL}}
	\acro{DEA}{deterministischer endlicher Automat}
	\acro{DFA}{\acroextra{engl.: }deterministic finite automaton}
	\acro{DPDA}{deterministischer Kellerautomat}
	\acro{DTM}{deterministische \acs*{TM}}
	\acro{EA}{endlicher Automat}
	\acro{LBA}{Linear Bounded Automaton}
	\acro{MPCP}{Das modifizierte \acs*{PCP}}
	\acro{ND}{Nicht-Determinismus}
	\acro{NEA}{nichtdeterministischer endlicher Automat}
	\acro{NFA}{\acroextra{engl.: }nondeterministic finite automaton}
	\acro{NPDA}{nichtdeterministischer Kellerautomat}
	\acro{NT}{Nichtterminal}
	\acroplural{NT}[NT]{Nichtterminale}
	\acro{NTM}{Eine nichtdeterministische \acs*{TM}}
	\acro{PCP}{Das Postsche Korrespondenzproblem}
	\acro{PDA}{pushdown automaton\acroextra{ (Kellerautomat)}}
	\acro{PL}{Pumping Lemma}
	\acro{RE}[\textit{RE}]{Menge der regulären Ausdrücke}
	\acro{REG}[\textit{REG}]{Menge der regulären Sprachen}
	\acro{RM}{Registermaschine}
	\acro{TM}{Turing-Maschine}
	\acro{TT}{Turingtabelle}
\end{acronym}
\listoffixmes

\end{document}
