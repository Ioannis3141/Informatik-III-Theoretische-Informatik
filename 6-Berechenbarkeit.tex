\section[Berechenbarkeit]{Berechenbarkeit} \draftnote{18.01.17}
\subsection{Typ-0 und Typ-1 Sprachen}
%\ptnote{checkpoint}
% \begin{Def}[name={[NTM]}]
% 	\acf{NTM} ist ein Tupel $(Q,\Sigma,\Gamma,\delta,q_0,\blank,F)$, wobei alles wie bei einer \ac{DTM} außer $\delta: Q\x\Gamma\-> \wp(Q\x\Gamma\x\{N,L,R\})$.\\
% 	Konfiguration und Berechnungsrelation wie gehabt.
% \end{Def}

\begin{Def}[name={[Akzeptanz, Entscheidbarkeit, Semi-Entscheidbarkeit]}]
	Sei $M$ \ac{TM}.
	\begin{itemize}
	\item $M$  \emph{akzeptiert} $w\in\Sigma^*$, falls $q_0w
          \vdash^* uq'v$ Haltekonfiguration und $q'\in F$
	\item $M$ \emph{akzeptiert} $L\subseteq\Sigma^*$, falls $M$ akzeptiert $w \<-> w\in L$
	\item $M$ \emph{entscheidet} $L\subseteq\Sigma^*$, falls $M$ akzeptiert $L$ und $M$ hält für jede Eingabe an.
	\item $L\subseteq\Sigma^*$ ist \emph{semi-entscheidbar} (rekursiv aufzählbar), falls $\exists M$, die $L$ akzeptiert.
	\item $L\subseteq\Sigma^*$ ist \emph{entscheidbar} (rekursiv), falls $\exists M$, die $L$ entscheidet.
	\end{itemize}
\end{Def}
\begin{Def}[name={[Laufzeit und Platzbedarf einer \acs*{TM}]}]
	Laufzeit und Platzbedarf einer \ac{TM} $M:$
	
	Laufzeit : $T_M(w) =
	\begin{cases}
		\parbox{.74\textwidth}{\raggedleft Anzahl der Schritte einer kürzesten Berechnung, die zur Akz. von $w$ führt (falls $\exists$)}\\[-1em]
		\text{1, sonst}
	\end{cases}
	$\\
	Platzbedarf : $S_M(w) =
	\begin{cases}
		\parbox{.7\textwidth}{\raggedleft geringster Platzbedarf (Länge einer Konf.) einer akz. Berechnung von $w$ (falls $\exists$)}\\[-1em]
		\text{1, sonst}
	\end{cases}
	$
\textbf{Zeitbeschränkt mit $t(n)$}: $\forall w\in\Sigma^*: |w|\leq n \=> T_M(w)\leq t(n)$,\\
platzbeschränkt analog.
\end{Def}
\begin{Satz}[name={[Zu jeder NTM gibt es \acs*{DTM}]}]\label{satz:6.1}
	Zu jeder NTM gibt es eine \ac{DTM} $M'$, so dass
	\begin{itemize}
	\item $M'$ akzeptiert $L(M)$
	\item $M'$ terminiert gdw. $M$ terminiert
	\item Falls $M$ zeit- und platzbeschränkt ist mit $t(n)$
          bzw. $s(n)$ ($n=$ Länge der Eingabe), dann ist $M'$
          zeitbeschränkt mit $2^{O(t(n))}$ und platzbeschränkt mit
          $O(s(n)\cdot t(n))$. 
	\end{itemize}
\end{Satz}\vspace{-2em}
\begin{proof}
	Die Konfigurationen von $M$ bilden einen Baum, dessen Kanten durch $\vdash$ gegeben sind. Er ist endlich verzweigt, hat aber ggf. unendlich lange Äste.
	
	Definiere eine (Mehrband-)\ac{DTM}, die den Konfigurationsbaum
        systematisch durchläuft und akzeptiert, sobald eine
        Haltekonfiguration erreicht ist, in der $M$ akzeptiert. 
	
	Die \ac{DTM} terminiert ebenfalls, wenn alle Blätter des Baumes besucht worden sind, ohne dass eine akzeptierende Konfiguration gefunden wurde.
	
	Baumsuche mit Kontrollinformation und bereits besuchten Konf. auf ein Extraband.
	\begin{itemize}
	\item Tiefensuche? Nicht geeignet, sie könnte in unendlichen Ast laufen.
	\item Breitensuche? OK, aber Platzbedarf $O(2^{t(n)}\cdot s(n))$
	\item iterative deepending\,: Tiefensuche mit vorgegebener Schranke, bei erfolgloser Suche Neustart mit erhöhter Schranke. \qedhere
	\end{itemize}
\end{proof}
Nächstes Ziel: Charakterisierung von Typ-1 Sprachen.
\begin{Def}[name={[$\DTAPE$ und $\NTAPE$]}]\
\begin{itemize}
\item $\DTAPE(s(n)):$ Menge der Sprachen, die von einer \ac{DTM} in Platz $s(n)$ akzeptiert werden können.
\item $\NTAPE(s(n)):$ Wie für $\DTAPE$, aber mit \ac{NTM}. \qedhere
\end{itemize}
\end{Def}
\begin{Bemerkung}\
	\newcommand{\underarrowset}[2]{%
		\underset{%
			\mathclap{%
				\overset{\displaystyle\uparrow}{\mathclap{#1}}%
			}%
		}{#2}%
	}
	\begin{enumerate}
	\item Für "`$s(n)\leq n$"' betrachte 2-Band \ac{TM}, bei denen die Eingabe read-only ist und nur das zweite Arbeitsband der Platzschranke unterliegt (so ist $s(n)$ sublinear möglich).
	\item Jede Platzbeschränkung impliziert Laufzeitschranke.\\
	Angenommen Platzschranke $s(n)$\\
	$\curvearrowright$ \ac{TM} hat nur endlich viele Konfigurationen
	\[ N := \underarrowset{%
			\parbox{\widthof{\scriptsize Eingabeband}}{\raggedright\scriptsize Kopfpos. im Eingabeband}\hspace{1cm}
		}{n\vphantom{|}}
		|Q| \quad\cdot\quad
		\underarrowset{\parbox{2.2cm}{\scriptsize\centering mögliche Inhalte des Arbeitsbands}}{|\Gamma|}^{s(n)}
		\ \cdot\
		\underarrowset{\hspace{1.7cm}\parbox{2cm}{\scriptsize Kopfpos. auf\\ Arbeitsband}}{s(n)}
		\in 2^{O(\log n + s(n))}
	\]
	\item \ac{DTM} mit Platzschranke\,: $M$ entscheidet,\\
	falls sie akzeptiert, dann in weniger als $N$ Schritten,\\
	falls nach $N$ Schritten keine Termination erfolgt\\
	\quad$\curvearrowright$ Endlosschleife -- Abbruch
	\item \ac{NTM}\,: nutze den \ac{ND} optimistisch aus\,:\\
	falls eine akzeptierende Berechnung existiert, dann muss es eine Berechnung ohne wiederholte Konfiguration geben.
	\end{enumerate}
\end{Bemerkung}
\begin{Satz}[name={[$L\in\DTAPE(n),\ L\in\NTAPE(n)$]}]\label{satz:6.2}\
	\begin{itemize}
	\item $L\in\DTAPE(n) \curvearrowright\ \exists$ \ac{DTM}, die $L$ in Zeit $2^{O(n)}$ entscheidet.
	\item $L\in\NTAPE(n)$ analog.
	\end{itemize}
\end{Satz}\vspace{-2em}
\begin{proof}
	siehe oben.
\end{proof}
\begin{Bemerkung}
	Die Klasse $\NTAPE (n)$ heißt auch \ac{LBA}.
\end{Bemerkung}
%
\begin{Satz}\label{satz:6.3}
	$\mathcal L_1=\mathrm{NTAPE}(n)$
\end{Satz}
\begin{proof}\
\begin{itemize}
	\item["`\=>"'\,:] Sei $G=(N,\Sigma,P,S)$ Typ-1 Grammatik für $L$.\\
		Konstruiere \ac{NTM} $M$ mit $L=L(M),\ \Gamma = \Sigma\cup N\cup\{\blank\}$
		\begin{enumerate}
		\item $M$ rät nicht deterministisch eine Position auf
                  dem Band und eine Produktion $\alpha\-> \beta$. Falls $\beta$ gefunden wird, ersetze durch $\alpha$, weiter bei 1.
		\item Falls Bandinhalt $=S$ \quad stop, akzeptiert.
		\end{enumerate}
		Dieses Verfahren terminiert.
	\item["`\<="'\,:] %
	Gegeben: \ac{NTM} $M$ linear beschränkt.\\
	Gesucht: Typ-1 Grammatik $\mathcal{G}$ mit $L(\mathcal{G})=L(M)$\\
	Idee: 
	\begin{align*}
		a_1\cdots a_n &\-->
			\pmqty{a_1\\a_1} \pmqty{a_2\\a_2} \pmqty{(q,a)\\a_3} \pmqty{a_4\\a_4} \pmqty{a_n\\a_n}
			&&\begin{matrix}\text{Spur 1}\\\text{Spur 2}\end{matrix}\\
	\shortintertext{ad\footnotemark\ Spur 1: Alphabet $\Gamma\cup(Q\x\Gamma) = \triangle$}
		P' &\begin{cases}
			(q,a) &\-> (q',a')   \quad q\in Q,a\in\Gamma\\
			(q,a)b &\-> a'(q',b) \quad b\in\Gamma\\
			b(q,a) &\-> (q',b)a'
		\end{cases}
		& &\begin{aligned}
			\delta(q,a) &\ni (q',a',N)\\
			\delta(q,a) &\ni (q',a',R)\\
			\delta(q,a) &\ni (q',a',L)
		\end{aligned}
	\end{align*}\footnotetext{ad $\approx$ zur}%
	Def. $\widetilde{uqav} = u(q,a)v\ ,\ u,v\in\Gamma^*,a\in\Gamma$\\
	Es gilt: $uqav\vdash^* k' \curvearrowleftright \widetilde{uqav} \overset{*}{\=>} \widetilde{k'}$ mit Produktion $P'$.
	
	Def. $\mathcal{G}$ durch $N = \{S\}\dotcup\triangle\x\Sigma$
	\begin{align*}
		\text{mit } P &= \\
		S &\-> \pmqty{(q_0,a)\\a} &&\forall a\in\Sigma\\
		S &\-> S\pmqty{a\\a} &&\forall a\in\Sigma\\
		\pmqty{\alpha\\a}
			&\-> \pmqty{\beta\\a}
			&&\begin{aligned}
				\forall \alpha\->\beta &\in P'\\
				\alpha,\beta &\in\triangle
			\end{aligned}\\
		\pmqty{\alpha_1\\a_1}\pmqty{\alpha_2\\a_2}
			&\-> \pmqty{\beta_1\\a_1}\pmqty{\beta_2\\a_2}
			&&\begin{aligned}
				\forall \alpha_1\alpha_2\->\beta_1\beta_2 &\in P'\\
				\alpha_i,\beta_i &\in\triangle
			\end{aligned}\\
		\pmqty{x\\a} &\-> a
			&&\begin{aligned}
				x&\in\Gamma\\
				a&\in\Sigma
			\end{aligned}\\
		\pmqty{(q',x)\\a} &\-> a
			&&\begin{aligned}
				x&\in\Gamma, q'\in F, \delta (q',x)=\emptyset\\
				a&\in\Sigma
			\end{aligned}
	\end{align*}
	\begin{align*}
		S &\xRightarrow{*} \pmqty{(q_0,a_1)\\a_1}\pmqty{a_2\\a_2}\dots\pmqty{a_n\\a_n}\\
		&\phantom{{}\xRightarrow{*}{}}\ \acs*{TM}\ \,\dots\\
		&\xRightarrow{*} \pmqty{x_1\\a_1}\dots\pmqty{(q',x_i)\\a_i}\dots\pmqty{x_n\\a_n}\\
		&\xRightarrow{*} a_1\dots a_i\dots a_n
	\end{align*}
	Damit gesehen $L(\mathcal{G})\subseteq L(M)$\\
	Rückrichtung: selbst \qedhere
	\end{itemize}
\end{proof}

\begin{Satz}
	Die Typ-1 Sprachen sind abgeschlossen unter {\thinmuskip=6mu$\cup,\cap,\cdot,{}^*$} und Komplement.
\end{Satz}
\begin{proof}
	Zu $\cup$ und $\cap$ betrachte \ac{NTM}.\\
	Für $\cdot$ und $^*$ konstruiere Grammatik.\\
	ad Komplement "`2. \acsu{LBA}-Problem\footnote{\acs*{LBA} = \acl*{LBA} -- 1964 Kuroda}"' bis 1987, dann gelöst durch Immerman und Szelepcsényi.
	
	1. \ac{LBA}-Problem (1964): Ist $\mathrm{NTAPE}(n) = \mathrm{DTAPE}(n)$? Bisher ungelöst.
\end{proof}
\begin{Satz}
	Das Wortproblem für Typ-1 Sprachen ist entscheidbar.
\end{Satz}
\begin{proof}
	\begin{align*}
		L\in\mathcal{L}_1 &\curvearrowleftright L\in\mathrm{NTAPE}(n)\\
		&\curvearrowright \text{\autoref{satz:6.2} $L$ entscheidbar}
	\end{align*}
	Nach \autoref{satz:6.1} sogar mit \ac{DTM}.
\end{proof}
Die Rückrichtung "`L entscheidbar. $\xcancel{\curvearrowright}\ L$ ist Typ-1 Sprache"' gilt nicht!

\begin{Satz}\label{satz:6.6}
	$\mathcal{L}_0 = \ac{NTM}$
\end{Satz}
\begin{proof}
	\begin{itemize}
	\item["'\=>"'] Kontruktion einer \ac{NTM} $M$ wie in \autoref{satz:6.3}, aber ohne Platzbeschränkung.
	\item["'\<="'] Konstruktion analog zu \autoref{satz:6.3} + Startsymbol $S'$
	\begin{align*}
		S' &\-> \pmqty{\blank\\\Eps} S' \pmqty{\blank\\\Eps} &&\text{Schaffe Platz für Berechnung von }M\\
		S' &\-> S\\
	\shortintertext{Erweitere $N$}
		&= \{S',S\}\cup\triangle\x(\Sigma\cup\{\Eps\})\\
	\shortintertext{Neue Löschregeln:}
		\pmqty{x\\ \Eps } &\-> \Eps &&\forall x\in\Gamma\\
		\rotatebox{90}{$\Rsh$}\ &\mathrlap{\text{die einzigen
                                          Regeln, die Typ-1 Bedingung verletzen.}} \tag*{\qedhere}
	\end{align*}
	\end{itemize}
\end{proof}

\begin{Satz}[name={[Abgeschlossenheit von Typ-0 Sprachen]}]\label{satz:Typ-0-abgeschl}
	Die Typ-0 Sprachen sind unter $\thinmuskip=6mu\cup,\cap,\cdot,{}^*$ abgeschlossen.
\end{Satz}
\begin{proof}
	Konstruiere \ac{NTM} für $\thinmuskip=6mu\cup,\cap$ ; Typ-0-Grammatiken für $\cdot$ und $^*$.
\end{proof}

\begin{Bem}
	Typ-0 Sprachen sind \emph{nicht} unter Komplement abgeschlossen!
\end{Bem}

\subsection{Universelle \acs*{TM} und das Halteproblem}
Ziel: Universelle \ac{TM} (eine \ac{TM}, die \ac{TM}s interpretiert) ist eine \ac{TM} $U$, die zwei Eingaben nimmt:
\begin{enumerate}
\item Kodierung einer (anderen) \ac{TM} $M_0$
\item Eingabe $w$ für $M_0$
\end{enumerate}
so dass
\begin{align*}
	w\in L(M_0) &\curvearrowleftright (M_0,w) \in L(U)\\
	\text{$M_0$ terminiert bei Eingabe $w$} &\curvearrowleftright
                                                  \text{$U$ terminiert
                                                  bei Eingabe $(M_0,w)$}
\end{align*}
Zur Vereinfachung:
\begin{align*}
	\Sigma &= \{0,1\}\\
	\Gamma &=\{\blank,0,1\}\\
\shortintertext{Die Kodierung von $M_0$}
	&= (Q,\dots,q_1,\delta,\dots,\{q_2\})
\end{align*}
mit $Q=\{q_1,q_2,\dots,q_t\}$ ist im wesentlichen die Kodierung von $\delta$. Dazu zwei Hilfsfunktionen:
\begin{tabular}{M{c} | M{c}}
	x\in\Gamma & f(x) \\ \hline
	0          & 1    \\
	1          & 2    \\
	\blank     & 3    \\
\end{tabular}
\quad
\begin{tabular}{M{c} | M{c}}
	D & g(D) \\
\hline
	L & 1    \\
	N & 2    \\
	R & 3
\end{tabular}

Kodiere $\delta$ Zeilenweise:
\begin{align*}
	\delta(q_i,X) &= (q_k,Y,D) \ (= \text{Zeile} z)\\
\shortintertext{durch}
	\mathrm{code}(z) &= 0^i10^{f(x)}10^k10^{f(y)}10^{g(D)}\\
\shortintertext{Wenn $\delta$ durch $s$ Zeilen $z_1\dots z_s$ gegeben,
  dann definiere die \emph{Gödelnummer} von $M_0$ durch}
	\ulcorner M_0 \urcorner &= 111 \mathrm{code}(z_1) 11 \mathrm{code}(z_2) 11 \dots 11 \mathrm{code}(z_s) 111\\
\end{align*}
Definiere $U$ als 3-Band Maschine mit
\begin{itemize}
\item[$B_1:$] Eingabe + Arbeitsband (für $M_0$)
\item[$B_2:$] $\ulcorner M_0 \urcorner$
\item[$B_3:$] $0^k$ für Zustand $q_k$
\end{itemize}
1. Schritt: Transformierte Eingabe
$\ulcorner M_0 \urcorner$
\begin{itemize}
\item Beginnt die Eingabe mit gültiger Gödelnummer?
\item Gleichzeitig: Verschiebe $\ulcorner M_0 \urcorner$ auf $B_2$
\item Wenn Ende von $\ulcorner M_0 \urcorner$ erreicht, schiebe 0 auf $B_3$.
\end{itemize}
Jetzt:
\begin{itemize}
\item[$B_1:$] $w$
\item[$B_2:$] $\ulcorner M_0 \urcorner$
\item[$B_3:$] $0' \sim Z,q$
\end{itemize}


 % Danke @ Lars Nitzke
\vspace{1em}

\begin{Satz}
	Es gibt eine universelle \ac{TM} $U$ mit $L(U) = \{ \ulcorner M \urcorner w\mid w \in L(M)\}$
\end{Satz}
\begin{proof}
	Initialisierung:
	\begin{itemize}
		\item[$B_1:$] $w$ Eingabewort/Arbeitsband
		\item[$B_2:$] $\ulcorner M \urcorner$ Gödelnummer
		\item[$B_3:$] 0 Zustand
	\end{itemize}
	Hauptschleife von U:
	\begin{itemize}
		\item Test auf Haltekonfiguration.
		\item Falls ja: 
		\begin{tabular}[t]{@{}r@{\,}l}
			Falls in $qz$ &: akzeptiert.\\
			sonst&: nicht
		\end{tabular}
		\item Ausführung des nächsten Schritts:\\
		Suche Zeile in $\ulcorner M \urcorner$ gemäß Zustand und aktuellem Symbol auf Arbeitsband.
		Ändere $B_1$ und $B_3$ gemäß $\delta$.\\
		$B_2, B_3$: Kopf zurück zum Anfang.
	\end{itemize}
\end{proof}
\begin{Def}[name={[length-lexicographic order]}]
	Die \emph{length-lexicographic order} auf $\{0, 1\}^*$ (mit $0<1$) ist definiert durch
	\begin{align*}
		v \leq w \<=> &\ |v| < |w|\\
		\vee &\ v = w\\
		\vee &\ |v| = |w| \text{ und } \exists u\in \{0, 1\}^*\\
		& \text{mit } v = u0v' \text{ und } w = u1w'
	\end{align*}
\end{Def}
\begin{Satz}[name={[$\leq:$ totale Ordnung]}]\
	\begin{itemize}
		\item $\leq$ ist totale Ordnung auf $\{0, 1\}^*$
		\item $\exists$ bijektive Abbildung $w: \N \to \{0, 1\}^*$ mit $i \leq j \to w(i) \leq w(j)$
	\end{itemize}
\end{Satz}
\begin{Def}[name={[Diagonalsprache]}]
	Sei $M_i$ die Turingmaschine mit Gödelnummer $\ulcorner M_i\urcorner = w(i)$.
	(falls $w(i)$ kein gültiger Code, dann sei $M_i$ eine beliebige \ac{TM} mit $L(M_i) = \varnothing$). Die \emph{Diagonalsprache}
	\[D = \{w(i) \mid w(i) \notin L(M_i)\}\]
	das heißt $M_i$ akzeptiert $w(i)$ nicht.
\end{Def}
\begin{Satz}[name={[$D$ ist nicht entscheidbar]}]\label{satz:6.10}
	$D$ ist nicht entscheidbar.
\end{Satz}
\begin{proof}
	Angenommen $\exists M$, die $D$ entscheidet.\\
	$M$ muss in Aufzählung vorkommen, das heißt $\exists j\in \N$, sodass $M = M_j$.
	Wende $M_j$ auf $w(j)$ an:
	\begin{itemize}
		\item $M_j$ stoppt/ja: $w(j) \in L(M_j) = D \qquad \lightning$ Def. von $D$
		\item $M_j$ stoppt/nein: $w(j) \notin L(M_j) = D \quad \lightning$ Def. von $D$ \qedhere
	\end{itemize}
\end{proof}
\begin{Korollar}\label{kor:6.11}
	$\overline{D} = \{w(i) \mid M_i\text{ akzepztiert }w(i) \}$ ist nicht entscheidbar.
\end{Korollar}
\begin{proof}
	Angenommen $\overline{D}$ sei entscheidbar durch $M$. Dann entscheidet $M'\ D$. $M'$ führt 
	zuerst $M$ aus und negiert das Ergebnis. $\lightning$ \autoref{satz:6.10}
\end{proof}
\begin{lemma}[name={[$\overline{D}$ ist semi-entscheidbar]}]
	$\overline{D}$ ist semi-entscheidbar.
\end{lemma}
\begin{proof}
	Bei Eingabe $w$
	\begin{itemize}
		\item Falls $w$ kein gültiger Code: stop mit Ergebnis nein.
		\item Falls $w$ gültiger Code, dann $\exists i$, sodass $ww = \left\ulcorner M_j
		\right\urcorner w(i) \qquad w = w(i)$
	\end{itemize}
	Wende $U$ auf $ww$ an.\\
	Insgesamt: \ac{TM}, die $\overline{D}$ akzeptiert.
\end{proof}
\begin{Satz}[name={[{$L,\overline{L}$ semi-entscheidbar $\=> L$ entscheidbar}]},restate={[name=Wiederholung]repeatSatz613}]\label{satz:6.13}
	Falls $L$ semi-entscheidbar und $\overline{L}$ semi-entscheidbar, dann ist $L$ entscheidbar.
\end{Satz}
\begin{proof}
	Sei $M$ die \ac{TM} für $L$, $\overline{M}$ die \ac{TM} für $\overline{L}$.\\
	Führe $M$ und $\overline{M}$ "`parallel"' mit der gleichen Eingabe aus.\\
	Falls $M$ akzeptiert $\Rightarrow$ Ja\\
	Falls $\overline{M}$ akzeptiert $\Rightarrow$ Nein.\\
	Eine muss anhalten, wegen Voraussetzung.
\end{proof}
$D$ \quad nicht entscheidbar\\
$\overline{D}$ \quad nicht entscheidbar\\
$\overline{D}$ \quad semi-entscheidbar (Typ-0)\\
$D$ \quad nicht-semientscheidbar (keine Typ-0)
\begin{Def}[name={[Halteproblem]}]
	Das \emph{Halteproblem} ist definiert durch
	\[H = \{\ulcorner M\urcorner w \mid M \text{ hält bei Eingabe } w \text{ an}\} \qedhere\]
\end{Def}
\begin{Satz}[name={[$H$ ist unentscheidbar]}]\label{satz:H ist unentscheidbar}
	$H$ ist unentscheidbar.
\end{Satz}
\begin{proof}
	Angenommen $M_0$ entscheidet $H$.\\
	Konstruiere $M'$ wie folgt:\\
	Bei Eingabe $w$ bestimme $i$, sodass $w = w(i)$\\
	Verwende $M_0$ um festzustellen, ob $\ulcorner M_i\urcorner w$ anhält.\\
	Antwort von $M_0:$\\
	nein: $\begin{aligned}[t]
		&\curvearrowright w(i) \notin L(M_i)\\
		&\curvearrowright M'\text{ akzeptiert }w\text{ nicht}.
	\end{aligned}$\\
	ja: Führe $\ulcorner M_i\urcorner w$ mit $U$ aus (muss ja terminieren) und akzeptiere entsprechend das Ergebnis von $U$.\\
	Insgesamt: $M'$ entscheidet $\overline{D} \qquad \lightning$ \autoref{kor:6.11}\\
	$\curvearrowright M_0$ existiert nicht.
\end{proof}
\begin{Satz}[name={[$H$ ist semi-entscheidbar]}]
	$H$ ist semi-entscheidbar.
\end{Satz}
\begin{proof}
	Modifiziere $U$, sodass sie jede Eingabe akzeptiert, bei der sie anhält.
\end{proof}
\begin{Korollar}
	$\mathcal{L}_0 \supsetneqq \mathcal{L}_1$
\end{Korollar}
\begin{proof}
	$\overline{D}$ ist unentscheidbar (also $\notin \mathcal{L}_1$), aber semi-entscheidbar (also $\in \mathcal{L}_0$)
\end{proof}
\emph{\textbf{Fragen:}}
\begin{enumerate}
	\item Ist $\mathcal{L}_1$ = Menge der entscheidbaren Sprachen?
	\item[--] Nein:\\
	\begin{minipage}[t]{.7\linewidth}
	Konstruiere eine Aufzählung aller Typ-1 Grammatiken. \medskip\\
	$D_1 = \{w(i) \mid w(i)\notin L(G_i)\}$ ist entscheidbar, weil das Wortproblem für Typ-1 
	Sprachen entscheidbar, aber es $\nexists j$, sodass $L(G_j) = D_1$s
	\end{minipage}\quad
	\begin{tabular}[t]{M{c} | *4{M{c}@{ }}}
		\ &w_1&w_2&w_3&\cdots\\\hline
		G_1 &\\
		G_2 &\\
		G_3 &\\
		\vdots&
	\end{tabular}
	\item Ist $\mathcal{L}_0$ = Menge aller Sprachen?
	\item[--] Nein: $D \notin \mathcal{L}_0$
\end{enumerate}
\begin{Def}[name={[Spezielles Halteproblem $H_\varepsilon$]}]
	Das spezielle Halteproblem $H_\varepsilon = \{\ulcorner M\urcorner \mid M 
	\text{ terminiert auf leeren Band}\}$
\end{Def}
\begin{Satz}[name={[$H_\varepsilon$ ist unentscheidbar]}]
	$H_\varepsilon$ ist unentscheidbar.
\end{Satz}
\begin{proof}
	Angenommen $M_\varepsilon$ entscheidet $H_\varepsilon$.\\
	Konstruiere $M'$ (mit Hilfe von $M_\varepsilon$), sodass $M'$ entscheidet $H$.\\
	\begin{tabular}{r@{ }l}
	$M':$ & Bei Eingabe $\ulcorner M\urcorner w$.\\
	& Konstruiere $M^*$, $M^*$ schreibt zuerst $w$ aufs (leere) Band und 
	startet dann $M$ auf $w$.\\
	& Wende $M_\varepsilon$ auf $\ulcorner M^*\urcorner$ an.
	\end{tabular}\\
	$\curvearrowright M'$ entscheidet $H \qquad \lightning$ \autoref{satz:H ist unentscheidbar}
\end{proof}

 % Latex-Qelle von M. Geffken
Nun betrachten wir \ac{TM}s vom Blickwinkel der von ihnen berechneten
(partiellen) Funktionen. Sei $R$ die Menge der von \ac{TM}s berechneten
Funktionen.

\begin{Satz}[Satz von Rice]\ \\
  Sei $R$ die Menge aller partiellen \ac{TM}-berechenbaren Funktionen und
  $\varnothing \neq S \subsetneq R$ eine nichttriviale (nicht-leere,
  echte) Teilmenge davon.\\
  Dann ist $L(S)=\{\ulcorner M \urcorner \mid M \text{ berechnet Funkt. aus
  }S\}$ unentscheidbar.
\end{Satz}
\begin{proof}
  Angenommen $M_S$ entscheidet $L(S)$.\\
  Sei $\Omega \in R$ die überall
  undefinierte Funktion. Wir nehmen an, dass $\Omega \in S$ (anderenfalls betrachten wir $\overline{L(S)}$).
  
  Da $R \setminus S \neq \varnothing$ folgt $\exists f \in R \setminus
  S$ und $f$ werde von \ac{TM} $M_f$ berechnet.
  
  Definiere $M'=M'_{(M, f)}$ wie folgt: $M'$ führt zunächst $M$ (beliebige \ac{TM}) auf leerer Eingabe aus. Falls $M$ anhält, wendet $M'$ dann $M_f$ auf die tatsächliche Eingabe an.\\
  Die von $M'$ berechnete Funktion ist also
  $f_{M'}=\begin{cases}
    f & \text{falls } M \text{ auf leerem Band hält,}\\
    \Omega & \text{sonst.}
  \end{cases}$
  
  Definiere nun $M''$ wie folgt:
  \begin{itemize}
  \item Bei Eingabe $\ulcorner M \urcorner$ berechne die Gödelnummer von $M'$.
  \item Wende nun $M_S$ auf $\ulcorner M' \urcorner$ an.
    \begin{align*}
      M_s\text{ akzeptiert }\ulcorner M' \urcorner
      &\iff M' \text{ berechnet Funktion in }S\\
      &\iff M' \text{ berechnet }\Omega \text{ ($f_{M'} \in \{\Omega, f\}, \Omega \in S, f \notin S$ )}\\
      &\iff M \text{ hält \emph{nicht} auf leerem Band an}
    \end{align*}
  \end{itemize}
  Also entscheidet $M''$ $H_\varepsilon$.$\qquad\lightning$
\end{proof}

\subsection{Eigenschaften von entscheidbaren und semi-entscheidbaren Sprachen}
\begin{Satz}[name={[Eigenschaften von Entscheidbarkeit]}]
\label{thm:eigensch-von-entsch}
  Seien $L_1$ und $L_2$ entscheidbar. Dann sind $\overline{L_1}$, $\overline{L_2}$, $L_1 \cup L_2$ und $L_1 \cap L_2$ entscheidbar.
\end{Satz}\vspace{-1.5em}
\begin{proof}
  Übung oder selbst.
\end{proof}
\begin{Satz}[name={[Eigenschaften von Semi-Entscheidbarkeit]}]
  Seien $L_1$ und $L_2$ semi-entscheidbar. Dann sind $L_1 \cup L_2$ und $L_1 \cap L_2$ semi-entscheidbar.
\end{Satz}
\begin{proof}
  vgl. \autoref{satz:Typ-0-abgeschl}.
\end{proof}
\csname repeatSatz613\endcsname*
\begin{Satz}
  Die Menge der semi-entscheidbaren Sprachen ist \emph{nicht}
  unter Komplement abgeschlossen.
\end{Satz}
\begin{proof}
  Laut \autoref{satz:6.10} und \autoref{kor:6.11} sind Diagonalsprache $D$ und $\overline{D}$ nicht entscheidbar.
  
  $\overline{D}$ ist semi-entscheidbar: Verwende universelle \ac{TM} $U$ um $\ulcorner M_i \urcorner w(i)$ auszuführen.
  
  Angenommen, $D=\overline{\overline{D}}$ ist ebenfalls semi-entscheidbar\\ $\curvearrowright$ (mit \autoref{satz:6.13}) $D$ ist entscheidbar. $\qquad\lightning$
\end{proof}


\subsection{Weitere unentscheidbare Probleme}
\paragraph[\acf*{PCP}]{\acf{PCP}}\ \\
\emph{Gegeben:}\\
Endliche Folge von Wortpaaren $K=((x_1, y_1), \dots, (x_k, y_k))$ mit $x_i, y_i \in \Sigma^+$

\emph{Gesucht:}\\
Indexfolge $i_1, \dots, i_n \in \{1, \dots, k\}\ (n \geq 1)$, so dass $x_{i_1} \cdots x_{i_n}=y_{i_1} \cdots y_{i_n}$

Die Folge $i_1, \dots, i_n$ (falls diese existiert) heißt \emph{Lösung} des Korrespondenzproblems $K$.
\begin{Bsp*}
\begin{align*}
	K &=((\underbrace{1,101}_{x_1, y_1}), (\underbrace{10, 00}_{x_2, y_2}), (\underbrace{011, 11}_{x_3, y_3}))\\
\shortintertext{besitzt die Lösung $(1,3,2,3)$, denn}
	x_1 x_3 x_2 x_3 &= \underbrace{1 \cdot 01}_{y_1}\underbrace{1 \cdot 1}_{y_3}\underbrace{0 \cdot 0}_{y_2}\underbrace{11}_{y_3} = y_1 y_3 y_2 y_3
\end{align*}
\end{Bsp*}\vspace{-1em}
\emph{Frage:}
\begin{align*}
  x_1&=001 & x_2&=01 & x_3&=01 & x_2&=10\\
  y_1&=0  & y_2&=011 & y_3&=101  & y_2&=001\\
\end{align*}
Besitzt dieses \ac{PCP} eine Lösung? [Schöning, S.124]
\begin{Bemerkung}\ \\
  Offensichtlich ist das \ac{PCP} semi-entscheidbar: Systematisches
  Ausprobieren von Indexfolgen findet Lösung nach endlicher Zeit,
  \emph{sofern es eine gibt}.
\end{Bemerkung}
Ziel: \ac{PCP} ist unentscheidbar. Vorbereitung:
Es interessiert uns ab hier nur, ob das Problem eine Lösung hat oder nicht.
\begin{Def}[Reduktion]\ \\
  Seien $U, V \subseteq \Sigma^*$ Sprachen.\\
  \emph{$U$ ist auf $V$ reduzierbar ($U \preceq V$)}, falls eine totale berechenbare Funktion
  $f:\Sigma^* \to \Sigma^*$ existiert, so dass $\forall x \in \Sigma^*:x \in U \iff f(x) \in V$.
\end{Def}
\begin{lemma}
  Falls $U \preceq V$ und $V$ \mbox{(semi-)entscheidbar}, dann ist auch $U$
  \mbox{(semi-)entscheidbar}.
\end{lemma}\vspace{-1.5em}
\begin{proof}
  Wenn $M$ ein (Semi-)Entscheidungsverfahren für $V$ ist, dann konstruiere $M'$ wie folgt
  \begin{itemize}
  \item wende erst $f$ auf die Eingabe $x$ an ($f$ ist Funktion gemäß Reduktion)
  \item führe $M$ auf dem Ergebnis $f(x)$ aus
  \end{itemize}
  $\curvearrowright$ $M'$ ist (Semi-)Entscheidungsverfahren für $U$
\end{proof}
\emph{Anwendung:} $U \preceq V$ und $U$
unentscheidbar $\curvearrowright$ $V$ unentscheidbar.
\paragraph[\acf*{MPCP}]{\acf{MPCP}}\ \\
\emph{Gegeben:} wie bei \ac{PCP}

\emph{Gesucht:} Lösung des \acsu{CP} mit $i_1=1$
\begin{lemma}[name={[MPCP $\preceq$ PCP]}]
    \ac{MPCP} $\preceq$ \ac{PCP}
\end{lemma}
\begin{proof}
	Betrachte \ac{MPCP} $K=((x_1,y_1),\dots(x_k,y_k))$ über $\Sigma$.\\
	Sei $\Sigma'=\Sigma\cup\{\#,\$\}$ mit neuen Symbolen.\\
	Für ein Wort $w=a_1\dots a_n\in\Sigma^+$ sei
	\begin{align*}
		\bar w &= \#a_1\#a_2\#\dots\#a_n\#\\
		\grave w &= \#a_1\#a_2\#\dots\#a_n &&\text{(am Ende kein \#)}\\
		\acute w &= \phantom{\#}a_1\#a_2\#\dots\#a_n\# &&(\text{am Anfang kein }\#)
	\end{align*}
	Definiere nun
	\[ f(K) = (\underbrace{(\bar x_1,\grave y_1)}_1, \underbrace{(\acute x_1, \grave y_1)}_2, \underbrace{(\acute x_2,\grave y_2)}_{2+1}, \dots, \underbrace{(\acute x_k,\grave y_k)}_{k+1}, \underbrace{(\$,\#\$)}_{k+2}) \]
	eine totale berechenbare Funktion.
	
	Zeige $K \in \ac{MPCP} \<==> f(K) \in \ac{PCP}$:
	
	"`$\==>$"': $1, i_2, \dots, i_n$ Lösung für $K$\\
	$\curvearrowright\ 1, i_{2}+1, \dots, i_{n}+1, k+2$ Lösung für $f(K)$
	
	"`$\<==$"': $i_1, \dots, i_n$ Lösung für $f(K)$\\
	$\curvearrowright i_1=1,\ i_n=k+2 \qquad,\ 1 < j < n: i_j \in \{2, \dots, k+1\}$\\
	$\curvearrowright 1, i_{2}-1, \dots, i_{n-1}-1$ Lösung für $K$
\end{proof}
Es reicht nun zu zeigen, dass \ac{MPCP} unentscheidbar ist!

\begin{lemma}[name={[H $\preceq$ \ac{MPCP}]}] 
	H $\preceq$ \ac{MPCP}
\end{lemma}
\begin{proof}
	\ac{TM} $M=(Q,\Sigma,\Gamma,\delta,q_0,\blank,F)$ und Eingabewort $w\in\Sigma^*$.
	
	Gesucht: totale berechenbare Funktion, die $(\ulcorner M \urcorner, w) \mapsto \underbrace{(x_1,y_1),\dots,(x_k,y_k)}_k$, sodass\\
	$\ulcorner M \urcorner w\in H \<=> K$ eine Lösung als \ac{MPCP} besitzt.
	
	Idee: Definiere $K$ so, dass die Berechnung von $M$ simuliert wird.\\
	Alphabet für $K: \triangle = \Gamma\cup Q\cup\{\#\}$\\
	$(x_1,y_1) = (\#,\#q_0w\#)$
	\begin{enumerate}
	\item Kopieren
		\[ (a,a)\quad, a\in\Gamma\cup\{\#\} \]
	\item Transition
		\begin{alignat*}{2}
			(qa,q'a') &\quad& \forall q,a:\delta(q,a) &\ni(q',a',N)\\
			(qa,a'q') && \dots\quad &\ni(q',a',R)\\
			(bqa,q'ba') &&  &\ni(q',a',L),b\in\Gamma\\
			(q\#,q'a'\#) && \forall q: \delta(q,\blank)&\ni(q',a',N)\\
			(q\#,a'q'\#) && &\ni(q',a',R)\\
			(bq\#,q'ba'\#) && &\ni(q',a',L),b\in\Gamma\\
			(\#qa,\#q'\blank a')&& \forall q,a:\delta(q,a) &\ni(q',a',L)\\
			(\#q\#,\#q'\blank a'\#)&& \forall q:\delta(q,\blank) &\ni(q',a',L)
		\end{alignat*}
	\item Löschen
		\begin{align*}
			\forall q\in \Gamma\\
			\forall q\in F &:(aq,a)\\
			&\phantom{{}:{}}(qa,a)
		\end{align*}
	\item Abschluss
		\[ \forall q\in F: (q\#\#,\#) \]
	\end{enumerate}
	$\ulcorner M \urcorner w\in H \<==>$ Folge von $\Konf$ von $M,\ k_0\dots k_t$ mit $k_0 =q_0w$ und $k_t = uq'v$ mit $q'\in F$\\
	mit $h_{i-1}\vdash k_i\quad \forall 1\leq i\leq t$
	\<==> Die Instanz $K$ von \ac{MPCP} besitzt Lösung und ein Lösungswort der Form
	\[ \#k_0\#k_1\# \dots \#h_t\#k_t^1\#k_t^2\# \dots \#q'\#\# \]
	wobei die $k_t^j$ durch Streichen eines Bandsymbols rechts oder links von $q'$ aus ihrem Vorgänger entsteht.
\end{proof}
\begin{Satz}[name={[\ac{PCP} ist unentscheidbar.]}]
	\ac{PCP} ist unentscheidbar.
\end{Satz}
\begin{proof}
	$\text{H}\leq\ac{MPCP}$ und $\ac{MPCP}\leq\ac{PCP}$
\end{proof}


\begin{Satz}
	Das Schnittproblem "`$L(\mathcal{G}_1)\cap L(\mathcal{G}_2)\neq \varnothing$?"' für \ac{CFL} ist unentscheidbar.
\end{Satz}
\begin{proof} Durch Reduktion $\ac{PCP}\leq{}$Schnittproblem.\\
	Sei $K=\{(x_i,y_i) \mid 1\leq i\leq k\}$ Instanz von \ac{PCP} über $\Sigma$.\\
	Berechne aus $K$ zwei \ac{CFG} $\mathcal{G}_1$ und $\mathcal{G}_2$, so dass $K$ eine Lösung hat $\<==> L(\mathcal{G}_1)\cap L(\mathcal{G}_2)\neq \varnothing$
	\begin{align*}
		\mathcal{G}_1: S_1\-> &1x_1 |\dots |kx_k &&\text{Alphabet}: \Sigma\cup\{1\dots k\}\\
		& |1S_1x_1|\dots|kS_1x_k\\
		\mathcal{G}_2: S_2\-> &1y_1 |\dots |ky_k\\
		&|1S_2x_1|\dots|kS_2y_k
	\end{align*}
	\begin{alignat*}{2}
		&&w&\in L(\mathcal{G}_1)\cap L(\mathcal{G}_2)\\
		\<==>&\ &  w &= k_n\dots k_1,xk_1\dots xk_n\\
		&&&= k_n\dots k_1,yk_1\dots yk_n\\
		\<==>&& \mathrlap{\hspace{-1ex}(k_1\dots k_n)\text{ ist Indexfolge zur Lösung von \ac{PCP} } k} \tag*{\qedhere}
	\end{alignat*}
\end{proof}
\paragraph{Folgerung :}Schnittproblem für Typ 1 und Typ 0 Sprachen ist ebenfalls unentscheidbar.

\begin{Korollar}
	Das Schnittproblem ist auch für \ac{DCFL} unentscheidbar.
\end{Korollar}
\begin{proof}
	$L(G_1)$ ist auch \ac{DPDA} erkennbar.
\end{proof}

\begin{Satz}
	Das Äquivalenzproblem für \ac{CFL} ist unentscheidbar.
\end{Satz}
\begin{proof}
	Sei $A=\{\mcG_1,\mcG_2 \mid L(\mcG_1)=L(\mcG_2)\}$\\
	Angenommen $\mcG_1,\mcG_2$ sind Typ 2 Grammatiken für \ac{DCFL}.\\
	Dann ist $(\mcG_1,\mcG_2)\in{}$Schnittproblem.
	\begin{align*}
		\<==> L(\mcG_1) &\cap L(\mcG_2)=\varnothing\\
		\<==> L(\mcG_1) &\subseteq \overline{L(\mcG_2)}
	\end{align*}
	Da $\mcG_2$ eine \ac{DCFG} $\exists\,\mcG_2$ mit $L(\mcG'_2)=\overline{L(\mcG_2)}$ (Abschluss unter Komplement).
	\begin{align*}
		\<==>\quad & L(\mcG_1) \subseteq L(\mcG'_2) \quad\leadsto\text{Inklusionsproblem} \tag{$*$}\label{eq:Inklusionsproblem}\\
		\<==>\quad & L(\mcG_1) \cup L(\mcG'_2) = L(\mcG'_2)\\
	\shortintertext{Wegen Abschluss unter $\cup:\exists\,\mcG_3\in\ac{CFG}$ mit $L(\mcG_3)=L(\mcG_1)\cup L(\mcG'_2)$}
		\<==>\quad & L(\mcG_3)=L(\mcG'_2)\\
		\<==>\quad & (\mcG_3,\mcG'_2)\in A
	\end{align*}
	$\curvearrowright$ Äquivalenzproblem ist unentscheidbar.\\
	\eqref{eq:Inklusionsproblem} \-> (Inklusionsproblem ist ebenfalls unentscheidbar.)
\end{proof}

\begin{Satz}
	Das Leerheitsproblem für Typ 1 Sprachen ist unentscheidbar.
\end{Satz}
\begin{proof}
	Reduktion auf Schnittproblem für \ac{CFL}.\\
	Sei $(\mcG_1,\mcG_2)\in{}$Schnittproblem (Typ 1).\\
	Insbesondere $\mcG_1,\mcG_2$ Typ 1 Grammatiken.\\
	Typ 1 Sprachen sind unter $\cap$ abgeschlossen, also $\exists\, \mcG$ Typ 1 Gramatik mit $L(\mcG)=L(\mcG_1)\cap L(\mcG_2)$\\
	Also "`$L(\mcG)=\varnothing$"' unentscheidbar.
\end{proof}


%%% Local Variables:
%%% mode: latex
%%% TeX-master: "Info_3_Skript_WS2016-17"
%%% End:
