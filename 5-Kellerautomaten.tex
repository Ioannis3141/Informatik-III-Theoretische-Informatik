\section[Kellerautomaten (\acs*{PDA})]{Kellerautomaten \quad\normalfont\normalsize \acf{PDA}}
Kellerautomat $\approx$ Endlicher Automat + Kellerspeicher von unbeschränkter Größe (Stack, push down)
\paragraph*{Neu:}
\begin{itemize}
	\item bei jedem Schritt darf der \ac{PDA} das oberste Kellersymbol inspizieren und durch beliebiges Kellerwort ersetzen.
	\item der \ac{PDA} darf auf dem Keller rechnen, ohne in der Eingabe weiter zu lesen. ($\epsilon$-Transition oder Spontantransition).
\end{itemize}
\begin{Bsp}
	\begin{align}
		\Sigma &= \{a,b,\#\} &\qquad&\text{Eingabealphabet} \notag\\
		\Gamma &= \{a,b,\bot\} &&\text{Kelleralphabet} \notag\\
		Q &= \{q_0,q_1\} &&\bot\,\hat=\,\text{Kellerbodensymbol} \notag\\
		\delta: (q_0,x,Z) &\mapsto (q_0,xZ) && x\in\{a,b\},Z\in\Gamma\\
		(q_0,\#,Z) &\mapsto (q_1,Z) \label{eq:5.2}\\
		(q_1,x,x) &\mapsto (q_1,\epsilon)\\
		(q_1,\epsilon,\bot) &\mapsto (q_1,\epsilon)\\
		\text{Konfiguration: } (q_0,&w,\bot) &&\text{Start}\\
		&\downarrow\\
		(q',&\,\epsilon,\epsilon) &&\text{akzeptiert!}
	\end{align}
	Erkannte Sprache $L=\{w \# w^R \mid w\in\{a,b\}^*\}$
	\begin{align*}
		&(q_0,abb\#bba,\bot)\\
		\vdash\quad &(q_0, bb\#bba, a\bot)\\
		\vdash\quad &(q_0, b\#bba, ba\bot) \vdash (q_0, \#bba, bba\bot)\\
		\vdash\quad &(q_1, bba, bba\bot)\\
		\vdash\quad &(q_1, ba, ba\bot) \vdash (q_1, a, a\bot)\\
		\vdash\quad &(q_1, \epsilon, \bot) \vdash (q_1,\epsilon,\epsilon) \qquad\text{akzeptiert!}
	\end{align*}
	deterministisch.
\end{Bsp}
Nicht deterministisch: $L'=\{ww^R \mid w\in \{a,b\}^*\}$\\
Ersetze \eqref{eq:5.2} durch $(q_0,x,Z) \mapsto (q_1,Z) \quad Z\in\Gamma$

\underline{Alle Palindrome} zusätzlich $(q_0,x,Z)\mapsto(q_1,Z) \ Z\in\Gamma,\ x\in\{a,b\}$ auch nicht deterministisch.

\begin{Def}[name={[NPDA]}]
	Ein \ac{NPDA} ist Tupel $(Q,\Sigma,\Gamma,q_0,Z_0,\delta)$
	\begin{itemize}
		\item $Q$ endl. Zustandsmenge
		\item $\Sigma$ endl. Eingabealphabet
		\item $\Gamma$ endl. Kelleralphabet
		\item $q_0\in Q$ Startzustand
		\item $Z_0\in\Gamma$ Kellerbodensymbol
		\item $\delta: Q\x(\Sigma\cup\{\epsilon\})\x\Gamma \-> \mathcal{P}(Q\x\Gamma^*)$ \qedhere
	\end{itemize}
\end{Def}
Im weiteren sei $M=(\dots)$ ein \ac{NPDA}.
\begin{Def}[name={[Menge der Konfigurationen eines \acs*{NPDA}]}]
	Die Menge der Konfigurationen von $M$ ist $\Konf(M) = Q\x\Sigma^*\x\Gamma^*$\\
	Die Schrittrelation von $M$ ist:
	\begin{align*}
		&\vdash\ \subseteq \Konf(M)^2 &&\text{definiert durch}\\
		(q,aw,Z\gamma) &\vdash (q',w,\beta\gamma) &&\text{falls }\delta(q,a,z)\ni(q',\beta)\\
		(q,w,Z\gamma) &\vdash (q',w,\beta\gamma) &&\text{falls }\delta(q,\epsilon,z)\ni(q',\beta)\\
	\shortintertext{Die von $M$ erkannte Sprache}
		L(M) = \{ w\in\Sigma^* \mid (q_0,w,Z_0) &\vdash^{\!\!*} (q',\epsilon,\epsilon) \} \tag*{\qedhere}
	\end{align*}
\end{Def}
\begin{Satz}\label{satz:5.1}
	\[
		L \in \ac{CFL} \xLeftrightarrow{\text{gdw.}}
		L =L(M) \quad\text{für \ac{NPDA} $M$.}
	\]
\end{Satz}
\begin{proof}
	\begin{itemize}
	\item["'\=>"'] Sei $L\in\ac{CFL}$\\
		Sei $\mathcal{G} = (N,\Sigma,P,S)$ Grammatik für $L\setminus \epsilon$ in \ac{CNF}. \\
		Def. \ac{NPDA} M durch
		\begin{alignat*}{2}
			Q &= \{q\} &\quad(=&\text{ Startzustand})\\
			\Gamma &= \Sigma\dotcup N\\
			Z_0 &= S\\
			\delta(q,a,a) &= \{(q,\epsilon)\} & a&\in\Sigma\\
			\delta(q,\epsilon,A) &= \{(q,\alpha)\} & A&\->\alpha\in P
		\end{alignat*}
		\begin{alignat*}{2}
		\text{f"ur alle } X&\in\Sigma\cup N \ ,\ v\in\Sigma^*\\ 
			\text{Zeige }&X\overset{*}{\==>} v &
			\curvearrowleftright\quad& \mathrlap{(q,v,X) \vdash^{\!\!*} (q,\epsilon,\epsilon)}
		\end{alignat*}
		per Induktion über Ableitungsbaum von $X\overset{*}{\==>}v$\\
		(Höhe von)
		\begin{description}
			\item[Höhe $0$:]
				\begin{align*}
					\bullet\quad  a &= X = v\\
					(q,a,a) &\vdash (q,\epsilon,\epsilon)\ \checkmark
				\end{align*}
			\item[Höhe $>0$:]\
			\begin{enumerate}
			\item Fall:
				\tikz[baseline=-.3em]{
					\Tree [.\node [label={right:$A=X$}] {$\bullet$};
						\node [label={right:$a$}] {$\bullet$};
					]
				}
				\quad $(q,a,A) \overset{A\-> a}{\vdash} (q,a,a) \vdash (q,\epsilon,\epsilon)$
			\item Fall:
				\tikz[baseline, level 1/.style={sibling distance=1em}]{
					\Tree [.\node[label={right:$A = X$}] {$\bullet$};
						[.\node[label={left:$B_1$}] {$\bullet$}; \edge[roof]; {$v_1 $} ]
						[.\node[label={right:$B_2$}] {$\bullet$}; \edge[roof]; {$v_2 $} ]
					]
				} Regel $A\-> B_1B_2$
				\begin{align*}
					\text{d.h. }v = v_1v_2 \quad B_1 &\overset{*}{\==>} v_1 \text{ mit kleinerem Abeitungsbaum}\\
					B_2 &\overset{*}{\==>} v_2 \text{ \ruleplaceholder{\widthof{mit kleinerem Abeitungsbaum}}}\\
					(q,v_1v_2,A) &\vdash (q,v_1v_2,B_1B_2)\\
					&\vdash^* (q,v_2,B_2) \quad \text{nach I.V. für }B_1\\
					&\vdash^* (q,\epsilon,\epsilon) \quad \text{nach I.V. für }B_2\\
					\text{D.h. } S\overset{*}{\==>} w &\<==> (q,w,S) \vdash^* (q,\epsilon,\epsilon)\\
					&\<==> w\in L(M)
				\end{align*}
			\item["'\<="'] Gegeben \ac{NPDA} $M=(Q,\dots)$ bei dem alle Transitionen pop, top oder push sind.
			\end{enumerate}
		\end{description}
	\end{itemize}
	\begin{figure}[H]\centering\vspace{-1em}
		\input{2015-12-22_diagramm.pdf_tex}
		{\setlength{\belowdisplayskip}{-.5em}
		\begin{align*}
			(q,Z,q') &\overset{*}{\Rightarrow} v\\
			(q,v,Z) &\vdash^* (q',\epsilon,\epsilon)
		\end{align*}}
		\caption{Beweis zu \autoref{satz:5.1}}
	\end{figure}\vspace{-2.5em}\qedhere\\*
    Grammatik $\mathcal{G}'$ leitet gerade $v$ ab, falls $Z$ im Stack.
\end{proof}

\begin{lemma}
	Zu jedem \ac{NPDA} gibt es einen äquivalenten \ac{NPDA}, so dass
	falls $\delta(q,x,Z)\ni (q',\alpha) \quad x\in\Sigma\cup\{\epsilon\}$
	dann ist entweder
	\begin{itemize}
		\item $\alpha = \epsilon$
		\item $\alpha = Z'$
		\item $\alpha = Z'Z''$
	\end{itemize}
	\vspace{-1.5em}
\end{lemma}
\begin{proof}
	Sei $(q',\alpha)\in\delta(q,x,Z)$ mit $\alpha = Z_n\dots Z_1$ f"ur $n>2$:
	\begin{itemize}
	\item 	neue Zustände $q_2\dots q_{n-1}$
	\item Ersetze $(q',\alpha)$ durch $(q_2, Z_2Z_1)$
	\item Definiere $\delta(q_i, \epsilon, Z_i) = \{ (q_{i+1}, Z_{i+1}Z_i) \}$, f"ur $2\le i < n-1$
	\item Definiere $\delta(q_{n-1}, \epsilon, Z_{n-1}) = \{ (q', Z_nZ_{n-1}) \}$
	\end{itemize}
	Wiederhole bis alle Transitionen die gew"unschte Form haben. \qedhere
\end{proof}

%
\datenote{22.12.15}
\vspace{3em}

%\rlerror{Anfang unvollständig (Aufschrieb fehlt!)}

$L=L(M)$ für \ac{NPDA} $M$\\
$\curvearrowright\ L$ ist \ac{CFL}

%\begin{figure}[H]\centering
%\vspace{2em}
%\rlerror*{Grafik fehlt}{<Bild>}% Handschr. Mitschrieb vorhanden.
%\caption{
%}
%\end{figure}

\begin{itemize}
\item Def. \ac{CFG} mit $N= Q\x \Gamma\x Q\cup \{S\}$, so dass $(q,Z,q')\overset{*}{\=>} v$ gdw.\ 
$(q, v, Z) \vdash^* (q',\epsilon,\epsilon)$
\\
	Transitionen von $M$ haben eine von drei Formen:
	\begin{align*}
		&(q,\epsilon) && (q,Z) && (q,Z_1 Z_2)
	\end{align*}
\item Def. $P$ durch $P \supseteq \{ S \-> (q_0,Z_0,q') \mid q'\in Q\}$ sowie die weiteren Produktion wie folgt
\end{itemize}
Produktion für $(q,Z,q')$
\begin{enumerate}[label={Fall \arabic*:},ref={Fall \arabic*},leftmargin=*]
\item\label{itm:Fall 1} $\delta(q,x,Z)\ni (q',\epsilon)\ , x\in \Sigma\cup \{\epsilon\}$\\
	$P \supseteq \{ (q,Z,q')\-> x \}$
\item $\delta(q,x,Z)\ni (q'',Z')\ , x\in \Sigma\cup \{\epsilon\}$\\
	$P \supseteq \{ (q,Z,q')\-> x(q'',Z',q') \mid q'\in Q\}$
\item $\delta(q,x,Z)\ni (q_1,Z_1Z_2)\ , x\in \Sigma\cup \{\epsilon\}$\\
	$P \supseteq \{ (q,Z,q')\-> x(q_1,Z_1,q_2)(q_2,Z_2,q') \mid q_1,q_2\in Q \}$
\end{enumerate}
Korrektheit: Zeige $(q,Z,q')\overset{*}{\=>} w\curvearrowleftright (q,w,Z) \vdash^* (q',\epsilon,\epsilon)$\\
Induktion über Ableitungsbaum von $(q,Z,q')\overset{*}{\=>} w$ (siehe oben)
\begin{itemize}[align=left]
\item[Höhe 1:] \tikz[baseline]{\Tree[.$(q,Z,q')$ $x$ ]} \quad \ref{itm:Fall 1} $\curvearrowleftright (q,x,Z)\vdash (q',\epsilon,\epsilon)$
\item[Höhe $>1$:] Zwei Möglichkeiten:
	\begin{enumerate}[label={Fall \arabic*:},ref={Fall \arabic*},leftmargin=*,start=2]
	\item \ \vspace{-1em}
	\begin{align*}
		\raisebox{0pt}[0pt][0pt]{\tikz[baseline]{\Tree[.$(q,Z,q')$ $x$ $(q'',Z,q')$ ]}}
			&\curvearrowright w = xw'\land (q'',Z',q')\overset{*}{\=>} w'\\
			&\begin{aligned}
				\curvearrowright (q,xw',Z) &\vdash (q'',w',Z')\\
				\text{I.V.}\quad &\vdash^* (q',\epsilon,\epsilon)
			\end{aligned}
	\end{align*}
	\item \tikz[baseline]{\Tree[.$(q,Z,q')$ $x$ $(q_1,Z_1,q_2)$ $(q_2,Z_2,q')$ ]}
		\begin{align*}
			w &= xw_1w_2\\
			(q_1,Z_1,q_2) &\overset{*}{\=>} w_1 \quad\text{mit kleinerem Ableitungsbaum}\\
			(q_2,Z_2,q') &\overset{*}{\=>} w_2 \quad\text{mit kleinerem Ableitungsbaum}\\
			\curvearrowright (q,xw_1w_2,Z) &\vdash (q_1,w_1w_2,Z_1Z_2)\\
			\text{I.V. für }w_1\text{ + Lemma} &\vdash^* (q_2,w_2,Z_2)\\
			\text{I.V. für }w_2\phantom{\text{ + Lemma}} &\vdash^* (q',\epsilon,\epsilon)
		\end{align*}
	\end{enumerate}
\end{itemize}

\begin{Def}[name={[DPDA]}]
	Ein \ac{DPDA} ist ein Tupel $(\underbrace{Q,\Sigma,\Gamma,q_0,Z_0}_{\text{wie gehabt}},\delta,F)$
	\vspace{-1em}
	\begin{itemize}
	\item $F\subseteq Q$ akzeptierende Zustände
	\item $\delta: Q\x (\Sigma\cup\{\epsilon\})\x \Gamma \-> \mathcal{P}(Q\x \Gamma^*)$\\
	$|\delta(q,a,Z)| + |\delta(q,\epsilon,Z)| \leq 1 \quad \forall q\in Q,a\in\Sigma,Z\in\Gamma$
	\item $\vdash$ wie gehabt
	\item $L(M) = \{w\in\Sigma^* \mid (q_0,w,Z_0) \vdash^* (q',\epsilon,\gamma) \land q'\in F \}$ \qedhere
	\end{itemize}
\end{Def}

\begin{lemma}[name={[\acs*{DPDA}, der gesamte Eingabe verarbeitet]}]
	\label{lem:DPDA ges. Eingabe}
	Zu jedem \ac{DPDA} gibt es einen äquivalenten \ac{DPDA}, der die gesamte Eingabe verarbeitet.
\end{lemma}
\begin{proof}
    Erste Möglichkeit: Die Transitionsrelation ist nicht total.

	Führe einen neuen Zustand $q_s\notin F$ ein.
	\begin{align*}
	\shortintertext{Falls $\exists a\in\Sigma, q, Z$}
		\delta(q,a,Z) &\cup \delta(q,\epsilon,Z) = \varnothing\\
	\shortintertext{dann erweitere $\delta$ um}
		\delta(q,a,Z) &= \{ (q_s,Z) \}\\
	\shortintertext{und $\forall a\in\Sigma, Z\in \Gamma: \delta(q_s,a,Z) = \{(q_s,Z)\}$}
	\end{align*}
	Weitere Möglichkeit: der Automat bleibt wegen leerem Keller stecken.\\
	Abhilfe: Neues Kellerbodensymbol $Z_0'$ und neuer Startzustand $q_0'$.
	\begin{align*}
		\delta(q_0',\epsilon,Z_0') &= \{(q_0,Z_0Z_0')\}\\
	\shortintertext{Erweitere $\delta$ für alle originalen zustände $q\in Q$ um}
		\delta(q,\epsilon,Z_0') &= \{(q_s,Z_0')\}
	\end{align*}
	"`Falls Keller abgeräumt, Wechsel nach $q_s$"'
\end{proof}

\begin{Satz}[name={[Abgeschlossenheit der deterministischen \acs*{CFL}]}]
	Die deterministischen \ac{CFL} sind unter Komplement abgeschlossen.
\end{Satz}
\begin{proof}
	Sei $L=L(M)$ für \ac{DPDA} $M$. Nach \autoref{lem:DPDA ges. Eingabe} liest $M$ die komplette Eingabe.\\
	Def. $M'$ mit $Q'= q\x \{0,1,2\}$\\
	Zustand $(q,0)$: seit Lesen des letzten Symbols wurde kein akz. Zustand durchlaufen.\\
	Zustand $(q,1)$ seit Lesen \dots$\geq 1$ akz. Zustand durchlaufen\\
	\phantom{Zustand} $(q,2)$ \dots akz. Zustände d.h. $F'=F\x \{2\}$
	\begin{align*}
		q_0' &= \begin{cases}
			(q_0,0) & q_0\notin F\\
			(q_0,1) & q_0\in F
		\end{cases}\\
		\text{Falls }\delta(q,\epsilon,Z) = \{(q',\gamma)\}\text{ dann}&\\
		\delta'((q,0),\epsilon,Z) &= \begin{cases}
				((q',0),\gamma) & q'\notin F\\
				((q',1),\gamma) & q'\in F
			\end{cases}\\
		\delta'((q,1),\epsilon,Z) &= ((q',1),\gamma)\\
		\text{Falls }\delta(q,a,Z) = \{(q',\gamma)\}& \\
		\delta'((q,0), \epsilon, Z) &= \{ ((q,2), Z) \} \\
		\delta'((q,2), a, Z) &=
		    \begin{cases}
		        ((q',0), \gamma) & q' \notin F\\
		        ((q',1), \gamma) & q' \in F
		    \end{cases}\\
		\delta'((q,1), a, Z) &=
            \begin{cases}
                ((q',0), \gamma & q' \notin F\\
                ((q',1), \gamma & q' \in F
            \end{cases}
        \\ \tag*{\qedhere}
	\end{align*}
\end{proof}
\begin{Satz}
    Die deterministischen \ac{CFL} sind \textbf{nicht} unter Vereinigung und Durchschnitt abgeschlossen.
\end{Satz}
\begin{proof}
    Betrachte
    \begin{align*}
        L_1 &= \{ a^nb^nc^m \mid n, m \ge 1 \} \\
        L_2 &= \{ a^mb^nc^n \mid n, m \ge 1 \}
    \end{align*}
    Sowohl $L_1$ als auch $L_2$ sind DCFL, aber $L_1 \cap L_2 = \{ a^nb^nc^n \mid n \ge 1\}$ ist nicht kontextfrei.
    
    DCFL ist nicht abgeschlossen unter Vereinigung. Angenommen doch: Seien $U, V$ DCFL. Dann sind auch $\overline{U}$ und $\overline{V}$ DCFL. Bei Abschluss unter Vereinigung wäre $\overline{U} \cup \overline{V}$ eine DCFL und somit auch $\overline{\overline{U} \cup \overline{V}} = U \cap V$, ein Widerspruch gegen den ersten Teil.
\end{proof}
\begin{Satz}
    DCFL ist abgeschlossen unter Schnitt mit REG.
\end{Satz}
\begin{proof}
    Sei $L$ DCFL und $R$ regulär.
    Bilde das Produkt aus einem DPDA für $L$ und einem DFA für $R$.
    Offenbar ist das Ergebnis ein DPDA, der $L\cap R$ erkennt.
\end{proof}
\begin{Satz}
    Sei $L$ DCFL und $R$ regulär.
    Es ist entscheidbar, ob $R=L$, $R\subseteq L$ und $L=\Sigma^*$.
\end{Satz}
\begin{proof}
    Es gilt $R\subseteq L$ gdw.\ $R \cap \overline{L} = \emptyset$.
    
    Weiter ist $R = L$ gdw.\ $R\subseteq L$ und $L \subseteq R$. Für den zweiten Teil betrachte $L\cap \overline{R}$.
    
    Für kontextfreie Sprachen ist $L\ne \emptyset$ entscheidbar, also betrachte $L=\Sigma^*$ gdw.\ $\overline{L}=\emptyset$.
\end{proof}

\begin{Satz} \textbf{DPDA Äquivalenzproblem}
    Seien $L_1, L_2$ DCFL. Dann ist $L_1 = L_2$ entscheidbar.
\end{Satz}

\begin{proof}
    Siehe Senizergues (2000) und Stirling (2001).
\end{proof}
